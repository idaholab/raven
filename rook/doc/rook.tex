\documentclass{article}

\title{Rook Test Description}

\begin{document}
\maketitle

\section{Testers}

The testers are used to run a test.  The tester can fail the test, or
it can be failed by one of the differs that are attached as subnodes
in the test input.

\subsection{Common Features}

These are the features that are common to all testers.

\begin{description}
  \item[type] The type of this test
  \item[skip] If true skip test
  \item[prereq] list of tests to run before running this one
  \item[max\_time] Maximum time that test is allowed to run in seconds
  \item[os\_max\_time] Maximum time by os. Example: Linux 20 Windows 300 OpenVMS 1000
  \item[heavy] If true, run only with heavy tests
  \item[output] Output of the test
  \item[expected\_fail] if true, then the test should fails, and if it passes, it fails
  \item[run\_types] The run types that this test is
  \item[output\_wait\_time] Number of seconds to wait for output
\end{description}

\subsection{GenericExecutable}

This tester runs an executable, and the test passes if the executable
returns a 0 as the return code.  This test can also use the differs to
check for output.

\begin{description}
  \item[executable] The executable to use
  \item[parameters] arguments to the executable
\end{description}

\section{Differs}

\subsection{Common Features}

\begin{description}
  \item[type] The type of this differ
  \item[output] Output files to check
  \item[gold\_files] Gold filenames
\end{description}

\subsection{Exists}

This checks that a file exists, and if it does, it passes.  There are
no specific tags for the Exists differ.

\subsection{XML}

This can compare two XML files to see if they are the same.

\begin{description}
  \item[unordered] if true allow the tags in any order
  \item[zero\_threshold] it represents the value below which a float is considered zero
  \item[remove\_whitespace] Removes whitespace before comparing xml node text if True
  \item[remove\_unicode\_identifier] if true, then remove u infront of a single quote
  \item[xmlopts] Options for xml checking
  \item[rel\_err] Relative Error for floating point numbers
\end{description}

\subsection{OrderedCSV}

This can compare two CSV files to see if they are the same.

\begin{description}
  \item[rel\_err] Relative Error for csv files
  \item[zero\_threshold] it represents the value below which a float is considered zero
  \item[ignore\_sign] if true, then only compare the absolute values
  \item[check\_absolute\_value] if true the values are compared to the tolerance directectly, instead of relatively.
\end{description}

\end{document}
