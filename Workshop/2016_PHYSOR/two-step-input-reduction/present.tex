\documentclass{beamer}

\mode<presentation>
{
  \usetheme{INL}
  % or ...

  \setbeamercovered{transparent}
  % or whatever (possibly just delete it)
}


\usepackage[english]{babel}
\usepackage[utf8]{inputenc}
\usepackage{times}
\usepackage[T1]{fontenc}
\usepackage{graphicx}
\usepackage[compatibility=false]{caption}
\usepackage{subcaption}
\usepackage{physics}
\usepackage{amsmath}
\usepackage{amssymb}

%\usepackage{multimedia}
%\usepackage{movie9}


\newcommand{\expv}[1]{\ensuremath{\mathbb{E}[ #1]}}
\newcommand{\xs}[2]{\ensuremath{\Sigma_{#1}^{(#2)}}}
\newcommand{\intO}{\ensuremath{\int\limits_{4\pi}}}
\newcommand{\intz}{\ensuremath{\int\limits_0^1}}
\newcommand{\intf}{\ensuremath{\int\limits_{-\infty}^\infty}}
\newcommand{\intzf}{\ensuremath{\int\limits_{0}^\infty}}

\title[Multistep Input Reduction]
{Multistep Input Reduction \\in RAVEN}

%\subtitle
%{A Term Project}

\author[Talbot] % (optional, use only with lots of authors)
{Paul Talbot\inst{}}


\institute[University of New Mexico] % (optional, but mostly needed)
{
  %\inst{1}%
  %University of New Mexico\\
  \inst{}
  paul.talbot@inl.gov%Idaho National Laboratory
}

\date[RAVEN Workshop] % (optional, should be abbreviation of conference name)
{RAVEN Workshop 2016}


\subject{Multistep Uncertainty Quantification}

%\pgfdeclareimage[height=0.75cm]{university-logo}{graphics/INL}
%\logo{\pgfuseimage{university-logo}}

\addtobeamertemplate{navigation symbols}{}{
  \usebeamerfont{footline}%
  \usebeamercolor[fg]{footline}%
  \hspace{1em}%
  \insertframenumber/\inserttotalframenumber
}

\begin{document}
\title[RAVEN Workshop]{RAVEN Workshop}
\subtitle{Stochastic Analysis Framework}
\institute[INL]{Nuclear Engineering Methods Development Department \\ Idaho National Laboratory}

\begin{titleframe}{RAVEN Workshop}
\end{titleframe}

\begin{frame}{Outline}{Discussion Points}\vspace{-20pt}
  \tableofcontents%[pausesections]
  % You might wish to add the option [pausesections]
\end{frame}

%                                                  %
%                 OVERVIEW                         %
%                                                  %
\section{Overview}
% UQ is useful in reactor physics
\begin{frame}{Overview}{What we'll learn in this session}\vspace{-30pt}
  %benefits, uses of UQ
  Goal:

  Reduce high-dimension input spaces
  \vspace{10pt}
  \begin{itemize}
    \item Use PCA to eliminate correlated inputs
    \item Use sensitivity to eliminate low-impact inputs
    \item Perform UQ on a reduced input space
  \end{itemize}
\end{frame}

\begin{frame}{Overview}{Assumptions}\vspace{-30pt}
  We assume:
  \begin{itemize}
    \item Simulation codes are expensive
    \item All inputs are initially perturbable
    \item Saving time and money is desirable
  \end{itemize}
\end{frame}
%                                                  %
%                MOTIVATION                        %
%                                                  %
\section{Motivation}
\begin{frame}{Motivation}{Sampling strategies}\vspace{-30pt}
  Two classes of sampling strategies in RAVEN:
  \begin{itemize}
    \item Grid-based (Latin Hypercube, ND-Grids, Limit Surface)
    \item Unstructured (Monte Carlo)
  \end{itemize}
  %TODO add picture examples
\end{frame}

\begin{frame}{Motivation}{Monte Carlo}\vspace{-30pt}
  Traditional Monte Carlo sampling
  \begin{itemize}
    \item Agnostic of dimensionality
    \item Consistent but slow convergence
  \end{itemize}
\end{frame}

\begin{frame}{Motivation}{Grids}\vspace{-30pt}
  Structured sampling strategies
  \begin{itemize}
    \item Orthogonal Grid
    \item Sparse Grid
    \item Stratified (Latin Hypercube)
  \end{itemize}
  All suffer from Curse of Dimensionality (to varying degrees)
\end{frame}

\begin{frame}{Motivation}{}\vspace{-30pt}
\end{frame}
%                                                  %
%               METHODOLOGIES                      %
%                                                  %
\section{Methods}


\subsection{Principle Component Analysis}


\subsection{Sensitivity Analysis}


%                                                  %
%                DEMONSTRATION                     %
%                                                  %
\section{Demonstration}

\end{document}






\begin{frame}{Uncertainty Quantification}{Results: pdf}
  \begin{figure}[h!]
    \centering
      \includegraphics[width=0.7\textwidth]{../../graphics/projectlePDF}
      \caption{Probability Distributions}
  \end{figure}
\end{frame}
