%
\documentclass[pdf,12pt]{article}

%\usepackage{times}
%\usepackage[FIGBOTCAP,normal,bf,tight]{subfigure}
\usepackage{amsmath}
\usepackage{amssymb}
%\usepackage{pifont}
\usepackage{enumerate}
\usepackage{listings}
\usepackage{fullpage}
\usepackage{xcolor}          % Using xcolor for more robust color specification
%\usepackage{ifthen}          % For simple checking in newcommand blocks
%\usepackage{textcomp}
%\usepackage{authblk}         % For making the author list look prettier
%\renewcommand\Authsep{,~\,}

% Custom colors
\definecolor{deepblue}{rgb}{0,0,0.5}
\definecolor{deepred}{rgb}{0.6,0,0}
\definecolor{deepgreen}{rgb}{0,0.5,0}
\definecolor{forestgreen}{RGB}{34,139,34}
\definecolor{orangered}{RGB}{239,134,64}
\definecolor{darkblue}{rgb}{0.0,0.0,0.6}
\definecolor{gray}{rgb}{0.4,0.4,0.4}

\lstset {
  basicstyle=\ttfamily,
  frame=single
}

\lstdefinestyle{XML} {
    language=XML,
    extendedchars=true,
    breaklines=true,
    breakatwhitespace=true,
%    emph={name,dim,interactive,overwrite},
    emphstyle=\color{red},
    basicstyle=\ttfamily,
%    columns=fullflexible,
    commentstyle=\color{gray}\upshape,
    morestring=[b]",
    morecomment=[s]{<?}{?>},
    morecomment=[s][\color{forestgreen}]{<!--}{-->},
    keywordstyle=\color{cyan},
    stringstyle=\ttfamily\color{black},
    tagstyle=\color{darkblue}\bf\ttfamily,
    morekeywords={name,type},
%    morekeywords={name,attribute,source,variables,version,type,release,x,z,y,xlabel,ylabel,how,text,param1,param2,color,label},
}
\lstset{language=xml}

\usepackage{titlesec}
\newcommand{\sectionbreak}{\clearpage}
\setcounter{secnumdepth}{4}


%%%%%%%% Begin comands definition to input python code into document
\usepackage[utf8]{inputenc}

% Default fixed font does not support bold face
\DeclareFixedFont{\ttb}{T1}{txtt}{bx}{n}{9} % for bold
\DeclareFixedFont{\ttm}{T1}{txtt}{m}{n}{9}  % for normal

\usepackage{listings}

% Python style for highlighting
%\newcommand\pythonstyle{\lstset{
%language=Python,
%basicstyle=\ttm,
%otherkeywords={self, none, return},             % Add keywords here
%keywordstyle=\ttb\color{deepblue},
%emph={MyClass,__init__},          % Custom highlighting
%emphstyle=\ttb\color{deepred},    % Custom highlighting style
%stringstyle=\color{deepgreen},
%frame=tb,                         % Any extra options here
%showstringspaces=false            %
%}}


% Python environment
%\lstnewenvironment{python}[1][]
%{
%$\pythonstyle
%\lstset{#1}
%}
%{}

% Python for external files
%\newcommand\pythonexternal[2][]{{
%\pythonstyle
%\lstinputlisting[#1]{#2}}}
%
%\lstnewenvironment{xml}
%{}
%{}

% Python for inline
%\newcommand\pythoninline[1]{{\pythonstyle\lstinline!#1!}}

%\def\DRAFT{} % Uncomment this if you want to see the notes people have been adding
% Comment command for developers (Should only be used under active development)
%\ifdefined\DRAFT
%  \newcommand{\nameLabeler}[3]{\textcolor{#2}{[[#1: #3]]}}
%\else
%  \newcommand{\nameLabeler}[3]{}
%\fi
% Commands for making the LaTeX a bit more uniform and cleaner
%\newcommand{\TODO}[1]    {\textcolor{red}{\textit{(#1)}}}
\newcommand{\xmlAttrRequired}[1] {\textcolor{red}{\textbf{\texttt{#1}}}}
\newcommand{\xmlAttr}[1] {\textcolor{cyan}{\textbf{\texttt{#1}}}}
\newcommand{\xmlNodeRequired}[1] {\textcolor{deepblue}{\textbf{\texttt{<#1>}}}}
\newcommand{\xmlNode}[1] {\textcolor{darkblue}{\textbf{\texttt{<#1>}}}}
\newcommand{\xmlString}[1] {\textcolor{black}{\textbf{\texttt{'#1'}}}}
\newcommand{\xmlDesc}[1] {\textbf{\textit{#1}}} % Maybe a misnomer, but I am
                                                % using this to detail the data
                                                % type and necessity of an XML
                                                % node or attribute,
                                                % xmlDesc = XML description
\newcommand{\default}[1]{~\\*\textit{Default: #1}}
\newcommand{\nb} {\textcolor{deepgreen}{\textbf{~Note:}}~}

% The bm package provides \bm for bold math fonts.  Apparently
% \boldsymbol, which I used to always use, is now considered
% obsolete.  Also, \boldsymbol doesn't even seem to work with
% the fonts used in this particular document...
\usepackage{bm}

% Define tensors to be in bold math font.
\newcommand{\tensor}[1]{{\bm{#1}}}

% Override the formatting used by \vec.  Instead of a little arrow
% over the letter, this creates a bold character.
\renewcommand{\vec}{\bm}

% Define unit vector notation.  If you don't override the
% behavior of \vec, you probably want to use the second one.
\newcommand{\unit}[1]{\hat{\bm{#1}}}

% Use this to refer to a single component of a unit vector.
\newcommand{\scalarunit}[1]{\hat{#1}}

% \toprule, \midrule, \bottomrule for tables
\usepackage{booktabs}

% \llbracket, \rrbracket
\usepackage{stmaryrd}

\usepackage{hyperref}
\hypersetup{
    colorlinks,
    citecolor=black,
    filecolor=black,
    linkcolor=black,
    urlcolor=black
}

% Compress lists of citations like [33,34,35,36,37] to [33-37]
\usepackage{cite}

% If you want to relax some of the SAND98-0730 requirements, use the "relax"
% option. It adds spaces and boldface in the table of contents, and does not
% force the page layout sizes.
% e.g. \documentclass[relax,12pt]{SANDreport}
%
% You can also use the "strict" option, which applies even more of the
% SAND98-0730 guidelines. It gets rid of section numbers which are often
% useful; e.g. \documentclass[strict]{SANDreport}

% The INLreport class uses \flushbottom formatting by default (since
% it's intended to be two-sided document).  \flushbottom causes
% additional space to be inserted both before and after paragraphs so
% that no matter how much text is actually available, it fills up the
% page from top to bottom.  My feeling is that \raggedbottom looks much
% better, primarily because most people will view the report
% electronically and not in a two-sided printed format where some argue
% \raggedbottom looks worse.  If we really want to have the original
% behavior, we can comment out this line...
\raggedbottom
\setcounter{secnumdepth}{5} % show 5 levels of subsection
\setcounter{tocdepth}{5} % include 5 levels of subsection in table of contents

% ---------------------------------------------------------------------------- %
%
% Set the title, author, and date
%
\title{'CashFlow' User Manual \\
        \large Economics plugin for RAVEN \\
}
%\author{%
%\begin{tabular}{c} Author 1 \\ University1 \\ Mail1 \\ \\
%Author 3 \\ University3 \\ Mail3 \end{tabular} \and
%\begin{tabular}{c} Author 2 \\ University2 \\ Mail2 \\ \\
%Author 4 \\ University4 \\ Mail4\\
%\end{tabular} }


\author{
 \\Aaron S. Epiney\\
}

% There is a "Printed" date on the title page of a SAND report, so
% the generic \date should [WorkingDir:]generally be empty.
\date{\today}

%\def\component#1{\texttt{#1}}

% ---------------------------------------------------------------------------- %
%\newcommand{\systemtau}{\tensor{\tau}_{\!\text{SUPG}}}

% Added by Sonat
%\usepackage{placeins}
%\usepackage{array}

%\newcolumntype{L}[1]{>{\raggedright\let\newline\\\arraybackslash\hspace{0pt}}m{#1}}
%\newcolumntype{C}[1]{>{\centering\let\newline\\\arraybackslash\hspace{0pt}}m{#1}}
%\newcolumntype{R}[1]{>{\raggedleft\let\newline\\\arraybackslash\hspace{0pt}}m{#1}}

% end added by Sonat
% ---------------------------------------------------------------------------- %
%
% Start the document
%

\begin{document}
    \maketitle

    % ------------------------------------------------------------------------ %
    % The table of contents and list of figures and tables
    % Comment out \listoffigures and \listoftables if there are no
    % figures or tables. Make sure this starts on an odd numbered page
    %
    \cleardoublepage		% TOC needs to start on an odd page
    \tableofcontents
    %\listoffigures
    %\listoftables
    % ---------------------------------------------------------------------- %

    % ---------------------------------------------------------------------- %
    % This is where the body of the report begins; usually with an Introduction
    %
    
\section{Installation}
\label{sec:Installation}

This section of the manual describes how to install the 'Economics for RAVEN'  plug-in and use it from within RAVEN \cite{RAVEN}.

\subsection{Installing the plug-in}
The CashFlow plugin is distributed with RAVEN. If you just want to use the plugin and are not interested in developing additional features or bug-fixes for it, just make sure that you have access to the plugin repository and you will automatically get it when installing RAVEN.
For more information, please refere to the RAVEN manual.

\subsection{Accessing the plug-in from within RAVEN}
The plugin can be accessed as a special subtype of the External model. The syntax is given in Listing \ref{lst:CashFlowfromRAVEN}.

\begin{lstlisting}[style=XML,morekeywords={anAttribute},caption=Call CashFlow from RAVEN input., label=lst:CashFlowfromRAVEN]
<ExternalModel name="Cash_Flow" subType="CashFlow.CashFlow">
  <variables> Input and output variables needed by CashFlow  </variables>
  <ExternalXML node="Economics" xmlToLoad="Cash_Flow_input.xml"/>
</ExternalModel>
\end{lstlisting}

\subsection{Running the plugin as a stand-alone python code}

In addition to accessing the plug-in from within RAVEN, it can also be run as a stand-alone python program. This is useful for example for testing. However, since CashFlow is still a RAVEN plugin, RAVEN needs to be installed and the plugin needs to be in the plugin-folder for that to work.

Assuming RAVEN is installed and the plugin is in the proper directory (execution will generate an error if its not), one can run it using the command shown in Listing \ref{lst:CashFlowAsCode}.

\small
\begin{lstlisting}[caption=CashFlow run as stand-alone python code, label=lst:CashFlowAsCode]
~/raven --> python plugins/CashFlow/src/CashFlow_ExtMode.py -h
usage: Cash_Flow.py [-h] -iXML inp_file -iINP inp_file -o out_file

Run RAVEN CashFlow plugin as stand-alone code

arguments:
  -h, --help      show this help message and exit
  -iXML inp_file  XML CashFlow input file name
  -iINP inp_file  CashFlow input file name with the input 
                  variable list
  -o out_file     Output file name
\end{lstlisting}
\normalsize

The XML CashFlow input is the one that contains the component and cash flow definitions, i.e. the one in \xmlAttr{xmlToLoad} in Listing \ref{lst:CashFlowfromRAVEN}.

The CashFlow input file contains the variables that RAVEN provides when using the CashFlow as a plugin, i.e. the cash flow drivers and multipliers. Listing \ref{lst:CashFlowInputFile} shows an example of the CashFlow input file for the cash flow definitions given in Listing \ref{lst:InputExample}.

\begin{lstlisting}[caption=CashFlow run as stand-alone python code, label=lst:CashFlowInputFile]
Cfdriver1 5.5
multiplier1 1.0
Cfdriver2 10.8
multiplier2 2.0
\end{lstlisting}


    \input{include/CashFlow.tex}

    \section*{Document Version Information}
    This document has been compiled using the following version of the plug-in git repository:
    \newline
    bc7dfe29dfa9b18998f61708f50af2de94fe874e Diego Mandelli Tue, 24 Apr 2018 10:07:59 -0600


    % ---------------------------------------------------------------------- %
    % References
    %
    \clearpage
    % If hyperref is included, then \phantomsection is already defined.
    % If not, we need to define it.
    \providecommand*{\phantomsection}{}
    \phantomsection
    \addcontentsline{toc}{section}{References}
    \bibliographystyle{ieeetr}
    \bibliography{user_manual}


    % ---------------------------------------------------------------------- %

\end{document}
