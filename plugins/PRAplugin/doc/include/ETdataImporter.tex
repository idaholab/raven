\section{ET Data Importer}
\label{sec:ETdataImporter}

This Post-Processor is designed to import an ET as a PointSet in RAVEN.
The ET must be specified in a specific format: the OpenPSA format (\href{<url>}{https://github.com/open-psa}). 
As an example, the ET of Fig.~\ref{fig:ET} is translated in the OpenPSA as shown in Listing~\ref{lst:ETModel}.
The work-flow is shown below:

\begin{lstlisting}[style=XML,morekeywords={anAttribute},caption=ET Importer input example., label=lst:ET_PP_InputExample]
  <Files>
    <Input name="eventTreeTest" type="">eventTree.xml</Input>
  </Files>
  
  <Models>
    ...
    <PostProcessor name="ETimporter" subType="ETImporter">
      <fileFormat>OpenPSA</fileFormat>
      <expand>False</expand>
    </PostProcessor> 
    ...  
  </Models>

  <Steps>
    ...
    <PostProcess name="import">
      <Input   class="Files"        type=""                >eventTreeTest</Input>
      <Model   class="Models"       type="PostProcessor"   >ETimporter</Model>
      <Output  class="DataObjects"  type="PointSet"        >ET_PS</Output>
    </PostProcess>
    ...
  </Steps>

  <DataObjects>
    ...
    <PointSet name="ET_PS">
      <Input>ACC,LPI,LPR</Input>
      <Output>sequence</Output>
    </PointSet>
    ...
  </DataObjects>
\end{lstlisting}

All the specifications of the ET importer are given in the 
\xmlNode{PostProcess} block. 
Inside the \xmlNode{PostProcess} block, the XML
nodes that belong to this models are:
\begin{itemize}
  \item  \xmlNode{fileFormat}, \xmlDesc{string, required parameter}, type of file format (e.g., OpenPSA)
  \item  \xmlNode{expand},\xmlDesc{bool, required parameter}, expand the ET branching conditions for all branches even if they are not queried
\end{itemize}

Each Point in the PointSet represents a unique accident sequence of the ET.
The PointSet is structured as follows: input variables are the ET branching conditions, output variable is the branch sequence ID.
The value for each input variable can have the following values:
\begin{itemize}
  \item 0: for the upper branch past the branching condition  
  \item 1: for the lower branch past the branching condition
  \item -1: if the branching condition is not queried in that particular accident sequence of the ET.
\end{itemize}
Note that the 0 or 1 values are specified in the \xmlNode{<path state="0">} or \xmlNode{<path state="1">} nodes in the ET OpenPSA file.

Provided this definition, the ET model of Fig.~\ref{fig:ET} and described in Listing~\ref{lst:ETModel}, 
the resulting model in RAVEN is characterized by these variables:
\begin{itemize}
	\item Input variables: statusACC, statusLPI, statusLPR
	\item Output variable: sequence
\end{itemize}
The corresponding PointSet if the \xmlNode{expand} node is set to False is as follows:
\begin{table}
    \centering
    \caption{PointSet generated by RAVEN by employing the ET Importer Post-Processor with \xmlNode{expand} 
             set to False for the ET of Fig.~\ref{fig:ET}.}
	\begin{tabular}{c | c | c | c} 
		\hline 
		ACC & LPI & LPR & sequence \\ 
		\hline 
		0.  &  0. &  0. & 1. \\
		0.  &  0. &  1. & 2. \\
		0.  &  1. & -1. & 3. \\
		1.  & -1. & -1. & 4. \\
		\hline 
	\end{tabular}
\end{table}
while is the following if it is set to True
\begin{table}
    \centering
    \caption{PointSet generated by RAVEN by employing the ET Importer Post-Processor with \xmlNode{expand} 
             set to True for the ET of Fig.~\ref{fig:ET}.}
	\begin{tabular}{c | c | c | c} 
		\hline 
		ACC & LPI & LPR & sequence \\ 
		\hline 
		0.  &  0. &  0. & 1. \\
		0.  &  0. &  1. & 2. \\
		0.  &  1. &  0. & 3. \\
		0.  &  1. &  1. & 3. \\
		1.  &  0. &  0. & 4. \\
		1.  &  0. &  1. & 4. \\
		1.  &  1. &  0. & 4. \\
		1.  &  1. &  1. & 4. \\
		\hline 
	\end{tabular}
\end{table}

Important notes and capabilities:
\begin{itemize}
	\item If the branching condition is not binary, then the ET Importer Post-Processor just follows 
	      the numerical value of the \xmlNode{state} attribute of the \xmlNode{<path>} node in the ET OpenPSA file. 
	\item If the ET is split in two or more ETs (and thus one file for each ET), then it is only required to list 
	      all files in the Step. RAVEN automatically detect links among ETs and merge all of them into a single PointSet.
	\item If the ET sequence IDs are not numerical but are string then RAVEN assign a numerical value to each sequence ID
	      and it generate an .xml file that report the mapping between the original (string) and the new (numerical) 
	      sequence IDs. As an example, if the ET of Fig.~\ref{fig:ET} specifies a string ID for each sequence 
	      (seq\_1,seq\_3,seq\_3,seq\_4) then RAVEN generates the following mapping (e.g., for seq\_2 branch the new 
	      numerical branch ID is set to 1):
\begin{lstlisting}[style=XML,morekeywords={anAttribute},caption=ET Importer mapping., label=lst:ET_PP_mapping]
  <map Tree="eventTreeMandd">
    <sequence ID="0">seq_1</sequence>
    <sequence ID="1">seq_2</sequence>
    <sequence ID="2">seq_3</sequence>
    <sequence ID="3">seq_4</sequence>
  </map>
\end{lstlisting}
    \item The ET can contain a branch that is defined as a separate block in the \xmlNode{define-branch} node and it is 
          replicated in the ET; in such case RAVEN automatically replicate such branch when generating the PointSet. 
\end{itemize}
 

\subsection{ET Importer reference tests}
\begin{itemize}
	\item test\_ETimporter.xml
	\item test\_ETimporterMultipleET.xml
	\item test\_ETimporterSymbolic.xml
	\item test\_ETimporter\_expand.xml
	\item test\_ETimporter\_DefineBranch.xml
	\item test\_ETimporter\_3branches.xml
	\item test\_ETimporter\_3branches\_NewNumbering.xml
	\item test\_ETimporter\_3branches\_NewNumbering\_expanded.xml
\end{itemize}
