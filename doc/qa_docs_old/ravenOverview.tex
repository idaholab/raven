\section{Introduction}

\section{Introduction}
\subsection{System Purpose}

RAVEN is a flexible and multi-purpose uncertainty quantification (UQ), regression analysis, probabilistic risk assessment 
(PRA), data analysis and model optimization software.  Depending on the tasks to be accomplished and on the 
probabilistic
 characterization of the problem, RAVEN perturbs (Monte-Carlo, latin hyper-cube, reliability surface search, etc.) the
 response of the system under consideration by altering its own parameters. The system is modeled by third party
 software (RELAP5-3D, MAAP5, BISON, etc.) and accessible to RAVEN either directly (software coupling) or
 indirectly (via input/output files). The data generated by the sampling process is analyzed using classical statistical
 and more advanced data mining approaches. RAVEN also manages the parallel dispatching (i.e. both on
 desktop/workstation and large High-Performance Computing machines) of the software representing the physical 
 model. RAVEN heavily relies on artificial intelligence algorithms to construct surrogate models of complex physical
 systems in order to perform uncertainty quantification, reliability analysis (limit state surface) and parametric studies.

\subsection{System Scope}

RAVEN’s scope is to provide a set of capabilities to build analysis flows based on UQ, PRA, Optimization and Data Analysis techniques to be applied to any physical model(s). The main objective of the software is to assist the engineer/user to:
\begin{itemize}
  \item identify the best design (on any physics/model), its safety and confidence;
  \item estimate the likelihood of undesired outcomes (risk analysis);
  \item identify main drivers/events to act on for reducing impact/consequences of anomalous dynamic behaviors of the 
         system under analysis;
  \item to construct analysis flows combining multiple physical models and analysis procedures.
\end{itemize}

In other words, the RAVEN software is aimed to be employed for:
\begin{itemize}
  \item Uncertainty Quantification;
  \item Sensitivity Analysis / Regression Analysis;
  \item Probabilistic Risk and Reliability Analysis (PRA);
  \item Data Mining Analysis;
  \item Model Optimization.
\end{itemize}

The combination of all the previously mentioned analysis capabilities is a key component to 
define safety margins in engineering design that are more representative of real prediction deficiencies. 
This could reduce 
cost and maintain a more coherent safety level of the system (no excess/no lack of safety margins in any operational 
condition).
The risk analysis, assisted by the data mining algorithms, is used to find engineering solutions to reduce costs, while 
preserving safety margins, or to increase safety at the minimum cost. These tasks can be automatically achieved by using 
optimization algorithms available in the RAVEN software.
Moreover, the knowledge of the relationship between input and system response uncertainties allows identifying effective 
experiments, which are the most suitable for increasing the accuracy of the model. This approach reduces time and cost 
of the deployment of complex engineering systems and new technologies.

The RAVEN software employs several novel and unique techniques, based on extensive usage of artificial intelligence 
algorithms, such as adaptive (smart) sampling, adaptive branching algorithms (Dynamic Event Tree), time-dependent 
statistical analysis and data mining. 
The overall set of algorithms implemented in the RAVEN software are designed to handle highly non-linear systems, 
characterized by system response discontinuities and discrete variables. These capabilities are crucial for handling 
complex system models, such as nuclear power plants.
For example, reliability surface analysis, as implemented in RAVEN, is unique and capable to handle non-linear, 
discontinuous systems, allowing for faster and more accurate assessing of failure risk for complex systems.

In addition, the RAVEN software provides the unique capability to combine any model (e.g. physical models, surrogate 
models, data analysis models, etc.) in a single entity (named Ensemble Model) where each model can feedback into others. This capability allows to analyze system that could be simulated only by using complex computational work-flows.

\subsection{Dependencies and Limitations}
The software should be designed with the fewest possible constraints. 
Ideally the software should run on a wide variety of evolving hardware, 
so it should follow well-adopted standards and guidelines. The software
 should run on any POSIX compliant system (including Windows POSIX 
 emulators such as MinGW). The software will also make use of artificial 
 intelligence and numerical libraries that run on POSIX systems as well. 
 The main interface for the software will be command line based with no 
 assumptions requiring advanced terminal capabilities such as coloring and line control. 

No evident limitations are envisioned for the RAVEN software design and its further expansions.


\subsection{Other Design Documentation}

In addition to this document, an automatic software documentation
is generated every time a new CR (see def.) is approved. This
documentation is automatically extracted from the source code using
doxygen (see def.) and is available to developers at
\url{https://hpcsc.inl.gov/ssl/RAVEN/docs/classes.html}.

In order to generate(locally) a hard copy in ``html'' or ``latex'', any user/developer can lunch the 
following command (in the raven directory):
\begin{lstlisting}[language=bash]
doxygen ./doc/doxygen/Doxyfile
\end{lstlisting}
Once the documentation is generated, any user/developer can navigate in the folder
\begin{lstlisting}[language=bash]
./doc/doxygen/latex
\end{lstlisting}
and type the following command:
\begin{lstlisting}[language=bash]
make refman.pdf
\end{lstlisting}
Once the command is executed, a ``pdf'' file named ``refman.pdf'' will be available.

The doxygen software is under configuration management process identified in
`` RAVEN Configuration Management '' PLN-5553.


\section{References}

\begin{itemize}

  \item ASME NQA 1 2008 with the NQA-1a-2009 addenda, ``Quality Assurance Requirements for Nuclear Facility Applications,'' First Edition, August 31, 2009.
  \item ISO/IEC/IEEE 24765:2010(E), ``Systems and software engineering Vocabulary,'' First Edition, December 15, 2010.
  \item LWP 13620, ``Managing Information Technology Assets''
\end{itemize}


\section{Definitions and Acronyms}

\subsection{Definitions}
\begin{itemize}
  \item \textbf{Baseline.} A specification or product (e.g., project plan, maintenance and operations [M\&O] plan, requirements, or 
design) that has been formally reviewed and agreed upon, that thereafter serves as the basis for use and further 
development, and that can be changed only by using an approved change control process. [ASME NQA-1-2008 with the 
NQA-1a-2009 addenda edited]
  \item \textbf{Validation.} Confirmation, through the provision of objective evidence (e.g., acceptance test), that the requirements 
for a specific intended use or application have been fulfilled. [ISO/IEC/IEEE 24765:2010(E) edited]
  \item \textbf{Verification.}
  \begin{itemize}
     \item The process of evaluating a system or component to determine whether the products of a given development 
     phase satisfy the conditions imposed at the start of that phase.
     \item  Formal proof of program correctness (e.g., requirements, design, implementation reviews, system tests). 
     [ISO/IEC/IEEE 24765:2010(E) edited]
  \end{itemize}
\end{itemize}

\subsection{Acronyms}
\begin{description}
\item[API] Application Programming Interfaces
\item[ASME] American Society of Mechanical Engineers
\item[DOE] Department of Energy
\item[INL] Idaho National Laboratory
\item[IT] Information Technology
\item[M\&O] Maintenance and Operations
\item[NQA] Nuclear Quality Assurance
\item[QA] Quality Assurance
\item[SDD] System Design Description
\end{description}

\subsection{Software infrastructure}
\subsubsection{Outlines}
RAVEN has been developed in a highly modular and pluggable way in order to enable easy integration of different programming languages (i.e., C++, Python) and, as already mentioned, coupling with any system code.

\subsubsection{Probabilistic and Parametric framework}
The probabilistic and parametric framework represents the core of the RAVEN analysis capabilities. The main idea behind the design of the system is the creation of a multipurpose framework characterized by high flexibility with respect to the possible performable analysis. The framework must be capable of constructing the analysis/calculation flow at run-time, interpreting the user-defined instructions and assembling the different analysis tasks following a user specified scheme.
In order to achieve such flexibility, combined with reasonably fast development, a programming language naturally suitable for this kind of approach was needed: $Python$.


\begin{figure}[ht]
  \centering
  \includegraphics[width=1.0\textwidth]  {pics/RavenFramework.png}
  \caption{RAVEN framework layout}
  \label{fig:RAVENframeworkLayout}
\end{figure}

Hence, RAVEN is coded in $Python$ and is characterized by an object-oriented design. The core of the analysis performable through RAVEN is represented by a set of basic components (objects) the user can combine, in order to create a custom analysis flow. A list of these components and a summary of their most important functionalities are reported as follows:
\begin{itemize}
\item	Distribution: In order to explore the input/output space, RAVEN requires the capability to perturb the input space (initial conditions and/or model coefficients of a system code). The input space is generally characterized by probability distribution functions (PDFs), which might need to be considered when a perturbation is applied. In this respect, a large library of PDFs is available.
\item Sampler: A proper approach to sample the input space is fundamental for the optimization of the computational time. In RAVEN, a ``sampler'' employs a unique perturbation strategy that is applied to the input space of a system. The input space is defined through the connection of uncertain variables and their relative probability distributions.
\item Model: A model is the representation of a physical system (e.g. Nuclear Power Plant); it is therefore capable of predicting the evolution of a system given a coordinate set in the input space. In addition it can represent an
action on a data in order to extract key features (e.g. Data mining).
\item Reduced Order Model (ROM): The evaluation of the system response, as a function of the coordinates in the input space, is very computationally expensive, especially when brute-force approaches (e.g. Monte Carlo methods) are chosen as the sampling strategy. ROMs are used to lower this cost, reducing the number of needed points and prioritizing the area of the input space that needs to be explored. They can be considered as an artificial representation of the link between the input and output spaces for a particular system.
\end{itemize}
The list above is not comprehensive of all the RAVEN framework components (visualization and storage infrastructure).
\\ Figure~\ref{fig:RAVENframeworkLayout} shows a schematic representation of the whole framework, highlighting the communication pipes among the different modules and engines. As can be seen, in the figure all the components reported above are schematically shown. In addition the data management, mining and processing modules are shown.

\subsubsection{Distribution}
As already mentioned, the perturbation of the input space, through the initial conditions (parameters) affected by uncertainties, needs to be performed by the proper distribution functions. RAVEN provides, through an interface to the BOOST library, the following variate (truncated and not) distributions: Bernoulli, Binomial, Exponential, Logistic, Lognormal, Normal, Poisson, Triangular, Uniform, Weibull, Gamma, and Beta.
\\The usage of uni-variate distributions for sampling initial conditions is based on the assumption that the uncertain parameters are not correlated with each other. Quite often uncertain parameters are subject to correlations and thus the uni-variate approach is not applicable. This happens when a generic outcome is dependent on different phenomena simultaneously (i.e. the outcome dependency description can not be collapsed to a function of a single variable). RAVEN currently supports both N-dimensional (N-D) PDFs. The user can provide the distribution values on either Cartesian or sparse grid, which determines the interpolation algorithm used in the evaluation of the imported CDF/PDF:
\begin{enumerate}
\item N-Dimensional Spline, for Cartesian grids
\item Inverse weight, for sparse grids
\end{enumerate}
Internally, RAVEN provides the needed N-D differentiation and integration algorithms to compute the PDF from the CDF and vice-versa.
\\As already mentioned, the sampling methods use the distributions in order to perform probability-weighted perturbations. For example, in the Monte Carlo approach, a random number $\in [0,1]$ is generated (probability threshold) and the CDF, corresponding to that probability, is inverted in order to retrieve the parameter value usable in the simulation. The existence of the inverse for variate distributions is guaranteed by the monotonicity of the CDF. For N-D distributions this condition is not sufficient since the $CDF(X)\longrightarrow [0,1],X \in  R^{N} $ and therefore it could not be a bijective function. From an application point of view, this means the inverse of a N-D CDF is not unique.
\\As an example, the Figure~\ref{fig:NDDistributionExample} shows a multivariate normal distribution for a pipe failure as function of the pressure and temperature. The plane identifies an isoprobability surface (in this case, a line) that represents a probability threshold of 50 \% in this example.  Hence, the inverse of this CDF is an infinite number of points.
 \\As easily inferable, the standard sampling approach cannot directly be employed. When multivariate distributions are used, RAVEN implements a surface search algorithm for identifying the iso-probability surface location. Once the location of the surface has been found, RAVEN chooses, randomly, one point on it.

\begin{figure}
  \centering
  \includegraphics[width=0.5\textwidth]  {pics/NDimensionalDistributionExample.png}
  \caption{2-D CDF, function of pressure and temperature}
  \label{fig:NDDistributionExample}
\end{figure}

\subsubsection{Sampler}

The sampler is probably the most important entity in the RAVEN framework. Indeed, it performs the driving of the specific sampling strategy and, hence, determines the effectiveness of the analysis, from both an accuracy and computational point of view.  The samplers, that are available in RAVEN, can be categorized in three main classes:
\begin{itemize}
 \item Forward
 \item Dynamic Event Tree (DET)
 \item Adaptive
\end{itemize}
\paragraph{Forward Samplers} ~\\
The Forward sampler category collects all the strategies that perform the sampling of the input space without exploiting, through dynamic learning approaches, the information made available from the outcomes of calculation previously performed (adaptive sampling) and the common system evolution (patterns) that different sampled calculations can generate in the phase space (dynamic event tree).
In the RAVEN framework, several different and well-known forward samplers are available:
\begin{itemize}
\item Monte Carlo (MC)
\item Stratified based, whose most known specialization is the Latin Hyper-Cube Sampling (LHS)
\item Grid Based
\item Response Surface Design of Experiment
\item Sparse Grid
\item Factorials
\item Etc.
\end{itemize}
As already mentioned, all these sampling strategies are well known, as well as their properties. Therefore, a detailed investigation of their application is not provided.
%%%%%%%%%%%%%%%%%%%%%%%%%%%%%%%%%%%%%%%%%%%%%%%%%%%%%%%%%%%%%%%%%%%%%%%%%%%%%%%%
\paragraph{Dynamic Event Tree Samplers}~\\
In order to clarify the idea behind the Dynamic Event Tree Sampler currently available in RAVEN, a small overview is needed.
\\In technological complex systems, as nuclear power plants, an accident scenario begins with an initiating event and then evolves over time through the interaction of dynamics and stochastic events. This mutual action leads to the production of infinitely many different scenarios, which define a continuous dynamic event tree with infinite branches. At each time point, the stochastic variability of the accident outcomes is determined by a multivariate probability distribution. The PRA analysis needs an approximation to this distribution for selected consequence variables. A way to achieve this goal is an Event Tree approach. In dynamic PRA analysis, Conventional Event Tree sequences are run simultaneously starting from a single initiating event. The branches occur at user specified times and/or when an action is required by the operator and/or the system, creating a deterministic sequence of events based on the time of their occurrence. One of the disadvantages of this method is that the timing/sequencing of events and system dynamics is not explicitly accounted for in the analysis. In order to overcome these limitations a “dynamic” approach is needed. The Dynamic Event Tree (DET) technique brings several advantages, among which is the fact that it simulates probabilistic system evolution in a way that is consistent with the severe accident model. This leads to a more realistic and mechanistically consistent analysis of the system taken into consideration. The dynamic PRA, in general, and the Dynamic Event Tree methodologies in particular, are designed to take the timing of events explicitly into account, which can become very important especially when uncertainties in complex phenomena are considered. Hence, the main idea of this methodology is to let a system code determine the pathway of an accident scenario.
\\From an application point of view, a N-D grid is built on the CDF space. A single simulation is spawned and a set of triggers is added to the system code control logic. Every time a trigger gets activated (one of the CDF thresholds in the grid is violated), a new set of simulations (branches) is spawned. Each branch carries its own probability.
\\Figure \ref{fig:DETschemeExample} shows a practical example. In this particular case, it is assumed that the
probability failure of a pipe depends on the fluid pressure magnitude. Three probability thresholds are defined on
the cumulative distribution function. One simulation is spawned (0). As soon as the pressure of the fluid reaches a
value corresponding to a 33\% probability (CDF), a stop signal is sent and the framework starts two new
simulations (branches). The branch in which the system evolved to the newer condition (pipe failed, red line)
carries 33\% of the probability, while the other the complementary amount. The same procedure is repeated at
point 2.
\\Generally, not all the input space can be explored using a DET approach. For instance, usually the parameters affected by aleatory uncertainty are sampled using a dynamic event tree approach, while the ones characterized by epistemic uncertainty are sampled through ``forward'' sampling strategies.
\\As already mentioned, this strategy requires a tight interaction between the system code and the sampling driver (i.e., RAVEN framework). In addition, the system code must have a control logic capability (i.e. trigger system). For these reasons, the application of this sampling approach to a generic code needs a larger effort when compared to the other Samplers available in RAVEN. Currently, the DET is fully available for the thermal-hydraulic codes RELAP-7 and RELAP5-3D.
\\In the RAVEN framework, several different DET-based samplers are available:
\begin{itemize}
\item Dynamic Event Tree (aleatory sampling);
\item Hybrid Dynamic Event Tree (aleatory and epistemic sampling);
\item Adaptive Dynamic Event Tree (goal-oriented sampling for aleatory sampling);
\item Adaptive Hybrid Dynamic Event Tree (goal-oriented sampling for aleatory and epistemic sampling).
\end{itemize}

\begin{figure}
  \centering
  \includegraphics[width=0.7\textwidth]  {pics/DETscheme.png}
  \caption{Dynamic Event Tree simulation pattern}
  \label{fig:DETschemeExample}
\end{figure}

%%%%%%%%%%%%%%%%%%%%%%%%%%%%%%%%%%%%%%%%%%%%%%%%%%%%%%%%%%%%%%%%%%%%%%%%%%%%%%%%
\paragraph{Adaptive Samplers}~\\
A key feature available within RAVEN is the possibility to perform smart sampling (also known as adaptive sampling) as an alternative to classical ``forward'' techniques.
\\The motivation is that nuclear simulations are often computationally expensive, time-consuming, and high dimensional with respect to the number of input parameters. Thus, exploring the space of all possible simulation outcomes is unfeasible using finite computing resources. During simulation-based probabilistic risk analysis, it is important to discover the relationship between a potentially large number of input parameters and the output of a simulation using as few simulation trials as possible.
\\This is a typical context for performing adaptive sampling where a few observations are obtained from the simulation, a reduced order model (ROM) is built to represent the simulation space, and new samples are selected based on the model constructed. The ROM is then updated based on the simulation results of the sampled points. In this way, it is attempted to gain the most information possible with a small number of carefully selected sampled points, limiting the number of expensive trials needed to understand features of the system space.
\\In the RAVEN framework, several different adaptive samplers are available:
\begin{itemize}
\item Limit Surface Search;
\item Adaptive Dynamic Event Tree;
\item Adaptive Hybrid Dynamic Event Tree ;
\item Adaptive Sparse Grid;
\item Adaptive Sobol Decomposition.
\end{itemize}

\subsubsection{Models}
The Model entity, in the RAVEN environment, represents a ``connection pipeline'' between the input and the output space. The RAVEN framework does not own any physical model (i.e. it does not posses the equations needed to simulate a generic physical system, such as Navier-Stocks equations, Maxwell equations, etc.), but implements Application Program Interfaces (APIs) by which any generic model can be integrated and interrogated. The RAVEN framework provides APIs for four different model categories: Codes, Externals, Post-Processors (PPs) and Reduced Order Models (ROMs). In the following paragraphs, a brief explanation of each of this Model categories is reported.
\paragraph{Code} ~\\
The \textit{Code} model represents the communication pipe between the RAVEN framework and any system and physical code. The communication between RAVEN and any driven code is performed through the implementation of interfaces directly operated by the framework.
\\The procedure of coupling a new code/application with RAVEN is a straightforward process. The coupling is performed through a \textit{Python}  interface that interprets the information coming from RAVEN and translates them to the input of the driven code. The coupling procedure does not require modifying RAVEN itself. Instead, the developer creates a new \textit{Python} interface that is going to be embedded in RAVEN at run-time (no need to introduce hard-coded coupling statements).  This interface needs to be placed in a folder (whatever name) located in (see figure~\ref{fig:CodeInterfaceLocation}):
\begin{lstlisting}[language=bash]
 path/to/raven/distribution/raven/framework/CodeInterfaces/
\end{lstlisting}

\begin{figure}
  \centering
  \includegraphics[width=0.8\textwidth]  {pics/CodeInterfaceLocation.png}
  \caption{Code Interface Location}
  \label{fig:CodeInterfaceLocation}
\end{figure}
At the initialization stage, RAVEN imports all the Interfaces that are contained in this directory and performs some preliminary cross-checks.
\\ If the coupled code is parallelized and/or multi-threaded, RAVEN is going to manage the system in order to optimize the computational resources in both workstations and High Performance Computing systems.
\\Currently, RAVEN has APIs for RELAP5-3D, RELAP-7, any MOOSE-based application, SASS and Modelica.
\paragraph{External Model} ~\\
The External model allows the user to create, in a \textit{Python} file (imported, at run-time, in the RAVEN framework), its own model (e.g. set of equations representing a physical model, connection to another code, control logic, etc.). This model will be interpreted/used by the framework and, at run-time, will become part of RAVEN itself.
\paragraph{Post-Processor} ~\\
The Post-Processor model represents the container of all the post-processing capabilities in the RAVEN code. This model is aimed to process the data (for example, derived from the sampling of a physical code) in order to identify representative Figure of Merits. This system has  been designed and, presently, is under heavy development by the whole RAVEN team.  Currently, the following post-processors are available:
\begin{itemize}
 \item \textit{Basic Statistics}, container of the algorithms to compute many of the most important statistical quantities. This post-processor is able to compute mean, sigma/variance, variation coefficient, skewness, kurtosis, median, percentiles and all the principal matrix quantities such as covariance, sensitivity (either leas-squared and variance weighted) and correlation matrices;
 \item \textit{Comparison Statistics}, aimed to employ validation and verification metrics;
 \item \textit{Limit Surface}, aimed to compute the limit surface in the input space (i.e. the hyper-surface that represents the boundary between the failure/success of the system);
 \item \textit{Limit Surface Integral}, intended to compute the integral of the limit surface that corresponds to the probability of failure;
  \item \textit{Safest Point}, provides the coordinates of the farthest point from the limit surface that is given as an input. The safest point coordinates are expected values of the coordinates of the farthest points from the limit surface in the space of the ``controllable'' variables based on the probability distributions of the ``non-controllable'' variables. The term ``controllable'' identifies those variables that are under control during the system operation, while the ``non-controllable'' variables are stochastic parameters affecting the system behavior randomly;
 \item \textit{External Post-Processor}, user-defined post-processor;
 \item \textit{Topological Decomposition}, aimed to compute an approximated hierarchical Morse-Smale complex which will add two columns to a data-set, performing a topological decomposition of such data-set;
 \item \textit{Data Mining}, container of all the RAVEN data mining, clustering and dimensionality reduction techniques aimed to identify dominant and common patterns in high-dimensionality data.
\end{itemize}
\paragraph{Reduced Order Model} ~\\
 As briefly mentioned, a ROM is a mathematical representation of a system, used to predict a selected output space of a physical system.
The ``training'' is a process that uses sampling of the physical model to improve the prediction capability (capability to predict the status of the system given a realization of the input space) of the ROM. More specifically, in RAVEN the Reduced Order Model is trained to emulate a high fidelity numerical representation (system codes) of the physical system. Two general characteristics of these models can be generally assumed (even if exceptions are possible):
\begin{enumerate}
   \item The higher the number of realizations in the training sets, the higher is the accuracy of the prediction performed by the reduced order model. This statement is true for most of the cases although some ROMs might be subject to the over-fitting issues. The over-fitting phenomenon is not discussed here, since its occurrence is highly dependent on the algorithm type, (and there is large number of ROM options available in RAVEN). Every time the user chooses a particular reduced order model algorithm to use, he should consult the relative literature;
   \item The smaller the size of the input domain with respect to the variability of the system response, the more likely the surrogate model will be able to represent the system output space.
\end{enumerate}
In most of the cases of interest, the information that is sought is related to defining the failure boundaries of a system with respect to perturbations in the input space. For this reason, in the development of RAVEN, it has been given priority to the introduction of a class of supervised learning algorithms, which are usually referred to as classifiers. A classifier is a reduced order model that is capable of representing the system behavior through a binary response (failure/success).
\\The first class of classifier introduced has been the Support Vector Machines (SVMs) [reference] with several different kernels (polynomial of arbitrary integer order, radial basis function kernel, sigmoid) followed by a nearest-neighbor based classification using a K-D tree search algorithm. Currently, RAVEN supports around 40 different ROM methodologies. All these supervised learning algorithms have been imported via an API from the scikit-learn [reference] library. In addition, the N-Dimensional spline and the inverse weight methods, that are currently available for the interpolation of N-Dimensional PDF/CDF, can also be used as ROMs.
\subsubsection{Simulation Environment}
RAVEN is perceived by the user as a pool of tools and data. Any action in which the tools are applied to the data is considered a calculation ``step'' in the RAVEN environment. Simplistically, a ``step'' can be seen as a \textbf{transfer function} between the input and output space through a Model (e.g. Code, External, ROM or Post-Processor). One of the most important step in the RAVEN framework is called ``multi-run'', that is aimed to handle calculations that involve multiple runs of a driven code (sampling strategies). Firstly, the RAVEN input file associates the variables to a set of PDFs and to a sampling strategy. The ``multi-run'' step is used to perform several runs in a block of a model (e.g. in a MC sampling).
\begin{figure}[ht]
  \centering
  \includegraphics[width=0.8\textwidth]  {pics/MultiRunCalculationFlow.png}
  \caption{Calculation flow for a multi-run sampling}
  \label{fig:multiRun}
\end{figure}
At the beginning of each sub sequential run, the sampler provides the new values of the variables to be perturbed. The code API places those values in the input file. At this point, the code API generates the run command and asks to be queued by the job handler. The job handler manages the parallel execution of as many runs as possible within a user prescribed range and communicates with the step controller when a new set of output files are ready to be processed. The code API receives the new input files and collects the data in the RAVEN internal format. The sampler is queried to assess if the sequence of runs is ended, if not, the step controller asks for a new set of values from the sampler and the sequence is restarted as described in Figure~\ref{fig:multiRun}.
The job handler is currently capable to run different run instances of the code in parallel and can also handle codes that are multi-threaded or using any form of parallel implementation.
RAVEN also has the capability to plot the simulation outcomes while the set of sampling is performed and to store the data for later recovery.
