\section{Forward Sampling Strategies}
\label{sec:forwardSamplingStrategies}
In order to perform UQ and dynamic
probabilistic risk assessment (DPRA),
a sampling strategy needs to be employed. The sampling strategy
perturbs the input space (domain of the uncertainties) to explore
the response of a complex system in relation to selected figure of merits (FOMs).

The most widely used strategies to perform UQ and PRA are generally
collected in RAVEN as \textit{\textbf{Forward}} samplers. \textit{\textbf{Forward}} samplers include
all the strategies that simply perform the sampling of the input space.  These strategies sample
without exploiting, through learning approaches,
the information made available from the outcomes of evaluation previously performed (adaptive sampling) and the
common system evolution (patterns) that different sampled calculations can generate in the phase space (Dynamic Event Tree).

RAVEN has several different \textit{\textbf{Forward}} samplers:
\begin{itemize}
  \item \textit{Monte-Carlo}
  \item \textit{Grid-based}
  \item \textit{Stratified} and its specialization named \textit{Latin Hyper Cube}.
\end{itemize}
In addition, RAVEN posses advanced \textit{\textbf{Forward}} sampling strategies that:
\begin{itemize}
  \item Build a grid in the input space selecting evaluation points
    based on characteristic quadratures as part of stochastic collocation
    for generalized polynomial chaos method (\textit{Sparse
    Grid Collocation} sampler);
  \item Use high-density model reduction (HDMR) a.k.a. Sobol
    decomposition to approximate a function as the sum of
    interactions with increasing complexity (\textit{Sobol} sampler).
\end{itemize}
In the following subsections, the theory behind these sampling
methodologies is explained.
%%%%%%%%%%%%%%%%%%%%%%%%%
%%%%%%%%  MONTE-CARLO %%%%%%%%
%%%%%%%%%%%%%%%%%%%%%%%%%
\subsection{Monte-Carlo}
\label{sub:MC}
The Monte-Carlo method is one of the most-used methodologies in several mathematic disciplines.
It approximates an expectation by the sample mean of a function of
simulated random variables. It is based on the laws of large numbers in order to approximate expectations.
In order words, it approximates the average response of multiple FOMs
relying on multiple random sampling of the input space.
\\Consider a random variable (eventually multidimensional) $X$ having probability mass function or probability density function $pdf_{X}(x)$,
which is greater than zero on a set of values $\chi$. Then the expected value of a function $f$ of $X$ is as follows:
\begin{equation}
\begin{matrix}
\mathbb{E}(f(X)) =\sum_{x \in \chi} f(x)pdf_{X}(x) & \text{if} X \, discrete \\
\\
\mathbb{E}(f(X)) =\int_{x \in \chi} f(x)pdf_{X}(x) & \, \text{if} X \, \, continuous
\end{matrix}
\end{equation}
Now consider $n$ samples of $X$, $(x_{1},...,x_{n})$, and compute the mean of $f(x)$ over the samples. This computation represents the Monte-Carlo estimate:
\begin{equation}
  \mathbb{E}(f(X)) \approx   \widetilde{f}_{n}(x) = \frac{1}{n} \sum_{i=1}^{n} f(x_{i})
\end{equation}
If $\mathbb{E}(f(X))$ exists, then the law of large numbers determines that for any arbitrarily small $\varepsilon$:
\begin{equation}
  \lim_{n\rightarrow \, \infty} P( \left | \widetilde{f}_{n}(X) - \mathbb{E}(f(X))  \right |\geq \varepsilon) = 0
\end{equation}
The above equation suggests that as $n$ gets larger, then the probability that $\widetilde{f}_{n}(X)$ deviates
from the $\mathbb{E}(f(X))$ becomes smaller. In other words, the more samples are spooned, the more closer the Monte-Carlo estimate of $X$ gets to the real value.
\\In addition, $\widetilde{f}_{n}(X)$ represent an unbiased estimate for $\mathbb{E}(f(X))$:
\begin{equation}
\mathbb{E}(\widetilde{f}_{n}(X)) = \mathbb{E} \left ( \frac{1}{n} \sum_{i=1}^{n} f(x_{i})   \right ) =
\frac{1}{n} \sum_{i=1}^{n} \mathbb{E}(f(x_{i})) =   \mathbb{E}(f(X))
\end{equation}
%After this brief introduction, it is important to understand how the Monte-Carlo method can be employed
%for the analysis of Dynamic Stochastic system.
%\\Referencing to the nomenclature defined in section~\ref{sub:mathBackground}, given:
%\begin{equation}
%\frac{\partial  \overline{\theta}^{c}\left ( t \right )}{\partial t}=f\left ( \overline{\theta}^{c},\overline{\theta}^{d}_{i}, \overline{\alpha}_{staz} ,\overline{\alpha}_{brow}, t \right )
%\end{equation}
% define a function $g_{i}$ that represents the solution of the previous equation (the trajectory in the $ \overline{\theta}^{c}$ space for a fixed $ \overline{\theta}^{d}_{i}$ and initial condition $\overline{\theta}^{c}_{0}$) :
%\begin{equation}
%  \overline{\theta}^{c}(t) = g_{i}(t,\overline{\theta}^{c}_{0})
%\end{equation}
%The Monte-Carlo analysis is performed as following:
%\begin{enumerate}
%  \item Sample:
%  \begin{itemize}
%    \item $\overline{\alpha}_{staz} $,$\overline{\alpha}_{DS}$, $\overline{t}$ depending on which
%    approximations are valid (see~\ref{sub:mathBackground});
%    \item The initial conditions $\overline{\theta}^{c}_{0}$,$\overline{\theta}^{d}_{0}$;
%    \item Transition conditions from $W(\overline{\theta}^{d}|\overline{\theta}^{d}_{i},\overline{\theta}^{c},t)$
%  \end{itemize}
%  \item Run the system simulator using the previously sampled values and affected by the intrinsic stochasticity
%  represented by $\overline{\alpha}_{brow}$;
%  \item Pause the simulation when a transition condition is reached and move from the current $\overline{\theta}^{d}_{0}$ to the new $\overline{\theta}^{d}$ (e.g. $\overline{\theta}^{d}_{1}$);
%  \item Run the simulation as performed in step 3, starting from the new coordinate and pause the simulation when a new transition is reached;
%  \item Repeat steps 3 and 4 until a stopping condition is reached;
%  \item Repeat 1 through 4 for a large number of runs $n$.
%\end{enumerate}
%%%%%%%%%%%%%%%%%%%%%%%%%
%%%%%%%%          GRID          %%%%%%%%
%%%%%%%%%%%%%%%%%%%%%%%%%
\subsection{Grid}
\label{sub:Grid}
The Grid sampling method (also known as Full Factorial Design of Experiment) represents one of the simplest methodologies that can be employed in order to explore the interaction of multiple random variables with respect
selected FOMs.
The goal of the Grid-based sampling strategy is to explore the interaction of multiple random
variables (i.e.,uncertainties) with respect to selected FOMs. Indeed, this method is mainly used to perform
parametric analysis of the system response rather than a probabilistic one. This method discretizes the
domain of the uncertainties in a user-defined number of intervals (see Figure~\ref{fig:GridDiscretization}) and
record the response of the model (e.g. a system code) at each coordinate (i.e., combination of the uncertainties) of the grid.
\\ This method starts from the assumption that each coordinate on the grid is representative, with respect to the FOMs of interest, of the surrounding grid cell. In other words, it is assumed that the response of a system does not significantly change within the hyper-volume surrounding each grid coordinate (red square in Figure~\ref{fig:GridDiscretization}).
 %%%%%%%%%%%%%%%%%%%%%%%%%%%%%%%%%%%%%%%%%%%%%%%%%%%%%%%%%%
 %figure history
 \begin{figure}[h!]
  \centering
  \includegraphics[scale=0.7]{pics/GridDiscretization.png}
  \caption{Example of 2-Dimensional grid discretization. }
  \label{fig:GridDiscretization}
 \end{figure}
 %%%%%%%%%%%%%%%%%%%%%%%%%%%%%%%%%%%%%%%%%%%%%%%%%%%%%%%%%%

Similarly to what has been already reported for the Monte-Carlo sampling, consider a random variable $X$ having PDF $pdf_{X}(x)$ and, consequentially, CDF $cdf_{X}(x)$ in the domain $\chi$. Then the expected value of a function $f$ of $X$ is as follows:
\begin{equation}
\begin{matrix}
\mathbb{E}(f(X)) =\sum_{x \in \chi} f(x)pdf_{X}(x) & if X \, discrete \\
\\
\mathbb{E}(f(X)) =\int_{x \in \chi} f(x)pdf_{X}(x) & \, if X \, \, continuous
\end{matrix}
\end{equation}
In the Grid approach, the domain of $X$ is discretized in a finite number of intervals. Recall that
each node of this discretization is representative of the surrounding hyper-volume. This means that a weight
needs to be associated with each coordinate of the resulting grid:
\begin{equation}
\begin{matrix}
  w_{i}= cdf_{X}(x_{i+1/2}) - cdf_{X}(x_{i-1/2})
\end{matrix}
\end{equation}
Now consider
a $n-$discretization of the domain of  $X$, $(x_{1},...,x_{n})$ and compute the mean of $f(x)$ over the discretization. Based on the previous equation, the computation of the expected value of $f(x)$ is as follows:
\begin{equation}
 \mathbb{E}(f(X)) \approx   \widetilde{f}_{n}(x) = \frac{1}{\sum_{i=1}^{n}w_{i}} \sum_{i=1}^{n} f(x_{i}) \times w_{i}
\end{equation}
If the number of uncertainties under consideration is greater than one ($m$), the above equation
becomes:
\begin{equation}
\mathbb{E}(f(\overline{X})) \approx   \widetilde{f}_{n}(\overline{x}) = \frac{1}{\sum_{i=1}^{n}\prod_{j=1}^{m}w_{i,j}} \sum_{i=1}^{n} f(\overline{x}_{i}) \times \prod_{j=1}^{m}w_{i,j}
\end{equation}
%%%%%%%%%%%%%%%%%%%%%%%%%
%%%%%%%%    STRATIFIED    %%%%%%%%
%%%%%%%%%%%%%%%%%%%%%%%%%
\subsection{Stratified}
\label{sub:Stratified}
The Stratified sampling is a class of methods that relies on the assumption that the input space (i.e.,uncertainties)
can be separated in regions (strata) based on similarity of the response of the system for input set within the
same strata. Following this assumption, the most rewarding (in terms of computational cost vs. knowledge gain)
sampling strategy would be to place one sample for each region. In this way, the same information is not
collected more than once and all the prototypical behavior are sampled at least once. In
Figure~\ref{fig:StratifiedSamplingExample}, the Stratified sampling approach is exemplified.
 \begin{figure}[h!]
  \centering
  \includegraphics[scale=0.55]{pics/StratifiedSamplingExample.png}
  \caption{Example of Stratified sampling approach.}
  \label{fig:StratifiedSamplingExample}
 \end{figure}
\\The Stratified sampling approach is a method for the exploration of the input space that consists of dividing the uncertain domain into subgroups before sampling. In the Stratified sampling, these subgroups must be:
\begin{itemize}
  \item Mutually exclusive: every element in the population must be assigned to only one stratum (subgroup)
  \item Collectively exhaustive: no population element can be excluded.
\end{itemize}
Then, simple random sampling or systematic sampling is applied within each stratum. The Latin Hyper-Cube sampling represents a specialized version of the stratified approach, when the domain strata are constructed in equally-probable CDF bins.
\\Similarly to what has been already reported for the Grid sampling, consider a set of  $m$ random variables $X_{j}, \, j=1,..,m$ having PDFs $pdf_{X_{j}}(x_{j})$ and, consequentially, CDF $cdf_{X_{j}}(x_{j})$ in the domain $\chi_{j}$. Then the expected value of a function $f$ of $X_{j}, \, j=1,..,m$ is as follows:
\begin{equation}
\begin{matrix}
\mathbb{E}(f(\overline{X})) =\sum f(x)   \prod_{j=1}^{m} pdf_{X_{j}}(x_{j}) & if \overline{X} \, discrete \\
\\
\mathbb{E}(f(\overline{X})) =\int f(x)\prod_{j=1}^{m} pdf_{X_{j}}(x_{j}) & \, if \overline{X} \, \, continuous
\end{matrix}
\end{equation}
In the Stratified approach, the domain of $\overline{X}$ is discretized in a set of strata. Consequentially, similarly to the Grid sampling, a weight needs to be associated with each realization in the input space:
\begin{equation}
\begin{matrix}
  w_{i}= \frac{\prod_{j=1}^{m} \left [  cdf_{X_{j}}(x_{i,j+1}) - cdf_{X_{j}}(x_{i,j}) \right ]}{\sum_{points}\prod_{j=1}^{m} \left [  cdf_{X_{j}}(x_{i,j+1}) - cdf_{X_{j}}(x_{i,j}) \right ]}
\end{matrix}
\end{equation}
Each realization carries a weight representative of each stratum.
\\Now consider
an $n-$strata of the domain of  $\overline{X}$, and compute the expected value of $f(x)$ over the discretization. Based on the previous equation, the computation of the expected value of $f(x)$ is as follows:
\begin{equation}
 \mathbb{E}(f(\overline{X})) \approx   \widetilde{f}_{n}(\overline{x}) = \frac{1}{\sum_{i=1}^{n}w_{i}} \sum_{i=1}^{n} f(\overline{x}_{i}) \times w_{i}
\end{equation}
%%%%%%%%%%%%%%%%%%%%%%%%%
\subsection{Sparse Grid Collocation}
\label{sub:SGC}
The Sparse Grid Collocation sampler represents an advanced methodology to perform Uncertainty Quantification. They aim
to explore the input space leveraging the information contained in the associated probability density functions. It builds on generic Grid sampling by selecting evaluation points based on characteristic quadratures as part of stochastic collocation for generalized polynomial chaos uncertainty quantification. In collocation an N-D grid is constructed, with each uncertain variable providing an axis. Along each axis, the points of evaluation correspond to quadrature points necessary to integrate polynomials. In the simplest (and most naive) case, a N-D tensor product of all possible combinations of points from each dimension’s quadrature is constructed as sampling points. The number of necessary samples can be reduced by employing Smolyak-like sparse grid algorithms, which use reduced combinations of polynomial orders to reduce the necessary sampling space.
\subsubsection{Generalized Polynomial Chaos}
In general, polynomial chaos expansion (PCE) methods seek to interpolate the simulation code as a combination of
polynomials of varying degree in each dimension of the input space.  Originally Wiener
proposed expanding in Hermite polynomials for Gaussian-normal distributed variables~\cite{wiener}.  Askey and
Wilson generalized Hermite polynomials to include Jacobi polynomials, including Legendre and Laguerre
polynomials~\cite{Wiener-Askey}.  Xiu and Karniadakis combines these concepts to perform PCE for a range of Gaussian-based
distributions with corresponding polynomials,
including Legendre polynomials for uniform distributions, Laguerre polynomials for Gamma distributions, and
Jacobi polynomials for Beta distributions \cite{xiu}.

In each of these cases, a probability-weighted
integral over the distribution can be cast in a way that the corresponding polynomials are orthogonal over the
same weight and interval.  These chaos Wiener-Askey polynomials were used by Xiu and Karniadakis to develop
the generalized polynomial chaos expansion method (gPC), including a transformation for applying the same
method to arbitrary distributions (as long as they have a known inverse CDF)~\cite{xiu}.  Two significant
methodologies have grown from gPC application.  The first makes use of Lagrange polynomials to expand the
original function or simulation code, as they can be made orthogonal over the same domain as the
distributions~\cite{SCLagrange}; the other uses the Wiener-Askey polynomials~\cite{xiu}.

Consider a simulation code that produces a quantity of interest $u$ as a function $u(Y)$ whose arguments are
the uncertain, distributed input
parameters $Y=(Y_1,\ldots,Y_n,\ldots,Y_N)$.  A particular realization $\omega$ of $Y_n$ is expressed by
$Y_n(\omega)$, and a single realization of the entire input space results in a solution to the function as
$u(Y(\omega))$. Obtaining a realization of $u(Y)$ may take considerable computation time and
effort.
$u(Y)$ gets expanded in orthonormal multidimensional polynomials $\Phi_k(Y)$, where $k$ is a multi-index tracking
the polynomial order in each axis of the polynomial Hilbert space, and $\Phi_k(Y)$ is constructed as
\begin{equation}\label{eq:gPC}
  \Phi_k(Y) = \prod_{n=1}^N \phi_{k_n}(Y_n),
\end{equation}
where $\phi_{k_n}(Y_n)$ is a single-dimension Wiener-Askey orthonormal polynomial of order $k_n$ and
$k=(k_1,\ldots,k_n,\ldots,k_N)$, $k_n\in\mathbb{N}^0$.  For example, given $u(y_1,y_2,y_3)$, $k=(2,1,4)$
is the multi-index of the
product of a second-order polynomial in $y_1$, a first-order polynomial in $y_2$, and a fourth-order
polynomial in $y_4$. The gPC for $u(Y)$ using this notation is
\begin{equation}
  u(Y) \approx \sum_{k\in\Lambda(L)} u_k\Phi_k(Y),
\end{equation}
where $u_k$ is a scalar weighting polynomial coefficient. The polynomials used in the expansion are determined
by the set of multi-indices $\Lambda(L)$, where $L$ is a truncation order.  In the limit
that $\Lambda$ contains all possible combinations of polynomials of any order, Eq. \ref{eq:gPC} is exact.
Practically, however, $\Lambda$ is truncated to some finite set of combinations, discussed in section
\ref{sec:indexSets}.

Using the orthonormal properties of the Wiener-Askey polynomials,
\begin{equation}
  \int_\Omega \Phi_k(Y)\Phi_{\hat k}(Y) \rho(Y) dY = \delta_{k\hat k},
\end{equation}
where $\rho(Y)$ is the combined PDF of $Y$, $\Omega$ is the multidimensional domain of $Y$, and $\delta_{nm}$
is the Dirac delta, an expression of the polynomial expansion coefficients can be isolated.
We multiply both sides of Eq. \ref{eq:gPC} by
$\Phi_{\hat k}(Y)$, integrate both sides over the probability-weighted input domain, and sum over all $\hat k$
to obtain the coefficients, sometimes referred to as polynomial expansion moments,
\begin{align}\label{eq:polycoeff}
  u_k &= \frac{\langle u(Y)\Phi_k(Y) \rangle}{\langle \Phi_k(Y)^2 \rangle},\\
      &= \langle u(Y)\Phi_k(Y) \rangle,
\end{align}
where we use the angled bracket notation to denote the probability-weighted inner product,
\begin{equation}
  \langle f(Y) \rangle \equiv \int_\Omega f(Y)\rho(Y) dY.
\end{equation}
When $u(Y)$ has an analytic form, these coefficients can be solved by integration; however, in general other
methods must be applied to numerically perform the integral.  While tools such as Monte Carlo integration can
be used to evaluate the integral, the properties of Gaussian quadratures, because of the
probability weights and domain, can be harnessed.  This stochastic collocation method is discussed in section \ref{sec:stoch
coll}.
\subsubsection{Polynomial Index Set Construction}\label{sec:indexSets}
The main concern in expanding a function in interpolating multidimensional polynomials is choosing appropriate polynomials to
make up the expansion.
There are many generic ways by which a polynomial set can be constructed.  Here  three static
approaches are presented:
\begin{itemize}
  \item Tensor Product
  \item Total Degree
  \item Hyperbolic Cross.
\end{itemize}
In the nominal tensor
product case, $\Lambda(L)$ contains all possible combinations of polynomial indices up to truncation order $L$ in each
dimension, as:
\begin{equation}
  \Lambda_\text{TP}(L)=\Big\{\bar p=(p_1,\cdots,p_N): \max_{1\leq n\leq N}p_n\leq L
\Big\}.
\end{equation}
The cardinality of this index set is $|\Lambda_\text{TP}(L)|=(L+1)^N$. For example, for a two-dimensional
input space ($N$=2) and truncation limit $L=3$, the index set $\Lambda_\text{TP}(3)$ is given in Table
\ref{tab:TP}, where the notation $(1,2)$ signifies the product of a polynomial that is first order in $Y_1$
and second order in $Y_2$.

\begin{table}[h]
  \centering
  \begin{tabular}{c c c c}
    (3,0) & (3,1) & (3,2) & (3,3) \\
    (2,0) & (2,1) & (2,2) & (2,3) \\
    (1,0) & (1,1) & (1,2) & (1,3) \\
    (0,0) & (0,1) & (0,2) & (0,3)
  \end{tabular}
  \caption{Tensor Product Index Set, $N=2,L=3$.}
  \label{tab:TP}
\end{table}

It is evident there is some inefficiencies in this index set.  First, it suffers dramatically from the
\emph{curse of dimensionality}; that is, the number of polynomials required grows exponentially with
increasing dimensions.  Second, the total order of polynomials is not considered.  Assuming the contribution of
each higher-order polynomial is smaller than lower-order polynomials, the (3,3) term is
contributing sixth-order corrections that are likely smaller than the error introduced by ignoring
fourth-order corrections (4,0) and (0,4).  This leads to the development of the \emph{total degree} (TD) and
\emph{hyperbolic cross} (HC) polynomial index set construction strategies \cite{hctd}.

In TD, only multidimensional polynomials whose \emph{total} order at most $L$ are permitted,
\begin{equation}
  \Lambda_\text{TD}(L)=\Big\{\bar p=(p_1,\cdots,p_N):\sum_{n=1}^N p_n \leq L
\Big\}.
\end{equation}
The cardinality of this index set is $|\Lambda_\text{TD}(L)|={L+N\choose N}$, which grows with increasing
dimensions much more slowly than TP.  For the same $N=2,L=3$ case above, the TD index set is given in Table
\ref{tab:TD}.

\begin{table}[h]
  \centering
  \begin{tabular}{c c c c}
    (3,0) &       &       &       \\
    (2,0) & (2,1) &       &       \\
    (1,0) & (1,1) & (1,2) &       \\
    (0,0) & (0,1) & (0,2) & (0,3)
  \end{tabular}
  \caption{Total Degree Index Set, $N=2,L=3$}
  \label{tab:TD}
\end{table}

In HC, the \emph{product} of polynomial orders is used to restrict allowed polynomials in the index set.  This
tends to polarize the expansion, emphasizing higher-order polynomials in each dimension but lower-order
polynomials in combinations of dimensions, as:
\begin{equation}
  \Lambda_\text{HC}(L)=\Big\{\bar p=(p_1,\ldots,p_N):\prod_{n=1}^N p_n+1 \leq L+1
\Big\}.
\end{equation}
The cardinality of this index set is bounded by $|\Lambda_\text{HC}(L)|\leq (L+1)(1+\log(L+1))^{N-1}$. It
grows even more slowly than TD with increasing dimension, as shown in Table \ref{tab:HC} for $N=2,L=3$.

\begin{table}[h]
  \centering
  \begin{tabular}{c c c c}
    (3,0) &       &       &       \\
    (2,0) &       &       &       \\
    (1,0) & (1,1) &       &       \\
    (0,0) & (0,1) & (0,2) & (0,3)
  \end{tabular}
  \caption{Hyperbolic Cross Index Set, $N=2,L=3$}
  \label{tab:HC}
\end{table}

It has been shown that the effectiveness of TD and HC as index set choices depends strongly on the regularity
of the responce \cite{hctd}.  TD tends to be most effective for infinitely-continuous response surfaces,
while HC is more effective for surfaces with limited smoothness or discontinuities.

\subsubsection{Anisotropy}
While using TD or HC to construct the polynomial index set combats the curse of dimensionality present in TP,
it is not eliminated and continues to be an issue for problems of large dimensionality.  Another method that can
be applied to mitigate this issue is index set anisotropy, or the unequal treatment of various dimensions.
In this strategy, weighting factors $\alpha=(\alpha_1,\ldots,\alpha_n,\ldots,\alpha_N)$ are applied in each
dimension to allow additional polynomials in some dimensions and less in others.  This change adjusts the TD
and HC construction rules as follows, where $|\alpha|_1$ is the one-norm of $\alpha$.
\begin{equation}
  \tilde\Lambda_{TD}(L)=\Big\{\bar p=(p_1,\ldots,p_N):\sum_{n=1}^{N} \alpha_n p_{n} \leq |\vec\alpha|_1 L\Big\},
\end{equation}

\begin{equation}
  \tilde\Lambda_\text{HC}(L)=\Big\{\bar p=(p_1,\cdots,p_N):\prod_{n=1}^N (p_n+1)^{\alpha_n} \leq
  (L+1)^{|\vec\alpha|_1} \Big\}
\end{equation}
As it is desirable to obtain the isotropic case from a reduction of the anisotropic cases, define the
one-norm for the weights is defined as:
\begin{equation}
  |\alpha|_1 = \frac{\sum_{n=1}^N \alpha_n}{N}.
\end{equation}
Considering the same case above ($N=2,L=3$), it can be applied weights $\alpha_1=5,\alpha_2=3$, and the resulting index
sets are Tables \ref{tab:aniTD} (TD) and \ref{tab:aniHC} (HC).

\begin{table}[h]
  \centering
  \begin{tabular}{c c c c c}
    (2,0) &       &       &       & \\
    (1,0) & (1,1) & (1,2) &       & \\
    (0,0) & (0,1) & (0,2) & (0,3) & (0,4)
  \end{tabular}
  \caption{Anisotropic Total Degree Index Set, $N=2,L=3$.}
  \label{tab:aniTD}
\end{table}

\begin{table}[h]
  \centering
  \begin{tabular}{c c c c}
    (1,0) &       &       &       \\
    (0,0) & (0,1) & (0,2) & (0,3)
  \end{tabular}
  \caption{Anisotropic Hyperbolic Cross Index Set, $N=2,L=3$.}
  \label{tab:aniHC}
\end{table}

There are many methods by which anisotropy weights can be assigned.  Often, if a problem is well-known to an
analyst, it may be enough to use heuristics to assign importance arbitrarily.  Otherwise, a smaller
uncertainty quantification solve can be used to roughly determine sensitivity coefficients (such as Pearson
coefficients), and the inverse of those can then be applied as anisotropy weights.  Sobol coefficients
obtained from first- or second-order HDMR, an additional sampling strategy present in RAVEN, could also serve as a basis for these weights.
A good choice of anisotropy weight can greatly speed up convergence; however, a
poor choice can slow convergence considerably, as computational resources are used to resolve low-importance
dimensions.

\subsubsection{Stochastic Collocation}\label{sec:stoch coll}
Stochastic collocation is the process of using collocated points to approximate integrals of stochastic space
numerically.  In particular consider using Gaussian quadratures (Legendre, Hermite, Laguerre, and Jacobi)
corresponding to the polynomial expansion polynomials for numerical integration.  Quadrature integration takes
the form:
\begin{align}
  \int_a^b f(x)\rho(x) &= \sum_{\ell=1}^\infty w_\ell f(x_\ell),\\
  &\approx \sum_{\ell=1}^{\hat L} w_\ell f(x_\ell),
\end{align}
where $w_\ell,x_\ell$ are corresponding points and weights belonging to the quadrature set, truncated at order
$\hat L$.  At this point, this $\hat L$ should not be confused with the polynomial expansion truncation order $L$.  This expression can be simplified using the operator notation
\begin{equation}\label{eq:quad op}
  q^{(\hat L)}[f(x)] \equiv \sum_{\ell=1}^{\hat L} w_\ell f(x_\ell).
\end{equation}
A nominal multidimensional quadrature is the tensor product of
individual quadrature weights and points, and can be written
\begin{align}
  Q^{(\vec{L})} &= q^{(\hat L_1)}_1 \otimes q^{(\hat L_2)}_2 \otimes \cdots,\\
                     &= \bigotimes_{n=1}^N q^{(\hat L_n)}_n.
\end{align}
It is worth noting that each quadrature may have distinct points and weights; they need to not be constructed using
the same quadrature rule.
In general, one-dimensional Gaussian
quadrature excels in exactly integrating polynomials of order $2p-1$ using $p$ points and weights;
equivalently, it requires $(p+1)/2$ points to integrate an order $p$ polynomial.

For convenience, the coefficient integral to be evaluated is here reported again, Eq.
\ref{eq:polycoeff}.
\begin{equation}
  u_k = \langle u(Y)\Phi_k(Y) \rangle.
\end{equation}
This integral can be approximated with the appropriate Gaussian quadrature as
\begin{align}
  u_k &\approx Q^{(\vec{\hat L})}[u(Y)\Phi_k(Y)],
\end{align}
where bold vector notation is used to note the order of each individual quadrature,
$\vec{\hat L} = [\hat L_1, \ldots,\hat L_n,\ldots,\hat L_N]$. For clarity, the bold notation is removed and
it is assumed a one-dimensional problem, which extrapolates as expected into the multidimensional case.
\begin{align}
  u_k &\approx q^{(\hat L)}[u(Y)\Phi_k(Y)],\\
      &= \sum_{\ell=1}^{\hat L} w_\ell u(Y_\ell)\Phi_k(Y_\ell).
\end{align}
To determine the quadrature order $\hat L$ needed to accurately integrate this expression, consider the
gPC formulation for $u(Y)$ in Eq. \ref{eq:gPC} and replace it in the sum,
\begin{equation}
  u_k\approx \sum_{\ell=1}^{\hat L} w_\ell \Phi_k(Y_\ell) \sum_{k\in\Lambda(L)}u_{\hat k}\Phi_{\hat k}(Y_\ell).
\end{equation}
Using orthogonal properties of the polynomials, this reduces as $\hat L\to\infty$ to
\begin{equation}
  u_k\approx \sum_{\ell=1}^{\hat L} w_\ell u_k \Phi_k(Y_\ell)^2.
\end{equation}
Thus, the integral, to the same error introduced by truncating the  gPC expansion, the quadrature is
approximating an integral of order $2k$. As a result, the quadrature order should be ordered:
\begin{equation}
  p=\frac{2k+1}{2}=k+\frac{1}{2}<k+1,
\end{equation}
so it  can conservatively used  $p=k+1$.  In the case of the largest polynomials with order
$k=L$, the quadrature size $\hat L$ is the same as $L+1$.  It is worth noting that if $u(Y)$ is effectively of
much higher-order polynomial than $L$, this equality for quadrature order does not hold true; however, it also
means that gPC of order $L$ will be a poor approximation.

While a tensor product of highest-necessary quadrature orders could serve as a suitable multidimensional
quadrature set, we can make use of Smolyak-like sparse quadratures to reduce the number of function
evaluations necessary for the TD and HC polynomial index set construction strategies.

\subsubsection{Smolyak Sparse Grids}
Smolyak sparse grids \cite{smolyak} are an attempt to discover the smallest necessary quadrature set to
integrate a multidimensional integral with varying orders of predetermined quadrature sets.  In RAVEN case, the
polynomial index sets determine the quadrature orders each one needs in each dimension to be integrated
accurately.  For example, the polynomial index set point (2,1,3) requires three points in $Y_1$, two in $Y_2$,
and four in $Y_3$,or
\begin{equation}
  Q^{(2,1,3)} = q^{(3)}_1 \otimes q^{(2)}_2 \otimes q^{(4)}_3.
\end{equation}
The full tensor grid of all collocation points would be the tensor product of all quadrature for all points,
or
\begin{equation}
  Q^{(\Lambda(L))} = \bigotimes_{k\in\Lambda}Q^{(k)}.
\end{equation}
Smolyak sparse grids consolidate this tensor form by adding together the points from tensor products of subset
quadrature sets.  Returning momentarily to a one-dimensional problem, introduce the notation
\begin{equation}
  \Delta_k^{(\hat L)}[f(x)] \equiv (q_k^{(\hat L)} - q_{k-1}^{(\hat L)})[f(x)],
\end{equation}
\begin{equation}
  q_0^{(\hat L)}[f(x)] = 0.
\end{equation}
A Smolyak sparse grid is then defined and applied to the desired integral in Eq. \ref{eq:polycoeff},
\begin{equation}
  S^{(\vec{\hat L})}_{\Lambda,N}[u(Y)\Phi_k(Y)] = \sum_{k\in\Lambda(L)} \left(\Delta_{k_1}^{(\hat L_1)} \otimes \cdots \otimes
  \Delta_{k_N}^{(\hat L_N)}\right)[u(Y)\Phi_k(Y)].
\end{equation}
Equivalently, and in a more algorithm-friendly approach,
\begin{equation}
  S^{(\vec{\hat L})}_{\Lambda,N}[u(Y)\Phi_k(Y)] = \sum_{k\in\Lambda(L)} c(k)\bigotimes_{n=1}^N
  q^{(\hat L_n)}_n[u(Y)\Phi_k(Y)]
\end{equation}
where
\begin{equation}
  c(k) = \sum_{\substack{j=\{0,1\}^N,\\k+j\in\Lambda}} (-1)^{|j|_1},
\end{equation}
using the traditional 1-norm for $|j|_1$.
The values for $u_k$ can then be calculated as
\begin{align}
  u_k &= \langle u(Y)\Phi_k(Y) \rangle,\\
      &\approx S^{(\vec{\hat L})}_{\Lambda,N}[u(Y)\Phi_k(Y)].
\end{align}
With this numerical method to determine coefficients,  a complete method for performing SCgPC
analysis in an algorithmic manner is obtained.








