\section{OutStreams Plot Plugins}
\label{sec:outstreams_plot}
New plots that are specific to particular applications are possible through \xmlNode{OutStreams}
\xmlNode{Plot} plugins.

These plotting plugins should inherit from the \texttt{PlotPlugin} base class defined in
\begin{lstlisting}[language=bash]
  raven/framework/PluginBaseClasses/OutStreamPlotPlugin.py
\end{lstlisting}
which sets up the plotting tool to be found when RAVEN runs.
When loading an ``OutStreams Plot Plugin'', RAVEN expects to find, in the class
representing the plugin, the following required methods:
\begin{lstlisting}[language=python, basicstyle=\scriptsize\ttfamily, breaklines=True, columns=fullflexible]
  from ravenframework.PluginBaseClasses.OutStreamPlotPlugin import PlotPlugin
  class NewPlugin(PlotPlugin):
    ...
    def run(self):
      """
        Generate the plot
        @ In, None
        @ Out, None
      """
\end{lstlisting}
In addition, the following optional methods can be specified:
\begin{lstlisting}[language=python, basicstyle=\scriptsize\ttfamily, breaklines=True, columns=fullflexible]
  from ravenframework.PluginBaseClasses.OutStreamPlotPlugin import PlotPlugin
  class NewPlugin(PlotPlugin):
    ...
    def __init__(self):
      """
        Constructor.
        @ In, None
        @ Out, None
      """
    ...
    @classmethod
    def getInputSpecification(cls):
      """
        Define the acceptable user inputs for this class.
        @ In, None
        @ Out, specs, InputData.ParameterInput,
      """
    ...
    def handleInput(self, spec):
      """
        Reads in data from the input file
        @ In, spec, InputData.ParameterInput, input information
        @ Out, None
      """
    ...
    def initialize(self, stepEntities):
      """
        Set up plotter for each run
        @ In, stepEntities, dict, entities from the Step
        @ Out, None
      """
\end{lstlisting}

A good example of the \texttt{PlotPlugin} can be found in the RAVEN ExamplePlugin, found at
\begin{lstlisting}[language=bash]
  raven/plugins/ExamplePlugin/src/CorrelationPlot.py
\end{lstlisting}

In the following sub-sections, these methods are explained in detail.

\subsection{Optional Methods}
The explanation for \texttt{\_\_init\_\_}, \texttt{getInputSpecification} and \texttt{handleInput}
methods can be found in \ref{subsubsec:init}, \ref{subsubsec:getInputSpecification},
\ref{subsubsec:handleInput}, respectively. All of these methods require a call to \texttt{super} to
function as expected.
%
%
\subsection{\texttt{run} method, required}
The \texttt{run} method is the primary execution method for the PlotPlugin. Here, whatever data
handling and plotting mechanics will be executed. Note that \texttt{run} does not receive any inputs;
often, the source for the data to be plotted will be identified in the \texttt{initialize} method.

The \texttt{run} method can perform many actions including data manipulation, creation of
\texttt{matplotlib} figures and axes, saving figures to file, and so forth. In the end, \texttt{run}
should not return anything.
%
%
\subsection{initialization, optional}
The aptly-named \texttt{initialize} method is used at the start of every RAVEN \xmlNode{Step} to prepare
for execution. This method has been simplified and it only accepts one input variable \texttt{stepEntities}.
A common task for this method is to find the source of the data to plot. To make this
process easier, RAVEN provides a \texttt{self.findSource} method that can search the input dictionary
provided to \texttt{initialize} and find a \xmlNode{DataObject} by string name. In the following
an example is reported:

\begin{lstlisting}[language=python]
  def initialize(self, stepEntities):
    """
      Set up plotter for each run
      @ In, stepEntities, dict, entities from the Step
      @ Out, None
    """
    src = self.findSource('DataObjectName', stepEntities)
\end{lstlisting}
%
%
