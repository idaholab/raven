%
% This is an example LaTeX file which uses the SANDreport class file.
% It shows how a SAND report should be formatted, what sections and
% elements it should contain, and how to use the SANDreport class.
% It uses the LaTeX article class, but not the strict option.
% ItINLreport uses .eps logos and files to show how pdflatex can be used
%
% Get the latest version of the class file and more at
%    http://www.cs.sandia.gov/~rolf/SANDreport
%
% This file and the SANDreport.cls file are based on information
% contained in "Guide to Preparing {SAND} Reports", Sand98-0730, edited
% by Tamara K. Locke, and the newer "Guide to Preparing SAND Reports and
% Other Communication Products", SAND2002-2068P.
% Please send corrections and suggestions for improvements to
% Rolf Riesen, Org. 9223, MS 1110, rolf@cs.sandia.gov
%
\documentclass[pdf,12pt]{INLreport}
% pslatex is really old (1994).  It attempts to merge the times and mathptm packages.
% My opinion is that it produces a really bad looking math font.  So why are we using it?
% If you just want to change the text font, you should just \usepackage{times}.
% \usepackage{pslatex}
\usepackage{times}
\usepackage[FIGBOTCAP,normal,bf,tight]{subfigure}
\usepackage{amsmath}
\usepackage{amssymb}
\usepackage{soul}
\usepackage{pifont}
\usepackage{enumerate}
\usepackage{listings}
\usepackage{fullpage}
\usepackage{xcolor}          % Using xcolor for more robust color specification
\usepackage{ifthen}          % For simple checking in newcommand blocks
\usepackage{textcomp}
\usepackage{mathtools}
\usepackage{lscape}
\usepackage[toc,page]{appendix}
\usepackage{RAVEN}

\newtheorem{mydef}{Definition}
\newcommand{\norm}[1]{\lVert#1\rVert}
%\usepackage[table,xcdraw]{xcolor}
%\usepackage{authblk}         % For making the author list look prettier
%\renewcommand\Authsep{,~\,}

% Custom colors
\definecolor{deepblue}{rgb}{0,0,0.5}
\definecolor{deepred}{rgb}{0.6,0,0}
\definecolor{deepgreen}{rgb}{0,0.5,0}
\definecolor{forestgreen}{RGB}{34,139,34}
\definecolor{orangered}{RGB}{239,134,64}
\definecolor{darkblue}{rgb}{0.0,0.0,0.6}
\definecolor{gray}{rgb}{0.4,0.4,0.4}

\lstset {
  basicstyle=\ttfamily,
  frame=single
}


\setcounter{secnumdepth}{5}
\lstdefinestyle{XML} {
    language=XML,
    extendedchars=true,
    breaklines=true,
    breakatwhitespace=true,
%    emph={name,dim,interactive,overwrite},
    emphstyle=\color{red},
    basicstyle=\ttfamily,
%    columns=fullflexible,
    commentstyle=\color{gray}\upshape,
    morestring=[b]",
    morecomment=[s]{<?}{?>},
    morecomment=[s][\color{forestgreen}]{<!--}{-->},
    keywordstyle=\color{cyan},
    stringstyle=\ttfamily\color{black},
    tagstyle=\color{darkblue}\bf\ttfamily,
    morekeywords={name,type},
%    morekeywords={name,attribute,source,variables,version,type,release,x,z,y,xlabel,ylabel,how,text,param1,param2,color,label},
}
\lstset{language=python,upquote=true}

\usepackage{titlesec}
\newcommand{\sectionbreak}{\clearpage}
\setcounter{secnumdepth}{4}

%\titleformat{\paragraph}
%{\normalfont\normalsize\bfseries}{\theparagraph}{1em}{}
%\titlespacing*{\paragraph}
%{0pt}{3.25ex plus 1ex minus .2ex}{1.5ex plus .2ex}

%%%%%%%% Begin comands definition to input python code into document
\usepackage[utf8]{inputenc}

% Default fixed font does not support bold face
\DeclareFixedFont{\ttb}{T1}{txtt}{bx}{n}{9} % for bold
\DeclareFixedFont{\ttm}{T1}{txtt}{m}{n}{9}  % for normal

\usepackage{listings}

% Python style for highlighting
\newcommand\pythonstyle{\lstset{
language=Python,
basicstyle=\ttm,
otherkeywords={self, none, return},             % Add keywords here
keywordstyle=\ttb\color{deepblue},
emph={MyClass,__init__},          % Custom highlighting
emphstyle=\ttb\color{deepred},    % Custom highlighting style
stringstyle=\color{deepgreen},
frame=tb,                         % Any extra options here
showstringspaces=false            %
}}


% Python environment
\lstnewenvironment{python}[1][]
{
\pythonstyle
\lstset{#1}
}
{}

% Python for external files
\newcommand\pythonexternal[2][]{{
\pythonstyle
\lstinputlisting[#1]{#2}}}

\lstnewenvironment{xml}
{}
{}

% Python for inline
\newcommand\pythoninline[1]{{\pythonstyle\lstinline!#1!}}

% Named Colors for the comments below (Attempted to match git symbol colors)
\definecolor{RScolor}{HTML}{8EB361}  % Sonat (adjusted for clarity)
\definecolor{DPMcolor}{HTML}{E28B8D} % Dan
\definecolor{JCcolor}{HTML}{82A8D9}  % Josh (adjusted for clarity)
\definecolor{AAcolor}{HTML}{8D7F44}  % Andrea
\definecolor{CRcolor}{HTML}{AC39CE}  % Cristian
\definecolor{RKcolor}{HTML}{3ECC8D}  % Bob (adjusted for clarity)
\definecolor{DMcolor}{HTML}{276605}  % Diego (adjusted for clarity)
\definecolor{PTcolor}{HTML}{990000}  % Paul

\def\DRAFT{} % Uncomment this if you want to see the notes people have been adding
% Comment command for developers (Should only be used under active development)
\ifdefined\DRAFT
  \newcommand{\nameLabeler}[3]{\textcolor{#2}{[[#1: #3]]}}
\else
  \newcommand{\nameLabeler}[3]{}
\fi
\newcommand{\alfoa}[1] {\nameLabeler{Andrea}{AAcolor}{#1}}
\newcommand{\cristr}[1] {\nameLabeler{Cristian}{CRcolor}{#1}}
\newcommand{\mandd}[1] {\nameLabeler{Diego}{DMcolor}{#1}}
\newcommand{\maljdan}[1] {\nameLabeler{Dan}{DPMcolor}{#1}}
\newcommand{\cogljj}[1] {\nameLabeler{Josh}{JCcolor}{#1}}
\newcommand{\bobk}[1] {\nameLabeler{Bob}{RKcolor}{#1}}
\newcommand{\senrs}[1] {\nameLabeler{Sonat}{RScolor}{#1}}
\newcommand{\talbpaul}[1] {\nameLabeler{Paul}{PTcolor}{#1}}
% Commands for making the LaTeX a bit more uniform and cleaner
\newcommand{\TODO}[1]    {\textcolor{red}{\textit{(#1)}}}
\newcommand{\xmlAttrRequired}[1] {\textcolor{red}{\textbf{\texttt{#1}}}}
\newcommand{\xmlAttr}[1] {\textcolor{cyan}{\textbf{\texttt{#1}}}}
\newcommand{\xmlNodeRequired}[1] {\textcolor{deepblue}{\textbf{\texttt{<#1>}}}}
\newcommand{\xmlNode}[1] {\textcolor{darkblue}{\textbf{\texttt{<#1>}}}}
\newcommand{\xmlString}[1] {\textcolor{black}{\textbf{\texttt{'#1'}}}}
\newcommand{\xmlDesc}[1] {\textbf{\textit{#1}}} % Maybe a misnomer, but I am
                                                % using this to detail the data
                                                % type and necessity of an XML
                                                % node or attribute,
                                                % xmlDesc = XML description
\newcommand{\default}[1]{~\\*\textit{Default: #1}}
\newcommand{\nb} {\textcolor{deepgreen}{\textbf{~Note:}}~}


%%%%%%%% End comands definition to input python code into document

%\usepackage[dvips,light,first,bottomafter]{draftcopy}
%\draftcopyName{Sample, contains no OUO}{70}
%\draftcopyName{Draft}{300}

% The bm package provides \bm for bold math fonts.  Apparently
% \boldsymbol, which I used to always use, is now considered
% obsolete.  Also, \boldsymbol doesn't even seem to work with
% the fonts used in this particular document...
\usepackage{bm}


% Define tensors to be in bold math font.
\newcommand{\tensor}[1]{{\bm{#1}}}

% Override the formatting used by \vec.  Instead of a little arrow
% over the letter, this creates a bold character.
\renewcommand{\vec}{\bm}

% Define unit vector notation.  If you don't override the
% behavior of \vec, you probably want to use the second one.
\newcommand{\unit}[1]{\hat{\bm{#1}}}
% \newcommand{\unit}[1]{\hat{#1}}

% Use this to refer to a single component of a unit vector.
\newcommand{\scalarunit}[1]{\hat{#1}}

% Aliases
\newcommand{\raven}{\texttt{RAVEN}}
\newcommand{\ravens}{\texttt{RAVEN}'s}
\newcommand{\plugin}{\raven{} Plugin}
\newcommand{\plugins}{\raven{} Plugins}


% \toprule, \midrule, \bottomrule for tables
\usepackage{booktabs}

% \llbracket, \rrbracket
\usepackage{stmaryrd}

\usepackage{hyperref}
\hypersetup{
    colorlinks,
    citecolor=black,
    filecolor=black,
    linkcolor=black,
    urlcolor=black
}

% Compress lists of citations like [33,34,35,36,37] to [33-37]
\usepackage{cite}

% If you want to relax some of the SAND98-0730 requirements, use the "relax"
% option. It adds spaces and boldface in the table of contents, and does not
% force the page layout sizes.
% e.g. \documentclass[relax,12pt]{SANDreport}
%
% You can also use the "strict" option, which applies even more of the
% SAND98-0730 guidelines. It gets rid of section numbers which are often
% useful; e.g. \documentclass[strict]{SANDreport}

% The INLreport class uses \flushbottom formatting by default (since
% it's intended to be two-sided document).  \flushbottom causes
% additional space to be inserted both before and after paragraphs so
% that no matter how much text is actually available, it fills up the
% page from top to bottom.  My feeling is that \raggedbottom looks much
% better, primarily because most people will view the report
% electronically and not in a two-sided printed format where some argue
% \raggedbottom looks worse.  If we really want to have the original
% behavior, we can comment out this line...
\raggedbottom
\setcounter{secnumdepth}{5} % show 5 levels of subsection
\setcounter{tocdepth}{5} % include 5 levels of subsection in table of contents

% ---------------------------------------------------------------------------- %
%
% Set the title, author, and date
%
\title{RAVEN Plugins Manual}
%\author{%
%\begin{tabular}{c} Author 1 \\ University1 \\ Mail1 \\ \\
%Author 3 \\ University3 \\ Mail3 \end{tabular} \and
%\begin{tabular}{c} Author 2 \\ University2 \\ Mail2 \\ \\
%Author 4 \\ University4 \\ Mail4\\
%\end{tabular} }


\author{
\\Paul W. Talbot
\\Congjian Wang
}

% There is a "Printed" date on the title page of a SAND report, so
% the generic \date should [WorkingDir:]generally be empty.
\date{}


% ---------------------------------------------------------------------------- %
% Set some things we need for SAND reports. These are mandatory
%
\SANDnum{INL/EXT-TODO}
\SANDprintDate{\today}
\SANDauthor{Paul W. Talbot, Congjian Wang}
\SANDreleaseType{Revision 0}


% ---------------------------------------------------------------------------- %
% Include the markings required for your SAND report. The default is "Unlimited
% Release". You may have to edit the file included here, or create your own
% (see the examples provided).
%
% \include{MarkOUO} % Not needed for unlimted release reports

\def\component#1{\texttt{#1}}

% ---------------------------------------------------------------------------- %
\newcommand{\systemtau}{\tensor{\tau}_{\!\text{SUPG}}}

% Added by Sonat
\usepackage{placeins}
\usepackage{array}

\newcolumntype{L}[1]{>{\raggedright\let\newline\\\arraybackslash\hspace{0pt}}m{#1}}
\newcolumntype{C}[1]{>{\centering\let\newline\\\arraybackslash\hspace{0pt}}m{#1}}
\newcolumntype{R}[1]{>{\raggedleft\let\newline\\\arraybackslash\hspace{0pt}}m{#1}}

% end added by Sonat
% ---------------------------------------------------------------------------- %
%
% Start the document
%

\begin{document}
    \sloppy
    \maketitle

    % ------------------------------------------------------------------------ %
    % An Abstract is required for SAND reports
    %
%    \begin{abstract}
%    \input abstract
%    \end{abstract}


    % ------------------------------------------------------------------------ %
    % An Acknowledgement section is optional but important, if someone made
    % contributions or helped beyond the normal part of a work assignment.
    % Use \section* since we don't want it in the table of context
    %
%    \clearpage
%    \section*{Acknowledgment}



%	The format of this report is based on information found
%	in~\cite{Sand98-0730}.


    % ------------------------------------------------------------------------ %
    % The table of contents and list of figures and tables
    % Comment out \listoffigures and \listoftables if there are no
    % figures or tables. Make sure this starts on an odd numbered page
    %
    \cleardoublepage		% TOC needs to start on an odd page
    \tableofcontents
    %\listoffigures
    %\listoftables


    % ---------------------------------------------------------------------- %
    % An optional preface or Foreword
%    \clearpage
%    \section*{Preface}
%    \addcontentsline{toc}{section}{Preface}
%	Although muggles usually have only limited experience with
%	magic, and many even dispute its existence, it is worthwhile
%	to be open minded and explore the possibilities.


    % ---------------------------------------------------------------------- %
    % An optional executive summary
    %\clearpage
    %\section*{Summary}
    %\addcontentsline{toc}{section}{Summary}
    %\input{Summary.tex}
%	Once a certain level of mistrust and skepticism has
%	been overcome, magic finds many uses in todays science



%	and engineering. In this report we explain some of the
%	fundamental spells and instruments of magic and wizardry. We
%	then conclude with a few examples on how they can be used
%	in daily activities at national Laboratories.


    % ---------------------------------------------------------------------- %
    % An optional glossary. We don't want it to be numbered
%    \clearpage
%    \section*{Nomenclature}
%    \addcontentsline{toc}{section}{Nomenclature}
%    \begin{description}
%          \item[alohomoral]
%           spell to open locked doors and containers
%          \item[leviosa]
%           spell to levitate objects
%    \item[remembrall]
%           device to alert you that you have forgotten something
%    \item[wand]
%           device to execute spells
%    \end{description}


    % ---------------------------------------------------------------------- %
    % This is where the body of the report begins; usually with an Introduction
    %
    \SANDmain		% Start the main part of the report





\section{Introduction}
% High-level RAVEN description
RAVEN~\cite{alfonsiMC} ~\cite{alfonsiPSA}~\cite{RAVENFY13}~\cite{ESREL2014} is a software framework that allows the user to perform parametric and stochastic
analysis based on the response of complex system codes.
The initial development was designed to provide dynamic probabilistic risk analysis
capabilities (DPRA) to the thermal-hydraulic code RELAP-7~\cite{relap7FY12}, currently under development
at Idaho National Laboratory (INL).
Now, RAVEN is not only a framework to perform DPRA but it is a
multi-purpose stochastic and uncertainty quantification platform, capable of communicating with any system code.

The provided Application Programming
Interfaces (APIs) allow RAVEN to interact with any code as long as all the parameters
that need to be perturbed are accessible by input files or via python
interfaces.
RAVEN is capable of investigating system response and explore input space using various
sampling schemes such as Monte Carlo, grid, or Latin Hypercube.
However, RAVEN strength lies in its system feature discovery capabilities such as: constructing
limit surfaces, separating regions of the input space leading to system failure,
and using dynamic supervised learning techniques.

The development of RAVEN started in 2012 when, within the Nuclear Energy
Advanced Modeling and Simulation (NEAMS) program~\cite{neams}, the need of a modern
risk evaluation framework arose.
RAVEN's principal assignment is to provide the necessary software and algorithms
in order to employ the concepts developed by the Risk Informed Safety Margin
Characterization (RISMC) Pathway.
RISMC is one of the pathways defined within the Light Water Reactor
Sustainability (LWRS) program~\cite{lwrs}.

The goal of the RISMC approach is  the identification not only of the frequency of an
event which can potentially lead to system failure, but also the proximity (or lack
thereof) to key safety-related events: the safety margin.
Hence, the approach is interested in identifying and increasing the safety
margins related to those events.
A safety margin is a numerical value quantifying the probability that a safety
metric (e.g. peak pressure in a pipe) is exceeded under certain conditions.
% Conclusion
Most of the capabilities, implemented having Reactor Excursion and Leak Analysis Program v.7
(RELAP-7) as a principal focus, are
easily deployable to other system codes.
%
For this reason, several side activates have been employed (e.g.  RELAP5-3D~\cite{RELAP5userManual}, any Multiphysics Object Oriented
Simulation Environment-based App, etc.)
or are currently ongoing for coupling RAVEN with several different software.
%
The aim of this document is to provide a set of detailed examples that can help the user to become familiar with the RAVEN code.

\section{Making a New RAVEN Plugin}

Creating a new plugin is a straightforward process. It involves setting up a repository,
establishing a basic structure, and installing in RAVEN for testing.

\subsection{Setting up a Repository}
TODO

\subsection{Plugin Structure}
The following directories must be present in the main directory of the plugin in order for RAVEN to
read it correctly:
\begin{itemize}
  \item \texttt{src}, where the entities for RAVEN to load are located;
  \item \texttt{doc}, where the documentation for the plugin and its entities is located.
  \item \texttt{tests}, where continuous integration tests are located;
\end{itemize}

\subsection{Additional Libraries}
If the plugin requires additional libraries, they can extend the \texttt{dependencies.ini} file in
the same manner as RAVEN's dependencies file. Libraries will be added like they are for RAVEN
itself, and a check will be performed to assure no base RAVEN (or other plugin) dependencies are
modified.

\subsection{Installing in RAVEN}
Use the installation script in \texttt{raven/install\_plugins.py}.

This process automatically registers the plugin in the plugin directory, and informs the plugin
about RAVEN (TODO more notes on this!)

Entities in RAVEN that are ready for new strategies via Plugins are
described in the following sections.

\section{ExternalModel Plugins}
\label{sec:newExternalModelPlugin}

The procedure of adding a plugin for the ExternalModel is a straightforward process.
At the initialization stage, RAVEN imports all the Plugins that are contained in this directory and performs some preliminary cross-checks.
\\It is important to notice that the name of class in the Plugin module is the one the user needs to specify when the new plugin
needs to be used. For example, if the Plugin module contains the class 	``NewPlugin'', the \textit{subType}
in the \xmlNode{ExternalModel} block will be 	``pluginName.NewPlugin'':
\begin{lstlisting}[language=python]
  class NewPlugin(ExternalModelPluginBase):
    ...
\end{lstlisting}
\begin{lstlisting}[style=XML,morekeywords={name,file}] %moreemph={name,file}]
  <Models>
    ...
    <ExternalModel name='whatever' subType='pluginName.NewPlugin'>
     ...
    </ExternalModel>
    ...
  </Models>
\end{lstlisting}

In the following sub-sections, a step-by-step procedure for creating a new ExternalModel plugin is outlined.

\subsection{ExternalModel Plugin Input}
\label{subsec:externalModelPluginInput}
When a new ExternalModel plugin is developed, its RAVEN input is almost identical
to the general ExternalModel entity (see \ref{subsec:models_externalModel}).
The specifications of an ExternalModel Plugin must be defined within the XML block
\xmlNode{ExternalModel}.
%
This XML node needs to contain the attributes:

\vspace{-5mm}
\begin{itemize}
  \itemsep0em
  \item \xmlAttr{name}, \xmlDesc{required string attribute}, user-defined name
  of this External Model.
  %
  \nb As with the other objects, this is the name that can be used to refer to
  this specific entity from other input blocks in the XML.
  \item \xmlAttr{subType}, \xmlDesc{required string attribute}, must be equal to the
  plugin type, e.g., \xmlAttr{PluginName.PluginModuleName}. For example, TEAL.CashFlow.
  \nb In case a plugin is requested (through the  \xmlAttr{subType} attribute) the
  attribute \xmlAttr{ModuleToLoad} must not be inputted.
  %
\end{itemize}
\vspace{-5mm}

In order to make the RAVEN code aware of the variables the user is going to
manipulate/use in her/his ExternalModel Plugin, the variables need to be specified
in the \xmlNode{ExternalModel} input block.
%
The user needs to input, within this block, only the variables that RAVEN needs
to be aware of (i.e. the variables are going to directly be used by the Plugin)
and not the local variables that the ExternalModel Plugin developer does not want to,
for example, store in a RAVEN internal object.
%
These variables are specified within a \xmlNode{inputs} and \xmlNode{outputs}  or
\xmlNode{variables} blocks:
\begin{itemize}
  \item \xmlNode{variables}, \xmlDesc{string, required parameter}.
  %
  Comma-separated list of variable names.
  %
  Each variable name needs to match a variable used/defined in the external python
  model. \nb this node is is deprecated and will be removed in future releas of RAVEN in
  favor of the two following nodes \xmlNode{inputs} and \xmlNode{outputs}
  %
  \item \xmlNode{inputs}, \xmlDesc{string, required parameter}.
  %
  Comma-separated list of input variable names.
  %
  Each variable name needs to match a variable used/defined in the external python
  model.
  %
  \item \xmlNode{outputs}, \xmlDesc{string, required parameter}.
  %
  Comma-separated list of output variable names.
  %
  Each variable name needs to match a variable used/defined in the external python
  model.
  %
\end{itemize}

In addition, if the user wants to use the alias system, the following XML block can be used:

\begin{lstlisting}[style=XML, basicstyle=\scriptsize\ttfamily, breaklines=True, columns=fullflexible]
<Models>
  ...
  <ExternalModel name='myName' subType='myPluginName.myPluginModel'>
    ...
    <inputs>ExternalModelInputVariableNameList</inputs>
    <outputs>ExternalModelOutputVariableNameList</outputs>
    <alias variable='RavenAliasForInput' type='input'>ExternalModelInputVariableName</alias>
    <alias variable='RavenAliasForOutput' type='output'>ExternalModelOutputVariableName</alias>
  </ExternalModel>
  ...
</Models>
\end{lstlisting}

The description of the alias is:
\begin{itemize}
\item \xmlNode{alias} \xmlDesc{string, optional field} specifies alias for
any variable of interest in the input or output space for the model.
%
In the body of this node the user specifies the name of the variable that the model is going to use
(during its execution).
%
The actual alias, usable throughout the RAVEN input, is instead defined in the
\xmlAttr{variable} attribute of this tag.
\\The user can specify aliases for both the input and the output space. As sanity check, RAVEN
requires an additional required attribute \xmlAttr{type}. This attribute can be either ``input'' or ``output''.
%
\nb The user can specify as many aliases as needed.
\end{itemize}

When the Plugin variables are defined, at run time, RAVEN initializes
them and tracks their values during the simulation.
%
Each variable defined in the \xmlNode{ExternalModel} block is available in the
Plugin class (in each implemented method ) as the object ``container'' that ``acts''
as a Python ``self''. For example,
\begin{lstlisting}[language=python]
  def run (self, container, inputs):
    print(container.variableA)
\end{lstlisting}

%%%%%%%
\subsection{ExternalModel Plugin Creation}
\label{subsec:externalModelPluginCreation}
As already mentioned, RAVEN imports all the ``ExternalModel Plugins'' at run-time.
In order to make RAVEN
able to drive a newer ExternalModel plugin, the developer needs to code a Python class
containing few methods (with strict syntax) that are called by RAVEN during the simulation.
\\ Every new ``ExternalModel Plugin'' must inherit from a RAVEN base class named
$ExternalModelPluginBase$:
\begin{lstlisting}[language=python]
  class NewPlugin(ExternalModelPluginBase):
    ...
\end{lstlisting}
This base class is needed by RAVEN to identify in the plugins folder which class must
be considered as an ``ExternalModel Plugin''.
\\ In addition, when loading an ``ExternalModel Plugin'', RAVEN expects to find, in the class representing the plugin,
 the following required methods:
\begin{lstlisting}[language=python, basicstyle=\scriptsize\ttfamily, breaklines=True, columns=fullflexible]
from ravenframework.PluginBaseClasses.ExternalModelPluginBase import ExternalModelPluginBase
class NewPlugin(ExternalModelPluginBase):
  def run (self, container, Inputs)
\end{lstlisting}
In addition, the following optional methods can be specified:
\begin{lstlisting}[language=python, basicstyle=\scriptsize\ttfamily, breaklines=True, columns=fullflexible]
from ravenframework.PluginBaseClasses.ExternalModelPluginBase import ExternalModelPluginBase
class NewPlugin(ExternalModelPluginBase):
  ...
  def createNewInput(self, container, inputs, samplerType, **Kwargs)
  def _readMoreXML(self, container, xmlNode)
  def initialize(self,container, runInfo, inputs)
\end{lstlisting}
In the following sub-sections all the methods are fully explained, providing examples.
\subsubsection{Method: \texttt{run}}
\label{subsubsec:runExternalModelPlugin}
\begin{lstlisting}[language=python]
def run (self, container, Inputs)
\end{lstlisting}
As stated previously, the only method that \emph{must} be present in an
ExternalModel Plugin is the \textbf{run} function.
%
In this function, the plugin developer needs to implement the algorithm that RAVEN will
execute.
%
The \texttt{\textbf{run}} method is generally called after having inquired the
``createNewInput'' method (either the internal RAVEN one or the one implemented by
the plugin developer).
%
The only two attributes this method is going to receive are a Python list of inputs
(the inputs coming from the \texttt{createNewInput} method) and a ``self-like'' object
named ``container''.
%
If the user wants RAVEN to collect the results of this method, the outcomes of
interest need to be stored in the above mentioned ``container'' object.
%
\nb RAVEN is trying to collect the values of the variables listed only in the
\xmlNode{ExternalModel} XML block.
%
In the following an example is reported:
\begin{lstlisting}[language=python, basicstyle=\scriptsize\ttfamily, breaklines=True, columns=fullflexible]
def run(self, container, Input):
  # in here the actual run of the
  # model is implemented
  input = Input[0]
  container.outcome = container.sigma*container.rho*input[``whatEver'']
\end{lstlisting}

\subsubsection{Method: \texttt{createNewInput}}
\label{subsubsec:createNewInputExternalModelPlugin}
\begin{lstlisting}[language=python]
def createNewInput(self, container, inputs, samplerType, **Kwargs)
\end{lstlisting}

The \textbf{createNewInput} method can be implemented by the ExternalModel Plugin
developer to create a new input with the information coming from the RAVEN framework.
%
In this function, the developer can retrieve the information coming from the RAVEN
framework, during the employment of a calculation flow, and use them to
construct a new input that is going to be transferred to the ``run'' method.
%
The new input created needs to be returned to RAVEN (i.e. ``return NewInput'').
\\This method expects that the new input is returned in a Python ``dictionary''.
%
RAVEN communicates, through a set of method attributes, all the information
that are generally needed to create a new input:

\begin{itemize}
  \item \texttt{inputs}, \xmlDesc{python list}, a list of all the inputs that
  have been defined in the ``Step'' using this model.
  \item \texttt{samplerType}, \xmlDesc{string}, the type of Sampler, if a
  sampling strategy is employed; will be None otherwise.
  \item \texttt{Kwargs}, \xmlDesc{dictionary}, a dictionary containing several
  pieces of information (that can change based on the ``Step'' type).
  %
  If a sampling strategy is employed, this dictionary contains another
  dictionary identified by the keyword ``SampledVars'', in which the variables
  perturbed by the sampler are reported.
\end{itemize}
\nb If the ``Step'' that is using this Model has as input(s) an object of main
class type ``DataObjects'' (see Section~\ref{sec:DataObjects}), the internal ``createNewInput''
method is going to convert it in a dictionary of values.
%
Here we present an example:
\begin{lstlisting}[language=python]
def createNewInput(self, container, inputs,samplerType,**Kwargs):
  # in here the actual createNewInput of the
  # model is implemented
  if samplerType == 'MonteCarlo':
    avariable = inputs['something']*inputs['something2']
  else:
    avariable = inputs['something']/inputs['something2']
  return avariable*Kwargs['SampledVars']['aSampledVar']
\end{lstlisting}

\subsubsection{Method: \texttt{\_readMoreXML}}
\label{subsubsec:externalReadMoreXMLExternalModelPlugin}
\begin{lstlisting}[language=python]
def _readMoreXML(self, container, xmlNode)
\end{lstlisting}
As already mentioned, the \textbf{\_readMoreXML} method can be implemented by the
ExternalModel Plugin developer if the XML input that belongs to this ExternalModel
plugin needs to be extended to contain other information.
%
The read information needs to be stored in the ``self-like'' object ``container''
 in order to be available to all the other methods (e.g. if the developer needs to add a
 couple of newer XML nodes with information needed by the algorithm implemented in
 the ``run'' method).
%
If this method is implemented in the \textbf{ExternalModel}, RAVEN is going to
call it when the node \xmlNode{ExternalModel} is found parsing the XML input
file.
%
The method receives from RAVEN an attribute of type ``xml.etree.ElementTree'',
containing all the sub-nodes and attribute of the XML block \xmlNode{ExternalModel}.
%

Example XML:
\begin{lstlisting}[style=XML,morekeywords={subType,ModuleToLoad}]
<Models>
  ...
  <ExternalModel name='AnExtModule' subType='NewPlugin'>
    <inputs>sigma,rho</inputs>
    <outputs>outcome</outputs>
    <!--
      here we define other XML nodes RAVEN does not read automatically.
      We need to implement, in the external model Plugin class the _readMoreXML
      method
    -->
    <newNodeWeNeedToRead>
      whatNeedsToBeRead
    </newNodeWeNeedToRead>
  </ExternalModel>
  ...
</Models>
\end{lstlisting}

Corresponding Python function:
\begin{lstlisting}[language=python]
def _readMoreXML(self, container, xmlNode):
  # the xmlNode is passed in by RAVEN framework
  # <newNodeWeNeedToRead> is unknown (in the RAVEN framework)
  # we have to read it on our own
  # get the node
  ourNode = xmlNode.find('newNodeWeNeedToRead')
  # get the information in the node
  container.ourNewVariable = ourNode.text
  # end function
\end{lstlisting}

\subsubsection{Method: \texttt{initialize}}
\label{subsubsec:externalInitializeExternalModelPlugin}
\begin{lstlisting}[language=python]
def initialize(self, container, runInfo, inputs)
\end{lstlisting}

The \textbf{initialize} method can be implemented in the \textbf{ExternalModel} Plugin
in order to initialize some variables needed by it. This function accepts similar inputs
to \textbf{initialize} method in section \ref{subsubsec:initialize}, except the \texttt{container}
input variable. As all the others method in the ExternalModel Plugin, the information \emph{must} be
stored in the ``self-like'' object ``container''.
%
If this method is implemented in the \textbf{ExternalModel} Plugin, RAVEN is going to
call it at the initialization stage of each ``Step'' (see RAVEN User Manual Steps section).
%

In the following an example is reported:
\begin{lstlisting}[language=python]
def initialize(self, container, runInfo, inputs):
 # Let's suppose we just need to initialize some variables
  container.sigma = 10.0
  container.rho   = 28.0
  # end function
\end{lstlisting}

\section{Plots}
\label{sec:outstreams_plot}
New plots that are specific to particular applications are possible through \xmlNode{OutStreams}
\xmlNode{Plot} plugins.

These plotting plugins should inherit from the \texttt{PlotPlugin} base class defined in
\begin{lstlisting}[language=bash]
  raven/framework/PluginBaseClasses/OutStreamPlotPlugin.py
\end{lstlisting}
which sets up the plotting tool to be found when RAVEN runs.

A good example of the \texttt{PlotPlugin} can be found in the RAVEN ExamplePlugin, found at
\begin{lstlisting}[language=bash]
  raven/plugins/ExamplePlugin/src/CorrelationPlot.py
\end{lstlisting}

There are a few methods that can or must be implemented for new plotting strategies.

%
%
\subsection{\texttt{run} method, required}
The \texttt{run} method is the primary execution method for the PlotPlugin. Here, whatever data
handling and plotting mechanics will be executed. Note that \texttt{run} does not receive any inputs;
often, the source for the data to be plotted will be identified in the \texttt{initialize} method.

The \texttt{run} method can perform many actions including data manipulation, creation of
\texttt{matplotlib} figures and axes, saving figures to file, and so forth. In the end, \texttt{run}
should not return anything.

%
%
\subsection{Constructor, optional}
As the constructor for Python classes, the \texttt{\_\_init\_\_} method should be extended to define
any instance variables used in the plugin class. If no instance variables are used, this may be ommitted.

Any call to \texttt{\_\_init\_\_} must include a call to the parent's constructor, such as
\begin{lstlisting}[language=python]
  def __init__(self):
    """ ... """
    super().__init__()
\end{lstlisting}
This assures access to basic RAVEN functionalities required to use the plugin.

%
%
\subsection{Input Handling methods, optional}
The class method \texttt{getInputSpecification} and corresponding instance method
\texttt{handleInput} are how RAVEN determines what user inputs are allowed and how those input
values get stored in the plugin. Acceptable inputs are defined in \texttt{getInputSpecification}
and then read in during \texttt{handleInput}. Both of these methods require a call to \texttt{super}
to function as expected.

%
%
\subsection{Initialization, optional}
The aptly-named \texttt{initialize} method is used at the start of every RAVEN \xmlNode{Step} to prepare
for execution. A common task for this method is to find the source of the data to plot. To make this
process easier, RAVEN provides a \texttt{self.findSource} method that can search the input dictionary
provided to \texttt{initialize} and find a \xmlNode{DataObject} by string name.

%
%

\section{Code Plugins}
\label{sec:codePlugins}

The procedure of creating a new code/application plugin with RAVEN is a straightforward process.
The plugin is performed through a Python interface that interprets the information coming from RAVEN
and translates them into the input of the driven code plugin. This procedure does not require
modifying RAVEN itself. Instead, the developer creates a new Python Code Plugin entity that is
going to be embedded in RAVEN at run-time (no need to introduce  hard-coded coupling statements).

At the initialization stage, RAVEN imports all the code plugin entity that are contained in
\texttt{plugin/src} directory and performs some preliminary cross-checks.
\\It is important to notice that the name of class in the plugin entity module is the one
the user needs to specify when the new code plugin needs to be used.
For example, if the new plugin module contains the class ``NewCode'',
the \textit{subType} in the \xmlNode{Code} block will be \xmlAttr{PluginName.NewCode}:
\begin{lstlisting}[language=python, basicstyle=\scriptsize\ttfamily, breaklines=True, columns=fullflexible]
  from ravenframework.PluginBaseClasses.CodePluginBase import CodePluginBase
  class NewCode(CodePluginBase):
    ...
\end{lstlisting}
\begin{lstlisting}[style=XML,morekeywords={name,file}] %moreemph={name,file}]
  <Models>
    ...
    <Code name='whatever' subType='PluginName.NewCode'>
     ...
    </Code>
    ...
  </Models>
\end{lstlisting}

When loading an ``Code Plugin'', RAVEN expects to find, in the class
representing the plugin, the following required methods:
\begin{lstlisting}[language=python, basicstyle=\scriptsize\ttfamily, breaklines=True, columns=fullflexible]
  from ravenframework.PluginBaseClasses.CodePluginBase import CodePluginBase
  class NewCode(CodePluginBase):
    ...
    def generateCommand(self, inputFiles, executable, clargs=None, fargs=None, preExec=None):
      """
        See base class.  Collects all the clargs and the executable to produce the command-line call.
        Returns tuple of commands and base file name for run.
        Commands are a list of tuples, indicating parallel/serial and the execution command to use.
        @ In, inputFiles, list, List of input files (lenght of the list depends on the number of inputs have been added in the Step is running this code)
        @ In, executable, string, executable name with absolute path (e.g. /home/path_to_executable/code.exe)
        @ In, clargs, dict, optional, dictionary containing the command-line flags the user can specify in the input (e.g. under the node < Code >< clargstype =0 input0arg =0 i0extension =0 .inp0/ >< /Code >)
        @ In, fargs, dict, optional, a dictionary containing the axuiliary input file variables the user can specify in the input (e.g. under the node < Code >< clargstype =0 input0arg =0 aux0extension =0 .aux0/ >< /Code >)
        @ In, preExec, string, optional, a string the command that needs to be pre-executed before the actual command here defined
        @ Out, returnCommand, tuple, tuple containing the generated command. returnCommand[0] is the command to run the code (string), returnCommand[1] is the name of the output root
      """
    def createNewInput(self, currentInputFiles, oriInputFiles, samplerType, **Kwargs):
      """
        Generate a new Projectile input file (txt format) from the original, changing parameters
        as specified in Kwargs['SampledVars']. In addition, it creaes an additional input file including the vector data to be
        passed to Dymola.
        @ In, currentInputFiles, list,  list of current input files (input files from last this method call)
        @ In, oriInputFiles, list, list of the original input files
        @ In, samplerType, string, Sampler type (e.g. MonteCarlo, Adaptive, etc. see manual Samplers section)
        @ In, Kwargs, dictionary, kwarded dictionary of parameters. In this dictionary there is another dictionary called "SampledVars"
              where RAVEN stores the variables that got sampled (e.g. Kwargs['SampledVars'] => {'var1':10,'var2':40})
        @ Out, newInputFiles, list, list of newer input files, list of the new input files (modified and not)
      """

\end{lstlisting}
In addition, the following optional methods can be specified:
\begin{lstlisting}[language=python, basicstyle=\scriptsize\ttfamily, breaklines=True, columns=fullflexible]
  from ravenframework.PluginBaseClasses.CodePluginBase import CodePluginBase
  class NewCode(CodePluginBase):
    ...
    def __init__(self):
      """
        Constructor
        @ In, None
        @ Out, None
      """
    def initialize(self, runInfo, oriInputFiles):
      """
        Method to initialize the run of a new step
        @ In, runInfo, dict,  dictionary of the info in the <RunInfo> XML block
        @ In, oriInputFiles, list, list of the original input files
        @ Out, None
      """
    def _readMoreXML(self, xmlNode):
      """
        Function to read the portion of the xml input that belongs to this specialized class and
        initialize some members based on inputs. This can be overloaded in specialized code interface in order
        to read specific flags
        @ In, xmlNode, xml.etree.ElementTree.Element, Xml element node
        @ Out, None
      """
    def getInputExtension(self):
      """
        Return a tuple of possible file extensions for a simulation initialization file (e.g., input.i).
        @ In, None
        @ Out, validExtensions, tuple, tuple of valid extensions
      """
    def finalizeCodeOutput(self, command, output, workingDir):
      """
        Called by RAVEN to modify output files (if needed) so that they are in a proper form.
        In this case, the default .mat output needs to be converted to .csv output, which is the
        format that RAVEN can communicate with.
        @ In, command, string, the command used to run the just ended job
        @ In, output, string, the Output name root
        @ In, workingDir, string, current working dir
        @ Out, output, string, optional, present in case the root of the output file gets changed in this method.
      """
    def checkForOutputFailure(self, output, workingDir):
      """
        This method is called by RAVEN at the end of each run if the return code is == 0.
        This method needs to be implemented for the codes that, if the run fails, return a return code that is 0
        This can happen in those codes that record the failure of the job (e.g. not converged, etc.) as normal termination (returncode == 0)
        This method can be used, for example, to parse the output file looking for a special keyword that testifies that a particular job got failed
        (e.g. in RELAP5 would be the keyword "********")
        @ In, output, string, the Output name root
        @ In, workingDir, string, current working dir
        @ Out, failure, bool, True if the job is failed, False otherwise
      """
\end{lstlisting}

The explanation for all above required and optional methods except the \texttt{\_readMoreXML} can be found in
RAVEN user manual section \textbf{Advanced Users: How to couple a new code}. We recommand the plugin developers
take a look at the RAVEN user manual for the detailed information and learn how to use these methods. In the
following subsection, the \texttt{\_readMoreXML} method is explained.

\subsubsection{Method: \texttt{\_readMoreXML}}
\label{subsubsec:codePluginReadMoreXML}
The \textbf{\_readMoreXML} method can be implemented by the
Code plugin developer if the XML input that belongs to this Code
plugin needs to be extended to contain other information.
%
If this method is implemented in the \textbf{NewCodePlugin}, RAVEN is going to
call it when the node \xmlNode{Code} is found parsing the XML input
file.
%
The method receives from RAVEN an attribute of type ``xml.etree.ElementTree'',
containing all the sub-nodes and attribute of the XML block \xmlNode{Code}.
%

Example XML:
\begin{lstlisting}[style=XML,morekeywords={subType,ModuleToLoad}]
<Models>
  ...
  <Code name='newCode' subType='NewPluginName.NewCodePluginEntity'>
    <!--
      here we define other XML nodes RAVEN does not read automatically.
      We need to implement, in the code Plugin class the _readMoreXML
      method
    -->
    <newNodeWeNeedToRead>
      whatNeedsToBeRead
    </newNodeWeNeedToRead>
  </ExternalModel>
  ...
</Models>
\end{lstlisting}

\section{PostProcessor Plugins}
\label{sec:postprocessorPlugins}

Similar to other entities plugins, the PostProcessor plugin accepts RAVEN DataObject(s) as input(s),
perform additional operations on the data stored in the DataObject(s), and save the calculations
results into a new RAVEN DataObject. This procedure does not require modifying RAVEN itself, and the
developed plugin entity is going to be embedded in RAVEN at run-time.

At the initialization stage, RAVEN imports all the \texttt{PostProcessor} plugin entity that are contained in
\texttt{plugin/src} directory and performs some preliminary cross-checks.
\\It is important to notice that the name of class in the plugin entity module is the one
the user needs to specify when the new postprocessor plugin entity needs to be used.
For example, if the new plugin module contains the class ``NewPostProcessor'',
the \textit{subType} in the \xmlNode{PostProcessor} block will be \xmlAttr{PluginName.NewPostProcessor}:
\begin{lstlisting}[language=python, basicstyle=\scriptsize\ttfamily, breaklines=True, columns=fullflexible]
  from ravenframework.PluginBaseClasses.PostProcessorPluginBase import PostProcessorPluginBase
  class NewPostProcessor(PostProcessorPluginBase):
    ...
\end{lstlisting}
\begin{lstlisting}[style=XML,morekeywords={name,file}] %moreemph={name,file}]
  <Models>
    ...
    <PostProcessor name='whatever' subType='PluginName.NewPostProcessor'>
     ...
    </PostProcessor>
    ...
  </Models>
\end{lstlisting}

When loading an ``PostProcessor Plugin entities'', RAVEN expects to find, in the class
representing the plugin, the following required methods:
\begin{lstlisting}[language=python, basicstyle=\scriptsize\ttfamily, breaklines=True, columns=fullflexible]
  from ravenframework.PluginBaseClasses.CodePluginBase import CodePluginBase
  class NewCode(CodePluginBase):
    ...
    def run(self,inputIn):
      """
       This method to perform operations on the data from input DataObject(s)
       @ In, inputIn, dict, dictionary which contains the data inside the input DataObject
       @ Out, outputDict, dict, the output dictionary, passing through HistorySet info
      """
\end{lstlisting}

Where the input \texttt{inputIn} for \textbf{run} method has the following format:
\begin{lstlisting}[language=python, basicstyle=\scriptsize\ttfamily, breaklines=True, columns=fullflexible]
inputIn = {'Data':listData, 'Files':listOfFiles},
listData has the following format if 'xrDataset' is passed to self.setInputDataType('xrDataset')
(listOfInputVars, listOfOutVars, xr.Dataset)
Otherwise listData has the following format: (listOfInputVars, listOfOutVars, DataDict) with
DataDict is a dictionary that has the format:
    dataDict['dims']     = dict {varName:independentDimensions}
    dataDict['metadata'] = dict {metaVarName:metaVarValue}
    dataDict['type'] = str TypeOfDataObject
    dataDict['inpVars'] = list of input variables
    dataDict['outVars'] = list of output variables
    dataDict['numberRealization'] = int SizeOfDataObject
    dataDict['name'] = str DataObjectName
    dataDict['metaKeys'] = list of meta variables
    dataDict['data'] = dict {varName: varValue(1-D or 2-D numpy array)}
\end{lstlisting}

In addition, the following optional methods can be specified:
\begin{lstlisting}[language=python, basicstyle=\scriptsize\ttfamily, breaklines=True, columns=fullflexible]
  from ravenframework.PluginBaseClasses.CodePluginBase import CodePluginBase
  class NewCode(CodePluginBase):
    ...
    @classmethod
    def getInputSpecification(cls):
      """
        Method to get a reference to a class that specifies the input data for
        class cls.
        @ In, cls, the class for which we are retrieving the specification
        @ Out, inputSpecification, InputData.ParameterInput, class to use for
          specifying input of cls.
      """
    def __init__(self):
      """
        Constructor
        @ In, None
        @ Out, None
      """
    def initialize(self, runInfo, inputs, initDict=None):
      """
        Method to initialize the DataClassifier post-processor.
        @ In, runInfo, dict, dictionary of run info (e.g. working dir, etc)
        @ In, inputs, list, list of inputs
        @ In, initDict, dict, optional, dictionary with initialization options
        @ Out, None
      """
    def _handleInput(self, paramInput):
      """
        Function to handle the parameter input.
        @ In, paramInput, ParameterInput, the already parsed input.
        @ Out, None
      """
\end{lstlisting}

The explanation for all above optional methods can be found in section \ref{subsec:commonMethods}.



\clearpage
\begin{appendices}
  \section{Document Version Information}
  d317572ffb80f564d27ec54e5db24d763705a8a5 Aaron Epiney - INL Tue, 17 Oct 2017 XX:XX:XX -0600

\end{appendices}
%\appendix
\section{Appendix: Example Primer}
\label{sec:examplePrimer}
In this Appendix, a set of examples are reported. In order to be as general as possible, the \textit{Model} type ``ExternalModel'' has been used.
%%%% EXAMPLE 1
\subsection{Example 1.}
\label{subsec:ex1}
This simple example is about the construction of a ``Lorentz attractor'', sampling the relative input space. The parameters that are sampled represent the initial coordinate (x0,y0,z0) of the attractor origin.

\begin{lstlisting}[style=XML,morekeywords={debug,re,seeding,class,subType,limit}]
<?xml version="1.0" encoding="UTF-8"?>
<Simulation verbosity="debug">
<!-- RUNINFO -->
<RunInfo>
    <WorkingDir>externalModel</WorkingDir>
    <Sequence>FirstMRun</Sequence>
    <batchSize>3</batchSize>
</RunInfo>
<!-- Files -->
<Files>
    <Input name='lorentzAttractor.py' type=''>lorentzAttractor</Input>
</Files>
<!-- STEPS -->
<Steps>
    <MultiRun name='FirstMRun'  re-seeding='25061978'>
        <Input   class='Files'     type=''               >lorentzAttractor.py</Input>
        <Model   class='Models'    type='ExternalModel'  >PythonModule</Model>
        <Sampler class='Samplers'  type='MonteCarlo'     >MC_external</Sampler>
        <Output  class='DataObjects'     type='HistorySet'      >testPrintHistorySet</Output>
        <Output  class='Databases' type='HDF5'           >test_external_db</Output>
        <Output  class='OutStreams' type='Print'   >testPrintHistorySet_dump</Output>
    </MultiRun >
</Steps>
<!-- MODELS -->
<Models>
    <ExternalModel name='PythonModule' subType='' ModuleToLoad='externalModel/lorentzAttractor'>
       <variables>sigma,rho,beta,x,y,z,time,x0,y0,z0</variables>
    </ExternalModel>
</Models>
<!-- DISTRIBUTIONS -->
<Distributions>
    <Normal name='x0_distrib'>
        <mean>4</mean>
        <sigma>1</sigma>
    </Normal>
    <Normal name='y0_distrib'>
        <mean>4</mean>
        <sigma>1</sigma>
    </Normal>
    <Normal name='z0_distrib'>
        <mean>4</mean>
        <sigma>1</sigma>
    </Normal>
</Distributions>
<!-- SAMPLERS -->
<Samplers>
    <MonteCarlo name='MC_external'>
      <samplerInit>
        <limit>3</limit>
      </samplerInit>
      <variable name='x0' >
        <distribution  >x0_distrib</distribution>
      </variable>
      <variable name='y0' >
        <distribution  >y0_distrib</distribution>
      </variable>
      <variable name='z0' >
        <distribution  >z0_distrib</distribution>
      </variable>
    </MonteCarlo>
</Samplers>
<!-- DATABASES -->
<Databases>
  <HDF5 name="test_external_db"/>
</Databases>
<!-- OUTSTREAMS -->
<OutStreams>
  <Print name='testPrintHistorySet_dump'>
    <type>csv</type>
    <source>testPrintHistorySet</source>
  </Print>
</OutStreams>
<!-- DATA OBJECTS -->
<DataObjects>
    <HistorySet name='testPrintHistorySet'>
        <Input>x0,y0,z0</Input>
        <Output>time,x,y,z</Output>
   </HistorySet>
</DataObjects>
</Simulation>
\end{lstlisting}
The Python \textit{ExternalModel} is reported below:
\begin{lstlisting}[language=python]
import numpy as np

def run(self,Input):
  max_time = 0.03
  t_step = 0.01

  numberTimeSteps = int(max_time/t_step)

  self.x = np.zeros(numberTimeSteps)
  self.y = np.zeros(numberTimeSteps)
  self.z = np.zeros(numberTimeSteps)
  self.time = np.zeros(numberTimeSteps)

  self.x0 = Input['x0']
  self.y0 = Input['y0']
  self.z0 = Input['z0']

  self.x[0] = Input['x0']
  self.y[0] = Input['y0']
  self.z[0] = Input['z0']
  self.time[0]= 0

  for t in range (numberTimeSteps-1):
    self.time[t+1] = self.time[t] + t_step
    self.x[t+1]    = self.x[t] +  self.sigma*
                      (self.y[t]-self.x[t]) * t_step
    self.y[t+1]    = self.y[t] + (self.x[t]*
                      (self.rho-self.z[t])-self.y[t]) * t_step
    self.z[t+1]    = self.z[t] + (self.x[t]*
                          self.y[t]-self.beta*self.z[t]) * t_step
\end{lstlisting}
%%%% EXAMPLE 2
\subsection{Example 2.}
\label{subsec:ex1}
This example shows a slightly more complicated example, that employs the usage of:
\begin{itemize}
    \item \textit{Samplers:} Grid and Adaptive;
    \item \textit{Models:} External, Reduce Order Models and Post-Processors;
    \item \textit{OutStreams:} Prints and Plots;
    \item \textit{Data Objects:} PointSets;
    \item \textit{Functions:} ExternalFunctions.
\end{itemize}
The goal of this input is to compute the ``SafestPoint''.
It provides the coordinates of the farthest
point from the limit surface that is given as an input.
%
The safest point coordinates are expected values of the coordinates of the
farthest points from the limit surface in the space of the ``controllable''
variables based on the probability distributions of the ``non-controllable''
variables.

The term ``controllable'' identifies those variables that are under control
during the system operation, while the ``non-controllable'' variables are
stochastic parameters affecting the system behavior randomly.

The ``SafestPoint'' post-processor requires the set of points belonging to the
limit surface, which must be given as an input.

\begin{lstlisting}[style=XML,morekeywords={debug,re,seeding,class,subType,limit}]
<Simulation verbosity='debug'>

<!-- RUNINFO -->
<RunInfo>
  <WorkingDir>SafestPointPP</WorkingDir>
  <Sequence>pth1,pth2,pth3,pth4</Sequence>
  <batchSize>50</batchSize>
</RunInfo>

<!-- STEPS -->
<Steps>
  <MultiRun name = 'pth1' pauseAtEnd = 'False'>
    <Sampler  class = 'Samplers'  type = 'Grid'           >grd_vl_ql_smp_dpt</Sampler>
    <Input    class = 'DataObjects'     type = 'PointSet'   >grd_vl_ql_smp_dpt_dt</Input>
    <Model    class = 'Models'    type = 'ExternalModel'  >xtr_mdl</Model>
    <Output   class = 'DataObjects'     type = 'PointSet'   >nt_phy_dpt_dt</Output>
  </MultiRun >

  <MultiRun name = 'pth2' pauseAtEnd = 'True'>
    <Sampler          class = 'Samplers'  type = 'Adaptive'      >dpt_smp</Sampler>
    <Input            class = 'DataObjects'     type = 'PointSet'  >bln_smp_dt</Input>
    <Model            class = 'Models'    type = 'ExternalModel' >xtr_mdl</Model>
    <Output           class = 'DataObjects'     type = 'PointSet'  >nt_phy_dpt_dt</Output>
    <SolutionExport   class = 'DataObjects'     type = 'PointSet'  >lmt_srf_dt</SolutionExport>
  </MultiRun>

  <PostProcess name='pth3' pauseAtEnd = 'False'>
    <Input    class = 'DataObjects'          type = 'PointSet'       >lmt_srf_dt</Input>
    <Model    class = 'Models'         type = 'PostProcessor'  >SP</Model>
    <Output   class = 'DataObjects'          type = 'PointSet'     >sfs_pnt_dt</Output>
  </PostProcess>

  <OutStreamStep name = 'pth4' pauseAtEnd = 'True'>
  	<Input  class = 'DataObjects'            type = 'PointSet'  >lmt_srf_dt</Input>
  	<Output class = 'OutStreams' type = 'Print'         >lmt_srf_dmp</Output>
    <Input  class = 'DataObjects'            type = 'PointSet'  >sfs_pnt_dt</Input>
  	<Output class = 'OutStreams' type = 'Print'         >sfs_pnt_dmp</Output>
  </OutStreamStep>
</Steps>

<!-- DATA OBJECTS -->
<DataObjects>
  <PointSet name = 'grd_vl_ql_smp_dpt_dt'>
    <Input>x1,x2,gammay</Input>
    <Output>OutputPlaceHolder</Output>
  </PointSet>

  <PointSet name = 'nt_phy_dpt_dt'>
    <Input>x1,x2,gammay</Input>
    <Output>g</Output>
  </PointSet>

  <PointSet name = 'bln_smp_dt'>
    <Input>x1,x2,gammay</Input>
    <Output>OutputPlaceHolder</Output>
  </PointSet>

  <PointSet name = 'lmt_srf_dt'>
    <Input>x1,x2,gammay</Input>
    <Output>g_zr</Output>
  </PointSet>

  <PointSet name = 'sfs_pnt_dt'>
    <Input>x1,x2,gammay</Input>
    <Output>p</Output>
  </PointSet>
</DataObjects>

<!-- DISTRIBUTIONS -->
<Distributions>
  <Normal name = 'x1_dst'>
    <upperBound>10</upperBound>
    <lowerBound>-10</lowerBound>
  	<mean>0.5</mean>
    <sigma>0.1</sigma>
  </Normal>

  <Normal name = 'x2_dst'>
    <upperBound>10</upperBound>
    <lowerBound>-10</lowerBound>
    <mean>-0.15</mean>
    <sigma>0.05</sigma>
  </Normal>

  <Normal name = 'gammay_dst'>
    <upperBound>20</upperBound>
    <lowerBound>-20</lowerBound>
    <mean>0</mean>
    <sigma>15</sigma>
  </Normal>
</Distributions>

<!-- SAMPLERS -->
<Samplers>
  <Grid name = 'grd_vl_ql_smp_dpt'>
    <variable name = 'x1' >
      <distribution>x1_dst</distribution>
      <grid type = 'value' construction = 'equal' steps = '10' upperBound = '10'>2</grid>
    </variable>
    <variable name='x2' >
      <distribution>x2_dst</distribution>
      <grid type = 'value' construction = 'equal' steps = '10' upperBound = '10'>2</grid>
    </variable>
    <variable name='gammay' >
      <distribution>gammay_dst</distribution>
      <grid type = 'value' construction = 'equal' steps = '10' lowerBound = '-20'>4</grid>
    </variable>
  </Grid>

  <Adaptive name = 'dpt_smp' verbosity='debug'>
    <ROM              class = 'Models'    type = 'ROM'           >accelerated_ROM</ROM>
    <Function         class = 'Functions' type = 'External'      >g_zr</Function>
    <TargetEvaluation class = 'DataObjects'     type = 'PointSet'  >nt_phy_dpt_dt</TargetEvaluation>
    <Convergence limit = '3000' forceIteration = 'False' weight = 'none' persistence = '5'>1e-2</Convergence>
      <variable name = 'x1'>
        <distribution>x1_dst</distribution>
      </variable>
      <variable name = 'x2'>
        <distribution>x2_dst</distribution>
      </variable>
      <variable name = 'gammay'>
        <distribution>gammay_dst</distribution>
      </variable>
  </Adaptive>
</Samplers>

<!-- MODELS -->
<Models>
  <ExternalModel name = 'xtr_mdl' subType = '' ModuleToLoad = 'SafestPointPP/safest_point_test_xtr_mdl'>
    <variables>x1,x2,gammay,g</variables>
  </ExternalModel>

  <ROM name = 'accelerated_ROM' subType = 'SciKitLearn'>
    <Features>x1,x2,gammay</Features>
    <Target>g_zr</Target>
    <SKLtype>svm|SVC</SKLtype>
    <kernel>rbf</kernel>
    <gamma>10</gamma>
    <tol>1e-5</tol>
    <C>50</C>
  </ROM>

  <PostProcessor name='SP' subType='SafestPoint'>
    <!-- List of Objects (external with respect to this PP) needed by this post-processor -->
    <Distribution     class = 'Distributions'  type = 'Normal'>x1_dst</Distribution>
    <Distribution     class = 'Distributions'  type = 'Normal'>x2_dst</Distribution>
    <Distribution     class = 'Distributions'  type = 'Normal'>gammay_dst</Distribution>
    <!- end of the list -->
    <controllable>
    	<variable name = 'x1'>
    		<distribution>x1_dst</distribution>
    		<grid type = 'value' steps = '20'>1</grid>
    	</variable>
    	<variable name = 'x2'>
    		<distribution>x2_dst</distribution>
    		<grid type = 'value' steps = '20'>1</grid>
    	</variable>
    </controllable>
    <non-controllable>
    	<variable name = 'gammay'>
    		<distribution>gammay_dst</distribution>
    		<grid type = 'value' steps = '20'>2</grid>
    	</variable>
    </non-controllable>
  </PostProcessor>
</Models>

<!-- FUNCTIONS -->
<Functions>
  <External name='g_zr' file='SafestPointPP/safest_point_test_g_zr.py'>
    <variable>g</variable>
  </External>
</Functions>

<!-- OUT-STREAMS -->
<OutStreams>
  <Print name = 'lmt_srf_dmp'>
  	<type>csv</type>
  	<source>lmt_srf_dt</source>
  </Print>

  <Print name = 'sfs_pnt_dmp'>
  	<type>csv</type>
  	<source>sfs_pnt_dt</source>
  </Print>
</OutStreams>

</Simulation>
\end{lstlisting}
The Python \textit{ExternalModel} is reported below:
\begin{lstlisting}[language=python]
def run(self,Input):
  self.g = self.x1+4*self.x2-self.gammay
\end{lstlisting}
The ``Goal Function'',the function that defines the transitions with respect the input space coordinates, is as follows:
\begin{lstlisting}[language=python]
def __residuumSign(self):
  if self.g<0 : return  1
  else        : return -1
\end{lstlisting}

%%%%% EXAMPLE 3
%\subsection{Example3}
%\label{subsec:ex1}
%example 3



    % ---------------------------------------------------------------------- %
    % References
    %
    \clearpage
    % If hyperref is included, then \phantomsection is already defined.
    % If not, we need to define it.
    \providecommand*{\phantomsection}{}
    \phantomsection
    \addcontentsline{toc}{section}{References}
    \bibliographystyle{ieeetr}
    \bibliography{raven_plugins_manual}


    % ---------------------------------------------------------------------- %
    %

    % \printindex

    %\include{distribution}

\end{document}
