\section{Global Sobol Sensitivity: Sudret}
Associated external model: \texttt{sudret\_sobol\_poly.py}

This model provides analytic Sobol sensitivities for a flexible number of input parameters.  It is taken from
\cite{sudret2007} and has the following form:
\begin{equation}
  u(Y) = \frac{1}{2^N} \prod_{n=1}^N \left(3y_n^2 + 1\right).
\end{equation}
The variables $y_n$ are distributed uniformly on [0,1].  For three input variables ($N=3$), the Sobol sensitivities are
as follows, to 12 digits of accuracy:
\begin{align}
  S_1 = S_2 = S_3 &= \frac{25}{91}\hspace{10pt} (0.2747), \\
  S_{1,2} = S_{1,3} = S_{2,3} &= \frac{5}{91}\hspace{10pt} (0.0549), \\
  S_{1,2,3} &= \frac{1}{91}\hspace{10pt} (0.0110).
\end{align}
The mean is 1.0 and the variance is 0.72*.

%
%
%
%
%
%
%
\section{Global Sobol Sensitivity: Ishigami}
Associated external model: \texttt{ishigami.py}

This model has interesting properties for its sensitivity indices, in that $y_3$ has zero impact alone but a
nonzero impact when coupled with $y_1$.  Additionally, the sinusoidal expression is not trivially represented
by polynomial expansion.  It is listed in \cite{saltelli2000} and has the following form:
\begin{equation}
  u(Y) = \sin(y_1) + a\sin^2(y_2) + b y_3^4\sin(y_1),
\end{equation}
where in this case $a=7$ and $b=0.1$, and all $y_n$ are uniformly distributed on $[-\pi,\pi]$.

The variance and partial variances are as follows:
\begin{align}
  D_\text{tot} &= \frac{a^2}{8} + \frac{b\pi^4}{5} + \frac{b^2\pi^8}{18} + \frac{1}{2}, \\
  D_1 &= \frac{b\pi^4}{5} + \frac{b^2\pi^8}{50} + \frac{1}{2} ,\\
  D_2 &= \frac{a^2}{8}, \\
  D_3 &= 0, \\
  D_{1,2} &= 0, \\
  D_{2,3} &= 0, \\
  D_{1,3} &= \frac{8b^2\pi^8}{225}, \\
  D_{1,2,3} &= 0.
\end{align}
The corresponding variance values and Sobol sensitivities are listed in Table \ref{tab:ishigami sens}.
\begin{table}[h]
  \centering
  \begin{tabular}{c|c|c}
    Variable & Partial Variance & Sobol Index \\ \hline
    (total) & 13.8446 & 1 \\
    $y_1$         & 4.34589 & 0.3138 \\
    $y_2$         & 6.125   & 0.4424 \\
    $y_3$         & 0       & 0      \\
    $y_1,y_2$     & 0       & 0      \\
    $y_1,y_3$     & 3.3737  & 0.2436 \\
    $y_2,y_3$     & 0       & 0      \\
    $y_1,y_2,y_3$ & 0       & 0      \\
  \end{tabular}
  \caption{Ishigami sensitivities and variances}
  \label{tab:ishigami sens}
\end{table}

%
%
%
%
%
%
\section{Sobol G-Function}
Associated external model: \texttt{gFunction.py}

This function developed by Sobol has the benefit of tuning factors $a_n$ that allow the importance of any
particular term to be increased or decreased.  Because of the absolute value, this function is quite
challenging for polynomial expansion.  Documentation can be found in \cite{sobol2003}.  The function is
represented by
\begin{equation}
  u(Y) = \prod_{n=1}^N \frac{|4y_n - 2|+a_n}{1+a_n},
\end{equation}
where $y_n$ are distributed uniformly on [0,1] and $a_n$ are non-negative.  $a_n$ are generally integers, and
smaller values lead to greater impact of corresponding $y_n$.  As in \cite{sudret2007} we use $N=8$ with
$a=[1,2,5,10,20,50,100,500]$.  The partial variances are given by
\begin{equation}
  D_n = \frac{1}{3(1+a_n)^2},
\end{equation}
\begin{equation}
  D_\text{tot} = \prod_{n=1}^N (D_n+1)-1.
\end{equation}
Analytic values for Sobol sensitivities are given in Table \ref{tab:gfunc sens}.
\begin{table}[h]
  \centering
  \begin{tabular}{c|c}
    Variable & Sobol sensitivity \\ \hline
    $y_1$ & 0.6037 \\
    $y_2$ & 0.2683 \\
    $y_3$ & 0.0671 \\
    $y_4$ & 0.0200 \\
    $y_5$ & 0.0055 \\
    $y_6$ & 0.0009 \\
    $y_7$ & 0.0002 \\
    $y_8$ & 0.0000 \\
  \end{tabular}
  \caption{G-Function sensitivities and variances}
  \label{tab:gfunc sens}
\end{table}
