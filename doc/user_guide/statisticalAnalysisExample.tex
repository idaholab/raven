\section{Statistical Analysis through RAVEN}
\label{sec:SAraven}
In order to perform a complete analysis of a system under uncertainties,
it is crucial to be able to compute all the statistical moments of one or even multiple
FOMs. In addition, it is essential to identify the correlation
among different FOMs toward a specific input space.

RAVEN is able to compute the most important statistical moments:
such as:
\begin{enumerate}
  \item \textit{Expected Value}
  \item \textit{Standard Deviation}
  \item \textit{Variance}
  \item \textit{variationCoefficient}
  \item \textit{Skewness}
  \item \textit{Kurtosis}
  \item \textit{Median}
  \item \textit{Percentile}.
\end{enumerate}
In addition, RAVEN fully supports the computation of all of the statistical moments defined to
``measure'' the correlation among variables/parameters/FOMs:
\begin{enumerate}
  \item \textit{Covariance matrix}
  \item \textit{Normalized Sensitivity  matrix}
  \item \textit{Variance Dependent Sensitivity  matrix}
  \item \textit{Sensitivity matrix}
  \item \textit{Pearson matrix}.
\end{enumerate}
The goals of this section is to show how to:
 \begin{enumerate}
   \item Set up a sampling strategy to perform a final statistical analysis
   perturbing a driven code
   \item Compute all the statistical moments and correlation/covariance
   metrics.
\end{enumerate}
In order to accomplish these tasks, the following RAVEN \textbf{Entities} (XML blocks in the input files) need to be defined:
\begin{enumerate}
   \item \textbf{\textit{RunInfo}}:
     \xmlExample{framework/user_guide/StatisticalAnalysis/statisticalAnalysis.xml}{RunInfo}
   As shown in the other examples, the \textit{RunInfo} \textbf{Entity} is intended  to set up the desired analysis . In this specific case, two steps  (\xmlNode{Sequence}) are  sequentially run
   using forty processors (\xmlNode{batchSize}).
   \\In the first step, the original physical model is sampled. The obtained results are  analyzed with the Statistical Post-Processor.
   \item \textbf{\textit{Files}}:
     \xmlExample{framework/user_guide/StatisticalAnalysis/statisticalAnalysis.xml}{Files}
   Since the driven code uses a single input file, in this section the original input is placed. As detailed in the user manual
   the attribute  \xmlAttr{name} represents the alias that is going to be
   used in all the other input blocks in order to refer to this file.
   \\In addition, the output file of the \textit{PostProcess} \textbf{Step} is
   here defined (XML format).
   \item \textbf{\textit{Models}}:
     \xmlExample{framework/user_guide/StatisticalAnalysis/statisticalAnalysis.xml}{Models}
 The goal of this example is to show how the
 principal statistical FOMs can be computed through RAVEN.
 \\Indeed, in addition to the previously explained Code
 model, a Post-Processor model (BasicStatistics) is here specified.
Note that the post-process step is
performed on all the variables with respect to the parameters used in this example ( $A,\, B,\, C \, and \, D$
with respect to $sigma-A,\,sigma-B,\, decay-A,$ and $decay-B$).
   \item \textbf{\textit{Distributions}}:
     \xmlExample{framework/user_guide/StatisticalAnalysis/statisticalAnalysis.xml}{Distributions}
  In the Distributions XML section, the stochastic models for the
  uncertainties are reported. In
  this case 2 distributions are defined:
  \begin{itemize}
    \item $sigma \sim \mathbb{U}(0,1000)$, used to model the uncertainties
    associated with  the Model \textit{sigma-A} and \textit{sigma-B}
    \item  $decayConstant \sim \mathbb{U}(1e-8,1e-7)$,  used to
    model the uncertainties
    associated with  the Model \textit{decay-A} and \textit{decay-B}.
  \end{itemize}
   \item \textbf{\textit{Samplers}}:
     \xmlExample{framework/user_guide/StatisticalAnalysis/statisticalAnalysis.xml}{Samplers}
  In order to obtained the data-set through which the statistical FOMs need to be computed, a \textit{MonteCarlo} sampling approach is here employed.
   \item \textbf{\textit{DataObjects}}:
     \xmlExample{framework/user_guide/StatisticalAnalysis/statisticalAnalysis.xml}{DataObjects}
  Int this block, two \textit{DataObjects} are defined:
  1) PointSet named ``samplesMC'' used to collect the final outcomes of
  the code,
  2) HistorySet named ``histories'' in which the full time responses of the
  variables $A,B,C,D$ are going to be stored.

   \item \textbf{\textit{Steps}}:
     \xmlExample{framework/user_guide/StatisticalAnalysis/statisticalAnalysis.xml}{Steps}

 %%%%%%%%%%%%%%%%%%%%%%%%%%%%%%%%%%%%%%%%%%%%%%%%%%%%%%%%%%
 %%%%%%%%%%%%%%%%%%%%%%%%%%%%%%%%%%%%%%%%%%%%%%%%%%%%%%%%%%
   Finally, all the previously defined \textbf{Entities} can be combined in
   the \xmlNode{Steps} block. As inferable,
   2 \xmlNode{Steps} have been inputted:
   \begin{itemize}
     \item \xmlNode{MultiRun} named ``sampleMC'', used to run the
     multiple
     instances of the driven code and
     collect the outputs in the two \textit{DataObjects}. As it can be
     seen, the \xmlNode{Sampler} is inputted to communicate to the
     \textit{Step} that the driven code needs to
     be perturbed through the Grid sampling strategy.
     \item \xmlNode{PostProcess} named ``statisticalAnalysisMC'', used
     compute all the statistical moments and FOMs based on the
     data obtained through the sampling strategy. As it can be noticed,
     the \xmlNode{Output} of the ``sampleMC'' \textit{Step} is the
     \xmlNode{Input} of the ``statisticalAnalysisMC''  \textit{Step}.
   \end{itemize}
\end{enumerate}

Tables \ref{ScalarMoments}-\ref{SensitivityComputed} show all the results of the \textit{PostProcess}
step.


\begin{landscape}
\begin{table}[h!]
\centering
\caption{Computed Moments and Cumulants.}
\label{ScalarMoments}
\begin{tabular}{|c|c|c|c|c|c|c|c|c|}
\hline
{\ul \textit{\textbf{Computed Quantities}}} & \textbf{A} & \textbf{B} & \textbf{C} & \textbf{D} & \textbf{decay-A} & \textbf{decay-B} & \textbf{sigma-A} & \textbf{sigma-B} \\ \hline
\textit{expected value}                     & 5.97E-02   & 3.97E-01   & 9.82E-01   & 1.50E+00   & 5.57E-08         & 5.61E-08         & 5.07E+02         & 4.73E+02         \\ \hline
\textit{median}                             & 2.45E-02   & 3.06E-01   & 9.89E-01   & 1.54E+00   & 5.73E-08         & 5.62E-08         & 5.11E+02         & 4.70E+02         \\ \hline
\textit{variance}                           & 8.19E-03   & 6.00E-02   & 1.19E-02   & 1.49E-02   & 7.00E-16         & 6.83E-16         & 8.52E+04         & 8.64E+04         \\ \hline
\textit{sigma}                              & 9.05E-02   & 2.45E-01   & 1.09E-01   & 1.22E-01   & 2.64E-08         & 2.61E-08         & 2.92E+02         & 2.94E+02         \\ \hline
\textit{variation coefficient}              & 1.52E+00   & 6.17E-01   & 1.11E-01   & 8.15E-02   & 4.75E-01         & 4.66E-01         & 5.75E-01         & 6.21E-01         \\ \hline
\textit{skewness}                           & 2.91E+00   & 9.88E-01   & -1.49E-01  & -9.64E-01  & -6.25E-02        & -5.75E-02        & -2.18E-02        & 7.62E-02         \\ \hline
\textit{kurtosis}                           & 9.56E+00   & -1.12E-01  & -6.98E-01  & -1.50E-01  & -1.24E+00        & -1.21E+00        & -1.21E+00        & -1.20E+00        \\ \hline
\textit{percentile 5\%}                     & 2.87E-03   & 1.48E-01   & 7.89E-01   & 1.24E+00   & 1.42E-08         & 1.45E-08         & 5.08E+01         & 2.97E+01         \\ \hline
\textit{percentile 95\%}                    & 2.51E-01   & 9.19E-01   & 1.16E+00   & 1.63E+00   & 9.54E-08         & 9.48E-08         & 9.59E+02         & 9.49E+02         \\ \hline
\end{tabular}
\end{table}
\begin{table}[h!]
\centering
\caption{Covariance matrix.}
\label{covarianceComputed}
\begin{tabular}{|c|c|c|c|c|c|c|c|c|}
\hline
{\ul \textit{\textbf{Covariance}}} & \textbf{A} & \textbf{B} & \textbf{C} & \textbf{D} & \textbf{decay-A} & \textbf{decay-B} & \textbf{sigma-A} & \textbf{sigma-B} \\ \hline
\textbf{A}                         & 8.19E-03   & -1.11E-03  & -3.09E-03  & -1.13E-04  & -1.28E-09        & 5.14E-11         & -1.49E+01        & -3.74E-01        \\ \hline
\textbf{B}                         & -1.11E-03  & 6.00E-02   & 2.26E-03   & -2.96E-02  & -7.80E-11        & -6.02E-09        & 7.00E+00         & -1.47E+00        \\ \hline
\textbf{C}                         & -3.09E-03  & 2.26E-03   & 1.19E-02   & 7.15E-04   & -1.44E-09        & -4.11E-12        & 2.63E+01         & 3.19E-01         \\ \hline
\textbf{D}                         & -1.13E-04  & -2.96E-02  & 7.15E-04   & 1.49E-02   & -1.21E-10        & 3.01E-09         & 1.12E+00         & 8.01E-01         \\ \hline
\textbf{decay-A}                   & -1.28E-09  & -7.80E-11  & -1.44E-09  & -1.21E-10  & 7.00E-16         & -1.73E-17        & -1.26E-07        & 2.07E-07         \\ \hline
\textbf{decay-B}                   & 5.14E-11   & -6.02E-09  & -4.11E-12  & 3.01E-09   & -1.73E-17        & 6.83E-16         & -1.86E-07        & 3.91E-08         \\ \hline
\textbf{sigma-A}                   & -1.49E+01  & 7.00E+00   & 2.63E+01   & 1.12E+00   & -1.26E-07        & -1.86E-07        & 8.52E+04         & 1.79E+03         \\ \hline
\textbf{sigma-B}                   & -3.74E-01  & -1.47E+00  & 3.19E-01   & 8.01E-01   & 2.07E-07         & 3.91E-08         & 1.79E+03         & 8.64E+04         \\ \hline
\end{tabular}
\end{table}
\begin{table}[h!]
\centering
\caption{Correlation matrix.}
\label{pearsonComputed}
\begin{tabular}{|c|c|c|c|c|c|c|c|c|}
\hline
{\ul \textit{\textbf{Correlation}}} & \textbf{A} & \textbf{B} & \textbf{C} & \textbf{D} & \textbf{decay-A} & \textbf{decay-B} & \textbf{sigma-A} & \textbf{sigma-B} \\ \hline
\textbf{A}                          & 1.00E+00   & -5.02E-02  & -3.13E-01  & -1.03E-02  & -5.35E-01        & 2.17E-02         & -5.63E-01        & -1.40E-02        \\ \hline
\textbf{B}                          & -5.02E-02  & 1.00E+00   & 8.47E-02   & -9.90E-01  & -1.20E-02        & -9.41E-01        & 9.80E-02         & -2.04E-02        \\ \hline
\textbf{C}                          & -3.13E-01  & 8.47E-02   & 1.00E+00   & 5.37E-02   & -4.98E-01        & -1.44E-03        & 8.25E-01         & 9.96E-03         \\ \hline
\textbf{D}                          & -1.03E-02  & -9.90E-01  & 5.37E-02   & 1.00E+00   & -3.75E-02        & 9.43E-01         & 3.14E-02         & 2.23E-02         \\ \hline
\textbf{decay-A}                    & -5.35E-01  & -1.20E-02  & -4.98E-01  & -3.75E-02  & 1.00E+00         & -2.50E-02        & -1.64E-02        & 2.67E-02         \\ \hline
\textbf{decay-B}                    & 2.17E-02   & -9.41E-01  & -1.44E-03  & 9.43E-01   & -2.50E-02        & 1.00E+00         & -2.44E-02        & 5.08E-03         \\ \hline
\textbf{sigma-A}                    & -5.63E-01  & 9.80E-02   & 8.25E-01   & 3.14E-02   & -1.64E-02        & -2.44E-02        & 1.00E+00         & 2.08E-02         \\ \hline
\textbf{sigma-B}                    & -1.40E-02  & -2.04E-02  & 9.96E-03   & 2.23E-02   & 2.67E-02         & 5.08E-03         & 2.08E-02         & 1.00E+00         \\ \hline
\end{tabular}
\end{table}
\begin{table}[h!]
\centering
\caption{Variance Dependent Sensitivity matrix.}
\label{VarDepSensitivityComputed}
\begin{tabular}{|c|c|c|c|c|c|c|c|c|}
\hline
{\ul \textit{\textbf{Variance Sensitivity}}} & \textbf{A} & \textbf{B} & \textbf{C} & \textbf{D} & \textbf{decay-A} & \textbf{decay-B} & \textbf{sigma-A} & \textbf{sigma-B} \\ \hline
\textbf{A}                                   & 1.00E+00   & -1.36E-01  & -3.77E-01  & -1.38E-02  & -1.56E-07        & 6.27E-09         & -1.82E+03        & -4.56E+01        \\ \hline
\textbf{B}                                   & -1.86E-02  & 1.00E+00   & 3.77E-02   & -4.94E-01  & -1.30E-09        & -1.00E-07        & 1.17E+02         & -2.45E+01        \\ \hline
\textbf{C}                                   & -2.60E-01  & 1.90E-01   & 1.00E+00   & 6.01E-02   & -1.21E-07        & -3.46E-10        & 2.21E+03         & 2.68E+01         \\ \hline
\textbf{D}                                   & -7.60E-03  & -1.99E+00  & 4.80E-02   & 1.00E+00   & -8.11E-09        & 2.02E-07         & 7.51E+01         & 5.37E+01         \\ \hline
\textbf{decay-A}                             & -1.83E+06  & -1.11E+05  & -2.05E+06  & -1.73E+05  & 1.00E+00         & -2.47E-02        & -1.81E+08        & 2.96E+08         \\ \hline
\textbf{decay-B}                             & 7.52E+04   & -8.82E+06  & -6.02E+03  & 4.40E+06   & -2.53E-02        & 1.00E+00         & -2.72E+08        & 5.72E+07         \\ \hline
\textbf{sigma-A}                             & -1.75E-04  & 8.22E-05   & 3.08E-04   & 1.32E-05   & -1.48E-12        & -2.19E-12        & 1.00E+00         & 2.10E-02         \\ \hline
\textbf{sigma-B}                             & -4.33E-06  & -1.70E-05  & 3.69E-06   & 9.27E-06   & 2.40E-12         & 4.52E-13         & 2.07E-02         & 1.00E+00         \\ \hline
\end{tabular}
\end{table}
\begin{table}[h!]
\centering
\caption{Sensitivity matrix.}
\label{SensitivityComputed}
\begin{tabular}{|c|c|c|c|c|}
\hline
{\ul \textit{\textbf{Sensitivity (I/O)}}} & \textbf{decay-A} & \textbf{decay-B} & \textbf{sigma-A} & \textbf{sigma-B} \\ \hline
\textbf{A}                                & 3.83E-06         & -1.78E-04        & -2.07E+04        & -1.86E+06        \\ \hline
\textbf{B}                                & -1.36E-05        & 6.28E-05         & -8.80E+06        & -3.14E+05        \\ \hline
\textbf{C}                                & 2.17E-06         & 3.05E-04         & 2.64E+04         & -2.00E+06        \\ \hline
\textbf{D}                                & 6.96E-06         & 2.25E-05         & 4.40E+06         & -6.19E+04        \\ \hline
\end{tabular}
\end{table}
\end{landscape}

