\section{Sampling from Restart}
\label{sec:samplingFromRestart}
In some instances, there are existing solutions stored that are useful to a new sampling calculation.  For
example, if a Monte Carlo run collects 1000 runs, then later the user decides to expand to 1500 runs, the
original 1000 should not be wasted.  In this case, it is desirable to restart sampling.

All \xmlNode{Sampler} entities in RAVEN accept the \xmlNode{Restart} node, which allows the user to provide a
\xmlNode{DataObject} from which sampling can draw.  The way each sampler interacts with this restart data is
dependent on the sampling strategy.

Random sampling strategies, such as the \xmlNode{MonteCarlo} and \xmlNode{Stratified} samplers, increment the
random number generator by the number of samples in the restart data, then continue sampling as normal.

Grid-based sampling strategies, such as \xmlNode{Grid}, \xmlNode{SparseGridCollocation}, and \xmlNode{Sobol},
require specific sampling points.  As each required point in the input space is determined, the sampler will
check the restart data for a match.  If a match is found, the corresponding output values are used instead of
sampling the \xmlNode{Model} for that point in the input space.  In order to determine a match, all of the
values in the restart point must be within a relative tolerance of the corresponding point required by the
sampler.  While RAVEN has a default tolerance of 1e-15, the user can adjust this tolerance using the
\xmlNode{restartNode} node in the \xmlNode{Sampler} block.

In order to demonstrate this restart method, we include here an example of restarting a \xmlNode{Grid}
sampler.  This example runs a simple example Python code from the command line using the \xmlNode{GenericCode}
interface.  Within the run the following steps occur:
\begin{enumerate}
  \item A grid is sampled that includes only the endpoints in each dimension.
  \item The results of the first sampling are written to file.
  \item The results in the CSV are read back in to a new \xmlNode{DataObject} called \xmlString{restart}.
  \item A second, more dense grid is sampled that requires the points of the first sampling, plus several
    more.  The results are added both to the original \xmlNode{DataObject} as well as a new one, for
    demonstration purposes.
  \item The results of only the new sampling can be written to CSV because we added the second data object in
    the last step.
  \item Lastly, the complete \xmlNode{DataObject} is written to file, including both the original and more
    dense sampling.
\end{enumerate}
By looking at the contents of \texttt{GRIDdump1.csv}, \texttt{GRIDdump2.csv}, and \texttt{GRIDdump3.csv}, the
progressive construction of the data object becomes clear.  \texttt{GRIDdump1.csv} contains only a few samples
corresponding to the endpoints of the distributions.  \texttt{GRIDdump3.csv} contains all the points necessary
to include the midpoints of the distributions as well as the endpoints.  \texttt{GRIDdump2.csv} contains only
those points that were not already obtained in the first sampling, but still needed for the more dense
sampling.

\xmlExample[coarse,fine,restart]{framework/Samplers/Restart/Truncated/grid.xml}{Simulation}

In order to restart an existing file that failed for some reason, the
data will need to be listed in the \xmlNode{Files} section and a
\xmlNode{IOStep} that loads that input into a different data object
will be needed to be added to the \xmlNode{Steps} section. After that a \xmlNode{Restart} node can be added.

\xmlExample{framework/Samplers/Restart/test_restart_Grid_part2.xml}{Files}

\xmlExample{framework/Samplers/Restart/test_restart_Grid_part2.xml}{Steps}

\xmlExample{framework/Samplers/Restart/test_restart_Grid_part2.xml}{Samplers}
