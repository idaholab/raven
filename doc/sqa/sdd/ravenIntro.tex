\section{Introduction}
\subsection{System Purpose}

RAVEN is a flexible and multi-purpose uncertainty quantification (UQ), regression analysis, probabilistic risk assessment 
(PRA), data analysis and model optimization software.  Depending on the tasks to be accomplished and on the 
probabilistic
 characterization of the problem, RAVEN perturbs (Monte-Carlo, latin hyper-cube, reliability surface search, etc.) the
 response of the system under consideration by altering its own parameters. The system is modeled by third party
 software (RELAP5-3D, MAAP5, BISON, etc.) and accessible to RAVEN either directly (software coupling) or
 indirectly (via input/output files). The data generated by the sampling process is analyzed using classical statistical
 and more advanced data mining approaches. RAVEN also manages the parallel dispatching (i.e. both on
 desktop/workstation and large High-Performance Computing machines) of the software representing the physical 
 model. RAVEN heavily relies on artificial intelligence algorithms to construct surrogate models of complex physical
 systems in order to perform uncertainty quantification, reliability analysis (limit state surface) and parametric studies.

\subsection{System Scope}

RAVEN’s scope is to provide a set of capabilities to build analysis flows based on UQ, PRA, Optimization and Data Analysis techniques to be applied to any physical model(s). The main objective of the software is to assist the engineer/user to:
\begin{itemize}
  \item identify the best design (on any physics/model), its safety and confidence;
  \item estimate the likelihood of undesired outcomes (risk analysis);
  \item identify main drivers/events to act on for reducing impact/consequences of anomalous dynamic behaviors of the 
         system under analysis;
  \item to construct analysis flows combining multiple physical models and analysis procedures.
\end{itemize}

In other words, the RAVEN software is aimed to be employed for:
\begin{itemize}
  \item Uncertainty Quantification;
  \item Sensitivity Analysis / Regression Analysis;
  \item Probabilistic Risk and Reliability Analysis (PRA);
  \item Data Mining Analysis;
  \item Model Optimization.
\end{itemize}

The combination of all the previously mentioned analysis capabilities is a key component to 
define safety margins in engineering design that are more representative of real prediction deficiencies. 
This could reduce 
cost and maintain a more coherent safety level of the system (no excess/no lack of safety margins in any operational 
condition).
The risk analysis, assisted by the data mining algorithms, is used to find engineering solutions to reduce costs, while 
preserving safety margins, or to increase safety at the minimum cost. These tasks can be automatically achieved by using 
optimization algorithms available in the RAVEN software.
Moreover, the knowledge of the relationship between input and system response uncertainties allows identifying effective 
experiments, which are the most suitable for increasing the accuracy of the model. This approach reduces time and cost 
of the deployment of complex engineering systems and new technologies.

The RAVEN software employs several novel and unique techniques, based on extensive usage of artificial intelligence 
algorithms, such as adaptive (smart) sampling, adaptive branching algorithms (Dynamic Event Tree), time-dependent 
statistical analysis and data mining. 
The overall set of algorithms implemented in the RAVEN software are designed to handle highly non-linear systems, 
characterized by system response discontinuities and discrete variables. These capabilities are crucial for handling 
complex system models, such as nuclear power plants.
For example, reliability surface analysis, as implemented in RAVEN, is unique and capable to handle non-linear, 
discontinuous systems, allowing for faster and more accurate assessing of failure risk for complex systems.

In addition, the RAVEN software provides the unique capability to combine any model (e.g. physical models, surrogate 
models, data analysis models, etc.) in a single entity (named Ensemble Model) where each model can feedback into others. This capability allows to analyze system that could be simulated only by using complex computational work-flows.

\subsection{Dependencies and Limitations}
The software should be designed with the fewest possible constraints. 
Ideally the software should run on a wide variety of evolving hardware, 
so it should follow well-adopted standards and guidelines. The software
 should run on any POSIX compliant system (including Windows POSIX 
 emulators such as MinGW). The software will also make use of artificial 
 intelligence and numerical libraries that run on POSIX systems as well. 
 The main interface for the software will be command line based with no 
 assumptions requiring advanced terminal capabilities such as coloring and line control. 

No evident limitations are envisioned for the RAVEN software design and its further expansions.


\subsection{Other Design Documentation}

In addition to this document, an automatic software documentation
is generated every time a new CR (see def.) is approved. This
documentation is automatically extracted from the source code using
doxygen (see def.) and is available to developers at
\url{https://hpcsc.inl.gov/ssl/RAVEN/docs/classes.html}.

In order to generate(locally) a hard copy in ``html'' or ``latex'', any user/developer can lunch the 
following command (in the raven directory):
\begin{lstlisting}[language=bash]
doxygen ./doc/doxygen/Doxyfile
\end{lstlisting}
Once the documentation is generated, any user/developer can navigate in the folder
\begin{lstlisting}[language=bash]
./doc/doxygen/latex
\end{lstlisting}
and type the following command:
\begin{lstlisting}[language=bash]
make refman.pdf
\end{lstlisting}
Once the command is executed, a ``pdf'' file named ``refman.pdf'' will be available.

The doxygen software is under configuration management process identified in
`` RAVEN Configuration Management '' PLN-5553.


\section{References}

\begin{itemize}

  \item ASME NQA 1 2008 with the NQA-1a-2009 addenda, ``Quality Assurance Requirements for Nuclear Facility Applications,'' First Edition, August 31, 2009.
  \item ISO/IEC/IEEE 24765:2010(E), ``Systems and software engineering Vocabulary,'' First Edition, December 15, 2010.
  \item LWP 13620, ``Managing Information Technology Assets''
\end{itemize}


\section{Definitions and Acronyms}

\subsection{Definitions}
\begin{itemize}
  \item \textbf{Baseline.} A specification or product (e.g., project plan, maintenance and operations [M\&O] plan, requirements, or 
design) that has been formally reviewed and agreed upon, that thereafter serves as the basis for use and further 
development, and that can be changed only by using an approved change control process. [ASME NQA-1-2008 with the 
NQA-1a-2009 addenda edited]
  \item \textbf{Validation.} Confirmation, through the provision of objective evidence (e.g., acceptance test), that the requirements 
for a specific intended use or application have been fulfilled. [ISO/IEC/IEEE 24765:2010(E) edited]
  \item \textbf{Verification.}
  \begin{itemize}
     \item The process of evaluating a system or component to determine whether the products of a given development 
     phase satisfy the conditions imposed at the start of that phase.
     \item  Formal proof of program correctness (e.g., requirements, design, implementation reviews, system tests). 
     [ISO/IEC/IEEE 24765:2010(E) edited]
  \end{itemize}
\end{itemize}

\subsection{Acronyms}
\begin{description}
\item[API] Application Programming Interfaces
\item[ASME] American Society of Mechanical Engineers
\item[DOE] Department of Energy
\item[INL] Idaho National Laboratory
\item[IT] Information Technology
\item[M\&O] Maintenance and Operations
\item[NQA] Nuclear Quality Assurance
\item[QA] Quality Assurance
\item[SDD] System Design Description
\end{description}