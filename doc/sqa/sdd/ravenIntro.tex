\section{Introduction}
\subsection{System Purpose}

RAVEN is a flexible and multi-purpose uncertainty quantification (UQ), regression analysis, probabilistic risk assessment 
(PRA), data analysis and model optimization software.  Depending on the tasks to be accomplished and on the 
probabilistic
 characterization of the problem, RAVEN perturbs (Monte-Carlo, latin hyper-cube, reliability surface search, etc.) the
 response of the system under consideration by altering its own parameters. The system is modeled by third party
 software (RELAP5-3D, MAAP5, BISON, etc.) and accessible to RAVEN either directly (software coupling) or
 indirectly (via input/output files). The data generated by the sampling process is analyzed using classical statistical
 and more advanced data mining approaches. RAVEN also manages the parallel dispatching (i.e. both on
 desktop/workstation and large High-Performance Computing machines) of the software representing the physical 
 model. RAVEN heavily relies on artificial intelligence algorithms to construct surrogate models of complex physical
 systems in order to perform uncertainty quantification, reliability analysis (limit state surface) and parametric studies.

\subsection{System Scope}

RAVEN’s scope is to provide a set of capabilities to build analysis flows based on UQ, PRA, Optimization and Data Analysis techniques to be applied to any physical model(s). The main objective of the software is to assist the engineer/user to:
\begin{itemize}
  \item identify the best design (on any physics/model), its safety and confidence;
  \item estimate the likelihood of undesired outcomes (risk analysis);
  \item identify main drivers/events to act on for reducing impact/consequences of anomalous dynamic behaviors of the 
         system under analysis;
  \item to construct analysis flows combining multiple physical models and analysis procedures.
\end{itemize}

In other words, the RAVEN software is aimed to be employed for:
\begin{itemize}
  \item Uncertainty Quantification;
  \item Sensitivity Analysis / Regression Analysis;
  \item Probabilistic Risk and Reliability Analysis (PRA);
  \item Data Mining Analysis;
  \item Model Optimization.
\end{itemize}

The combination of all the previously mentioned analysis capabilities is a key component to 
define safety margins in engineering design that are more representative of real prediction deficiencies. 
This could reduce 
cost and maintain a more coherent safety level of the system (no excess/no lack of safety margins in any operational 
condition).
The risk analysis, assisted by the data mining algorithms, is used to find engineering solutions to reduce costs, while 
preserving safety margins, or to increase safety at the minimum cost. These tasks can be automatically achieved by using 
optimization algorithms available in the RAVEN software.
Moreover, the knowledge of the relationship between input and system response uncertainties allows identifying effective 
experiments, which are the most suitable for increasing the accuracy of the model. This approach reduces time and cost 
of the deployment of complex engineering systems and new technologies.

The RAVEN software employs several novel and unique techniques, based on extensive usage of artificial intelligence 
algorithms, such as adaptive (smart) sampling, adaptive branching algorithms (Dynamic Event Tree), time-dependent 
statistical analysis and data mining. 
The overall set of algorithms implemented in the RAVEN software are designed to handle highly non-linear systems, 
characterized by system response discontinuities and discrete variables. These capabilities are crucial for handling 
complex system models, such as nuclear power plants.
For example, reliability surface analysis, as implemented in RAVEN, is unique and capable to handle non-linear, 
discontinuous systems, allowing for faster and more accurate assessing of failure risk for complex systems.

In addition, the RAVEN software provides the unique capability to combine any model (e.g. physical models, surrogate 
models, data analysis models, etc.) in a single entity (named Ensemble Model) where each model can feedback into others. This capability allows the user to analyze system that could be simulated only by using complex computational work-flows.

\subsection{User Characteristics}

The users of the RAVEN software are expected to be part of any of the
following categories:
\begin{itemize}
  \item \textbf{Core developers (RAVEN core team)}: These are the developers of the RAVEN software. They will be responsible for following
    and enforcing the appropriate software development standards. They will be responsible for designing, implementing and 
    maintaining the software.
  \item \textbf{External developers}: A Scientist or Engineer that utilizes the RAVEN framework and wants to extend its capabilities (new interface to external
 applications, new data analysis tecniques, new sampling strategies, etc).This user will typically have a background in modeling and 
simulation techniques and/or numerical analysis but may only have a limited skill-set when it comes to object-oriented coding, C++/Python languages.
  \item \textbf{Analysts}:  These are users that will run the code and perform various analysis on the simulations they perform. These users may interact with developers of the system requesting new features and reporting bugs found and will typically make heavy use of the input file format.
\end{itemize}
