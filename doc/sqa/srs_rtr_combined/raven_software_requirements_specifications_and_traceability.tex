%
% This is an example LaTeX file which uses the SANDreport class file.
% It shows how a SAND report should be formatted, what sections and
% elements it should contain, and how to use the SANDreport class.
% It uses the LaTeX article class, but not the strict option.
% ItINLreport uses .eps logos and files to show how pdflatex can be used
%
% Get the latest version of the class file and more at
%    http://www.cs.sandia.gov/~rolf/SANDreport
%
% This file and the SANDreport.cls file are based on information
% contained in "Guide to Preparing {SAND} Reports", Sand98-0730, edited
% by Tamara K. Locke, and the newer "Guide to Preparing SAND Reports and
% Other Communication Products", SAND2002-2068P.
% Please send corrections and suggestions for improvements to
% Rolf Riesen, Org. 9223, MS 1110, rolf@cs.sandia.gov
%
\documentclass[pdf,12pt]{../../user_manual/INLreport}
% pslatex is really old (1994).  It attempts to merge the times and mathptm packages.
% My opinion is that it produces a really bad looking math font.  So why are we using it?
% If you just want to change the text font, you should just \usepackage{times}.
% \usepackage{pslatex}
\usepackage{times}
%\usepackage{longtable}
\usepackage[FIGBOTCAP,normal,bf,tight]{subfigure}
\usepackage{amsmath}
\usepackage{tabularx}
\usepackage{longtable}
\usepackage{amssymb}
\usepackage[labelfont=bf]{caption}
\usepackage{pifont}
\usepackage{enumerate}
\usepackage{listings}
\usepackage{fullpage}
\usepackage{xcolor}          % Using xcolor for more robust color specification
\usepackage{ifthen}          % For simple checking in newcommand blocks
\usepackage{textcomp}
%\usepackage{authblk}         % For making the author list look prettier
%\renewcommand\Authsep{,~\,}

% Custom colors
\definecolor{deepblue}{rgb}{0,0,0.5}
\definecolor{deepred}{rgb}{0.6,0,0}
\definecolor{deepgreen}{rgb}{0,0.5,0}
\definecolor{forestgreen}{RGB}{34,139,34}
\definecolor{orangered}{RGB}{239,134,64}
\definecolor{darkblue}{rgb}{0.0,0.0,0.6}
\definecolor{gray}{rgb}{0.4,0.4,0.4}

\lstset {
  basicstyle=\ttfamily,
  frame=single
}

\setcounter{secnumdepth}{5}
\lstdefinestyle{XML} {
    language=XML,
    extendedchars=true,
    breaklines=true,
    breakatwhitespace=true,
%    emph={name,dim,interactive,overwrite},
    emphstyle=\color{red},
    basicstyle=\ttfamily,
%    columns=fullflexible,
    commentstyle=\color{gray}\upshape,
    morestring=[b]",
    morecomment=[s]{<?}{?>},
    morecomment=[s][\color{forestgreen}]{<!--}{-->},
    keywordstyle=\color{cyan},
    stringstyle=\ttfamily\color{black},
    tagstyle=\color{darkblue}\bf\ttfamily,
    morekeywords={name,type},
%    morekeywords={name,attribute,source,variables,version,type,release,x,z,y,xlabel,ylabel,how,text,param1,param2,color,label},
}
\lstset{language=python,upquote=true}

\usepackage{titlesec}
\newcommand{\sectionbreak}{\clearpage}
\setcounter{secnumdepth}{4}

%\titleformat{\paragraph}
%{\normalfont\normalsize\bfseries}{\theparagraph}{1em}{}
%\titlespacing*{\paragraph}
%{0pt}{3.25ex plus 1ex minus .2ex}{1.5ex plus .2ex}

%%%%%%%% Begin comands definition to input python code into document
\usepackage[utf8]{inputenc}

% Default fixed font does not support bold face
\DeclareFixedFont{\ttb}{T1}{txtt}{bx}{n}{9} % for bold
\DeclareFixedFont{\ttm}{T1}{txtt}{m}{n}{9}  % for normal

\usepackage{listings}

% Python style for highlighting
\newcommand\pythonstyle{\lstset{
language=Python,
basicstyle=\ttm,
otherkeywords={self, none, return},             % Add keywords here
keywordstyle=\ttb\color{deepblue},
emph={MyClass,__init__},          % Custom highlighting
emphstyle=\ttb\color{deepred},    % Custom highlighting style
stringstyle=\color{deepgreen},
frame=tb,                         % Any extra options here
showstringspaces=false            %
}}


% Python environment
\lstnewenvironment{python}[1][]
{
\pythonstyle
\lstset{#1}
}
{}

% Python for external files
\newcommand\pythonexternal[2][]{{
\pythonstyle
\lstinputlisting[#1]{#2}}}

\lstnewenvironment{xml}
{}
{}

% Python for inline
\newcommand\pythoninline[1]{{\pythonstyle\lstinline!#1!}}

% Named Colors for the comments below (Attempted to match git symbol colors)
\definecolor{RScolor}{HTML}{8EB361}  % Sonat (adjusted for clarity)
\definecolor{DPMcolor}{HTML}{E28B8D} % Dan
\definecolor{JCcolor}{HTML}{82A8D9}  % Josh (adjusted for clarity)
\definecolor{AAcolor}{HTML}{8D7F44}  % Andrea
\definecolor{CRcolor}{HTML}{AC39CE}  % Cristian
\definecolor{RKcolor}{HTML}{3ECC8D}  % Bob (adjusted for clarity)
\definecolor{DMcolor}{HTML}{276605}  % Diego (adjusted for clarity)
\definecolor{PTcolor}{HTML}{990000}  % Paul

\def\DRAFT{} % Uncomment this if you want to see the notes people have been adding
% Comment command for developers (Should only be used under active development)
\ifdefined\DRAFT
  \newcommand{\nameLabeler}[3]{\textcolor{#2}{[[#1: #3]]}}
\else
  \newcommand{\nameLabeler}[3]{}
\fi
\newcommand{\alfoa}[1] {\nameLabeler{Andrea}{AAcolor}{#1}}
\newcommand{\cristr}[1] {\nameLabeler{Cristian}{CRcolor}{#1}}
\newcommand{\mandd}[1] {\nameLabeler{Diego}{DMcolor}{#1}}
\newcommand{\maljdan}[1] {\nameLabeler{Dan}{DPMcolor}{#1}}
\newcommand{\cogljj}[1] {\nameLabeler{Josh}{JCcolor}{#1}}
\newcommand{\bobk}[1] {\nameLabeler{Bob}{RKcolor}{#1}}
\newcommand{\senrs}[1] {\nameLabeler{Sonat}{RScolor}{#1}}
\newcommand{\talbpaul}[1] {\nameLabeler{Paul}{PTcolor}{#1}}
% Commands for making the LaTeX a bit more uniform and cleaner
\newcommand{\TODO}[1]    {\textcolor{red}{\textit{(#1)}}}
\newcommand{\xmlAttrRequired}[1] {\textcolor{red}{\textbf{\texttt{#1}}}}
\newcommand{\xmlAttr}[1] {\textcolor{cyan}{\textbf{\texttt{#1}}}}
\newcommand{\xmlNodeRequired}[1] {\textcolor{deepblue}{\textbf{\texttt{<#1>}}}}
\newcommand{\xmlNode}[1] {\textcolor{darkblue}{\textbf{\texttt{<#1>}}}}
\newcommand{\xmlString}[1] {\textcolor{black}{\textbf{\texttt{'#1'}}}}
\newcommand{\xmlDesc}[1] {\textbf{\textit{#1}}} % Maybe a misnomer, but I am
                                                % using this to detail the data
                                                % type and necessity of an XML
                                                % node or attribute,
                                                % xmlDesc = XML description
\newcommand{\default}[1]{~\\*\textit{Default: #1}}
\newcommand{\nb} {\textcolor{deepgreen}{\textbf{~Note:}}~}

%%%%%%%% End comands definition to input python code into document

%\usepackage[dvips,light,first,bottomafter]{draftcopy}
%\draftcopyName{Sample, contains no OUO}{70}
%\draftcopyName{Draft}{300}

% The bm package provides \bm for bold math fonts.  Apparently
% \boldsymbol, which I used to always use, is now considered
% obsolete.  Also, \boldsymbol doesn't even seem to work with
% the fonts used in this particular document...
\usepackage{bm}

% Define tensors to be in bold math font.
\newcommand{\tensor}[1]{{\bm{#1}}}

% Override the formatting used by \vec.  Instead of a little arrow
% over the letter, this creates a bold character.
\renewcommand{\vec}{\bm}

% Define unit vector notation.  If you don't override the
% behavior of \vec, you probably want to use the second one.
\newcommand{\unit}[1]{\hat{\bm{#1}}}
% \newcommand{\unit}[1]{\hat{#1}}

% Use this to refer to a single component of a unit vector.
\newcommand{\scalarunit}[1]{\hat{#1}}

% \toprule, \midrule, \bottomrule for tables
\usepackage{booktabs}

% \llbracket, \rrbracket
\usepackage{stmaryrd}

\usepackage{hyperref}
\hypersetup{
    colorlinks,
    citecolor=black,
    filecolor=black,
    linkcolor=black,
    urlcolor=black
}
%\usepackage[table,xcdraw]{xcolor}
\newcommand{\wiki}{\href{https://github.com/idaholab/raven/wiki}{RAVEN wiki}}

% Compress lists of citations like [33,34,35,36,37] to [33-37]
\usepackage{cite}

% If you want to relax some of the SAND98-0730 requirements, use the "relax"
% option. It adds spaces and boldface in the table of contents, and does not
% force the page layout sizes.
% e.g. \documentclass[relax,12pt]{SANDreport}
%
% You can also use the "strict" option, which applies even more of the
% SAND98-0730 guidelines. It gets rid of section numbers which are often
% useful; e.g. \documentclass[strict]{SANDreport}

% The INLreport class uses \flushbottom formatting by default (since
% it's intended to be two-sided document).  \flushbottom causes
% additional space to be inserted both before and after paragraphs so
% that no matter how much text is actually available, it fills up the
% page from top to bottom.  My feeling is that \raggedbottom looks much
% better, primarily because most people will view the report
% electronically and not in a two-sided printed format where some argue
% \raggedbottom looks worse.  If we really want to have the original
% behavior, we can comment out this line...
\raggedbottom
\setcounter{secnumdepth}{5} % show 5 levels of subsection
\setcounter{tocdepth}{5} % include 5 levels of subsection in table of contents

% ---------------------------------------------------------------------------- %
%
% Set the title, author, and date
%
\title{RAVEN Software Requirements Specification and Traceability Matrix}
%\author{%
%\begin{tabular}{c} Author 1 \\ University1 \\ Mail1 \\ \\
%Author 3 \\ University3 \\ Mail3 \end{tabular} \and
%\begin{tabular}{c} Author 2 \\ University2 \\ Mail2 \\ \\
%Author 4 \\ University4 \\ Mail4\\
%\end{tabular} }


\author{Andrea Alfonsi}
 

% There is a "Printed" date on the title page of a SAND report, so
% the generic \date should [WorkingDir:]generally be empty.
\date{}


% ---------------------------------------------------------------------------- %
% Set some things we need for SAND reports. These are mandatory
%
\SANDnum{SPC-2366}
\SANDprintDate{December 2018}
\SANDauthor{Andrea Alfonsi}
\SANDreleaseType{Revision 0}

% ---------------------------------------------------------------------------- %
% Include the markings required for your SAND report. The default is "Unlimited
% Release". You may have to edit the file included here, or create your own
% (see the examples provided).
%
% \include{MarkOUO} % Not needed for unlimted release reports

\def\component#1{\texttt{#1}}

% ---------------------------------------------------------------------------- %
\newcommand{\systemtau}{\tensor{\tau}_{\!\text{SUPG}}}

% Added by Sonat
\usepackage{placeins}
\usepackage{array}

\newcolumntype{L}[1]{>{\raggedright\let\newline\\\arraybackslash\hspace{0pt}}m{#1}}
\newcolumntype{C}[1]{>{\centering\let\newline\\\arraybackslash\hspace{0pt}}m{#1}}
\newcolumntype{R}[1]{>{\raggedleft\let\newline\\\arraybackslash\hspace{0pt}}m{#1}}

% end added by Sonat
% ---------------------------------------------------------------------------- %
%
% Start the document
%

\begin{document}
    \maketitle

    % ------------------------------------------------------------------------ %
    % An Abstract is required for SAND reports
    %
%    \begin{abstract}
%    \input abstract
%    \end{abstract}


    % ------------------------------------------------------------------------ %
    % An Acknowledgement section is optional but important, if someone made
    % contributions or helped beyond the normal part of a work assignment.
    % Use \section* since we don't want it in the table of context
    %
%    \clearpage
%    \section*{Acknowledgment}



%	The format of this report is based on information found
%	in~\cite{Sand98-0730}.


    % ------------------------------------------------------------------------ %
    % The table of contents and list of figures and tables
    % Comment out \listoffigures and \listoftables if there are no
    % figures or tables. Make sure this starts on an odd numbered page
    %
    \cleardoublepage		% TOC needs to start on an odd page
    \tableofcontents
    %\listoffigures
    %\listoftables


    % ---------------------------------------------------------------------- %
    % An optional preface or Foreword
%    \clearpage
%    \section*{Preface}
%    \addcontentsline{toc}{section}{Preface}
%	Although muggles usually have only limited experience with
%	magic, and many even dispute its existence, it is worthwhile
%	to be open minded and explore the possibilities.


    % ---------------------------------------------------------------------- %
    % An optional executive summary
    %\clearpage
    %\section*{Summary}
    %\addcontentsline{toc}{section}{Summary}
    %\input{Summary.tex}
%	Once a certain level of mistrust and skepticism has
%	been overcome, magic finds many uses in todays science



%	and engineering. In this report we explain some of the
%	fundamental spells and instruments of magic and wizardry. We
%	then conclude with a few examples on how they can be used
%	in daily activities at national Laboratories.


    % ---------------------------------------------------------------------- %
    % An optional glossary. We don't want it to be numbered
%    \clearpage
%    \section*{Nomenclature}
%    \addcontentsline{toc}{section}{Nomenclature}
%    \begin{description}
%          \item[alohomoral]
%           spell to open locked doors and containers
%          \item[leviosa]
%           spell to levitate objects
%    \item[remembrall]
%           device to alert you that you have forgotten something
%    \item[wand]
%           device to execute spells
%    \end{description}


    % ---------------------------------------------------------------------- %
    % This is where the body of the report begins; usually with an Introduction
    %
    \SANDmain		% Start the main part of the report

\section{Introduction}
RAVEN is a flexible and multi-purpose uncertainty quantification (UQ), regression analysis, probabilistic risk assessment 
(PRA), data analysis and model optimization software.  
Its broad spectrum of application determined the need of an integrated design (see RAVEN SDD document for  details)
of the software aimed to integrate multiple requirements.
\\This document is aimed to report and explain the RAVEN software requirements.

\subsection{Dependencies and Limitations}
The software should be designed with the fewest possible constraints. 
Ideally the software should run on a wide variety of evolving hardware, 
so it should follow well-adopted standards and guidelines. The software
 should run on any POSIX compliant system (including Windows POSIX 
 emulators such as MinGW). The software will also make use of artificial 
 intelligence and numerical libraries that run on POSIX systems as well. 
 The main interface for the software will be command line based with no 
 assumptions requiring advanced terminal capabilities such as coloring and line control. 
 \\In order to be functional, RAVEN depends on the following software/libraries.
\begin{itemize}
  \item h5py-2.7.1
  \item numpy-1.12.1
  \item scipy-1.1.0
  \item scikit-learn-0.19.1
  \item pandas-0.20.3
  \item xarray-0.10.3
  \item netcdf4-1.4.0
  \item matplotlib-2.1.1
  \item statsmodels-0.8.0
  \item python-2.7
  \item hdf5-1.8.18
  \item swig
  \item pylint
  \item coverage
  \item lxml
  \item psutil
  \item pyside
  \item pillow
\end{itemize}

In addition, RAVEN (for its automatic regression test system), depends on MOOSE software (no libMesh or PTSC required).

\section{References}

\begin{itemize}

  \item ASME NQA 1 2008 with the NQA-1a-2009 addenda, ``Quality Assurance Requirements for Nuclear Facility Applications,'' First Edition, August 31, 2009.
  \item ISO/IEC/IEEE 24765:2010(E), ``Systems and software engineering Vocabulary,'' First Edition, December 15, 2010.
  \item LWP 13620, ``Managing Information Technology Assets''
\end{itemize}


\section{Definitions and Acronyms}

\subsection{Definitions}
\begin{itemize}
  \item \textbf{Baseline.} A specification or product (e.g., project plan, maintenance and operations [M\&O] plan, requirements, or 
design) that has been formally reviewed and agreed upon, that thereafter serves as the basis for use and further 
development, and that can be changed only by using an approved change control process. [ASME NQA-1-2008 with the 
NQA-1a-2009 addenda edited]
  \item \textbf{Validation.} Confirmation, through the provision of objective evidence (e.g., acceptance test), that the requirements 
for a specific intended use or application have been fulfilled. [ISO/IEC/IEEE 24765:2010(E) edited]
  \item \textbf{Verification.}
  \begin{itemize}
     \item The process of evaluating a system or component to determine whether the products of a given development 
     phase satisfy the conditions imposed at the start of that phase.
     \item  Formal proof of program correctness (e.g., requirements, design, implementation reviews, system tests). 
     [ISO/IEC/IEEE 24765:2010(E) edited]
  \end{itemize}
\end{itemize}

\subsection{Acronyms}
\begin{description}
\item[API] Application Programming Interfaces
\item[ASME] American Society of Mechanical Engineers
\item[CDF]  Comulative Distribution Functions
\item[DET] Dynamic Event Tree
\item[DOE] Department of Energy
\item[HDF5] Hierarchical Data Format (5)
\item[LWRS] Light Water Reactor Sustainability
\item[NEAMS] Nuclear Energy Advanced Modeling and Simulation
\item[NHES] Nuclear-Renewable Hybrid Energy Systems 
\item[INL] Idaho National Laboratory
\item[IT] Information Technology
\item[M\&O] Maintenance and Operations
\item[MC] Monte Carlo
\item[MOOSE] Multiphysics Object Oriented Simulation Environment
\item[NQA] Nuclear Quality Assurance
\item[POSIX]  Portable Operating System Interface
\item[PDF]  Probability Distribution (Density) Functions
\item[PP]  Post-Processor
\item[PRA]  Probabilistic Risk Assessment
\item[QA] Quality Assurance
\item[RAVEN] Risk Analysis and Virtual ENviroment
\item[ROM] Reduced Order Model
\item[SDD] System Design Description
\item[XML] eXtensible Markup Language 
\end{description}
\documentclass{article}
\usepackage{hyperref}
\newcommand{\requirement}[5]{\item[Requirement: #1] #2 \\Source: #3\\Explanation: #4\\Regression Test(s): #5}

\newcommand{\futurerequirement}[6]{\item[Future Requirement: #1 #6] #2 \\Source: #3\\Explanation: #4\\Regression Test(s): #5}

\title{RAVEN Requirements-DRAFT}

\begin{document}
\maketitle

\section{Introduction}

\subsection{System Purpose}

The RAVEN code is a generic software framework to perform parametric
and probabilistic analysis based on the response of complex system
codes. RAVEN is capable of investigating the system response as well
as the input space using Monte Carlo, Grid, or Latin Hyper Cube
sampling schemes, but its strength is focused toward system feature
discovery, such as limit surfaces, separating regions of the input
space leading to system failure, using dynamic supervised learning
techniques.

The development of RAVEN started in 2012 to satisfy the need to
provide a modern risk evaluation framework. RAVEN's principal
assignment is to provide the necessary software and algorithms in
order to employ the concept developed by the Risk Informed Safety
Margin Characterization (RISMC) program. RISMC is one of the pathways
defined within the Light Water Reactor Sustainability (LWRS)
program. In the RISMC approach, the goal is not just specifically
identifying the frequency of an event potentially leading to a system
failure, but the closeness (or not) to key safety-related events. This
approach may be used in identifying and increasing the safety margins
related to those events. A safety margin is a numerical value
quantifying the probability that a safety metric (e.g. as peak
pressure in a pipe) is exceeded under certain conditions. The initial
development of RAVEN has been focused on providing dynamic risk
assessment capability to RELAP-7, currently under development at the
INL and, the likely future replacement of the RELAP5-3D code. Most of
the capabilities implemented using RELAP-7 are easily deployable for
other system codes.

\subsection{System Scope}

The produced product is the RAVEN software.  It is a computer code
designed for probabilistic analysis.  RAVEN is a statistical analysis
tool that is used to estimate risk by computing real numbers to
determine what can go wrong, how likely is it, and what are its
consequences.  RAVEN takes in input (such as input files for
subprograms, or CSV files of data) and then can run subprograms with
perturbed input parameters to calculate the result of physical
simulations with varying input parameters.  Then RAVEN takes the
output of those program or the data provided and performs statistical
analysis on the data.

\subsection{System Overview}

\subsubsection{User Characteristics}

The users of the system are expected to be analysts performing
analysis of simulations of systems.  The users will use the outputs
for planning and design of the systems.  The system can be whatever
system the subprograms are simulating.  This can be a nuclear power
plant, but there is nothing specific in RAVEN that requires that.  The
system should have a simulation program that takes inputs and returns
output data that can be read and written by RAVEN.  The users will
need to understand both RAVEN and the subprogram(s) that are used.

\subsubsection{Assumptions and Dependencies}

The software depends on the Python programming language being
available.  The software uses C++ to compile some Python modules.  The
software depends on other software:

\begin{enumerate}
\item MOOSE (libMesh is not required)
\item numpy
\item hdf5
\item Cython
\item h5py
\item scipy
\item scikit-learn
\item matplotlib
\item swig
\end{enumerate}

The software uses subprograms to run simulations.  RAVEN lets the user
write interfaces for this purpose.  There are existing interfaces for
RELAP-5, RELAP-7, and generic MOOSE based applications.

The software runs on Linux, Mac OSX, and Windows.  The computer system
needs to have the necessary Python libraries and be able to compile
the RAVEN Python modules.

For running on distributed memory machines, RAVEN uses PBSPro by
default.  Otherwise a Custom Mode will need to be programmed.

It is assumed that other methods will be used to provide security
(such as running in a single user's account with the input and output
of the program stored in that user's account).

\section{References}

\begin{flushleft}
ASME NQA-1-2008 with the NQA-1a-2009 addenda, ``Quality Assurance
Requirements for Nuclear Facility Applications,'' First Edition, August
31, 2009.

INL/EXT-15-34123, ``RAVEN User Manual,'' Rev 3, October 13, 2015

LWP-13620, ``Managing Information Technology Assets,'' Rev. 16, December
23, 2013.
\end{flushleft}

\section{Definitions and Acronyms}

\begin{description}
\item[CDF] cumulative distribution function
\item[CSV] comma separated variable (a simple spreadsheet format)
\item[RAVEN] Risk Analysis Virtual ENvironment
\item[PDF] probability distribution function
\end{description}

\section{Risk Evaluation}

\begin{description}

\requirement{R-RE-1}{RAVEN shall support 1-Dimensional probability distributions including generating random numbers from them.}
{INL/EXT-15-34123 Rev 3 RAVEN User Manual 8.1 1-Dimensional Probability Distributions}
{RAVEN needs to create different input parameters for the simulations that it runs.  For the non-adaptive sampling, the value of those input parameters are generated according to their probability distributions functions (including uniform distributions).  In order to do this, the distributions need to be able to calculate things like PDFs and CDFs and inverse CDFs.}
{test\_distributions}
%tests/framework/tests

\requirement{R-RE-2}{RAVEN shall support N-Dimensional probability distributions.  It shall support multivariate normal distributions and distributions defined by tabular data.}
{INL/EXT-15-34123 Rev 3 RAVEN User Manual 8.2 N-Dimensional Probability Distributions}
{The N-Dimensional probability distributions allow the user to model stochastic dependencies between input parameters.}
{ND\_external\_MC}
%tests/framework/tests

\requirement{R-RE-3}{RAVEN shall support a variety of samplers that use probability distributions to sample the input space.}
{INL/EXT-15-34123 Rev 3 RAVEN User Manual 9.1 Once-through Samplers}
{Forward samplers allow sampling strategies such as Grid sampling, Monte Carlo and Latin Hypercube sampling.  These samplers allow the analyses to be performed.  The samplers are Python code that is included with RAVEN.}
{testGrid}
%tests/framework/tests

\end{description}

\section{Risk Analysis}

\begin{description}

\requirement{R-RA-1}{RAVEN shall support adaptive sampling that use already gathered samples to determine where to do new samples.}
{INL/EXT-15-34123 Rev 3 RAVEN User Manual 9.3 Adaptive Samplers}
{The adaptive samplers support sampling the input space, but in a more efficient manner.  Examples of these samplers is limit surface search and adaptive stochastic collocation polynomial chaos.}
{testLimitSurfacePostProcessor}
%tests/framework/tests

\requirement{R-RA-2}{RAVEN shall support inputting/outputting data in CSV format.}
{INL/EXT-15-34123 Rev 3 RAVEN User Manual 12.1 Printing system}
{The user needs to be able to get the data from other programs or to analyze it in other software. The data will be read or written to disk space that RAVEN has access to. Inputting/Outputting the data in CSV files allows this use to be done.  The CSV files will be simple tables (header line, followed by numeric data for each column).}
{test\_iostep\_load}
%tests/framework/tests

\requirement{R-RA-3}{RAVEN shall support generating plots from the data it generates.}
{INL/EXT-15-34123 Rev 3 RAVEN User Manual 12.2 Plotting system}
{The user needs to be able to see the progress of the algorithms, and what the results are graphically.  As well, plots to be used in documentation and reports need to be outputted.  The plotting capability of RAVEN is used for this.  Example plots include scatter plots or wireframe plots relating input parameters to output parameters.}
{test\_output}
%tests/framework/tests

\requirement{R-RA-4}{RAVEN shall be able to generate Reduced Order Models from its data and use them to predict responses from a system.}
{INL/EXT-15-34123 Rev 3 RAVEN User Manual 13.3 ROM}
{Often the physical model is computationally expensive.  For some models the relevant output parameters can be captured by a much simpler model that can be quickly calculated.  This is the purpose for the Reduced order model.}
{test\_rom\_trainer}
%tests/framework/tests

\requirement{R-RA-5}{RAVEN shall be able to perform basic statistical analysis of generated data.}
{INL/EXT-15-34123 Rev 3 RAVEN User Manual 13.5}
{One of the main tasks of the RAVEN code is to assess to how the probabilistic distribution of the input parameters reflects in the figure of merits characterizing the system response.}
{framework/PostProcessors/BasicStatistics.testBasicStatisticsGeneral}
%tests/framework/PostProcessors/BasicStatistics/tests

\requirement{R-RA-6}{RAVEN shall be able to perform advanced post processing of generated data, using classical data mining methodologies}
{INL/EXT-15-36632 Developing and Implementing tho Data Mining Algorithms in RAVEN Rev 0}
{Often engineers need to understand the system response to the variation of several input and output parameters which mutually interacts. Such a scenario is of difficult interpretation and data mining algorithms help in this task.  Data mining methodologies include clustering, principal component analysis, etc.}
{framework/PostProcessors/DataMiningPostProcessor\-/DimensionalityReduction.ExactPCA}


\end{description}

\section{Risk Mitigation}

\begin{description}

\requirement{R-RM-1}{RAVEN shall be able to choose the values of a set of input parameters that minimize/maximize a goal function that depends on system output figure of merits and input parameters.}
{INL/EXT-15-34123 Rev 3 RAVEN User Manual Chapter 13}
{Mitigation of risk is related to choose a set of operational or design parameters so that, in a probabilistic sense the risk driven goal function is minimized. In addition, parametric minimization (not risk/probability weighted) is essential for design activities.}
{framework/Optimizers.Beale}
\end{description}

\section{Infrastructure Support}

\begin{description}
\requirement{R-IS-1}{RAVEN shall be able to parallelize running external codes.}
{INL/EXT-15-34123 Rev 3 RAVEN User Manual 6.2 RunInfo: Input of Queue Modes}
{RAVEN runs external codes, and sometimes they are not parallelized.  RAVEN will run faster if it can run multiple codes at the same time when multiple cores are available.  Even for parallelized codes it usually will be more efficient to run multiple instances in parallel than run one code parallelized.}
{testLHSBisonParallel}
%tests/framework/CodeInterfaceTests/tests

\requirement{R-IS-2}{RAVEN shall be able to run external codes by supplying them with the needed input files and collecting the output data.}
{INL/EXT-15-34123 Rev 3 RAVEN User Manual 7 Files}
{RAVEN runs external codes, and each instance may need a different input file that needs to be generated from the sampler choices.  RAVEN also may need to read the output files in. (possibly with application specific code that is user provided.)  The output files will often be CSV files.  For MOOSE based application, there is code to generate MOOSE input files, and to read MOOSE output.  For other external codes, the user many need to provide interface code to create input files and read in the output files the external code uses.}
{simple\_framework}
%tests/framework/test

\requirement{R-IS-3}{RAVEN shall support storing and retrieving data in a HDF5 database.}
{INL/EXT-15-34123 Rev 3 RAVEN User Manual 11 Databases}
{RAVEN uses HDF5 databases to store inputs and results for simulations, as well as other auxiliary information.  This allows using the data generated in sub-sequential steps.  This stores the input parameters used and the numeric output parameters that the simulation produces.}
{2steps\_same\_db}
%tests/framework/tests

\requirement{R-IS-4}{RAVEN shall be able to provide data to a user provided python function, and retrieve the data from that.}
{INL/EXT-15-34123 Rev 3 RAVEN User Manual 13.4 External Model}
{Sometimes all that is needed for the simulation is a function that can be calculated in Python.  The external model allows this.  This executes a python function to determine the result.}
{testExternalModel}
%tests/framework/tests

\requirement{R-IS-5}{RAVEN shall be able to perform various calculation tasks (simulation and post processing), and transfer data to the next task.}
{INL/EXT-15-34123 Rev 3 RAVEN User Manual 15 Steps}
{Sequences of calculation are one of the main uses of RAVEN.  For example, a initial calculation can be used to generate data to train a ROM, and then later calculations can use the ROM for faster calculation.  As well, steps allow various post processing to be done.}
{calculate\_and\_transfer}
%tests/framework/tests

\requirement{R-IS-6}{RAVEN shall be able to run external codes in parallel on shared memory machines.}
{INL/EXT-15-34123 Rev 3 RAVEN User Manual 6.1 batchSize}
{RAVEN will run on shared memory machines, and should be able to run external codes in parallel on them.  This can be done by running multiple processes.}
{test\_bison\_mc\_simple\_\&\_alias\_system}
%tests/framework/tests

\requirement{R-IS-7}{RAVEN shall be able to run external codes in parallel on distributed memory machines.}
{INL/EXT-15-34123 Rev 3 RAVEN User Manual 6.1 batchSize}
{RAVEN will run on distributed memory machines, and should be able to run external codes in parallel on them.  This can be done by running with mpiexec.}
{cluster\_tests/test\_mpiqsub\_local.xml}
%tests/cluster_tests/test_qsubs.sh

\requirement{R-IS-8}{RAVEN shall be able to run internal models in parallel on shared memory machines.}
{INL/EXT-15-34123 Rev 3 RAVEN User Manual 6.1 internalParallel}
{RAVEN will run on shared memory machines, and it would be useful to run internal codes in parallel on them.}
{cluster\_tests/InternalParallel/test\_internal\_parallel\_extModel.xml}
%tests/cluster_tests/test_qsubs.sh

\requirement{R-IS-9}{RAVEN shall be able to run internal models in parallel on distributed memory machines.}
{INL/EXT-15-34123 Rev 3 RAVEN User Manual 6.1 internalParallel}
{RAVEN will run on distributed memory machines, and it would be useful to run internal codes in parallel on them.}
{cluster\_tests/InternalParallel/test\_internal\_parallel\_ROM\_scikit.xml}
%tests/cluster_tests/test_qsubs.sh

\end{description}

\section*{Document Version Information}

d317572ffb80f564d27ec54e5db24d763705a8a5 Aaron Epiney - INL Tue, 17 Oct 2017 XX:XX:XX -0600


\end{document}

\subsection{System Operations}
\subsubsection{Human System Integration Requirements}
The command line interface shall support the ability to toggle any supported coloring schemes on or off pursuant to section 
508 of the Rehabilitation Act of 1973.
\subsubsection{Maintainability}
\begin{itemize}
  \item The latest working version (defined as the version that passes all tests in the current regression test suite) shall be 
           publicly available at all times through the repository host provider.
  \item  Flaws identified in the system shall be reported and tracked in a ticket or issue based system. The technical lead or 
            any COB member will 
            determine the severity and priority of all reported issues. The technical lead will assign resources at his or her 
            discretion to resolve identified issues.
  \item  The software maintainers will entertain all proposed changes to the system in a timely manner 
           (within two business days).        
  \item  The RAVEN software in its entirety will be made publicly available under the Apache version 2.0 license.     
\end{itemize}
\subsubsection{Human System Integration Requirements}
The regression test suite will cover at least 80\% of all lines of code at all times. 
The results of the regression tests will be stored in the Continuous Integration System.

\subsection{Information Management}
The RAVEN software in its entirety will be made publicly available on an appropriate repository hosting site (e.g. GitHub).
Backups and security services will be provided by the hosting service.

\section{Verification}
The regression test suite shall employ several verification tests using comparison against analytic 
solutions (when possible) and convergence rate analysis. 
 \section{RAVEN:SYSTEM REQUIREMENTS} 
 \subsection{Requirements Traceability Matrix} 
 This section contains all of the requirements, requirements' description, and 
 requirement test cases. The requirement tests are automatically tested for each 
 CR (Change Request) by the CIS (Continuous Integration System). 
 \newcolumntype{b}{X} 
 \newcolumntype{s}{>{\hsize=.5\hsize}X} 
 \subsubsection{Minimum Requirements} 
\begin{tabularx}{\textwidth}{|s|s|b|} 
\hline 
\textbf{Requirment ID} & \textbf{Requirment Description} & \textbf{Test(s)}  \\ \hline 
\hline 
 \hspace{0pt}R-M-1 & \hspace{0pt}Computer: Any POSIX (and POSIX-like) system & \hspace{0pt}1)``RAVEN User Manual'', INL/EXT-15-34123 2)Continous Integration System \\ \hline 
\hline 
 \hspace{0pt}R-M-2 & \hspace{0pt}RAM: 2 GB per core execution (depending on the type of analysis and data genarated) & \hspace{0pt}1)``RAVEN User Manual'', INL/EXT-15-34123 2)Continous Integration System \\ \hline 
\hline 
 \hspace{0pt}R-M-3 & \hspace{0pt}Disk: 10 GB (size depending on the type of analysis and data generated) & \hspace{0pt}1)``RAVEN User Manual'', INL/EXT-15-34123 2)Continous Integration System \\ \hline 
\hline 
 \hspace{0pt}R-M-4 & \hspace{0pt}Compilers: GCC, Clang, or Intel & \hspace{0pt}1)``RAVEN User Manual'', INL/EXT-15-34123 2)Continous Integration System \\ \hline 
\hline 
 \hspace{0pt}R-M-5 & \hspace{0pt}Language: Python 2.7 & \hspace{0pt}1)``RAVEN User Manual'', INL/EXT-15-34123 2)Continous Integration System \\ \hline 
\hline 
 \hspace{0pt}R-M-6 & \hspace{0pt}Version Control: Git & \hspace{0pt}1)``RAVEN User Manual'', INL/EXT-15-34123 2)Continous Integration System \\ \hline 
\hline 
\caption*{Minimum Requirements}
\end{tabularx} 
 \subsubsection{Functional Requirements} 
\begin{tabularx}{\textwidth}{|s|s|b|} 
\hline 
\textbf{Requirment ID} & \textbf{Requirment Description} & \textbf{Test(s)}  \\ \hline 
\hline 
 \hspace{0pt}R-F-1 & \hspace{0pt}RAVEN shall allow support for user-defined instructions for controlling the execution stages of a simulation. & \hspace{0pt}1)/raven/tests/framework/test\_rom\_trainer.xml 2)/raven/tests/framework/test\_random.xml \\ \hline 
\hline 
 \hspace{0pt}R-F-2 & \hspace{0pt}RAVEN shall allow for user-defined resource allocation for driving external applications. & \hspace{0pt}1)/raven/tests/framework/CodeInterfaceTests/generic\_parallel.xml \\ \hline 
\hline 
 \hspace{0pt}R-F-3 & \hspace{0pt}RAVEN shall support a programmatic method for building up and/or downloading the necessary compiled objects/dependencies necessary for a simulation. & \hspace{0pt}1)RAVEN User Manual, INL/EXT-15-34123 \\ \hline 
\hline 
 \hspace{0pt}R-F-4 & \hspace{0pt}RAVEN shall provide the ability to resume a previous simulation using data generated and exported by RAVEN itself. & \hspace{0pt}1)/raven/tests/framework/Samplers/Restart/test\_restart\_MC.xml 2)/raven/tests/framework/Samplers/Restart/test\_restart\_csv.xml 3)/raven/tests/framework/Samplers/Restart/test\_restart\_constant.xml \\ \hline 
\hline 
 \hspace{0pt}R-F-5 & \hspace{0pt}RAVEN shall allow for user-defined output types for simulation data. & \hspace{0pt}1)/raven/tests/framework/test\_output.xml 2)/raven/tests/framework/ROM/TimeSeries/DMD/test\_traditional\_dmd.xml \\ \hline 
\hline 
 \hspace{0pt}R-F-6 & \hspace{0pt}RAVEN shall allow for a standardized method for importing simulation data not previosly generated by the system itself. & \hspace{0pt}1)/raven/tests/framework/test\_output.xml 2)/raven/tests/framework/test\_iostep\_load.xml 3)/raven/tests/framework/Databases/test\_load\_and\_push\_reusing\_same\_hdf5.xml \\ \hline 
\hline 
\caption*{Framework, I/O, Execution Control}
\end{tabularx} 
 \subsubsection{Usability Requirements} 
\begin{tabularx}{\textwidth}{|s|s|b|} 
\hline 
\textbf{Requirment ID} & \textbf{Requirment Description} & \textbf{Test(s)}  \\ \hline 
\hline 
 \hspace{0pt}R-RE-1 & \hspace{0pt}RAVEN shall support 1-Dimensional probability distributions including generating random numbers from them. & \hspace{0pt}1)/raven/tests/framework/unit\_tests/Distributions/TestDistributions.py \\ \hline 
\hline 
 \hspace{0pt}R-RE-2 & \hspace{0pt}RAVEN shall support N-Dimensional probability distributions. It shall support multivariate normal distributions and distributions defined by tabular data. & \hspace{0pt}1)/raven/tests/framework/test\_simple\_ND\_external\_MC.xml \\ \hline 
\hline 
 \hspace{0pt}R-RE-3 & \hspace{0pt}RAVEN shall support a variety of samplers that use probability distributions to sample the input space. & \hspace{0pt}1)/raven/tests/framework/test\_Grid\_Sampler.xml \\ \hline 
\hline 
\caption*{Risk Evaluation}
\end{tabularx} 
\begin{tabularx}{\textwidth}{|s|s|b|} 
\hline 
\textbf{Requirment ID} & \textbf{Requirment Description} & \textbf{Test(s)}  \\ \hline 
\hline 
 \hspace{0pt}R-RA-1 & \hspace{0pt}RAVEN shall support adaptive sampling that use already gathered samples to determine where to locate new samples. & \hspace{0pt}1)/raven/tests/framework/PostProcessors/LimitSurface/test\_LimitSurface.xml \\ \hline 
\hline 
 \hspace{0pt}R-RA-2 & \hspace{0pt}RAVEN shall support importing/exporting data in CSV format. & \hspace{0pt}1)/raven/tests/framework/test\_iostep\_load.xml \\ \hline 
\hline 
 \hspace{0pt}R-RA-3 & \hspace{0pt}RAVEN shall support generating plots from the data it generates. & \hspace{0pt}1)/raven/tests/framework/test\_output.xml \\ \hline 
\hline 
 \hspace{0pt}R-RA-4 & \hspace{0pt}RAVEN shall be able to generate Reduced Order Models from its data and use them to predict responses from a system. & \hspace{0pt}1)/raven/tests/framework/test\_rom\_trainer.xml \\ \hline 
\hline 
 \hspace{0pt}R-RA-5 & \hspace{0pt}RAVEN shall be able to perform basic statistical analysis of generated data. & \hspace{0pt}1)/raven/tests/framework/PostProcessors/BasicStatistics/test\_BasicStatistics.xml \\ \hline 
\hline 
 \hspace{0pt}R-RA-6 & \hspace{0pt}RAVEN shall be able to perform advanced post processing of generated data, using data mining methodologies. & \hspace{0pt}1)/raven/tests/framework/PostProcessors/DataMiningPostProcessor/DimensionalityReduction/test\_dataMiningExactPCA.xml \\ \hline 
\hline 
 \hspace{0pt}R-RA-7 & \hspace{0pt}RAVEN shall be able to compute probability of failure based on generated data and goal functions & \hspace{0pt}1)/raven/tests/framework/PostProcessors/LimitSurface/test\_LimitSurface.xml 2)/raven/tests/framework/PostProcessors/LimitSurface/test\_LimitSurface\_and\_integral.xml \\ \hline 
\hline 
\caption*{Risk Analysis}
\end{tabularx} 
\begin{tabularx}{\textwidth}{|s|s|b|} 
\hline 
\textbf{Requirment ID} & \textbf{Requirment Description} & \textbf{Test(s)}  \\ \hline 
\hline 
 \hspace{0pt}R-RM-1 & \hspace{0pt}RAVEN shall be able to choose the values of a set of input parameters that minimize/maximize a goal function that depends on system output figure of merits and input parameters. & \hspace{0pt}1)/raven/tests/framework/Optimizers/beale.xml \\ \hline 
\hline 
\caption*{Risk Mitigation}
\end{tabularx} 
 \subsubsection{Performance Requirements} 
\begin{tabularx}{\textwidth}{|s|s|b|} 
\hline 
\textbf{Requirment ID} & \textbf{Requirment Description} & \textbf{Test(s)}  \\ \hline 
\hline 
 \hspace{0pt}R-IS-1 & \hspace{0pt}RAVEN shall be able to parallelize running external codes. & \hspace{0pt}1)/raven/tests/framework/CodeInterfaceTests/test\_LHS\_Sampler\_Bison\_parallel.xml \\ \hline 
\hline 
 \hspace{0pt}R-IS-2 & \hspace{0pt}RAVEN shall be able to run external codes by supplying them with the needed input files and collecting the output data. & \hspace{0pt}1)/raven/tests/framework/test\_simple.xml \\ \hline 
\hline 
 \hspace{0pt}R-IS-3 & \hspace{0pt}RAVEN shall support storing and retrieving data in a HDF5 database. & \hspace{0pt}1)/raven/tests/framework/Databases/test\_2steps\_same\_db.xml \\ \hline 
\hline 
 \hspace{0pt}R-IS-4 & \hspace{0pt}RAVEN shall be able to provide data to a user provided python function, and retrieve the data from that. & \hspace{0pt}1)/raven/tests/framework/test\_Lorentz.xml \\ \hline 
\hline 
 \hspace{0pt}R-IS-5 & \hspace{0pt}RAVEN shall be able to perform various calculation tasks (simulation and post processing), and transfer data to the next task. & \hspace{0pt}1)/raven/tests/framework/test\_calc\_and\_transfer.xml \\ \hline 
\hline 
 \hspace{0pt}R-IS-6 & \hspace{0pt}RAVEN shall be able to run external codes in parallel on shared memory machines. & \hspace{0pt}1)/raven/tests/framework/test\_bison\_mc\_simple.xml 2)/raven/tests/framework/CodeInterfaceTests/test\_generic\_interface.xml 3)/raven/tests/framework/CodeInterfaceTests/test\_generic\_interface\_custom\_out\_file.xml \\ \hline 
\hline 
 \hspace{0pt}R-IS-7 & \hspace{0pt}RAVEN shall be able to run external codes in parallel on distributed memory machines. & \hspace{0pt}1)raven/cluster\_tests/test\_mpi.xml 2)raven/cluster\_tests/test\_mpiqsub\_local.xml 3)raven/cluster\_tests/test\_pbs.xml \\ \hline 
\hline 
 \hspace{0pt}R-IS-8 & \hspace{0pt}RAVEN shall be able to run internal models in parallel on shared memory machines. & \hspace{0pt}1)/raven/tests/framework/InternalParallelTests/test\_internal\_parallel\_ROM\_scikit.xml 2)/raven/tests/framework/InternalParallelTests/test\_internal\_parallel\_extModel.xml 3)/raven/tests/framework/InternalParallelTests/test\_internal\_parallel\_PP\_LS.xml \\ \hline 
\hline 
 \hspace{0pt}R-IS-9 & \hspace{0pt}RAVEN shall be able to run internal models in parallel on distributed memory machines. & \hspace{0pt}1)raven/cluster\_tests/InternalParallel/test\_internal\_parallel\_extModel.xml 2)raven/cluster\_tests/InternalParallel/test\_internal\_parallel\_PP\_LS.xml 3)raven/cluster\_tests/InternalParallel/test\_internal\_parallel\_ROM\_scikit.xml \\ \hline 
\hline 
\caption*{Infrastructure Support}
\end{tabularx} 
 \subsubsection{System Interfaces} 
\begin{tabularx}{\textwidth}{|s|s|b|} 
\hline 
\textbf{Requirment ID} & \textbf{Requirment Description} & \textbf{Test(s)}  \\ \hline 
\hline 
 \hspace{0pt}R-SI-1 & \hspace{0pt}RAVEN shall be able to be coupled with external applications via input files. & \hspace{0pt}1)/raven/tests/framework/CodeInterfaceTests/RAVEN/rom.xml 2)/raven/tests/framework/CodeInterfaceTests/RELAP5/test\_relap5\_code\_interface.xml \\ \hline 
\hline 
 \hspace{0pt}R-SI-2 & \hspace{0pt}RAVEN shall be able to be coupled with external applications via Python API. & \hspace{0pt}1)/raven/plugins/ExamplePlugin/tests/test\_example\_plugin.xml 2)/raven/plugins/ExamplePlugin/tests/test\_raven\_running\_raven\_plugin.xml 3)/raven/tests/plugins/ExamplePlugin/test\_example\_plugin.xml 4)/raven/tests/plugins/ExamplePlugin/test\_raven\_running\_raven\_plugin.xml \\ \hline 
\hline 
\caption*{Interface with external applications}
\end{tabularx} 
\end{document}
 

    % ---------------------------------------------------------------------- %
    % References
    %

%\addcontentsline{toc}{section}{References}
%\bibliographystyle{ieeetr}
%\bibliography{raven_software_requirements_specifications}

\section*{Document Version Information}

d317572ffb80f564d27ec54e5db24d763705a8a5 Aaron Epiney - INL Tue, 17 Oct 2017 XX:XX:XX -0600


\end{document}
