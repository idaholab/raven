\section{Models  \\ \vspace{2 mm} {\small }}
\label{sec:models}
In the RAVEN code a crucial entity is represented by a Model. A model is an object that employs a mathematical model used to re

The xml section "models" contains the information regarding the code employed in the analysis (e.g., RAVEN/RELAP-7, RELAP-5 or an external model). A model is something that given an input will return an output reproducing some physical model it could be as complex as a stand alone code, a reduced order model trained somehow or something externally build and imported by the user.
The available models are:
\begin{description}
\item [Dummy:] it is a dummy model that just return the effect of the sampler. The values reported as input in the output are the output of the sampler and the output is the counter of the performed sampling
\item [ROM:] ROM stands for Reduced Order Model. All the models here, first learn than predict the outcome
\item [ExternalModel:] this model allows to interface with an external python module
\item [Code:] generic class that imports an external code into the framework
\item [Projector:] generic data manipulator
\item [PostProcessor:] an Action System. All the models here, take an input and perform an action
\end{description}

\subsection{Dummy}
\label{sec:models_dummy}

Description

Summary

Example

\subsection{ROM}
\label{subsec:models_ROM}

Description

Summary 
\begin{itemize}
\item name: name of the ROM model
\item subType: 'SciKitLearn'. Imports the libraries from scikitlearn
\item Features: input the variables set in the Samplers block separated by a comma
\item SKLtype: input a model from http://scikit-learn.org/stable/modules/classes.html (LinearRegression in the example)
\item Target: Input a set of variables inside the output space which the ROM has to be developed for. 
\item Parameters: From the SKLtype class are defined all the parameters required ($fit\_intercept$ and $normalize$ in the exemple) 
\end{itemize}

Example

\begin{lstlisting}[style=XML]
<Models>
 <ROM name='***' subType='***'>
  <Features>***,***</Features>
  <SKLtype>linear_model|LinearRegression</SKLtype>
  <Target>***</Target>
  <fit_intercept>***</fit_intercept>
  <normalize>***</normalize>
 </ROM>
\end{lstlisting}




\subsection{External Model}
\label{subsec:models_externalModel}

Description

Summary

Example
As an example we use the external model shown in lorentzAttractor.py which, given the 3-dimensional initial coordinates (x0, y0, z0), calculate the trajectory of a Lorentz attractor in the time interval $[0.0,0.03]$ seconds.
We want to perform sampling of the 3-dimensional initial conditions of the attractor using classical Monte-Carlo sampling.
The user is required to specify:
\begin{itemize}
\item the initialize function: def initialize(self,runInfoDict,inputFiles)
\item the function which create a new input: def createNewInput(self,myInput,samplerType,**Kwargs)
\item the function which perform the actual calculation: def run(self,Input)
\end{itemize}

\begin{python}
def initialize(self,runInfoDict,inputFiles):
  self.SampledVars = None
  self.sigma = 10.0
  self.rho   = 28.0
  self.beta  = 8.0/3.0
  return

def createNewInput(self,myInput,samplerType,**Kwargs):
  return Kwargs['SampledVars']

def run(self,Input):
   ...
\end{python}


\begin{lstlisting}[style=XML]
<Models>
    <ExternalModel name='PythonModule' subType='' ModuleToLoad='externalModel/lorentzAttractor'>  
       <variable type='float'>sigma</variable>
       <variable type='float'>rho</variable>
       <variable type='float'>beta</variable>
       <variable type='numpy.ndarray'>x</variable>
       <variable type='numpy.ndarray'>y</variable>
       <variable type='numpy.ndarray'>z</variable>
       <variable type='numpy.ndarray'>time</variable>
       <variable type='float'>x0</variable>
       <variable type='float'>y0</variable>
       <variable type='float'>z0</variable>
    </ExternalModel>
</Models> 
\end{lstlisting}

\subsection{Code}
\label{sec:models_code}

Description: This is the generic class that import an external code into the framework

Summary

Example

\subsection{Projector}
\label{sec:models_projector}

Description

Summary

Example

\subsection{PostProcessor}
\label{sec:models_postProcessor}

Description

List variable, Input Data, 
Keyword sul tipo analisi statistica!!

Summary

Example
