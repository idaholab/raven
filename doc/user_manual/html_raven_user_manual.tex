%
% This is an example LaTeX file which uses the SANDreport class file.
% It shows how a SAND report should be formatted, what sections and
% elements it should contain, and how to use the SANDreport class.
% It uses the LaTeX article class, but not the strict option.
% ItINLreport uses .eps logos and files to show how pdflatex can be used
%
% Get the latest version of the class file and more at
%    http://www.cs.sandia.gov/~rolf/SANDreport
%
% This file and the SANDreport.cls file are based on information
% contained in "Guide to Preparing {SAND} Reports", Sand98-0730, edited
% by Tamara K. Locke, and the newer "Guide to Preparing SAND Reports and
% Other Communication Products", SAND2002-2068P.
% Please send corrections and suggestions for improvements to
% Rolf Riesen, Org. 9223, MS 1110, rolf@cs.sandia.gov
%
\documentclass[pdf,12pt]{report}
% pslatex is really old (1994).  It attempts to merge the times and mathptm packages.
% My opinion is that it produces a really bad looking math font.  So why are we using it?
% If you just want to change the text font, you should just \usepackage{times}.
% \usepackage{pslatex}
\usepackage{times}
\usepackage[FIGBOTCAP,normal,bf,tight]{subfigure}
\usepackage{amsmath}
\usepackage{amssymb}
\usepackage{pifont}
\usepackage{enumerate}
\usepackage{listings, color}
\definecolor{forestgreen}{RGB}{34,139,34}
\definecolor{orangered}{RGB}{239,134,64}
\definecolor{darkblue}{rgb}{0.0,0.0,0.6}
\definecolor{gray}{rgb}{0.4,0.4,0.4}
\setcounter{secnumdepth}{5}
\lstdefinestyle{XML} {
    language=XML,
    extendedchars=true,
    breaklines=true,
    breakatwhitespace=true,
    emph={name,dim,interactive,overwrite},
    emphstyle=\color{red},
    basicstyle=\ttfamily,
    columns=fullflexible,
    commentstyle=\color{gray}\upshape,
    morestring=[b]",
    morecomment=[s]{<?}{?>},
    morecomment=[s][\color{forestgreen}]{<!--}{-->},
    keywordstyle=\color{cyan},
    stringstyle=\ttfamily\color{black}\normalfont,
    tagstyle=\color{darkblue}\bf,
    morekeywords={attribute,source,variables,version,type,release,x,z,y,xlabel,ylabel,how,text,param1,param2,color,label},
}


%%%%%%%% Begin comands definition to input python code into document
\usepackage[utf8]{inputenc}

% Default fixed font does not support bold face
\DeclareFixedFont{\ttb}{T1}{txtt}{bx}{n}{9} % for bold
\DeclareFixedFont{\ttm}{T1}{txtt}{m}{n}{9}  % for normal

% Custom colors
\usepackage{color}
\definecolor{deepblue}{rgb}{0,0,0.5}
\definecolor{deepred}{rgb}{0.6,0,0}
\definecolor{deepgreen}{rgb}{0,0.5,0}

\usepackage{listings}

% Python style for highlighting
\newcommand\pythonstyle{\lstset{
language=Python,
basicstyle=\ttm,
otherkeywords={self, none, return},             % Add keywords here
keywordstyle=\ttb\color{deepblue},
emph={MyClass,__init__},          % Custom highlighting
emphstyle=\ttb\color{deepred},    % Custom highlighting style
stringstyle=\color{deepgreen},
frame=tb,                         % Any extra options here
showstringspaces=false            %
}}


% Python environment
\lstnewenvironment{python}[1][]
{
\pythonstyle
\lstset{#1}
}
{}

% Python for external files
\newcommand\pythonexternal[2][]{{
\pythonstyle
\lstinputlisting[#1]{#2}}}

% Python for inline
\newcommand\pythoninline[1]{{\pythonstyle\lstinline!#1!}}
%%%%%%%% End comands definition to input python code into document

%\usepackage[dvips,light,first,bottomafter]{draftcopy}
%\draftcopyName{Sample, contains no OUO}{70}
%\draftcopyName{Draft}{300}

% The bm package provides \bm for bold math fonts.  Apparently
% \boldsymbol, which I used to always use, is now considered
% obsolete.  Also, \boldsymbol doesn't even seem to work with
% the fonts used in this particular document...
\usepackage{bm}

% Define tensors to be in bold math font.
\newcommand{\tensor}[1]{{\bm{#1}}}

% Override the formatting used by \vec.  Instead of a little arrow
% over the letter, this creates a bold character.
\renewcommand{\vec}{\bm}

% Define unit vector notation.  If you don't override the
% behavior of \vec, you probably want to use the second one.
\newcommand{\unit}[1]{\hat{\bm{#1}}}
% \newcommand{\unit}[1]{\hat{#1}}

% Use this to refer to a single component of a unit vector.
\newcommand{\scalarunit}[1]{\hat{#1}}

% \toprule, \midrule, \bottomrule for tables
\usepackage{booktabs}

% \llbracket, \rrbracket
\usepackage{stmaryrd}

\usepackage{hyperref}
\hypersetup{
    colorlinks,
    citecolor=black,
    filecolor=black,
    linkcolor=black,
    urlcolor=black
}

% Compress lists of citations like [33,34,35,36,37] to [33-37]
\usepackage{cite}

% If you want to relax some of the SAND98-0730 requirements, use the "relax"
% option. It adds spaces and boldface in the table of contents, and does not
% force the page layout sizes.
% e.g. \documentclass[relax,12pt]{SANDreport}
%
% You can also use the "strict" option, which applies even more of the
% SAND98-0730 guidelines. It gets rid of section numbers which are often
% useful; e.g. \documentclass[strict]{SANDreport}

% The INLreport class uses \flushbottom formatting by default (since
% it's intended to be two-sided document).  \flushbottom causes
% additional space to be inserted both before and after paragraphs so
% that no matter how much text is actually available, it fills up the
% page from top to bottom.  My feeling is that \raggedbottom looks much
% better, primarily because most people will view the report
% electronically and not in a two-sided printed format where some argue
% \raggedbottom looks worse.  If we really want to have the original
% behavior, we can comment out this line...
\raggedbottom
\setcounter{secnumdepth}{5} % show 5 levels of subsection
\setcounter{tocdepth}{5} % include 5 levels of subsection in table of contents

% ---------------------------------------------------------------------------- %
%
% Set the title, author, and date
%
    \title{RAVEN User Manual}

    \author{%
      \\ \textit{The RAVEN team:}
       \\
      \\\ \underline{\textit{\textbf{Principal Investigator}}:}
       \\
      \\  \textbf{Cristian Rabiti}, \textit{the Boss}
      \\
      \\\underline{\textit{\textbf{Developers}}:}
       \\
      \\\textbf{Andrea Alfonsi},   \textit{the funny guy}
      \\\textbf{Joshua Cogliati},   \textit{the computer ace}
      \\\textbf{Diego Mandelli},   \textit{born old}
      \\\textbf{Robert Kinoshita}, \textit{he is not Japanese}
    }

    % There is a "Printed" date on the title page of a SAND report, so
    % the generic \date should [WorkingDir:]generally be empty.
    \date{}


% ---------------------------------------------------------------------------- %
% Set some things we need for SAND reports. These are mandatory
%
%\SANDnum{INL/EXT-14-xxxxx}
%\SANDprintDate{August 2014}
%\SANDauthor{ }
%\SANDreleaseType{Draft Release}

% ---------------------------------------------------------------------------- %
% Include the markings required for your SAND report. The default is "Unlimited
% Release". You may have to edit the file included here, or create your own
% (see the examples provided).
%
% \include{MarkOUO} % Not needed for unlimted release reports

\def\component#1{\texttt{#1}}

% ---------------------------------------------------------------------------- %
\newcommand{\systemtau}{\tensor{\tau}_{\!\text{SUPG}}}

% ---------------------------------------------------------------------------- %
%
% Start the document
%
\begin{document}
    \maketitle[Title]

    % ------------------------------------------------------------------------ %
    % An Abstract is required for SAND reports
    %
%    \begin{abstract}
%    \input abstract
%    \end{abstract}


    % ------------------------------------------------------------------------ %
    % An Acknowledgement section is optional but important, if someone made
    % contributions or helped beyond the normal part of a work assignment.
    % Use \section* since we don't want it in the table of context
    %
%    \clearpage
%    \section*{Acknowledgment}
%	Thanks to Ron Weasly for valuable discussions and helping
%	us finding new uses of magic.
%
%	The format of this report is based on information found
%	in~\cite{Sand98-0730}.


    % ------------------------------------------------------------------------ %
    % The table of contents and list of figures and tables
    % Comment out \listoffigures and \listoftables if there are no
    % figures or tables. Make sure this starts on an odd numbered page
    %
    \cleardoublepage		% TOC needs to start on an odd page
    \tableofcontents
    \listoffigures
    \listoftables


    % ---------------------------------------------------------------------- %
    % An optional preface or Foreword
%    \clearpage
%    \section*{Preface}
%    \addcontentsline{toc}{section}{Preface}
%	Although muggles usually have only limited experience with
%	magic, and many even dispute its existence, it is worthwhile
%	to be open minded and explore the possibilities.


    % ---------------------------------------------------------------------- %
    % An optional executive summary
    \clearpage
    \section*{Summary}
    \addcontentsline{toc}{section}{Summary}
    \input{Summary.tex}
%	Once a certain level of mistrust and skepticism has
%	been overcome, magic finds many uses in todays science
%	and engineering. In this report we explain some of the
%	fundamental spells and instruments of magic and wizardry. We
%	then conclude with a few examples on how they can be used
%	in daily activities at national Laboratories.


    % ---------------------------------------------------------------------- %
    % An optional glossary. We don't want it to be numbered
%    \clearpage
%    \section*{Nomenclature}
%    \addcontentsline{toc}{section}{Nomenclature}
%    \begin{description}
%	   \item[alohomoral]
%	    spell to open locked doors and containers
%	   \item[leviosa]
%	    spell to levitate objects
%    \item[remembrall]
%	    device to alert you that you have forgotten something
%    \item[wand]
%	    device to execute spells
%    \end{description}


    % ---------------------------------------------------------------------- %
    % This is where the body of the report begins; usually with an Introduction
    %
    %\SANDmain		% Start the main part of the report

\section{Introduction}

% High-level RAVEN description
RAVEN is a generic software framework to perform parametric and probabilistic
analysis based on the response of complex system codes.
%
The initial development was aimed at providing dynamic risk analysis
capabilities to the thermohydraulic code RELAP-7, currently under development
at Idaho National Laboratory (INL).
%
Although the initial goal has been fully accomplished, RAVEN is now a
multi-purpose probabilistic and uncertainty quantification platform, capable of
agnostically communicating with any system code.

% RAVEN's flexibility
The agnosticism of RAVEN is inherent in its provision of Application Programming
Interfaces (APIs).
%
These APIs allow RAVEN to interact with any code as long as all the parameters
that need to be perturbed are accessible by input files or via python
interfaces.
%
%RAVEN is capable of investigating the system response, and investigating the
%input space using Monte Carlo, Grid, or Latin Hyper Cube sampling schemes, but
%its strength is focused towards system feature discovery, such as limit
%surfaces, separating regions of the input space leading to system failure, and
%using dynamic supervised learning techniques.
RAVEN is capable of investigating system response and input space using various
sampling schemes such as Monte Carlo, grid, or Latin hypercube.
%
However, its strength lies in system feature discovery such as: constructing
limit surfaces, separating regions of the input space leading to system failure,
and using dynamic supervised learning techniques.

% Historical Context
The development of RAVEN started in 2012, when, within the Nuclear Energy
Advanced Modeling and Simulation (NEAMS) program, the need to provide a modern
risk evaluation framework arose.
%
RAVEN's principal assignment is to provide the necessary software and algorithms
in order to employ the concepts developed by the Risk Informed Safety Margin
Characterization (RISMC) program.
%
RISMC is one of the pathways defined within the Light Water Reactor
Sustainability (LWRS) program.

% RISMC
In the RISMC approach, the goal is not just to identify the frequency of an
event potentially leading to a system failure, but the proximity (or lack
thereof) to key safety-related events.
%
Hence, the approach is interested in identifying and increasing the safety
margins related to those events.
%
A safety margin is a numerical value quantifying the probability that a safety
metric (e.g. peak pressure in a pipe) is exceeded under certain conditions.

% Conclusion
The initial development of RAVEN has been focused on providing dynamic risk
assessment capability to RELAP-7, currently under development at INL.
%
Most of the capabilities, implemented having RELAP-7 as a principal focus, are
easily deployable to other system codes.
%
For this reason, several side activates have been employed (e.g.  RELAP5-3D, any MOOSE-based App, etc.)
or are currently ongoing for coupling RAVEN with several different software.
%
The aim of this document is to detail the input requirements for RAVEN focusing
on the input structure.

\section{RAVEN Concepts}
\label{sec:RAVENconcept}
After the brief overview of the RAVEN code, it is necessary to focus on the main concepts that are behind the design of the framework, overall focalizing to the following key subjects:
\begin{itemize}
    \item \textit{Mathematical Background}: Section ~\ref{sub:mathBackground} provides an description of the mathematical background of RAVEN, 
     overall focalizing on the probabilistic dynamics;
    \item \textit{RAVEN entities}: Section ~\ref{sub:EntitiesAndFlow} is aimed to provide an overview on how the different objects in
    RAVEN can interact with each other, generating the user-dependent analysis flow;
    \item \textit{RAVEN input main components}: Section ~\ref{sub:InputStructure} provides a brief introduction of the input structure, introducing
    some of the input ``structure'' that are going to be used in this manual. A detailed explanation of the
    input structure and keywords is reported in the user manual ~\cite{RAVENuserManual}.
\end{itemize}
\subsection{RAVEN mathematical background}
\label{sub:mathBackground}
\subsubsection{System and Control model}
\label{subsub:controlAndSystem}
The first step is the derivation of the mathematical model representing, with a high 
level of abstraction, the plant and control system model. Let $\overline{\theta}\left ( t 
\right )$ be a vector describing the system status in the phase space, characterized 
by the following governing equation:
\begin{equation}
\label{eq:dThetaOverDT}
\frac{\partial \overline{\theta} }{\partial t}=\overline{H}\left (  \overline{\theta}\left ( t \right ),t \right )
\end{equation}
In the equation above, the assumption of time differentiability of the trajectory equation $\overline{H}\left (  \overline{\theta}\left ( t \right ),t \right )$ in the phase space has been taken. This assumption is not fully correct and generally required and it is used here, without missing of generality, for compactness of the notation.
\\It can now be performed an arbitrary decomposition of the phase space:
\begin{equation}
\label{eq:thetaDecomposition}
  \overline{\theta} = \left (\frac{\overline{x}}{\overline{v}}  \right )
\end{equation}
The decomposition is made in such a way that $\overline{x}$ represent the unknowns 
solved by a system code (such as RELAP5-3D~\cite{relap5}, 
RELAP7~\cite{relap7FY12}, etc.) while $\overline{v}$ are the variables directly 
controlled by the control system (e.g. automatic mitigation systems, operator actions, 
etc.).
\\The governing equation can be now casted in the following system of equations:
\begin{equation}
\label{eq:governingEquations}
\left\{\begin{matrix}
\frac{\partial \overline{x} }{\partial t} = \overline{F}\left (  \overline{x}, \overline{v}, t \right )  \\ 
\frac{\partial \overline{v} }{\partial t} = \overline{V}\left (  \overline{x}, \overline{v}, t \right )
\end{matrix}\right.
\end{equation}
Consequentially to this splitting, $\overline{x}$ contains the state variables of the 
phase space that are continuous while $\overline{v}$ contains discrete state 
variables that are usually handled by the control system (consequentially, named 
\textbf{``control variables''}). It can be noticed that the
function  $ \overline{V}\left (  \overline{x}, \overline{v}, t \right )$, representing the 
control system, does not depend on the  knowledge of the complete status of the 
system but on a restricted subset that can be named \textbf{``monitored variables''} $\overline{C}$:

\begin{equation}
\label{eq:controlVars}
\left\{\begin{matrix}
\frac{\partial \overline{x} }{\partial t} = \overline{F}\left (  \overline{x}, \overline{v}, t \right )  \\ 
 \overline{C} =  \overline{G}(\overline{x},t)     \\
\frac{\partial \overline{v} }{\partial t} = \overline{V}\left (  \overline{x}, \overline{v}, t \right )
\end{matrix}\right.
\end{equation}
where $\overline{C}$ is a vector of smaller dimensionality than $\overline{x}$ and,
therefore, more convenient to handle.
\\As it can be noticed, the standard nomenclature of \textit{signals} (monitored variables) and \textit{status} (control variables) is not adopted . Two principal reasons 
justify this decision:
\begin{itemize}
  \item the definition of signals is tight to the definition of the 
  control logic for each component and, therefore, relative rather than absolute in the 
  overall system analysis. For example, it is possible the the \textit{signals} for a 
  component represent \textit{status} of another one, determining an in-unique 
  definition;
  \item the standard nomenclature becomes meaningless when this derivation is 
  applied to Uncertainty Quantification (UQ).
\end{itemize}

\paragraph{ Splitting Approach for the Simulation of the Control System}
Equation~\ref{eq:controlVars} represents a fully coupled system of Partial Differential 
Equations(PDEs). In order to solve this system, an  \textit{operator splitting} approach 
is employed. This method is preferable in this context for several reasons, among which the following:
\begin{itemize}
  \item in reality, the control system (automatic mitigation systems, operator actions, 
  etc.) is always characterized by an intrinsic delay;
  \item the reaction of the control system might make the system ``move'' among 
  different discrete states and, therefore, numerical errors will be always of first order
  unless the discontinuity is explicitly treated.
\end{itemize}
Employing the \textit{operator splitting} approach, Eq.~\ref{eq:controlVars}  can be 
casted as follows: 
\begin{equation}
\label{eq:operatorSplitting}
\left\{\begin{matrix}
\frac{\partial \overline{x} }{\partial t} = \overline{F}\left (  \overline{x},\overline{v}_{t_{i-1}}, t \right )  & \\ 
\overline{C} =  \overline{G}(\overline{x},t)  & t_{i-1} \leq t \leq  t_{i} =  t_{i-1} + \Delta  t_{i}\\ 
\frac{\partial \overline{v} }{\partial t} = \overline{V}\left (  \overline{x}, \overline{v}_{t_{i-1}}, t \right ) & 
\end{matrix}\right.
\end{equation}
Hence, the system of equations in solved decomposing it into simpler sub-problems that are treated individually 
using specialized numerical algorithms.

\paragraph{Definition of the monitored variable space}
The contraction of the information from the $\overline{x}$ space to the $\overline{C}$ space is a crucial step. 
Since $\overline{C}$ represents an arbitrary middle step, it is needed to define a set of rules that make this 
choice unique. $\overline{C}$ is chosen such that:
\begin{itemize}
  \item the solution of  $ \left.\begin{matrix} \frac{\partial \overline{v} }{\partial t} 
  \end{matrix}\right| =\overline{V}\left (  \overline{x},\overline{v}_{t_{i-1}}, t \right )$
  can be carried along without any knowledge of the solution algorithm of
   $ \left.\begin{matrix} 
  \frac{\partial \overline{x} }{\partial t} =  \end{matrix}\right| \overline{F}\left (  
  \overline{x},\overline{v}_{t_{i-1}}, t \right )  
  $. This requirement determines the minimum information contraction from  $\overline{x}$ to  
  $\overline{C}$.
  \item all the actions represented by $\overline{C} = \overline{G}(\overline{x},t)$ require knowledge of the 
  solution algorithm of 
  $ \left.\begin{matrix} 
  \frac{\partial \overline{x} }{\partial t} =  \end{matrix}\right| \overline{F}\left (  
  \overline{x},\overline{v}_{t_{i-1}}, t \right )  $. This requirement determines  the maximum  information contraction from  $\overline{x}$ to  $\overline{C}$.
\end{itemize}
The intersection of the two sub-spaces defined above create a minimal unique set.
\paragraph{Definition of the auxiliary variable space}
In the previous sections, it has been determined that the needed information to model the dynamic system
is contained in the solution vectors $\overline{x}$ and $\overline{v}$. Even if $\overline{x}$ and $\overline{v}$ 
are sufficient to assess the system status at every point in time, it can result in an unpractical way to model
the eventual control system:
Let's suppose to model a component of a particular system that presents different behavior depending on 
other systems or operation behaviors. In order to define the status of this component in every point in time, the 
whole history of the system needs to be tracked. In order to remove these inefficiency, a set of auxiliary variables 
$\overline{a}$ can be introduced. These variables are the ones that in the analysis of stochastic dynamics 
are artificially added into the phase space to a non-Markovian system to obtain back a Markovian behavior. In this
way only the previous time-step information is needed to determine the status of the system. 
\\ Adding this additional system of variables, Eq.~\ref{eq:operatorSplitting} can be casted as follows:

\begin{equation}
\label{eq:auxiliaryVariables}
\left\{\begin{matrix}
\frac{\partial \overline{x} }{\partial t} = \overline{F}\left (  \overline{x},\overline{v}_{t_{i-1}}, t \right )  & \\ 
\overline{C} =  \overline{G}(\overline{x},t)  & t_{i-1} \leq t \leq  t_{i} =  t_{i-1} + \Delta  t_{i}\: \\ 
\frac{\partial \overline{a} }{\partial t} = \overline{A}\left (  \overline{x},\overline{C},\overline{a},\overline{v}_{t_{i-1}}, t \right ) \\
\frac{\partial \overline{v} }{\partial t} = \overline{V}\left (  \overline{C},\overline{a}, \overline{v}_{t_{i-1}}, t \right )  & 
\end{matrix}\right.
\end{equation}

\subsubsection{Dynamic Systems stochastic modeling}
%
\paragraph{General system of equations and variable classification}
In section ~\ref{subsub:controlAndSystem} the derivation of the governing equations for a controllable system
have been reported. In this section, the mathematical framework of the modeling of dynamic stochastic systems,
under uncertainties,  is derived. 
\\ Dynamic stochastic systems are the ones whose dynamic is characterized by intrinsic randomness. Random 
behaviors, although present in nature, are often artificially introduced into physical models to account for 
incapability of fully modeling part of the nature of the system behavior and/or of the phenomena bounding the
physical problem.
\\The distinction between variables that are artificially considered aleatory and the ones intrinsically aleatory 
corresponds with the classical definition of epistemic (artificial) and aleatory (intrinsic) uncertainties. From a 
system simulation point of view it is more relevant how these variables, the sources of aleatory behavior, change 
in time.
Possible examples of random elements are:
\begin{itemize}
 \item random variability of parameters (e.g. uncertainty in physical parameters);
 \item presence of noise (background noise due to intrinsically stochastic behaviors or lack of detail in the 
 simulation);
 \item Uncertainty in the initial and boundary conditions;
 \item Random failure of components;
 \item aging effects.
\end{itemize}
Before introducing the mathematical models for uncertainty,  it can beneficial to recall 
Eq.~\ref{eq:dThetaOverDT}, adding the initial conditions:
\begin{equation}
\label{eq:dThetaOverDTWithBoundary}
\left\{\begin{matrix}
\frac{\partial  \overline{\theta}\left ( t \right )}{\partial t}=\overline{H}\left (  \overline{\theta}\left ( t \right ),t \right ) \\ 
 \overline{\theta}\left ( t_{0} \right ) = \overline{\theta}_{0}
\end{matrix}\right.
\end{equation}
At this point, each source of uncertainty or stochastic behavior is considered and progressively added in 
Eq.~\ref{eq:dThetaOverDTWithBoundary}.
For the scope of this derivation, it is convenient to split the phase space into \textit{continuous} (e.g. temperature,
pressure, hentalpy, etc.) and discrete (e.g. status of components, such as operational and failure states) variables
as follows: 
\begin{itemize}
 \item $ \overline{\theta}^{c} \in \Phi \subseteq \mathbb{R}^{C}$, the set of continuous variables;
 \item $ \overline{\theta}^{d} \in \Psi \subseteq \mathbb{N}^{D}$, the set of discrete variables.
 \item $\overline{\theta}(t) = \overline{\theta}^{c} \oplus \overline{\theta}^{d}$
\end{itemize}
Consequentially, Eq.~\ref{eq:dThetaOverDTWithBoundary} assumes the following form:
\begin{equation}
\label{eq:systemThetaContAndDescrete}
\left\{\begin{matrix}
\frac{\partial  \overline{\theta}^{c}\left ( t \right )}{\partial t}=f\left ( \overline{\theta}^{c},\overline{\theta}^{d},t \right ) \\ 
\frac{\partial  \overline{\theta}^{d}\left ( t \right )}{\partial t}=g\left ( \overline{\theta}^{c},\overline{\theta}^{d},t \right )\\
 \overline{\theta}^{c}\left ( t_{0} \right ) = \overline{\theta}^{c}_{0}\\
 \overline{\theta}^{d}\left ( t_{0} \right ) = \overline{\theta}^{d}_{0}
\end{matrix}\right.
\end{equation}
As it can be noticed, the time derivative operator has been also used for the time discontinuous variables, even
if this is allowed only introducing complex extension of the time derivative operator. In this context, the $\frac{\partial  }{\partial t}$ on the discontinuous space is employed for simplifying the notation only.

\paragraph{Probabilistic Nature of the Parameters Characterizing the Equation}
The first stochastic behaviors to be introduced are the uncertainties associated with the:
\begin{itemize}
  \item initial conditions (i.e. $\overline{\theta}^{c}$ and $\overline{\theta}^{d}$ at time $t_{0}$), and
  \item parameters characteristic of  $f\left ( \overline{\theta}^{c},\overline{\theta}^{d},t \right )$ and $g\left ( \overline{\theta}^{c},\overline{\theta}^{d},t \right )$.
\end{itemize}
as shown in Eq.~\ref{eq:systemThetaContAndDescreteStaz}.
\begin{equation}
\label{eq:systemThetaContAndDescreteStaz}
\left\{\begin{matrix}
\frac{\partial  \overline{\theta}^{c}\left ( t \right )}{\partial t}=f\left ( \overline{\theta}^{c},\overline{\theta}^{d}, \overline{\alpha}_{staz} ,      t \right ) \\ 
\frac{\partial  \overline{\theta}^{d}\left ( t \right )}{\partial t}=g\left ( \overline{\theta}^{c},\overline{\theta}^{d},\overline{\alpha}_{staz},t \right )\\
\Pi \left ( \overline{\theta}^{c},t_{0} \right ) \sim pdf\left ( \overline{\theta}^{c}_{0}|,\sigma_{c}^{2} \right )\\ 
\Pi \left ( \overline{\theta}^{d},t_{0} \right ) \sim pdf\left ( \overline{\theta}^{d}_{0}|,\sigma_{d}^{2} \right ) \\
\overline{\alpha}_{staz}\left ( t \right )=\overline{\alpha}_{staz}\left ( t_{0} \right ) \sim pdf\left ( \overline{\alpha}_{staz}^{0}|, \sigma_{staz}^{2} \right )
\end{matrix}\right.
\end{equation}
In Eq.~\ref{eq:systemThetaContAndDescreteStaz}, $\Pi \left ( \overline{\theta}^{c},t_{0} \right )$ indicates the 
probability distribution of $\overline{\theta}^{c}$ at the initial time $t=t_{0}$ while
$pdf\left ( \mu|, \sigma^{2} \right )$ represents a generic probability distribution function having mean value 
$\mu$ and sigma $\sigma$.The term $\overline{\alpha}_{staz}$ is the vector of parameters affected by 
uncertainty but not varying over time.
\\As already mentioned, Eq.~\ref{eq:systemThetaContAndDescreteStaz} considers uncertainties whose values
do not change during the dynamic evolution of the system. This set of uncertainties accounts for most of the 
common source of aleatory behaviors. Examples of this kind of uncertainties are:
\begin{itemize}
  \item \textit{Uncertainty associated with the heat conduction coefficient}:  This value is known (but uncertain) and has no physical reason to change during the simulation;
  \item \textit{Uncertainty on failure temperature for a pipe}: This value is usually characterized by a probability distribution function but once the value has been set (like through random sampling) it will not change during the simulation.
\end{itemize}
From a modeling perspective, all the probabilistic behaviors connected to $\Pi \left ( \overline{\theta}^{c},t_{0} 
\right ) $, $\Pi \left ( \overline{\theta}^{d},t_{0} \right )$ and $\overline{\alpha}_{staz}(t)$ can be modeled without 
changing the dimensionality of the phase space (hence, no alteration of the solution algorithm is required), simply performing sampling of the input space. In addition, the Markovian assumption is still preserved.

\paragraph{Variables subject to Random Motion}
The next aleatory component to be accounted for is the set of parameters that continuously change over time (i.e. $\overline{\alpha}_{brow}$). 
To make an easy parallel,  these parameters are referred as if they behave like a \textit{Brownian motion}. 
While what commonly is indicated as \textit{Brownian motion} has not impact at the character
the space and time scales (characteristic of a physical system), there are parameters that have (or \textbf{appear} 
to have) such behavior. The  \textit{Brownian motion} characteristic of some variables can be completely 
synthetic, due to the lack of modeling details in the simulation model.
\\For instance, two examples of these randomly varying variables are:
\begin{itemize}
  \item \textit{Cumulative damage growth in material}: Experimental data and models representing this
  phenomenon show large uncertainties. There is also an intrinsic natural stochasticity driving
  the accumulation of the damage (natural Brownian motion);
  \item \textit{Heat conductivity in the fuel gap during heating of fuel}: During some transients there are
  situations where the fuel is in contact with the clad while in others where there is the presence of a gap. While in   
  nature this is a discontinuous transition, it is not usually possible to model in such a way, especially if vibrations 
  of the fuel lead to high frequency oscillations. In this case, it would be helpful to introduce directly into the 
  simulation a random noise characterizing the thermal conductivity when these transitions occur (synthetic 
  Brownian motion).
\end{itemize}
The system of equations~\ref{eq:systemThetaContAndDescreteStaz} can be rewritten in the following form:

\begin{equation}
\label{eq:systemThetaContAndDescreteStazAndBrow}
\left\{\begin{matrix}
\frac{\partial  \overline{\theta}^{c}\left ( t \right )}{\partial t}=f\left ( \overline{\theta}^{c},\overline{\theta}^{d}, \overline{\alpha}_{staz} ,\overline{\alpha}_{brow},      t \right ) \\ 
\frac{\partial  \overline{\theta}^{d}\left ( t \right )}{\partial t}=g\left ( \overline{\theta}^{c},\overline{\theta}^{d},\overline{\alpha}_{staz},\overline{\alpha}_{brow},t \right )\\
\frac{\partial \overline{\alpha}_{brow} }{\partial t}=b\left ( \overline{\theta}^{c},\overline{\theta}^{d},\overline{\alpha}_{staz},\overline{\alpha}_{brow},t \right )\Gamma \left ( t \right )
\\
\Pi \left ( \overline{\theta}^{c},t_{0} \right ) \sim pdf\left ( \overline{\theta}^{c}_{0}|,\sigma_{c}^{2} \right )\\ 
\Pi \left ( \overline{\theta}^{d},t_{0} \right ) \sim pdf\left ( \overline{\theta}^{d}_{0}|,\sigma_{d}^{2} \right ) \\
\overline{\alpha}_{staz}\left ( t \right )=\overline{\alpha}_{staz}\left ( t_{0} \right ) \sim pdf\left ( \overline{\alpha}_{staz}^{0}|, \sigma_{staz}^{2} \right ) \\
\overline{\alpha}_{brow}\left ( t_{0} \right ) \sim  \overline{\alpha}_{brow}^{0} \Gamma \left ( t_{0} \right )
\end{matrix}\right.
\end{equation}
where $\Gamma \left ( t \right )$ is 0-mean random noise and $\overline{\alpha}_{brow}$ is the set of parameters subject to \textit{Brownian motion}.
\\Clearly, the equation referring to the time change of the parameters subject to the \textit{Brownian motion} should be interpreted in the \textbf{Ito} sense [C. Gardiner, Stochastic Methods, Springer (2009)].

\paragraph{Discontinuously and Stochastically varying variables}
The last and probably most difficult step is the introduction of parameters that are neither constant during the simulation nor continuously vary over time. As an example, consider a valve that, provided set of operating conditions, opens or closes. If this set of conditions is reached n times during the simulation, the probability of the valve correctly operating should be sampled n times. It is also foreseeable that the history of failure/success of the valve will impact future probability of failure/success.  In this case the time evolution of such parameters (discontinuously stochastic changing parameters  $\overline{\alpha}_{DS}$) is governed by the following equation.

\begin{equation}
\label{eq:systemDiscAndStochVaryVars}
\frac{\partial  \overline{\alpha }_{DS}\left ( t \right )}{\partial t}=  \overline{\delta}\left ( \overline{\alpha }_{DS}, \overline{\theta}^{c},\overline{\theta}^{d},\overline{\alpha}_{staz},\overline{\alpha}_{brow},t \right ) \times \overline{V}\left ( \overline{\alpha }_{DS}, \overline{\theta}^{c},\overline{\theta}^{d},\overline{\alpha}_{staz},\overline{\alpha}_{brow},t \right ) \times \overline{p}\left ( \int_{t_{0}}^{t}  S\left ( \overline{\theta}\left ( t^{'} \right ),t^{'} \right )dt^{'} \right )
\end{equation}
where:
\begin{itemize}
  \item The function $\overline{\delta}\left ( \overline{\alpha }_{DS}, \overline{\theta}^{c},\overline{\theta}^{d},
  \overline{\alpha}_{staz},\overline{\alpha}_{brow},t \right )$ is the delta of Dirac of the instant on which the 
  transition need to be evaluated (control logic signaling to the valve to open/close);
  \item The term $\overline{p}\left ( \int_{t_{0}}^{t}  S\left ( \overline{\theta}\left ( t^{'} \right ),t^{'} \right )\right ) 
  = \overline{p}\left ( \int_{t_{0}}^{t}  \overline{\alpha }_{DS}, \overline{\theta}^{c},\overline{\theta}^{d},
  \overline{\alpha}_{staz},\overline{\alpha}_{brow},t dt\right )$ represents the transition probability
  between different states (in case of the valve: open/close). Note that the time integral of the
  parameter history accounts for the memory of the component via the kernel $S\left ( \overline{\theta}\left ( t^{'} 
  \right ),t^{'} \right )$;
  \item The term $\overline{V}\left ( \overline{\alpha }_{DS}, \overline{\theta}^{c},\overline{\theta}^{d},
  \overline{\alpha}_{staz},\overline{\alpha}_{brow},t \right )$ is the rate of change of $\overline{\alpha }_{DS}$. 
  For a discrete parameter, it is defined as the value of the instantaneous $\overline{\alpha }_{DS}$ change
\end{itemize}
The introduction of the history dependency introduced in the term $\overline{p}$ determines that the system can
not be considered Markovian, if ``countermeasures'' are not taken. In order to make the system return to be Markovian, the phase space needs to be expanded (i.e. increase its dimensionality): the time at which the 
parameters changed status and their corresponding values $\left \{  \left (\overline{\alpha}_{DS}, t \right )_{i} \right \} = \left \{  \overline{\alpha}_{DS}, t_{i} \right \} = \overline{\overline{\alpha}}_{DS}, \overline{t}\, \left ( for\, i=1,...,n \right )$.
\\Equation~\ref{eq:systemDiscAndStochVaryVars} now assumes the form:
\begin{equation}
\label{eq:systemDiscAndStochVaryVarsExpanded}
\begin{matrix}
\frac{\partial  \overline{\alpha }_{DS}\left ( t \right )}{\partial t}=  \overline{\delta}\left ( \overline{\alpha }_{DS}, \overline{\theta}^{c},\overline{\theta}^{d},\overline{\alpha}_{staz},\overline{\alpha}_{brow},t \right ) \times \overline{V}\left ( \overline{\alpha }_{DS}, \overline{\theta}^{c},\overline{\theta}^{d},\overline{\alpha}_{staz},\overline{\alpha}_{brow},t \right ) \times \overline{p}\left ( \overline{\overline{\alpha}}_{DS},\overline{t},\overline{\theta}^{c},\overline{\theta}^{d},\overline{\alpha}_{staz},\overline{\alpha}_{brow},t  \right ) \\ \! \! \! \! \! \! \! \! \! \! \! \! \! \! \! \! \! \! \! \! \! \! \! \! \! \! \! \! \! \! \! \! \! \! \! \! \! \! \! \! \! \! \! \! \! \! \! \! \! \! \! \! \! \! \! \! \! \! \! \! \! \! \! \! \! \! \! \! \! \! \! \! \! \! \! \! \! \! \! \! \! \! \! \! \! \! \! \! \! \! \! \! \! \! \! \! \! \! \! \! \! \! \! \! \! \! \! \! \! \! \! \! \! \! \! \! \! \! \! \! \! \! \! \! \! \! \! \! \! \! \! \! \! \! \! \! \! \! \! \! \! \! \! \! \! \! \! \! \! \! \! \! \! \! \! \! \! \! \! \! \! \! \! \! \! \! \! \!  
for \, t\geq t_{n}
\end{matrix}
\end{equation}
This formulation introduces a phase space that is continuously growing over time $n \rightarrow \infty$. In this respect, it is useful to introduce and discuss possible assumptions:
\begin{enumerate}
  \item The memory of the past is not affected by the time distance; in this case:
  \begin{equation}
   \overline{p}\left ( \overline{\overline{\alpha}}_{DS},\overline{t},\overline{\theta}^{c},\overline{\theta}^{d},\overline{\alpha}_{staz},\overline{\alpha}_{brow},t  \right ) =  \overline{p}\left ( \overline{\overline{\alpha}}_{DS},\overline{\theta}^{c},\overline{\theta}^{d},\overline{\alpha}_{staz},\overline{\alpha}_{brow},t  \right )
  \end{equation}
  The dimensionality of the phase space is still growing during the simulation since more and more sampling is 
  performed, but the time integral is removed from the transition probability. A simple example of this situation is 
  a component activated on demand in which failure is a function of all previous sampling, but not of when the 
  component was sampled or in which sequence the outcome occurred.
  \item  The number of samples is determined before the simulation itself takes place (e.g. $n$ times) In this case 
  the different $\overline{\alpha}_{DS_{i}}$ could be treated explicitly as $\overline{\alpha}_{staz}$   while 
  $\overline{t}$ would still remain a variable to be added to the phase space (if simplification 1 is not valid) but of 
  fixed dimension. In this case $\overline{t}$ still needs to be computed and its expression is:
  \begin{equation}
   \overline{t} \left ( t \right ) = \int_{t_{0}}^{t} \overline{t}  \, \overline{\delta }\left ( \overline{\alpha }_{DS}, 
   \overline{\theta}^{c},\overline{\theta}^{d},\overline{\alpha}_{staz},\overline{\alpha}_{brow},t \right )  dt
  \end{equation}
  The transition probability becomes:
  \begin{equation}
     \overline{p}\left ( \int_{t_{0}}^{t} dt\, S\left ( \overline{t} \right ), \overline{\alpha}_{DS}, \overline{\theta}^{c},
     \overline{\theta}^{d},\overline{\alpha}_{staz},\overline{\alpha}_{brow},t \right )
  \end{equation}
  This is the case, for example, of a component that is sampled a fixed number of times for a given simulation 
  while the contribution of the history to the transition probability might decay exponentially over time. This 
  approximation might eliminate the memory from the system by adding n variables to the phase space $t_{i} \, \, 
  (for \, \, i=1,...,n)$ thus restoring the Markovian characteristic.
  \item Another possible approximation alternative to the previous one is that the memory of the system (here 
  explicitly represented by $ \int_{t_{0}}^{t}  \overline{\alpha }_{DS} dt$) is limited 
  only to a fixed number of steps back in the past. In this case $n$ is always bounded. Therefore adding  $\left \{  
  \overline{\alpha}_{DS_{i}},t_{i} \right \}, \left ( for\: i=1,...,n \right )$ would possibly preserve the system 
  Markovian properties of the system. This approximation allows for eliminating the memory from the system by 
  expanding the phase space $2n$ variables. From a software implementation point of view, this is the most 
  complex  situation since without any simplification we would have to deal with a system that is never reducible 
  to a Markovian one and therefore forced to use the whole history of the system to forecast its evolution at each 
  time step.
\end{enumerate}
Assumption 1 limits this cost by restraining it to the set of values assumed by the 
variable but would still lead to very difficult to deal with situation. Assumption 2 would 
require an expansion of phase space to introduce the time at which the transitions 
happens but the value that the parameter will assume at each sampling could be 
treated as initial condition. Assumption 3 would instead require the expansion of the 
phase space for both the time and the values of the 
transitioning variables.
\\Based on the this simplifications, the system of 
equations~\ref{eq:systemThetaContAndDescreteStazAndBrow}, accounting also for $ \overline{\alpha}_{DS}$ can be casted into the form:
\begin{equation}
\label{eq:fullSystem}
\begin{split}
\left\{\begin{matrix}
\frac{\partial  \overline{\theta}^{c}\left ( t \right )}{\partial t}=f\left ( \overline{\theta}^{c},\overline{\theta}^{d}, \overline{\alpha}_{staz} ,\overline{\alpha}_{brow},      t \right ) \\ 
\frac{\partial  \overline{\theta}^{d}\left ( t \right )}{\partial t}=g\left ( \overline{\theta}^{c},\overline{\theta}^{d},\overline{\alpha}_{staz},\overline{\alpha}_{brow},t \right )\\
\frac{\partial \overline{\alpha}_{brow} }{\partial t}=b\left ( \overline{\theta}^{c},\overline{\theta}^{d},\overline{\alpha}_{staz},\overline{\alpha}_{brow},t \right )\Gamma \left ( t \right ) \\
\frac{\partial  \overline{\alpha }_{DS}\left ( t \right )}{\partial t}=  \overline{\delta}\left ( \overline{\alpha }_{DS}, \overline{\theta}^{c},\overline{\theta}^{d},\overline{\alpha}_{staz},\overline{\alpha}_{brow},t \right ) \times \overline{V}\left ( \overline{\alpha }_{DS}, \overline{\theta}^{c},\overline{\theta}^{d},\overline{\alpha}_{staz},\overline{\alpha}_{brow},t \right ) \times 
\\ \times  \overline{p}\left ( \int_{t_{0}}^{t}  dt\;   \overline{\alpha }_{DS}, \overline{\theta}^{c},\overline{\theta}^{d},
  \overline{ \alpha}_{staz},\overline{\alpha}_{brow},t \right )
\\
\Pi \left ( \overline{\theta}^{c},t_{0} \right ) \sim pdf\left ( \overline{\theta}^{c}_{0}|,\sigma_{c}^{2} \right )\\ 
\Pi \left ( \overline{\theta}^{d},t_{0} \right ) \sim pdf\left ( \overline{\theta}^{d}_{0}|,\sigma_{d}^{2} \right ) \\
\overline{\alpha}_{staz}\left ( t \right )=\overline{\alpha}_{staz}\left ( t_{0} \right ) \sim pdf\left ( \overline{\alpha}_{staz}^{0}|, \sigma_{staz}^{2} \right ) \\
\overline{\alpha}_{brow}\left ( t_{0} \right ) \sim  \overline{\alpha}_{brow}^{0} \Gamma \left ( t_{0} \right ) \\
\overline{\alpha}_{DS} \left ( t_{0} \right ) = \overline{\alpha}_{DS} ^{0}
\end{matrix}\right.
\end{split}
\end{equation}
Introducing the simplifications \textbf{1} and \textbf{3} ( the most 
appropriated in this context), Eq.~\ref{eq:fullSystem} becomes:
\begin{equation}
\label{eq:fullSystemApprox1-3}
\begin{split}
\left\{\begin{matrix}
\frac{\partial  \overline{\theta}^{c}\left ( t \right )}{\partial t}=f\left ( \overline{\theta}^{c},\overline{\theta}^{d}, \overline{\alpha}_{staz} ,\overline{\alpha}_{brow},      t \right ) \\ 
\frac{\partial  \overline{\theta}^{d}\left ( t \right )}{\partial t}=g\left ( \overline{\theta}^{c},\overline{\theta}^{d},\overline{\alpha}_{staz},\overline{\alpha}_{brow},t \right )\\
\frac{\partial \overline{\alpha}_{brow} }{\partial t}=b\left ( \overline{\theta}^{c},\overline{\theta}^{d},\overline{\alpha}_{staz},\overline{\alpha}_{brow},t \right )\Gamma \left ( t \right ) \\
\frac{\partial  \overline{\alpha }_{DS}\left ( t \right )}{\partial t}=  \overline{\delta}\left ( \overline{\alpha }_{DS}, \overline{\theta}^{c},\overline{\theta}^{d},\overline{\alpha}_{staz},\overline{\alpha}_{brow},t \right ) \times \overline{V}\left ( \overline{\alpha }_{DS}, \overline{\theta}^{c},\overline{\theta}^{d},\overline{\alpha}_{staz},\overline{\alpha}_{brow},t \right ) \times 
\\ \times  \overline{p}\left ( \overline{\alpha }_{DS}, \overline{\theta}^{c},\overline{\theta}^{d},\overline{\alpha}_{staz},\overline{\alpha}_{brow},t \right )
\\
\Pi \left ( \overline{\theta}^{c},t_{0} \right ) \sim pdf\left ( \overline{\theta}^{c}_{0}|,\sigma_{c}^{2} \right )\\ 
\Pi \left ( \overline{\theta}^{d},t_{0} \right ) \sim pdf\left ( \overline{\theta}^{d}_{0}|,\sigma_{d}^{2} \right ) \\
\overline{\alpha}_{staz}\left ( t \right )=\overline{\alpha}_{staz}\left ( t_{0} \right ) \sim pdf\left ( \overline{\alpha}_{staz}^{0}|, \sigma_{staz}^{2} \right ) \\
\overline{\alpha}_{brow}\left ( t_{0} \right ) \sim  \overline{\alpha}_{brow}^{0} \Gamma \left ( t_{0} \right ) \\
\overline{\alpha}_{DS} \left ( t_{0} \right ) = \overline{\alpha}_{DS} ^{0}
\end{matrix}\right.
\end{split}
\end{equation}
This dissertation does not cover all the possible phenomena, but it 
provides a sufficient mathematical framework for extrapolating toward cases that are not explicitly treated. 
\\ Given the presence of all these sources of stochastic behaviors, every 
exploration of the uncertainties (through sampling strategies) only 
represents a possible trajectory of the system in the phase space. Hence, 
it is much more informative the assessment of the probability of a 
particular response, rather than the response itself.
\\The explanation of these concepts is demanded to next section.
%
%
% Formulation of the equation set in a statistical framework
%
%
\subsubsection{Formulation of the equation set in a statistical framework}
Based on the premises reported in the previous sections and assuming 
that at least one of the simplifications mentioned in 
section~\ref{sub:discAndStochVars} is applicable (i.e. the system can be
casted as Markovian), it is needed to investigate the system evolution 
in terms of its probability density function in the global phase space 
$\overline{\theta}$ via the Chapman-Kolmogorov 
equation~\cite{ProbReactoDynamicsDevooght}.
\\The integral form of the Chapman-Kolmogorov is the following: 
\begin{equation}
\label{eq:chapKolmogIntegralForm}
\begin{matrix}
\Pi \left (\overline{\theta}_{3},t_{3}|\overline{\theta}_{1},t_{1}  \right ) = \int 
d\overline{\theta}_{2} \Pi\left (\overline{\theta}_{2},t_{2}|
\overline{\theta}_{1},t_{1}  \right )   \Pi\left (\overline{\theta}_{3},t_{3}|
\overline{\theta}_{2},t_{2}  \right )   & 
where \: \:   t_{1} < t_{2} < t_{3}
\end{matrix}
\end{equation}
while its differential form is:
\begin{equation}
\label{eq:chapKolmogDiffForm}
\frac{\partial \Pi \left (\overline{\theta},t|\overline{\theta}_{0},t_{0}  \right )  }{\partial t} =
\mathcal{L}_{CK}\left (   \Pi \left (\overline{\theta},t|\overline{\theta}_{0},t_{0}  \right ) \right )
\end{equation}
The transition from the integral to the differential form is possible under the following assumptions: 
\begin{equation}
\label{eq:chapKolmogAssump1}
\lim_{\Delta t \to 0} \frac{1}{\Delta t}  \int_{|
\overline{\theta}_{2}-\overline{\theta}_{1}|<\varepsilon }   \Pi \left 
(\overline{\theta}_{2},t+\Delta t|\overline{\theta}_{1},t  \right ) 
d\overline{\theta}_{2} = 0
\end{equation}

\begin{equation}
\label{eq:chapKolmogAssump2}
\lim_{\Delta t \to 0} \frac{1}{\Delta t} \Pi \left (\overline{\theta}_{2},t+\Delta t|
\overline{\theta}_{1},t  \right ) = W\left ( \overline{\theta}_{2}|
\overline{\theta}_{1},t \right )
\end{equation}

\begin{equation}
\label{eq:chapKolmogAssump3}
\lim_{\Delta t \to 0} \frac{1}{\Delta t}  \int_{|
\overline{\theta}_{2}-\overline{\theta}_{1}|<\varepsilon }   
\left ( \overline{\theta}_{2,i} - \overline{\theta}_{1,i} \right )
\Pi \left (\overline{\theta}_{2},t+\Delta t|\overline{\theta}_{1},t  \right ) 
d\overline{\theta}_{2} = A_{i}\left ( \overline{\theta}_{1},t \right ) + 
\mathcal{O}\left ( \varepsilon \right )
\end{equation}

\begin{equation}
\label{eq:chapKolmogAssump4}
\lim_{\Delta t \to 0} \frac{1}{\Delta t}  \int_{|
\overline{\theta}_{2}-\overline{\theta}_{1}|<\varepsilon }   
\left ( \overline{\theta}_{2,i} - \overline{\theta}_{1,i} \right ) \left ( 
\overline{\theta}_{2,j} - \overline{\theta}_{1,j} \right )
\Pi \left (\overline{\theta}_{2},t+\Delta t|\overline{\theta}_{1},t  \right ) 
d\overline{\theta}_{2} = B_{i,j}\left ( \overline{\theta}_{1},t \right ) + 
\mathcal{O}\left ( \varepsilon \right )
\end{equation}

The first condition guarantees the continuity of $\Pi \left (\overline{\theta},t|\overline{\theta}_{0},t_{0}  \right )$, while the other three force the finite existence of three parameters that will be described in Section 8.1.3. 
Equation 25 can be furthermore decomposed into the continuous and discrete components:

\begin{equation}
\label{eq:chapKolmogIntegralFormContDisct}
\left\{\begin{matrix}
\Pi \left (\overline{\theta}_{3}^{c},t_{3}|\overline{\theta}_{1}^{c},t_{1}  \right ) 
= \int \Pi \left (\overline{\theta}_{2}^{c},t_{2}|\overline{\theta}_{1}^{c},t_{1}  
\right ) \Pi \left (\overline{\theta}_{3}^{c},t_{3}|\overline{\theta}_{2}^{c},t_{2}  
\right ) d\overline{\theta}_{2}^{c}
\\ 
\Pi \left (\overline{\theta}_{3}^{d},t_{3}|\overline{\theta}_{1}^{d},t_{1}  \right ) 
= \int \Pi \left (\overline{\theta}_{2}^{d},t_{2}|\overline{\theta}_{1}^{d},t_{1}  
\right ) \Pi \left (\overline{\theta}_{3}^{d},t_{3}|\overline{\theta}_{2}^{d},t_{2}  
\right ) d\overline{\theta}_{2}^{d}
\end{matrix}\right.
\: \: \: where \:\:   t_{1}<t_{2}<t_{3}
\end{equation}

and its differential form is as follows:
\begin{equation}
\label{eq:chapKolmogDiffFormContDisct}
\left\{\begin{matrix}
\frac{\partial \Pi \left (\overline{\theta}^{c},t|\overline{\theta}_{0}^{c},t_{0}  
\right ) }{\partial t} = \mathcal{L}_{CK}^{c}  \left (     \Pi \left 
(\overline{\theta}^{c},t|\overline{\theta}_{0}^{c},t_{0} \right ), 
\overline{\theta}^{d},\overline{\alpha}_{brow},\overline{\alpha}_{staz},
\overline{\alpha}_{DS},t  \right )
\\ 
\frac{\partial \Pi \left (\overline{\theta}^{d},t|\overline{\theta}_{0}^{d},t_{0}  
\right ) }{\partial t} = \mathcal{L}_{CK}^{d}  \left (     \Pi \left 
(\overline{\theta}^{d},t|\overline{\theta}_{0}^{d},t_{0} \right ), 
\overline{\theta}^{c},t  \right )
\end{matrix}\right.
\end{equation}
where:
\begin{itemize}
 \item  $\Pi \left (\overline{\theta}^{c},t|\overline{\theta}_{0}^{c},t_{0} \right 
 )$ of the system to be in state $\overline{\theta}^{c}$ at time $t$ given that 
 the system was in $\overline{\theta}^{c}_{0}$ at time $t_{0}$;
 \item $\Pi \left (\overline{\theta}^{d},t|\overline{\theta}_{0}^{d},t_{0} \right 
 )$ of the system to be in state $\overline{\theta}^{d}$ at time $t$ given 
 that the system was in $\overline{\theta}^{d}_{0}$ at time $t_{0}$;
 \item $\mathcal{L}_{CK}^{c} \left ( \cdot  \right )$ and 
 $\mathcal{L}_{CK}^{d} \left ( \cdot  \right )$  are specific Chapman-
 Kolmogorov operators that will be described in the following section.
\end{itemize}
%
%
% The Chapman-Kolmogorov Equation
%
%
% COPY FROM HERE
%
\subsubsection{The Chapman-Kolmogorov Equation}
\label{sec:ChapmanKolmogorov }
As mentioned in the previous paragraph, the system of equations~\ref{eq:thetaDecomposition}, written in integral 
form, can be solved in a differential form through the Chapman-Kolmogorov (C-K) 
operator~\cite{ProbReactoDynamicsDevooght}:

\begin{equation}
\label{eq:CK}
\begin{matrix}
\frac{\partial \Pi \left (\overline{\theta}^{c},t|\overline{\theta}_{0}^{c},t_{0}  
\right ) }{\partial t} = - \sum_i \frac{\partial }{\partial \overline{\theta}_{i}^{c}} \left ( A_{i}\left (  \overline{\theta}^{c}, \overline{\theta}^{d}, t\right ) \Pi \left (\overline{\theta}^{c},t|\overline{\theta}_{0}^{c},t_{0}  
\right )  \right ) + 
\\
+ \frac{1}{2}\sum_{i,j} \frac{\partial^2 }{\partial \overline{\theta}^{c}_{i} \partial \overline{\theta}^{c}_{j}}\left ( B_{i,j}\left (  \overline{\theta}^{c}, \overline{\theta}^{d}, t\right ) \Pi \left (\overline{\theta}^{c},t|\overline{\theta}_{0}^{c},t_{0}  
\right )  \right ) +
\\
+ \bigintss \left (  W\left ( \overline{\theta}^{c}|
\overline{\theta}^{'c},\overline{\theta}^{d},t \right )\Pi \left (\overline{\theta}^{'c},t|\overline{\theta}_{0}^{c},t_{0}  
\right ) - W\left ( \overline{\theta}^{'c}|
\overline{\theta}^{c},\overline{\theta}^{d},t \right )\Pi \left (\overline{\theta}^{c},t|\overline{\theta}_{0}^{c},t_{0}  
\right )  \right )d\overline{\theta}^{'c}
\end{matrix}
\end{equation}

\begin{equation}
\frac{\partial \Pi \left (\overline{\theta}^{d},t|\overline{\theta}_{0}^{d},t_{0}  
\right ) }{\partial t} =
\sum_{i} W\left ( \overline{\theta}^{d}|
\overline{\theta}^{d}_{i},\overline{\theta}^{c},t \right ) \Pi \left (\overline{\theta}^{d}_{i},t|\overline{\theta}^{d},t_{0}  
\right ) - W\left ( \overline{\theta}^{d}_{i}|
\overline{\theta}^{d},\overline{\theta}^{c},t \right ) \Pi \left (\overline{\theta}^{d},t|\overline{\theta}^{d}_{0},t_{0}  
\right ) 
\end{equation}
where:
\begin{equation}
\begin{matrix}
A_{i}\left ( \overline{\theta}, t \right ) = \left\{\begin{matrix}
0 & \: \: if  \: \: \overline{\theta}_{i} \in \overline{\theta}^{d} 
\\ 
f\left ( \overline{\theta}^{c},\overline{\theta}^{d},\alpha_{staz},\alpha_{brow},t \right ) +\frac{1}{2}\frac{\partial b\left ( \overline{\theta}^{c},t \right )}{\partial \overline{\theta}^{c}}Qb\left ( \overline{\theta}^{c},t \right ) \: \: & if  \: \: \overline{\theta}_{i} \notin \overline{\theta}^{d} 
\end{matrix}\right.
\\ 
B_{i,j}\left ( \overline{\theta}, t \right ) = \left\{\begin{matrix}
0 & \: \: if  \: \: \overline{\theta}_{i} \: \: or \: \: \overline{\theta}_{j}  \in \overline{\theta}^{d} 
\\ 
b\left ( \overline{\theta}^{c},t \right )Qb^{T}\left (  \overline{\theta}^{c},t \right ) \: \: & \: \: if  \: \: \overline{\theta}_{i} \: \: or \: \: \overline{\theta}_{j}  \notin \overline{\theta}^{d} 
\end{matrix}\right.
\end{matrix}
\end{equation}
This system of equations is composed of four main terms that identify four different types of processes:
\begin{itemize}
  \item Drift process;
  \item Diffusion process;
  \item Jumps in continuous space;
  \item Jumps in discrete space (component state transitions).
\end{itemize}
 These four processes are described in the following sub-sections.
%
%
% Drift process
%
%
\paragraph{Drift process}
\label{sec:CKDrift}
The drift process is defined by the Lioville’s equation:
\begin{equation}
\label{eq:lioville}
  \frac{\partial \, \Pi \left (\overline{\theta}^{c},t|\overline{\theta}^{c}_{0},t_{0}  \right ) }{\partial t} = \sum_{i}\frac{\partial }{\partial \overline{\theta}^{c}_{i}}\left ( A_{i}\left ( \overline{\theta}^{c},\overline{\theta}^{d},t \right ) \Pi \left (\overline{\theta}^{c},t|\overline{\theta}^{c}_{0},t_{0}  \right ) \right )
\end{equation}
It is important to notice that this equation describes a completely deterministic motion, indicated by the equation:
\begin{equation}
\label{eq:determLioville}
   \frac{\partial \, \overline{\theta}^{c}\left ( t \right ) }{\partial t} = A_{i}\left ( \overline{\theta}^{c},\overline{\theta}^{d},t \right ) 
\end{equation}
If $\overline{\theta}^{c} \left (\overline{\theta}^{c}_{0},\overline{\theta}^{d},t  \right )$ is the solution of Eq.~\ref{eq:determLioville}, then then the solution of the Lioville's equation is:
\begin{equation}
\label{eq:solLioville}
\Pi \left (\overline{\theta}^{c},t|\overline{\theta}^{c}_{0},t_{0}  \right ) = \delta\left ( \overline{\theta}^{c} - \overline{\theta}^{c}\left ( \overline{\theta}^{c}_{0},\overline{\theta}^{d},t \right ) \right )
\end{equation}
provided the initial condition:
\begin{equation}
\label{eq:solLiovilleInitCond}
\Pi \left (\overline{\theta}^{c},t|\overline{\theta}^{c}_{0},t_{0}  \right ) = \delta\left ( \overline{\theta}^{c} - \overline{\theta}^{c}_{0} \right )
\end{equation}
%
%
% Diffusion process
%
%
\paragraph{Diffusion process}
\label{subsec:CKDiffusion}
This process is described by the Fokker-Plank equation:
\begin{equation}
\begin{matrix}
\frac{\partial \, \Pi \left (\overline{\theta}^{c},t|\overline{\theta}^{c}_{0},t_{0}  \right ) }{\partial t} = 
\sum_{i}\frac{\partial }{\partial \overline{\theta}^{c}_{i}}\left ( A_{i}\left ( \overline{\theta}^{c},\overline{\theta}^{d},t \right ) \Pi \left (\overline{\theta}^{c},t|\overline{\theta}^{c}_{0},t_{0}  \right ) \right ) +
\\
+ \frac{1}{2}\sum_{i,j} \frac{\partial^2 }{\partial \overline{\theta}^{c}_{i} \partial \overline{\theta}^{c}_{j}}\left ( B_{i,j}\left (  \overline{\theta}^{c}, \overline{\theta}^{d}, t\right ) \Pi \left (\overline{\theta}^{c},t|\overline{\theta}_{0}^{c},t_{0}  
\right )  \right ) 
\end{matrix}
\end{equation}
where $A_{i}\left ( \overline{\theta}^{c},\overline{\theta}^{d},t \right )$ is the drift vector and $B_{i,j}\left (  \overline{\theta}^{c}, \overline{\theta}^{d}, t\right ) $  is the diffusion matrix. 
\\Provided the initial condition in Eq.~\ref{eq:solLiovilleInitCond}, the Fokker-Plank equation describes a system moving with drift whose velocity is 
 $A\left ( \overline{\theta}^{c},\overline{\theta}^{d},t \right )$ on which is imposed a Gaussian fluctuation with covariance matrix $B\left (  \overline{\theta}^{c}, \overline{\theta}^{d}, t\right ) $.
%
%
% Jumps in continuous space 
%
%
\paragraph{Jumps in continuous space }
\label{subsec:CKJumpsCont}
This process is described by the Master equation:
\begin{equation}
\frac{\partial \, \Pi \left (\overline{\theta}^{c},t|\overline{\theta}^{c}_{0},t_{0}  \right ) }{\partial t} =  \int \left (  W\left ( \overline{\theta}^{c}|
\overline{\theta}^{'c},\overline{\theta}^{d},t \right )\Pi \left (\overline{\theta}^{'c},t|\overline{\theta}_{0}^{c},t_{0}  
\right ) - W\left ( \overline{\theta}^{'c}|
\overline{\theta}^{c},\overline{\theta}^{d},t \right )\Pi \left (\overline{\theta}^{c},t|\overline{\theta}_{0}^{c},t_{0}  
\right )  \right )d\overline{\theta}^{'c}
\end{equation}
Provided the initial condition  in Eq.~\ref{eq:solLiovilleInitCond}, it describes a process characterized by 
straight lines interspersed with discontinuous jumps whose distribution is given by $W\left ( \overline{\theta}^{c}|
\overline{\theta}^{'c},\overline{\theta}^{d},t \right )$

%
%
% Jumps in discrete space 
%
%
\paragraph{Jumps in discrete space }
\label{subsec:CKJumpsDiscrete}
Transitions in the discrete space can occur in terms of jumps, then the formulation of 
\begin{equation}
\frac{\partial \, \Pi \left (\overline{\theta}^{d},t|\overline{\theta}^{d}_{0},t_{0}  \right ) }{\partial t} =  
\mathcal{L}_{CK}^{d}\left ( \Pi \left (\overline{\theta}^{d},t|\overline{\theta}_{0}^{d},t_{0}  
\right )  \right )
\end{equation}
is similar to the Master equation, recasted for a discrete phase space:
\begin{equation}
\frac{\partial \, \Pi \left (\overline{\theta}^{d},t|\overline{\theta}^{d}_{0},t_{0}  \right ) }{\partial t} =  \sum_{i} \left (  W\left ( \overline{\theta}^{d}|
\overline{\theta}^{d}_{i},\overline{\theta}^{c},t \right )\Pi \left (\overline{\theta}^{d}_{i},t|\overline{\theta}^{d}_{0},t_{0}  
\right ) - W\left ( \overline{\theta}^{d}_{i}|
\overline{\theta}^{d},\overline{\theta}^{c},t \right )\Pi \left (\overline{\theta}^{d},t|\overline{\theta}_{0}^{d},t_{0}  
\right )  \right ) 
\end{equation}

%
%
%
%
%
%
%
%
%
%
%
%
\subsection{RAVEN entities and analysis flow}
\label{sub:EntitiesAndFlow}
In the RAVEN code, the the number of analyses that can be performed is virtually ``infinite'',
being the code highly modular and object oriented. 
In the RAVEN code the number and types of possible analyses is potentially large (it is customization to many problem types).
\\Each basic action (e.g. sampling, printing, etc.) is encapsulated in
a dedicated object (named ``\textbf{Entity}''). Each object is inactive till it is connected with 
other objects in order to perform a more complex process. For example,
the \textit{Sampler} entity, aimed to employ a perturbation action, becomes active only in case
it gets associated with a \textit{Model}, that is the internal representation of a physical model (e.g. a system code).
\\RAVEN provides support for several \textbf{Entities}, which branch in several different categories/algorithms:
\begin{itemize}
  \item \textit{\textbf{RunInfo}}: 
    \\The RunInfo \textbf{Entity} represents the container of the information regarding how the overall computation should
      be performed. This \textbf{Entity}  accepts several input settings that define how to drive the calculation and set up, 
      when needed, particular settings for the machine the code needs to run on (e.g. queue system, if not PBS, etc.).  
  \item \textit{\textbf{Files}}: 
  \\ The Files \textbf{Entity}  defines any files that might be needed within the RAVEN run. This could include inputs to the   
      Model, pickled ROM files, or CSV files for post-processors, to name a few.
  \item \textit{\textbf{DataObjects}}: 
    \\The DataObjects system is a container of data objects of various types that can be constructed during the execution of 
     a particular calculation flow. These data objects can be used as input or output for a particular \textit{Model} 
     \textbf{Entity}. Currently, RAVEN supports 4 different data types, each with a particular conceptual meaning:
     \begin{itemize}
        \item \textit{Point}, describes the state of the system at a certain point (e.g. in time). In other words, it can be
                                       considered a mapping between a set of parameters in the input space and the resulting
                                       outcomes in the output space at a particular point in the phase space (e.g. in time);
        \item \textit{PointSet}, is a collection of individual Point objects. It can be considered a mapping between multiple
                                            sets of parameters in the input space and the resulting sets of outcomes in the output space
                                            at a particular point (e.g. in time);
        \item \textit{History},  describes the temporal evolution of the state of the system within a certain input domain;
        \item \textit{HistorySet}, is a collection of individual History objects. It can be considered a map- ping between
                                               multiple sets of parameters in the input space and the resulting sets of temporal evolution
                                               in the output space.
     \end{itemize}
     The DataObjects represent the preferred way to transfer the information coming from a Model (e.g., the driven code) to
      all the other RAVEN systems (e.g. Out-Stream system, Reduced Order Modeling component, etc.). 
  \item \textit{\textbf{Databases}}:
      \\ RAVEN provides the capability to store and retrieve data to/from an external database. Currently RAVEN supports
       only a database type called \textit{HDF5}. This database, depending on the data format it is receiving, will 
       organize itself in a ``parallel'' or ``hierarchical'' fashion. The user can create as many database \textbf{Entities} as needed. 
  \item \textit{\textbf{Samplers}}:
  \\ The Samplers  \textbf{Entity} is the container of all the algorithms to perform the perturbation of the input space.
      The Samplers can be categorized into 3 main classes:
      \begin{itemize}
        \item  \textit{Forward} : Sampling strategies that do not leverage the information coming from already evaluated 
        realizations in the input space. For example, Monte-Carlo, Stratified (LHS), Grid, Response Surface, Factorial Design, 
        Sparse Grid, etc.
        \item  \textit{Adaptive}:  Sampling strategies that take advantages of the information coming from already evaluated 
        realizations of the input space, adapting the sampling strategies to key figures of merits. For example, Limit Surface 
        search, Adaptive HDMR, etc.
        \item \textit{Dynamic Event Tree}: Sampling strategies that perform the exploration of the input space based on the 
        dynamic evolution of the system, employing branching techniques. For example, Dynamic Event Tree, Hybrid 
        Dynamic Event Tree, etc.
      \end{itemize}
  \item \textit{\textbf{OutStreamManager}}:
  \\ The OutStreamManager is the \textbf{Entity} used  for data exporting and dumping. The OutStreamManager supports
   2 actions:
      \begin{itemize}
       \item \textit{Print}: This Out-Stream is able to print out (in a Comma Separated Value format) all the information 
         contained in:
         \begin{itemize}
          \item DataObjects
          \item Reduced Order Models
         \end{itemize}
       \item \textit{Plot}: This Out-Stream is able to plot 2-Dimensional, 3-Dimensional, 4-Dimensional (using color 
       mapping), 5-Dimensional (using marker size). Several types of plot are available, such as scatter, line, surfaces, 
       histograms, pseudo-colors, contours, etc. 
      \end{itemize}
  \item \textit{\textbf{Distributions}}:
  \\ The Distributions \textbf{Entity} is the container of all the stochastic representations of random variables. Currently, 
  RAVEN supports:
      \begin{itemize}
       \item \textit{1-Dimensional} continuous and discrete distributions, such as Normal, Weibull, Binomial, etc.
       \item \textit{N-Dimensional} distributions, such as Multivariate Normal, user-inputted N-Dimensional distributions.
      \end{itemize}
\begin{figure}[h!]
  \includegraphics[scale=1]{pics/ExampleStepEntity.png}
  \caption{Example of the Steps \textbf{Entity}  and its connection in the input file.}
  \label{fig:ExampleStepEntity}
\end{figure}
  \item \textit{\textbf{Models}}:
  \\ The Models \textbf{Entity}  represents the projection from the input to the output space. Currently, RAVEN defines the 
  following sub-categories:
      \begin{itemize}
       \item \textit{Code}, the sub-~\textbf{Entity} that represent the driven code, through external code interfaces (see~\cite{RAVENuserManual})
       \item  \textit{ExternalModel}, the sub-~\textbf{Entity} that represents a physical or mathematical model that is 
       directly implemented by the user in a Python module
      \item \textit{ROM}, the sub-~\textbf{Entity} that represent the Reduced Order Model, interfaced with several algorithms
       \item \textit{PostProcessor}, the sub-~\textbf{Entity} that is used to perform action on data, such as computation of
       statistical moments, correlation matrices, etc.
      \end{itemize}
      The Model \textbf{Entity} can be seen as a transfer function between the input and output space.
  \item \textit{\textbf{Functions}}:
   \\ The Functions \textbf{Entity} is the container of all the user-defined functions, such as Goal Functions in adaptive 
   sampling strategies, etc.
\end{itemize}
All these action-objects are combined together in order to create a peculiar analysis flow, which is specified
by the user in an additional \textbf{Entity} named \textit{\textbf{Steps}}. This \textbf{Entity} represents the core of the analysis, since it is the location where the multiple objects get finally linked in order to perform a combined action on a certain \textit{Model} (see Fig.~\ref{fig:ExampleStepEntity}). In order to perform this linking, each \textbf{Entity} defined in the Step needs to ``play'' a Role:
\begin{itemize}
  \item \textit{Input}
  \item \textit{Output}
  \item \textit{Model}
  \item \textit{Sampler}
  \item \textit{Function}
  \item \textit{ROM}
  \item \textit{SolutionExport}, the \textbf{Entity} that is used to export the solution of a \textit{Sampler}.
\end{itemize}
Currently, RAVEN supports 4 different types of \textit{\textbf{Steps}}:
\begin{itemize}
  \item \textit{SingleRun}, perform a single run of a model
  \item \textit{MultiRun}, perform multiple runs of a model
  \item \textit{RomTrainer}, perform the training of a Reduced Order Model (ROM)
  \item \textit{PostProcess}, post-process data or manipulate RAVEN entities
  \item \textit{IOStep}, step aimed to perform multiple actions:
  \begin{itemize}
    \item construct/update a Database from a DataObjects and vice-versa
    \item construct/update a Database or a DataObjects object from CSV files 
    \item stream the content of a Database or a DataObjects out through an OutStream 
    \item store/retrieve a ROM to/from an external File using Pickle module of Python
  \end{itemize}
\end{itemize}

\subsection{Raven Input Structure}
\label{sub:InputStructure}
The RAVEN code does not have a fixed calculation flow, since all of its basic
objects can be combined in order to create a user-defined calculation flow.
%
Thus, its input (XML format) is organized in different XML blocks, each with a
different functionality.
%
The main input blocks are as follows:
\begin{itemize}
  \item \textbf{\textless Simulation\textgreater}: The root node containing the
  entire input, all of
  the following blocks fit inside the \emph{Simulation} block.
  %
  \item \textbf{\textless RunInfo\textgreater}: Specifies the calculation
  settings (number of parallel simulations, etc.).
  %
  \item \textbf{\textless Files\textgreater}: Specifies the files to be
  used in the calculation.
  %
  \item \textbf{\textless Distributions\textgreater}: Defines distributions
  needed for describing parameters, etc.
  %
  \item \textbf{\textless Samplers\textgreater}: Sets up the strategies used for
  exploring an uncertain domain.
  %
  \item \textbf{\textless DataObjects\textgreater}: Specifies internal data objects
  used by RAVEN.
  %
  \item \textbf{\textless Databases\textgreater}: Lists the HDF5 databases used
  as input/output to a
  RAVEN run.
  %
  \item \textbf{\textless OutStreamManager\textgreater}: Visualization and
  Printing system block.
  %
  \item \textbf{\textless Models\textgreater}: Specifies codes, ROMs,
  post-processing analysis, etc.
  %
  \item \textbf{\textless Functions\textgreater}: Details interfaces to external
  user-defined functions and modules.
  %
  the user will be building and/or running.
  %
  \item \textbf{\textless Steps\textgreater}: Combines other blocks to detail a
  step in the RAVEN workflow including I/O and computations to be performed.
  %
\end{itemize}

Each of these blocks are explained in dedicated sections in the user manual ~\cite{RAVENuserManual}.
%
\section{RunInfo  \\ \vspace{2 mm} {\small }}
The $RunInfo$ block is the place where the user specifiy how the calculation needs to be performed. In this input block, several settings can be inputted, in order to define how to drive the calculation and set up, when needed, particular settings for the machine the code needs to run on (queue system if not PBS, etc.).
In the following subsections, all the keywords are explained in detail.
\subsection{RunInfo: input of calculation flow.}
\label{subsec:runinfoCalcFlow}
AAGKBAGKHAGKHHGAKAGAG

\begin{itemize}
\item $<WorkingDir>$\textbf{\textit{, string, required field.}} in this block the user needs to specify the absolute or relative (with respect to the location where RAVEN is run from) path to a directory that is going to be used to store all the results of the calculations and where RAVEN looks for the files specified in the block $<Files>$. \textit{Default = None};

\item $<NumNode>$\textbf{\textit{, integer, optional field.}}  this xml node is used to specify the number of nodes RAVEN should request when running in High Performance Computing (HPC) systems. \textit{Default = None};

\item $<batchSize>$\textbf{\textit{, integer, required field.}}. This parameter specifies the number of parallel runs need to be run simultaneously (e.g., the number of driven code instances, e.g. RELAP5-3D, that RAVEN will spoon at the same time). \textit{Default = 1};

\item $<NumThreads>$\textbf{\textit{, integer, optional field.}} this section can be used to specify the number of threads RAVEN should associate when running the driven software. For example, if RAVEN is driving a code named "FOO", and this code has multi-threading support, in here the user specify how many threads each instance of FOO should use (e.g. FOO --n-threads=$NumThreads$). \textit{Default = 1 (or None when the driven code does not have multi-threading support)};

\item $<totalNumCoresUsed>$\textbf{\textit{, integer, optional field.}}  global number of cpus RAVEN is going to use for performing the calculation. When the driven code has MPI and/or  Multi-threading support and the user decides to input $NumThreads > 1$  and $NumMPI > 1$, the totalNumCoresUsed = NumThreads*NumMPI*batchSize. \textit{Default = 1};

\item $<NumMPI>$\textbf{\textit{, integer, optional field.}}  this section can be used to specify the number of MPI cpus RAVEN should associate when running the driven software. For example, if RAVEN is driving a code named "FOO", and this code has MPI support, in here the user specifies how many mpi cpus each instance of FOO should use (e.g. mpiexec FOO -np $NumMPI$). \textit{Default = 1 (or None when the driven code does not have MPI support)};

\item $<precommand>$\textbf{\textit{, string, optional field.}} in here the user can specifies a command that needs to be inserted before the actual command that is used to run the external model (e.g., mpiexec -n 8 $precommand$ ./externalModel.exe (...)). \textit{Default = None};  

\item $<postcommand>$\textbf{\textit{, string, optional field.}} in here the user can specifies a command that needs to be appended after the actual command that is used to run the external model (e.g., mpiexec -n 8  ./externalModel.exe (...) $postcommand$). \textit{Default = None};

\item $<MaxLogFileSize>$\textbf{\textit{, integer, optional field.}}  every time RAVEN drives a code/software, it creates a logfile of the code screen output. In this block, the user can input the maximum size of log file in bytes. \textit{Defautl = Inf}. NB. This flag is not implemtend yet; 

\item $<deleteOutExtension>$\textbf{\textit{, comma separated string, optional field.}} if a run of an external model has not failed delete the outut files with the listed extension (e.g., $<deleteOutExtension>txt,pdf</deleteOutExtension>$). \textit{Default = None}.

\item $<delSucLogFiles>$\textbf{\textit{, boolean, optional field.}} if a run of an external model has not failed (return code = 0), delete the associated log files. \textit{Default = False};

\item $<Files>$\textbf{\textit{, comma separated string, required field.}} these are the paths to the files required by the code, string from the $WorkingDir$; 

\item $<Sequence>$\textbf{\textit{, comma separated string, required field.}} ordered list of the step names that RAVEN will run (see Section~\ref{sec:steps});

\item $<DefaultInputFile>$\textbf{\textit{, string, optional field.}} In this block the user can change the default xml input file RAVEN is going to look for if none has been provided as command-line argument. \textit{Default = ``test.xml''}.

\end{itemize}

\subsection{RunInfo: input of queue modes.}
\label{subsec:runinfoModes}
In this sub-section all  the keyword (xml nodes) for setting the queue system are reported.
\begin{itemize}
%%%%%% MODE
\item $<mode>$\textbf{\textit{, string, optional field.}} In this xml block, the user might specify which kind of protocol the parallel enviroment should use. By instance, RAVEN currently supports two pre-defined ``modes'':
  \begin{itemize}
    \item pbsdsh: this ``mode'' uses the pbsdsh protocol to distribute the program running; more information regarding this protocol can be found in ~ref{}. 
    \item mpi: this ``mode'' uses mpiexec to distribute the program running; more information regarding this protocol can be found in ~ref{}
   \end{itemize}
Both methods can submit a qsub command or can be run from an already submitted interactive qsub command:
     \begin{itemize}
        \item Mode ``pbsdsh'' automatically ``understands'' when it needs to generate the ``qsub'' command, inquiring the ``machine eviroment'': 
         \begin{itemize}         
           \item If RAVEN is executed in the HEAD node of an HPC system, RAVEN generates the ``qsub'' command, instantiates and submits itself to the queue system; 
           \item If the user decides to execute RAVEN from an ``interactive node'' (a certain number of nodes that have been reserved in interactive PBS mode), RAVEN, using the ``pbsdsh'' system, is going to utilize the reserved resources (cpus and nodes) to distribute the jobs, but, obviously, it's not going to generate the ``qsub'' command. 
         \end{itemize}
          \item Mode ``MPI'' needs an additional keyword (xml sub-node) in order to understand when it needs to generate the ``qsub'' commnad:
         \begin{itemize}         
           \item If RAVEN is executed in the HEAD node of an HPC system, the user needs to input a sub-node, $<runQSUB/>$, right after the specification of the mpi mode (i.e. $<mode>mpi<runQSUB/></mode>$). If the keyword is provided, RAVEN generates the ``qsub'' command, instantiates and submits itself to the queue system; 
           \item If the user decides to execute RAVEN from an ``interactive node'' (a certain number of nodes that have been reserved in interactive PBS mode), RAVEN, using the ``mpi'' system, is going to utilize the reserved resources (cpus and nodes) to distribute the jobs, but, obviously, it's not going to generate the ``qsub'' command. 
         \end{itemize}
     \end{itemize}
     NB. Mode ``MPI'' can be used without any PBS support.

%%%%%% CUSTOM MODE
\item $<CustomMode>$\textbf{\textit{, xml node, optional field.}} In this xml node, the ``advanced'' users can implement a newer ``mode''. Please refer to sub-section~\ref{subsec:runinfoadvanced} for advanced users.

%%%%%% QUEUE SOFTWARE
\item $<quequingSoftware>$\textbf{\textit{, string, optional field.}} RAVEN has support for PBS quequing system. If the platform provides a different quequing system, the user can specify its name here (e.g., PBS PROFESSIONAL, etc.). \textit{Default = PBS PROFESSIONAL};

%%%%%% EXPECTED TIME
\item $<expectedTime>$\textbf{\textit{colum separated string, requested field (pbsdsh mode) }}. In this block the user specifies the time the whole calculation is expected to last. The syntax of this node is $hours:minutes:seconds$ (e.g. 40:10:30 => 40 hours, 10 minutes, 30 seconds). After this period of time the HPC system will automatically stop the simulation (even if the simulation is not completed). It is preferable to rationally overstimate the needed time. \textit{Default = None};
\end{itemize}

\begin{itemize}
\item $<WorkingDir>$\textbf{\textit{, string, required field.}} in this block the user needs to specify the absolute or relative (with respect to the location where RAVEN is run from) path to a directory that is going to be used to store all the results of the calculations and where RAVEN looks for the files specified in the block $<Files>$. \textit{Default = None};



\item $<CustomMode>$\textbf{\textit{, xml node, optional field.}} In this xml node, the ``advanced'' users can implement a newer ``mode''. Please refer to sub-section~\ref{subsec:runinfoadvanced} for advanced users.



\item $<NumNode>$\textbf{\textit{, integer, optional field.}}  this xml node is used to specify the number of nodes RAVEN should request when running in High Performance Computing (HPC) systems. \textit{Default = None};

\item $<batchSize>$\textbf{\textit{, integer, required field.}}. This parameter specifies the number of parallel runs need to be run simultaneously (e.g., the number of driven code instances, e.g. RELAP5-3D, that RAVEN will spoon at the same time). \textit{Default = 1};

\item $<NumThreads>$\textbf{\textit{, integer, optional field.}} this section can be used to specify the number of threads RAVEN should associate when running the driven software. For example, if RAVEN is driving a code named "FOO", and this code has multi-threading support, in here the user specify how many threads each instance of FOO should use (e.g. FOO --n-threads=$NumThreads$). \textit{Default = 1 (or None when the driven code does not have multi-threading support)};

\item $<totalNumCoresUsed>$\textbf{\textit{, integer, optional field.}}  global number of cpus RAVEN is going to use for performing the calculation. When the driven code has MPI and/or  Multi-threading support and the user decides to input $NumThreads > 1$  and $NumMPI > 1$, the totalNumCoresUsed = NumThreads*NumMPI*batchSize. \textit{Default = 1};

\item $<NumMPI>$\textbf{\textit{, integer, optional field.}}  this section can be used to specify the number of MPI cpus RAVEN should associate when running the driven software. For example, if RAVEN is driving a code named "FOO", and this code has MPI support, in here the user specifies how many mpi cpus each instance of FOO should use (e.g. mpiexec FOO -np $NumMPI$). \textit{Default = 1 (or None when the driven code does not have MPI support)};

\item $<precommand>$\textbf{\textit{, string, optional field.}} in here the user can specifies a command that needs to be inserted before the actual command that is used to run the external model (e.g., mpiexec -n 8 $precommand$ ./externalModel.exe (...)). \textit{Default = None};  

\item $<postcommand>$\textbf{\textit{, string, optional field.}} in here the user can specifies a command that needs to be appended after the actual command that is used to run the external model (e.g., mpiexec -n 8  ./externalModel.exe (...) $postcommand$). \textit{Default = None};

\item $<MaxLogFileSize>$\textbf{\textit{, integer, optional field.}}  every time RAVEN drives a code/software, it creates a logfile of the code screen output. In this block, the user can input the maximum size of log file in bytes. \textit{Defautl = Inf}. NB. This flag is not implemtend yet; 

\item $<deleteOutExtension>$\textbf{\textit{, comma separated string, optional field.}} if a run of an external model has not failed delete the outut files with the listed extension (e.g., $<deleteOutExtension>txt,pdf</deleteOutExtension>$). \textit{Default = None}.

\item $<delSucLogFiles>$\textbf{\textit{, boolean, optional field.}} if a run of an external model has not failed (return code = 0), delete the associated log files. \textit{Default = False};

\item $<Files>$\textbf{\textit{, comma separated string, required field.}} these are the paths to the files required by the code, string from the $WorkingDir$; 

\item $<Sequence>$\textbf{\textit{, comma separated string, required field.}} ordered list of the step names that RAVEN will run (see Section~\ref{sec:steps});

\item $<DefaultInputFile>$\textbf{\textit{, string, optional field.}} In this block the user can change the default xml input file RAVEN is going to look for if none has been provided as command-line argument. \textit{Default = ``test.xml''}.

\end{itemize}
% source: Simulation.py

\subsection{RunInfo for Advanced Users.}
\label{subsec:runinfoadvanced}
aaaaaaaaaaaaaaaa

\subsection{RunInfo examples.}
eccolo:

The example:
\begin{lstlisting}[style=XML]
<RunInfo>
    <WorkingDir>externalModel</WorkingDir>
    <Files>lorentzAttractor.py</Files>
    <Sequence>MonteCarlo</Sequence>
    <batchSize>100</batchSize>
    <NumThreads>4</NumThreads>    
    <mode>mpi</mode>
    <NumMPI>2</NumMPI>
</RunInfo>
\end{lstlisting}
Specifies the working directory (WorkingDir) where are located the files necessary (Files) to run a series of 100 (batchSize) Monte-Carlo calculations (Sequence).
MPI (mode) mode is used along with 4 threads (NumThreads) and 2 mpi process per run (NumMPI).

\section{Distributions \\ \vspace{2 mm}}
\label{sec:distributions}
\newcommand{\distname}[1]{\textbf{#1}}
\newcommand{\distattrib}[1]{\textit{#1}}
RAVEN provides support for several probability distributions. Currently, the user can choose among all the most important 1-Dimensional distributions and N-Dimensional ones, either custom or Multi-Variate.  
\\ The user needs to specify the probability distributions, that need to be used during the simulation, within the $<Distributions>$ xml block:
\begin{lstlisting}[style=XML]
<Simulation>
   ...
  <Distributions>
    <!-- here all the distributions, that need to be used, are listed -->
  </Distributions>
  ...
</Simulation>
\end{lstlisting}
In the following two sub-sections, the input requirements for all of them are reported.
%%%%%% 1-Dimensional Probability distributions
\subsection{1-Dimensional Probability Distributions}
\label{subsec:1dDist}

This sub-section is organized in two different parts: 1) Continuous 1-D distributions; 2) Discrete 1-D distributions. These two chapters cover all the  requirements for using the different distribution entities. 
%%%%%% paragraph 1-Dimensional Continuous Distributions.
\subsubsection{1-Dimensional Continuous Distributions.}
\label{subsubsec:1DContinuous}

In this paragraph all the 1-D distributions', currently available in RAVEN, are reported.
\\ Firstly, all the probability distributions functions in the code can be truncated by the following keywords:
\begin{lstlisting}[style=XML]
<lowerBound>***</lowerBound>
<upperBound>***</upperBound>
\end{lstlisting}
Obviously, each distribution already defines its validity domain (e.g. Normal distribution, [-inf,+inf]).
\\ RAVEN currently provides support for 12 1-Dimensional distributions.  In the following paragraphs, all the input requirements are reported and commented.

%%%%%% Beta
\paragraph{Beta Distribution}
\label{Beta}
The \distname{Beta} distribution is a continuous distribution  defined on the interval $[0, 1]$ parametrized by two positive shape parameters, denoted by $\alpha$ and $\beta$, that appear as exponents of the random variable and control the shape of the distribution. The distribution domain  can be changed,specifying new boundaries, to fit the user needs.  Its support is $x \in (0, 1)$.
\\ The specifications of this distribution must be defined within the xml block $<Beta>$. This xml-node needs to contain the attribute:
\vspace{-5mm}
\begin{itemize}
\itemsep0em
\item \textbf{name}, \textit{required string attribute}, user-defined name of this distribution. N.B. As for the other objects, this is the name that can be used to refer to this specific entity from other input blocks (xml).   
\end{itemize}
\vspace{-5mm}
This distribution can be initialized through the following keyword/s:
\begin{itemize}
\item $<alpha>$, float, required parameter, first shape parameter;
\item $<beta>$, float, required parameter, second shape parameter;
\item $<low>$, float, optional parameter,  lower domain boundary;  
\item $<high>$, float, required parameter, upper domain boundary.
\end{itemize}

\begin{lstlisting}[style=XML]
----------------------------
Example:
----------------------------
<Distributions>
  ...
  <Beta name='...'>
     <low>***</low>
     <high>***</high>
     <alpha>***</alpha>
     <beta>***</beta>
  </Beta>
  ...
</Distributions>
----------------------------
\end{lstlisting}



%%%%%% Exponential
\paragraph{Exponential Distribution}
\label{Exponential}
The \distname{Exponential} distribution is a continuous distribution that can be used to model the time between independent events that
happen at a constant average time.  Its support is $x \in [0, +\inf)$.
\\ The specifications of this distribution must be defined within the xml block $<Exponential>$. This xml-node needs to contain the attribute:
\vspace{-5mm}
\begin{itemize}
\itemsep0em
\item \textbf{name}, \textit{required string attribute}, user-defined name of this distribution. N.B. As for the other objects, this is the name that can be used to refer to this specific entity from other input blocks (xml).   
\end{itemize}
\vspace{-5mm}
This distribution can be initialized through the following keyword/s:
\begin{itemize}
\item $<lambda>$, float, required parameter,  rate parameter.
\end{itemize}

\begin{lstlisting}[style=XML]
----------------------------
Example:
----------------------------
<Distributions>
  ...
  <Exponential name='...'>
    <lambda>***</lambda>
  </Exponential>
  ...
</Distributions>
----------------------------
\end{lstlisting}

%%%%%% Gamma
\paragraph{Gamma Distribution}
\label{Gamma}
The \distname{Gamma} distribution is a two-parameter family of continuous probability distributions. The common exponential distribution and chi-squared distribution are special cases of the gamma distribution.  Its support is $x \in (0,+\inf)$.
\\ The specifications of this distribution must be defined within the xml block $<Gamma>$. This xml-node needs to contain the attribute:
\vspace{-5mm}
\begin{itemize}
\itemsep0em
\item \textbf{name}, \textit{required string attribute}, user-defined name of this distribution. N.B. As for the other objects, this is the name that can be used to refer to this specific entity from other input blocks (xml).   
\end{itemize}
\vspace{-5mm}
This distribution can be initialized through the following keyword/s:
\begin{itemize}
\item $<alpha>$, float, required parameter, shape parameter;
\item $<beta>$, float, required parameter, 1/scale or the inverse scale parameter;
\item $<low>$, float, optional parameter,  lower domain boundary.
\end{itemize}

\begin{lstlisting}[style=XML]
----------------------------
Example:
----------------------------
<Distributions>
  ...
  <Gamma name='...'>
    <alpha>***</alpha>
    <beta>***</beta>
    <low>***</low>
  </Gamma>
  ...
</Distributions>
----------------------------
\end{lstlisting}

%%%%%% Logistic
\paragraph{Logistic Distribution}
\label{Logistic}
The \distname{Logistic} distribution is a continuous distribution
similar to the normal distribution with a CDF that is an instance of a
logistic function. It resembles the normal distribution in shape but has heavier tails (higher kurtosis). Its support is $x \in (-\inf,+\inf)$.
\\ The specifications of this distribution must be defined within the xml block $<Logistic>$. This xml-node needs to contain the attribute:
\vspace{-5mm}
\begin{itemize}
\itemsep0em
\item \textbf{name}, \textit{required string attribute}, user-defined name of this distribution. N.B. As for the other objects, this is the name that can be used to refer to this specific entity from other input blocks (xml).   
\end{itemize}
\vspace{-5mm}
This distribution can be initialized through the following keyword/s:
\begin{itemize}
\item $<location>$, float, required parameter, it is the distribution mean;
\item $<scale>$, float, required parameter, scale parameter that is proportional to the standard deviation.
\end{itemize}

\begin{lstlisting}[style=XML]
----------------------------
Example:
----------------------------
<Distributions>
  ...
  <Logistic name='...'>
    <location>***</location>
    <scale>***</scale>
  </Logistic>
  ...
</Distributions>
----------------------------
\end{lstlisting}
 

%%%%%% LogNormal
\paragraph{LogNormal Distribution}
\label{LogNormal}
The \distname{LogNormal} distribution is a continuous distribution
with the logarithm of the random variable being normally distributed. Its support is $x \in (0, +\inf)$.
\\ The specifications of this distribution must be defined within the xml block $<LogNormal>$. This xml-node needs to contain the attribute:
\vspace{-5mm}
\begin{itemize}
\itemsep0em
\item \textbf{name}, \textit{required string attribute}, user-defined name of this distribution. N.B. As for the other objects, this is the name that can be used to refer to this specific entity from other input blocks (xml).   
\end{itemize}
\vspace{-5mm}
This distribution can be initialized through the following keyword/s:
\begin{itemize}
\item $<mean>$, float, required parameter, it is the distribution mean or expected value (in log-scale);
\item $<sigma>$, float, required parameter, standard deviation.
\end{itemize}

\begin{lstlisting}[style=XML]
----------------------------
Example:
----------------------------
<Distributions>
  ...
  <LogNormal name='...'>
    <mean>***</mean>
    <sigma>***</sigma>
  </LogNormal>
  ...
</Distributions>
----------------------------
\end{lstlisting}

%%%%%% Normal
\paragraph{Normal Distribution}
\label{Normal}
The \distname{Normal} distribution (or Gaussian) distribution is a
continuous distribution. It is extremely useful because of the central limit theorem, which states that, under mild conditions, the mean of many random variables independently drawn from the same distribution is distributed approximately normally, irrespective of the form of the original distribution. Its support is $x \in (-\inf, +\inf)$.
\\ The specifications of this distribution must be defined within the xml block $<Normal>$. This xml-node needs to contain the attribute:
\vspace{-5mm}
\begin{itemize}
\itemsep0em
\item \textbf{name}, \textit{required string attribute}, user-defined name of this distribution. N.B. As for the other objects, this is the name that can be used to refer to this specific entity from other input blocks (xml).   
\end{itemize}
\vspace{-5mm}
This distribution can be initialized through the following keyword/s:
\begin{itemize}
\item $<mean>$, float, required parameter, it is the distribution mean or expected value;
\item $<sigma>$, float, required parameter, standard deviation.
\end{itemize}

\begin{lstlisting}[style=XML]
----------------------------
Example:
----------------------------
<Distributions>
  ...
  <Normal name='...'>
    <mean>***</mean>
    <sigma>***</sigma>
  </Normal>
  ...
</Distributions>
----------------------------
\end{lstlisting}




%%%%%% Triangular
\paragraph{Triangular Distribution}
\label{Triangular}
The \distname{Triangular} distribution is a continuous distribution that has a triangular shape for the Pdf. It is often used where the distribution is only vaguely known, but, like the uniform distribution, upper and lower limits are ``known'', but a ``best guess'', the mode or center point, is also added. It has been recommended as a ``proxy'' for the beta distribution. Its support is $lower \le x \le upper$.
\\ The specifications of this distribution must be defined within the xml block $<Triangular>$. This xml-node needs to contain the attribute:
\vspace{-5mm}
\begin{itemize}
\itemsep0em
\item \textbf{name}, \textit{required string attribute}, user-defined name of this distribution. N.B. As for the other objects, this is the name that can be used to refer to this specific entity from other input blocks (xml).   
\end{itemize}
\vspace{-5mm}
This distribution can be initialized through the following keyword/s:
\begin{itemize}
\item $<apex>$, float, required parameter, ``best guess'', also called, peak factor;
\item $<min>$, float, required parameter, domain lower boundary;
\item $<max>$, float, required parameter, domain upper boundary.
\end{itemize}

\begin{lstlisting}[style=XML]
----------------------------
Example:
----------------------------
<Distributions>
  ...
  <Triangular name='...'>
    <apex>***</apex>
    <min>***</min>
    <max>***</max>
  </Triangular>
  ...
</Distributions>
----------------------------
\end{lstlisting}

%%%%%% Uniform
\paragraph{Uniform Distribution}
\label{Uniform}
The \distname{Uniform} distribution is a continuous distribution with a rectangular shaped Pdf. It is often used where the distribution is only vaguely known, but upper and lower limits are ``known''. Its support is $lower \le x \le upper$.
\\ The specifications of this distribution must be defined within the xml block $<Uniform>$. This xml-node needs to contain the attribute:
\vspace{-5mm}
\begin{itemize}
\itemsep0em
\item \textbf{name}, \textit{required string attribute}, user-defined name of this distribution. N.B. As for the other objects, this is the name that can be used to refer to this specific entity from other input blocks (xml).   
\end{itemize}
\vspace{-5mm}
This distribution can be initialized through the following keyword/s:
\begin{itemize}
\item $<low>$, float, required parameter, domain lower boundary;
\item $<high>$, float, required parameter, domain upper boundary.
\end{itemize}

\begin{lstlisting}[style=XML]
----------------------------
Example:
----------------------------
<Distributions>
  ...
  <Uniform name='...'>
    <low>***</low>
    <high>***</high>
  </Uniform>
  ...
</Distributions>
----------------------------
\end{lstlisting}

%%%%%% Weibull
\paragraph{Weibull Distribution}
\label{Weibull}
The \distname{Weibull} distribution is a continuous distribution that is often used in the field of failure analysis; in particular it can mimic distributions where the failure rate varies over time. If the failure rate is:
\vspace{-5mm}
\begin{itemize}
\itemsep0em
\item constant over time, then $k = 1$, suggests that items are failing from random events;   
\item decreases over time, then $k < 1$, suggesting ``infant mortality''; 
\item increases over time, then $k > 1$, suggesting ``wear out'' - more likely to fail as time goes by.
\end{itemize}
\vspace{-5mm}
 Its support is $x \in [0, +\inf)$.
\\ The specifications of this distribution must be defined within the xml block $<Weibull>$. This xml-node needs to contain the attribute:
\vspace{-5mm}
\begin{itemize}
\itemsep0em
\item \textbf{name}, \textit{required string attribute}, user-defined name of this distribution. N.B. As for the other objects, this is the name that can be used to refer to this specific entity from other input blocks (xml).   
\end{itemize}
\vspace{-5mm}
This distribution can be initialized through the following keyword/s:
\begin{itemize}
\item $<k>$, float, required parameter, shape parameter;
\item $<lambda>$, float, required parameter, scale parameter.
\end{itemize}

\begin{lstlisting}[style=XML]
----------------------------
Example:
----------------------------
<Distributions>
  ...
  <Weibull name='...'>
    <lambda>***</lambda>
    <k>***</k>
  </Weibull>
  ...
</Distributions>
----------------------------
\end{lstlisting}

%%%%%% paragraph 1-Dimensional Discrete Distributions.
\subsubsection{1-Dimensional Discrete Distributions.}
\label{subsubsec:1DDiscrete}
RAVEN currently supports 3 discrete distributions. In the following paragraphs, the input requirements are reported.
%%%%%% Bernoulli
\paragraph{Bernoulli Distribution}
\label{Bernoulli}
The \distname{Bernoulli} distribution is a discrete distribution of the outcome of a single trial with only two results, 0 (failure) or 1 (success), with a probability of success \distattrib{p}. It is the simplest building block on which other discrete distributions of sequences of independent Bernoulli trials can be based. Basically, it is the binomial distribution (k = 1, \distattrib{p}) with only one trial.  Its support is $k \in {0, 1}$.
\\ The specifications of this distribution must be defined within the xml block $<Bernoulli>$. This xml-node needs to contain the attribute:
\vspace{-5mm}
\begin{itemize}
\itemsep0em
\item \textbf{name}, \textit{required integer attribute}, Name of this distribution. As for the other objects, this is the name that can be used to refer to this specific entity in other input blocks (xml).   
\end{itemize}
\vspace{-5mm}
This distribution can be initialized through the following keyword/s:
\begin{itemize}
\item $<p>$, float, required parameter, probability of success.
 \end{itemize}
\begin{lstlisting}[style=XML]
----------------------------
Example:
----------------------------
<Distributions>
  ...
  <Bernoulli name='...'>
    <p>***</p>
  </Bernoulli>
  ...
</Distributions>
----------------------------
\end{lstlisting}

%%%%%% Binomial
\paragraph{Binomial Distribution}
\label{Binomial}
The \distname{Binomial} distribution is the discrete probability distribution of the number of successes in a sequence of \distattrib{n} independent yes/no experiments, each of which yields success with probability \distattrib{p}. Its support is $k \in {0, 1, 2, ..., n}$.
\\ The specifications of this distribution must be defined within the xml block $<Binomial>$. This xml-node needs to contain the attribute:
\vspace{-5mm}
\begin{itemize}
\itemsep0em
\item \textbf{name}, \textit{required string attribute}, user-defined name of this distribution. N.B. As for the other objects, this is the name that can be used to refer to this specific entity from other input blocks (xml).   
\end{itemize}
\vspace{-5mm}
This distribution can be initialized through the following keyword/s:
\begin{itemize}
\item $<p>$, float, required parameter,  probability of success;
\item $<n>$, integer, required parameter, number of experiment.
\end{itemize}

\begin{lstlisting}[style=XML]
----------------------------
Example:
----------------------------
<Distributions>
  ...
  <Binomial name='...'>
    <n>***</n>
    <p>***</p>
  </Binomial>
  ...
</Distributions>
----------------------------
\end{lstlisting}

%%%%%% Poisson
\paragraph{Poisson Distribution}
\label{Poisson}
The \distname{Poisson} distribution is a discrete probability distribution that expresses the probability of a given number of events occurring in a fixed interval of time and/or space if these events occur with a known average rate and independently of the time since the last event. Its support is $k \in {1, 2, 3, 4, ...}$.
\\ The specifications of this distribution must be defined within the xml block $<Poisson>$. This xml-node needs to contain the attribute:
\vspace{-5mm}
\begin{itemize}
\itemsep0em
\item \textbf{name}, \textit{required string attribute}, user-defined name of this distribution. N.B. As for the other objects, this is the name that can be used to refer to this specific entity from other input blocks (xml).   
\end{itemize}
\vspace{-5mm}
This distribution can be initialized through the following keyword/s:
\begin{itemize}
\item $<mu>$, float, required parameter, mean rate of events/time.
\end{itemize}

\begin{lstlisting}[style=XML]
----------------------------
Example:
----------------------------
<Distributions>
  ...
  <Poisson name='...'>
    <mu>***</mu>
  </Poisson>
  ...
</Distributions>
----------------------------
\end{lstlisting}

%%%%%% N-Dimensional Probability distributions
\subsection{N-Dimensional Probability Distributions}
\label{subsec:NdDist}
We have the MultiVariate Normal distributions and 3 different type of user-input ND distribution. These types depend on the type of interpolation scheme that the user request. The inputs requirements are explained in the following:
\\Diego I choose you!


\section{Samplers  \\ \vspace{2 mm} {\small }}
\label{sec:Samplers}
The sampler is probably the most important entity in the RAVEN framework. Indeed, it performs the driving of the specific sampling strategy and, hence, determines the effectiveness of the analysis, from both an accuracy and computational point of view.  The samplers, that are available in RAVEN, can be categorized in three main classes:
\begin{itemize}
\item \textbf{Once-through}
\item \textbf{Dynamic Event Tree (DET)}
\item \textbf{Adaptive}
\end{itemize}
%%% Once-Through Samplers
\subsection{Once-through Samplers.}
\label{subsec:onceThroughSamplers}
The once-through sampler category collects all the strategies that perform the sampling of the input space without exploiting, through dynamic learning approaches, the information made available from the outcomes of calculation previously performed (adaptive sampling) and the common system evolution (patterns) that different sampled calculations can generate in the phase space (dynamic event tree). 
In the RAVEN framework, five different and well-known “once-through” samplers are available: 
\begin{itemize}
\item \textbf{Monte Carlo (MC)}
\item \textbf{Stratified}
\item \textbf{Grid Based}
\item \textbf{Response Surface Design of Experiment}
\item \textbf{Factorial Design of Experiment}
\end{itemize}
From a pratical point of view, these sampling strategies represent different ways to perturb the input space. In the following paragraphs, the input requirements and a small explaination of the different sampling methodologies are reported.
%%% Once-Through Samplers: MonteCarlo
\subsubsection{Monte Carlo.}
\label{subsubsubsec:MC}







Samplers own the sampling strategy (Type) and they generate the input values using the associate distribution. They do not have distributions inside.

There are different kind of samplers:
\begin{itemize}
\item \textbf{MonteCarlo}
\item \textbf{Grid}
\item \textbf{LHS}
\item \textbf{Adaptive}
\item \textbf{DynamicEventTree}
\item \textbf{AdaptiveDynamicEventTree}
\end{itemize}

For MonteCarlo method: 
\begin{itemize}
\item \textbf{MonteCarlo}
\begin{itemize}
\item name: name of sampling method used
\item limit: number of samplings the MonteCarlo method will use
\item initial seed: (optional) initial number of the iterations
\end{itemize}
\item variable 
\begin{itemize}
\item name: name of the variable that the code will do the sampling of
\end{itemize}
\item distribution: 
\begin{itemize}
\item what kind of distribution the variable follows (it is chosen from the distribution block)
\end{itemize}
\end{itemize}

Example:

\begin{lstlisting}[style=XML]
<MonteCarlo name='MC_Sampler' limit='1000'> 
 <variable name='pressure'> 
  <distribution>***</distribution> 	
 </variable> 
</MonteCarlo> 
\end{lstlisting}
In the example the code is sampling 1000 times (limit) the variable "pressure" with a distribution taken from the distribution block.

For the Grid method:
\begin{itemize}
\item \textbf{Grid}
\begin{itemize}
\item name: name of the sampling method used
\item initial seed (optional)
\end{itemize}
\item variable
\begin{itemize}
\item name: name of the variable that the code will do the sampling of
\end{itemize}
\item distribution: what kind of distribution the variable follows (it is chosen from the distribution block) 
\item grid
\begin{itemize}
\item type
\begin{itemize}
\item value
\item CDF
\end{itemize}
\item construction 
\begin{itemize}
\item equal
\begin{itemize}
\item the size of the step, given in the input node, is the same for all steps
\item requires lowerbound or upperbound
\item requires steps
\end{itemize}
\item custom
\begin{itemize} 
\item no lowerbound or upperbound 
\item no number of steps
\item in the input node it is necessary to input each step of the grid separated by a space
\end{itemize}
\end{itemize}
\item lowerbound or upperbound
\item steps
\item input node
\end{itemize}
\end{itemize}

Example: 

\begin{lstlisting}[style=XML]
<Grid name='Grid_Sampler'> 
 <variable name='pressure'>  
  <distribution>***</distribution> 
  <grid	type='value' construction='equal' steps='100' lowerBound='1.0'>0.2</grid>   
 </variable> 
</Grid> 
\end{lstlisting}
In the example the code is sampling the variable "pressure" with a distribution "***" which was chosen by the distribution block. The grid ranges from 1.0 (lowerbound) to 21.0 (100 steps of equal size of 0.2).  
% % % % % % % % % % % % % % % % % % % % % % % % % % % % % % % % % %
\\
For the LHS method:
\begin{itemize}
\item \textbf{LHS}
\begin{itemize}
\item name: name of the sampling method used
\item initial seed (optional)
\end{itemize}
\item variable
\begin{itemize}
\item name: name of the variable that the code will do the sampling of
\end{itemize}
\item distribution: what kind of distribution the variable follows (it is chosen from the distribution block) 
\item grid
\begin{itemize}
\item type
\begin{itemize}
\item value
\item CDF
\end{itemize}
\item construction 
\begin{itemize}
\item equal
\begin{itemize}
\item the size of the step, given in the input node, is the same for all steps
\item requires lowerbound or upperbound
\item requires steps
\end{itemize}
\item custom
\begin{itemize} 
\item no lowerbound or upperbound 
\item no number of steps
\item in the input node it is necessary to input each step of the grid separated by a space
\end{itemize}
\end{itemize}
\item lowerbound or upperbound
\item steps
\item input node
\end{itemize}
\end{itemize}

Example:
\begin{lstlisting}[style=XML]
<LHS name='***' initial_seed='***'> 
 <variable name='***'>  
  <distribution>***</distribution> 	
  <grid	type='***' construction='***' steps='***' lowerBound='***'>****</grid>   
 </variable> 
</LHS> 
\end{lstlisting}
The input structure is the same as the \textbf{Grid} input structure

For the Adaptive method:
\begin{itemize}
\item \textbf{Adaptive}
\begin{itemize}
\item name
\item initial seed (optional)
\end{itemize}
\item Convergence
\begin{itemize}
\item limit: 'Integer'
\item persistence: 'Integer'
\item weight='probability' or 'None': 
\item subGridTol='None' or 'Float' :This is the tolerance used to construct the testing sub grid
\item forcelteration='True' or 'False': this flag control if at least a self.limit number of iteration should be done
\end{itemize}
\item variable
\item distribution
\end{itemize}

\begin{lstlisting}[style=XML]
<Adaptive name='***' initial_seed='***'> 
 <Convergence limit='***' persistence='***' weight='***' subGridTol='***' forcelteration='***'>***</Convergence>  
  <variable name='***'>
   <distribution>***</distribution>
  </variable> 
</Adaptive>   
\end{lstlisting}



For DynamicEventTree:
***

For AdaptiveDynamicEventTree:
***



























\section{Functions  \\ \vspace{2 mm} {\small }}

This module contains interfaces to import external functions

\begin{itemize}
\item \textbf{External}
\begin{itemize}
\item name=name of the function
\item file=file name where the function is
\end{itemize}
\item variable
	\begin{itemize}
	\item type=numpy.float64
	\item variable inside the function that has been defined
	\end{itemize}
\end{itemize}






\begin{lstlisting}[style=XML]
<Functions>
 <External name='***' file='***'>
  <variable type='***'>***</variable>
 </External>
</Functions>
\end{lstlisting}



\section{Models  \\ \vspace{2 mm} {\small }}
\label{sec:models}
In the RAVEN code a crucial entity is represented by a Model. A model is an object that employs a mathematical model used to re

The xml section "models" contains the information regarding the code employed in the analysis (e.g., RAVEN/RELAP-7, RELAP-5 or an external model). A model is something that given an input will return an output reproducing some physical model it could be as complex as a stand alone code, a reduced order model trained somehow or something externally build and imported by the user.
The available models are:
\begin{description}
\item [Dummy:] it is a dummy model that just return the effect of the sampler. The values reported as input in the output are the output of the sampler and the output is the counter of the performed sampling
\item [ROM:] ROM stands for Reduced Order Model. All the models here, first learn than predict the outcome
\item [ExternalModel:] this model allows to interface with an external python module
\item [Code:] generic class that imports an external code into the framework
\item [Projector:] generic data manipulator
\item [PostProcessor:] an Action System. All the models here, take an input and perform an action
\end{description}

\subsection{Dummy}
\label{sec:models_dummy}

Description

Summary

Example

\subsection{ROM}
\label{subsec:models_ROM}

Description

Summary 
\begin{itemize}
\item name: name of the ROM model
\item subType: 'SciKitLearn'. Imports the libraries from scikitlearn
\item Features: input the variables set in the Samplers block separated by a comma
\item SKLtype: input a model from http://scikit-learn.org/stable/modules/classes.html (LinearRegression in the example)
\item Target: Input a set of variables inside the output space which the ROM has to be developed for. 
\item Parameters: From the SKLtype class are defined all the parameters required ($fit\_intercept$ and $normalize$ in the exemple) 
\end{itemize}

Example

\begin{lstlisting}[style=XML]
<Models>
 <ROM name='***' subType='***'>
  <Features>***,***</Features>
  <SKLtype>linear_model|LinearRegression</SKLtype>
  <Target>***</Target>
  <fit_intercept>***</fit_intercept>
  <normalize>***</normalize>
 </ROM>
\end{lstlisting}




\subsection{External Model}
\label{subsec:models_externalModel}

Description

Summary

Example
As an example we use the external model shown in lorentzAttractor.py which, given the 3-dimensional initial coordinates (x0, y0, z0), calculate the trajectory of a Lorentz attractor in the time interval $[0.0,0.03]$ seconds.
We want to perform sampling of the 3-dimensional initial conditions of the attractor using classical Monte-Carlo sampling.
The user is required to specify:
\begin{itemize}
\item the initialize function: def initialize(self,runInfoDict,inputFiles)
\item the function which create a new input: def createNewInput(self,myInput,samplerType,**Kwargs)
\item the function which perform the actual calculation: def run(self,Input)
\end{itemize}

\begin{python}
def initialize(self,runInfoDict,inputFiles):
  self.SampledVars = None
  self.sigma = 10.0
  self.rho   = 28.0
  self.beta  = 8.0/3.0
  return

def createNewInput(self,myInput,samplerType,**Kwargs):
  return Kwargs['SampledVars']

def run(self,Input):
   ...
\end{python}


\begin{lstlisting}[style=XML]
<Models>
    <ExternalModel name='PythonModule' subType='' ModuleToLoad='externalModel/lorentzAttractor'>  
       <variable type='float'>sigma</variable>
       <variable type='float'>rho</variable>
       <variable type='float'>beta</variable>
       <variable type='numpy.ndarray'>x</variable>
       <variable type='numpy.ndarray'>y</variable>
       <variable type='numpy.ndarray'>z</variable>
       <variable type='numpy.ndarray'>time</variable>
       <variable type='float'>x0</variable>
       <variable type='float'>y0</variable>
       <variable type='float'>z0</variable>
    </ExternalModel>
</Models> 
\end{lstlisting}

\subsection{Code}
\label{sec:models_code}

Description: This is the generic class that import an external code into the framework

Summary

Example

\subsection{Projector}
\label{sec:models_projector}

Description

Summary

Example

\subsection{PostProcessor}
\label{sec:models_postProcessor}

Description

List variable, Input Data, 
Keyword sul tipo analisi statistica!!

Summary

Example

\section{Steps  \\ \vspace{2 mm} {\small }}
\label{sec:steps}
The core of the RAVEN calculation flow is employed in the so called \textbf{Step} system. The \textbf{Step} is in charge to assemble the different ``entities'' in RAVEN (e.g. Samplers, Models, DataBases, etc.) in order to perform a task that is defined by the kind of Step is under usage. A sequence of different \textbf{Steps} represents the calculation flow and core of the analysis the user wants to perform. 
\\Before analyzing each \textbf{Step} type, it is worth to briefly explain how a general Step entity is organized and the concept of ``role'' in its definition.
\\In the following example, a general example of a Step is reported:
\begin{lstlisting}[style=XML]
--------------------------------------------
<Simulation>
  ...
  <Steps>
    ...
    <WhatEverStepType name='***'>
        <Role1 class='***'     type='***'   >***</Role1>
        <Role2 class='***'     type='***'   >***</Role2>
        <Role3 class='***'     type='***'   >***</Role3>
        <Role4 class='***'     type='***'   >***</Role4>
    </WhatEverStepType>
    ...
  </Steps>
  ...
</Simulation>
--------------------------------------------
\end{lstlisting}
As it can be noticed above, independently of the type, each textbf{Step}  is basically constituted by a list of entities organized in ``Roles''.  Each role represents a ``behavior'' the entity (object) is going to take during the evolution of the Step.
In RAVEN several different ``roles'' are available:
\begin{itemize}
\item \textbf{Input.} As the name suggests, it represents the input of the Step. The objects usable as Inputs depends on the type of \textbf{Model} in the Step;
\item \textbf{Output.} The Output represents the container of an action performed by the \textbf{Model}. It can generally be of type \textbf{Datas}, \textbf{DataBases}, \textbf{OutStreamManager};
\item \textbf{Model.} The Model is the actual entity that represents a physical or mathematical representation of a system or behavior. The object that is used in this role, defines the compatibility of the different Inputs and Outputs listed in this step;
\item \textbf{Sampler.} It is the role that defines the Sampling strategy that needs to be performed. 
\item \textbf{Function.} The Function role is extremely important, for example, when performing Adaptive Sampling to represent the metric of the transition regions. This role is the role used, for example, to collapse information coming from a Model.
\item \textbf{ROM.} Acceleration Reduced Order Model;
\item \textbf{SolutionExport,} It represents the container of the eventual output of a Sampler.
\end{itemize}
As understandable, depending on the \textbf{Step} type, different combinations of these roles are allowed to be used.
For this reason, it is important to analyze each \textbf{Step} type in details.
%__interFaceDict['MultiRun'         ] = MultiRun
%__interFaceDict['IOStep'           ] = IOStep
%__interFaceDict['IODataBase'       ] = IOStep
%__interFaceDict['RomTrainer'       ] = RomTrainer
%__interFaceDict['PostProcess'      ] = SingleRun
%__interFaceDict['OutStreamStep'    ] = IOStep
%%%%%%%%%%%%%%%%%%%%
 %%%%% SINGLERUN %%%%%
%%%%%%%%%%%%%%%%%%%%
\subsection{SingleRun}
\label{subsec:stepSingleRun}
The  \textbf{SingleRun} is the simplest step the user can use to assemble a calculation flow. It is aimed to perform a single  action, employed by a \textbf{Model}. For example, it can be used to run a single job (Code Model) and collect the outcomes in a ``Datas'' of type ``TimePoint'' or ``History''.
\\ The specifications of this Step must be defined within the xml block $<SingleRun>$. This xml-node needs to contain the attributes:
\vspace{-5mm}
\begin{itemize}
\itemsep0em
\item \textbf{name}, \textit{required string attribute}, user-defined name of this Step. N.B. As for the other objects, this is the name that can be used to refer to this specific entity in the \textit{RunInfo} block, under the xml-node $<Sequence>$;
\item \textbf{pauseAtEnd}, \textit{optional boolean/string attribute}, if True (True values = True, yes, y, t, si, dajie), in case one, or more, of the Outputs is/are of type ``OutStreamManager'', the code is going to pause at the end of the step, waiting for an user signal to continue. For example, it can be used when a OutStreamManager of type Plot is supposed to be output on the screen, in order to be able to interact with the Plot itself (e.g. rotate the figure, change the scale, etc.).  \textit{Default = False};
\end{itemize}
\vspace{-5mm}
In the \textbf{SingleRun} input block, the user needs to specify the objects that need to be used for the different allowable roles. This step accepts the following roles:
\begin{itemize}
\item $<Input>$, string, required paramete . Name of the ``entity'' that is going to be used as input for the model specified in this step. This xml node needs to contain the following attributes:
\begin{itemize}
  \item \textbf{class}, \textit{required string attribute}, main object class type. The string, required here, corresponds to the tag of the main objects type used in the input. For example, ``Files'', ``Datas'', ``DataBases'', etc;
  \item \textbf{type}, \textit{required string attribute}, the actual entity type. This attribute needs to specify the object type within the main object class. For example, if the  \textbf{class} attribute is ``Datas'', the \textbf{type} attribute might be ``TimePointSet''. NB. The class ``Files'' has no type (i.e. \textbf{type = ``''}).
\end{itemize}
NB. The \textbf{class} and, consequentially,  the \textbf{type} usable for this role depends on the particular $<Model>$ is going to be used. In addition, the user can specify as many $<Input>$ as needed by the model;
\item $<Model>$, string, required parameter. Name of the ``entity'' that is going to be used as Model. This xml node needs to contain the following attributes:
\begin{itemize}
  \item \textbf{class}, \textit{required string attribute}, main object class type. The string, required here, corresponds to the tag of the main objects type used in the input. For this role, only ``Models'' can be used;
  \item \textbf{type}, \textit{required string attribute}, the actual entity type. This attribute needs to specify the object type within the ``Models'' object class. For example, the \textbf{type} attribute might be ``Code'', ``ROM'', etc.
\end{itemize}
\item $<Output>$, string, required parameter. Name of the ``entity'' that is going to be used as Output for the Model. This xml node needs to contain the following attributes:
\begin{itemize}
  \item \textbf{class}, \textit{required string attribute}, main object class type. The string, required here, corresponds to the tag of the main objects type used in the input. For this role, only ``Datas'', ``DataBases'' and ``OutStreamManager'' can be used;
  \item \textbf{type}, \textit{required string attribute}, the actual entity type. This attribute needs to specify the object type within the main object class. For example, if the  \textbf{class} attribute is ``Datas'', the \textbf{type} attribute might be ``TimePointSet''.
\end{itemize}
NB. The number of $<Output>$ nodes is unlimited.
\end{itemize}

\begin{lstlisting}[style=XML]
---------------------------------------------------------
Example:
---------------------------------------------------------
<Steps>
  ...
  <SingleRun name='StepName' pauseAtEnd='false'> 
        <Input   class='Files'     type=''>anInputFile.i</Input>
        <Input   class='Files'     type=''>anotherFileNeededByTheCode</Input>
        <Model   class='Models'    type='Code'      >aCode</Model>
        <Output  class='DataBases' type='HDF5' >aDataBase</Output>
        <Output  class='Datas'     type='History' >aData</Output>
  </SingleRun>
  ...
</Steps>
---------------------------------------------------------
\end{lstlisting}
%%%%%%%%%%%%%%%%%%%
 %%%%% MULTIRUN %%%%%
%%%%%%%%%%%%%%%%%%%
\subsection{SingleRun}
\label{subsec:stepSingleRun}
The  \textbf{SingleRun} is the simplest step the user can use to assemble a calculation flow. It is aimed to perform a single  action, employed by a \textbf{Model}. For example, it can be used to run a single job (Code Model) and collect the outcomes in a ``Datas'' of type ``TimePoint'' or ``History''.
\\ The specifications of this Step must be defined within the xml block $<SingleRun>$. This xml-node needs to contain the attributes:
\vspace{-5mm}
\begin{itemize}
\itemsep0em
\item \textbf{name}, \textit{required string attribute}, user-defined name of this Step. N.B. As for the other objects, this is the name that can be used to refer to this specific entity in the \textit{RunInfo} block, under the xml-node $<Sequence>$;
\item \textbf{pauseAtEnd}, \textit{optional boolean/string attribute}, if True (True values = True, yes, y, t, si, dajie), in case one, or more, of the Outputs is/are of type ``OutStreamManager'', the code is going to pause at the end of the step, waiting for an user signal to continue. For example, it can be used when a OutStreamManager of type Plot is supposed to be output on the screen, in order to be able to interact with the Plot itself (e.g. rotate the figure, change the scale, etc.).  \textit{Default = False};
\end{itemize}
\vspace{-5mm}
In the \textbf{SingleRun} input block, the user needs to specify the objects that need to be used for the different allowable roles. This step accepts the following roles:
\begin{itemize}
\item $<Input>$, string, required paramete . Name of the ``entity'' that is going to be used as input for the model specified in this step. This xml node needs to contain the following attributes:
\begin{itemize}
  \item \textbf{class}, \textit{required string attribute}, main object class type. The string, required here, corresponds to the tag of the main objects type used in the input. For example, ``Files'', ``Datas'', ``DataBases'', etc;
  \item \textbf{type}, \textit{required string attribute}, the actual entity type. This attribute needs to specify the object type within the main object class. For example, if the  \textbf{class} attribute is ``Datas'', the \textbf{type} attribute might be ``TimePointSet''. NB. The class ``Files'' has no type (i.e. \textbf{type = ``''}).
\end{itemize}
NB. The \textbf{class} and, consequentially,  the \textbf{type} usable for this role depends on the particular $<Model>$ is going to be used. In addition, the user can specify as many $<Input>$ as needed by the model;
\item $<Model>$, string, required parameter. Name of the ``entity'' that is going to be used as Model. This xml node needs to contain the following attributes:
\begin{itemize}
  \item \textbf{class}, \textit{required string attribute}, main object class type. The string, required here, corresponds to the tag of the main objects type used in the input. For this role, only ``Models'' can be used;
  \item \textbf{type}, \textit{required string attribute}, the actual entity type. This attribute needs to specify the object type within the ``Models'' object class. For example, the \textbf{type} attribute might be ``Code'', ``ROM'', etc.
\end{itemize}
\item $<Output>$, string, required parameter. Name of the ``entity'' that is going to be used as Output for the Model. This xml node needs to contain the following attributes:
\begin{itemize}
  \item \textbf{class}, \textit{required string attribute}, main object class type. The string, required here, corresponds to the tag of the main objects type used in the input. For this role, only ``Datas'', ``DataBases'' and ``OutStreamManager'' can be used;
  \item \textbf{type}, \textit{required string attribute}, the actual entity type. This attribute needs to specify the object type within the main object class. For example, if the  \textbf{class} attribute is ``Datas'', the \textbf{type} attribute might be ``TimePointSet''.
\end{itemize}
NB. The number of $<Output>$ nodes is unlimited.
\end{itemize}

\begin{lstlisting}[style=XML]
---------------------------------------------------------
Example:
---------------------------------------------------------
<Steps>
  ...
  <SingleRun name='StepName' pauseAtEnd='false'> 
        <Input   class='Files'     type=''>anInputFile.i</Input>
        <Input   class='Files'     type=''>anotherFileNeededByTheCode</Input>
        <Model   class='Models'    type='Code'      >aCode</Model>
        <Output  class='DataBases' type='HDF5' >aDataBase</Output>
        <Output  class='Datas'     type='History' >aData</Output>
  </SingleRun>
  ...
</Steps>
---------------------------------------------------------
\end{lstlisting}



\begin{itemize}
%%%%%%%%%%%%%%%%%%%%%%%%%%%%%%%%%%%%%%%%%%%%%%%%%%%%%%
\item MultiRun: This class implements one step of the simulation pattern where several runs are, needed without being adaptive.  
\begin{itemize}
\item
	\begin{itemize}
	\item name = name of the step (sequence) defined in the RunInfo block, under the Sequence card
	\end{itemize}
\item pauseAtEnd= if True the code will not go to next step until plots are closed manually by the user
\end{itemize}
\begin{itemize}
\item Sampler: 
\begin {itemize}
\item class=Samplers 
\item type: the type of sampler used in the Samplers block 
\item name of the sampler
\end{itemize}
\item Model: 
\begin{itemize}
\item class= (Models) 
\item type=(dummy, ROM, External Model, Code, Projector, PostProcessor) 
\item name of the model
\end{itemize}
\item Input:  
\begin{itemize}
\item class=(Data e Files)
\item type:
\begin{itemize}
\item if class = files ----$>$ none
\item if class = Data  ----$>$ timepoint, timepointset, historie, histories
\end{itemize}
\end{itemize}
\item Output: 
\begin{itemize}
\item class: they are the output destinations
\begin{itemize}
\item Datas 
\item OutStreamManager
\end{itemize} 
\item type:
\begin{itemize}
\item if class=Datas ----$>$(timepoint, timepointset, historie, histories) 
\item if class=OutstreamManager ---$>$ type:print, plot
\end{itemize} 
\item name of the output  
\end{itemize}

\end{itemize}
\begin{lstlisting}[style=XML]
<MultiRun name='***' pauseAtEnd='***'>
 <Sampler class='Samplers' type='***'>***</Sampler>
 <Input class='***' type='***'>***</Input>
 <Model class='Models' type='***'>***</Model>
 <Output class='Datas'*** type='***'</Output>
 <Output class='OutStreamManager' type='***'>***</Output>
</MultiRun>
\end{lstlisting}
%%%%%%%%%%%%%%%%%%%%%%%%%%%%%%%%%%%%%%%%%%%%%%%%%%%%
\item Adaptive: this class implement one step of the simulation pattern where several runs are needed in an adaptive scheme
\begin{itemize}
\item Sampler: class=samplers type=Adaptive>???<
\item Model: class=Models type=(dummy, ROM, External Model, Code, Projector, PostProcessor)
\item Function: it takes in a datas and generate the value of the goal functions, it gives the criteria for which i represent the limit surface (class=Functions)
\item Input:
\item TargetEvaluation: is the output datas that is used for the evaluation of the goal function, it represents the sampling points. It has to be declared among the outputs. 
\item SolutionExport: if declared it is used to export the location of the goal functions = 0, it exports the limit surface
\item ROM: is boolean, it selects values of input (through sampling) to find the limit surface through the values that were given by the function. 
\item Output:
\end{itemize}
\begin{lstlisting}[style=XML]
<Adaptive name='***' pauseAtEnd='***'> 
 <Input class = 'Datas' type = 'TimePointSet'>***</Input>
 <Sampler class = 'Samplers' type = 'Adaptive'>***</Sampler>
 <TargetEvaluation class = 'Datas' type = '***'>***</TargetEvaluation>
\end{lstlisting}
%%%%%%%%%%%%%%%%%%%%%%%%%%%%%%%%%%%%%%%%%%%%%%%%%%%%%%%
\item IODataBase: This step type is used only to extract or push information from/into a DataBase. If in the Databases block the Database is created (no directory or filename) the then the Databse will be an output of this block, otherwise, if it is uploaded then in this block it will be a Input
\begin{itemize}
\item Input: class=(Datas or Databases) type=(if Databases, HDF5, if Datas:timepoint, timepointset, historie, histories) 
\item Output: class=(Datas or Databases) type=(if Databases, HDF5, if Datas:timepoint, timepointset, historie, histories) 
\end{itemize}
%%%%%%%%%%%%%%%%%%%%%%%%%%%%%%%%%%%%%%%%%%%%%%%%%%%%%%%%
\label{subsec:stepTraining}
\item RomTrainer: This step type is used only to train a ROM. 
\begin{lstlisting}[style=XML]
<RomTrainer name='***'>
 <Input   class='Datas' type='TimePointSet'>***</Input>
\end{lstlisting}
%%%%%%%%%%%%%%%%%%%%%%%%%%%%%%%%%%%%%%%%%%%%%%%%%%%%%%%%
\item PostProcess:
\item OutStreamStep:
\end{itemize}












\section{Datas  \\ \vspace{2 mm} {\small }}
\label{sec:Datas}
As it could infer from the previous chapters,  in the RAVEN code different entities interact to each other in order to create, ideally, an infinite number of different calculation flows. These interactions are performed through a system that is ``understandable'' by each ``entity''. This system, neglecting the grammar imprecision,  is called ``Datas'' system. The ``Datas'' system is a container of Data objects of different type that can be constructed during the evolution of the particular calculation flow, can be used as input or output of particular Model (see Roles' meaning in section \ref{sec:models}), etc. 
Currently, RAVEN support 4 different data types, each with a particular conceptual meaning:
\begin{itemize}
\item \textbf{TimePoint}: The \textit{TimePoint} entity, as the name suggests, ``describes'' the state of the system in a certain point in time. In other words, it can be considered a mapping between a set of parameters in the input space and the resulting outcomes in the output space at a particular time;
\item \textbf{TimePointSet}: The \textit{TimePointSet} object is, obviously, a collection of single \textit{TimePoint(s)}. It can be considered a mapping between multiple sets of parameters in the input space and the resulting sets of outcomes in the output space at a particular point in time;
\item \textbf{History}:  The \textit{History} entity ``describes'' the temporal evolution of the state of the system within a certain input domain;
\item \textbf{Histories}:  The \textit{Histories} object is, obviously, a collection of single \textit{History(ies)}. t can be considered a mapping between multiple sets of parameters in the input space and the resulting sets of temporal evolutions in the output space.
\end{itemize}
As inferable  from the brief description reported above, each ``Data'' object represents a mapping between a set of parameters and the resulting outcomes.
The Data objects are defined within the main XML block called $<Datas>$:
\begin{lstlisting}[style=XML]
-----------------------------
<Simulation>
   ...
  <Datas> 
    <WhatEverData name='***'>  
     ... 
    </WhatEverData> 
  </Datas>
   ...
</Simulation>
-----------------------------
\end{lstlisting}

The specifications of each ``Data'' type must be defined within:
\begin{itemize}
   \item \textbf{TimePoint} $=>$ $<TimePoint>$
   \item \textbf{TimePointSet} $=>$ $<TimePointSet>$
   \item \textbf{History} $=>$ $<History>$
   \item \textbf{Histories} $=>$ $<Histories>$
\end{itemize}
Independently on the type of Data, the respective XML node needs (or not) to contain the attribute:
\vspace{-5mm}
\begin{itemize}
\itemsep0em
\item \textbf{name}, \textit{required string attribute}, user-defined name of this Data. N.B. As for the other objects, this is the name that can be used to refer to this specific entity from other input blocks (xml);
% Regarding the time attribute, we need to take a better decision... Now it is very confusing.
%\item \textbf{time}, \textit{optional float or string attribute}, time attribute. The user can here specify either the time (value) at which the outcomes need to be taken (History-like object, it represents the time from which the outcomes' evolution need to be tracked) or a string  that can be either ``end'', at the end of the hystory, or ``all'', consider     . \textit{Default = random seed};
\item \textbf{inputTs}, \textit{optional integer attribute}, This attribute can be used to specify at which ``time step'' the input space needs to be retrieved. NB. If the user wants to take those conditions from the end of the simulation, it can directly input ``-1''. \textit{Default = 0};
\item \textbf{operator}, \textit{optional string attribute}, The operator attribute is aimed to perform simple operations on the data to be stored. The 3 options currently available are \textit{operator = max}, \textit{operator = min}, \textit{operator = average}, with obvious meaning. \textit{Default = None};
\item \textbf{hierarchical}, \textit{optional boolean attribute}, If True this data is going to be constructed, if possible, in an hierarchical fashion. \textit{Default = False};
\end{itemize}
\vspace{-5mm}
In each XML node (e.g. $<TimePoint>$ or $<Histories>$), the user needs to specify the following sub nodes:
\begin{itemize}
 \item $<Inputs>$\textbf{\textit{, comma separated string, required field.}}.  List of input parameters this data is connected to. NB. In case the ``Data'' type is either \textit{TimePoint} or  \textit{History}, this XML node must contain the attribute \textbf{history}, where the name of the associated history needs to be placed;
 \item $<Outputs>$\textbf{\textit{, comma separated string, required field.}}.  List of output parameters this data is connected to;
\end{itemize}

\begin{lstlisting}[style=XML]
----------------------------------------------------------
<Datas> 
   <TimePoint name='***' inputTs='-1' operator='max' hierarchical='False'>  
      <Input history='a_history_name'>***,***,***</Input>
      <Output>***,***</Output>
   </TimePoint> 
   <TimePointSet name='***'>  
      <Input>***,***,***</Input>
      <Output>***,***</Output>
    </TimePointSet> 
    <History name='***'>  
      <Input history='a_history_name'>***,***,***</Input>
      <Output>***,***</Output>
    </History> 
    <Histories name='***'>  
      <Input>***,***,***</Input>
      <Output>***,***</Output>
    </Histories> 
</Datas>
----------------------------------------------------------
\end{lstlisting}

\section{Databases}
\label{sec:Databases}
The RAVEN framework provides the capability to store and retrieve data to/from an external database. Currently RAVEN has support for only a database type called \textbf{HDF5}. This database, depending on the data format is receiving, will organize itself in a ``parallel'' or ``hierarchical'' fashion. The user can create as many database objects as needed.
The DataBase objects are defined  within the main XML block called $<DataBases>$:
\begin{lstlisting}[style=XML]
-----------------------------
<Simulation>
   ...
  <DataBases>
        ...
        <HDF5 name="***"/>
        <HDF5 name="***"/>
        ...
  </DataBases>
   ...
</Simulation>
-----------------------------
\end{lstlisting}
The specifications of each DataBase of type HDF5 needs to be defined within the XML block $<HDF5>$, that  needs (or not) to contain  the attributes:
\vspace{-5mm}
\begin{itemize}
\itemsep0em
\item \textbf{name}, \textit{required string attribute}, user-defined name of this Data. N.B. As for the other objects, this is the name that can be used to refer to this specific entity from other input blocks (xml);
\item \textbf{directory}, \textit{optional string attribute}, this attribute can be used to specify a particular directory path where to create the database, if no \textit{filename} is specified, or from where open an already existing one, if \textit{filename} is provided. \textit{Default = raven/framework/DataBaseStorage};
\item \textbf{filename}, \textit{optional string attribute}, this attribute can be used to specified the filename of an HDF5 that already exists in the \textit{directory}. This is the only way to let RAVEN know that an HDF5 should be opened and not overwritten. NB. When this attribute is not specified, the newer database filename is going to be named \textit{name}.h5. \textit{Default = None};
\item \textbf{compression}, \textit{optional string attribute}, compression algorithm to be used. Available are:
   \begin{itemize}
      \item \textit{compression = gzip}, best where portability is required. Good compression, moderate speed;
      \item \textit{compression = lzf}, Low to moderate compression, very fast.
   \end{itemize}
  \textit{Default = no compression};
\end{itemize}

\begin{lstlisting}[style=XML]
----------------------------------------------------------
<DataBases> 
   <HDF5 name="***" directory=''path_to_a_dir'' compression=''lzf''/>
   <HDF5 name="***" filename=''existing_hdf5.h5''/>
</DataBases>
----------------------------------------------------------
\end{lstlisting}


\section{OutStream system \\ \vspace{2 mm} }
\label{sec:outstream}
The PRA and UQ framework provides the capabilities to visualize and dump out the
data that are generated, imported (from a system code) and post-processed during
the analysis.
%
These capabilities are contained in the "OutStream" system.
%
Actually, two different OutStream types are available:
\vspace{-5mm}
\begin{itemize}
  \itemsep0em
  \item \textbf{Print}, module that lets the user dump the data contained in the
  internal objects;
  \item \textbf{Plot}, module, based on MatPlotLib~\cite{MatPlotLib}, aimed to
  provide advanced plotting capabilities.
  %
\end{itemize}
\vspace{-5mm}
Both the types listed above only accept as ``input'' a \textit{Data} object
type.
%
This choice has been taken since the ``\textit{Datas}'' system (see
section~\ref{sec:Datas}) has the main advantages, among the others, of ensuring
a standardized approach for exchanging the data/meta-data among the different
framework entities.
%
Every module can project its outcomes into a \textit{Data} object.
%
This provides, to the user, the capability to visualize/dump all the modules'
results.
%
As already mentioned [put reference to the xml input section], the RAVEN
framework input is based on the \textbf{E}xtensible \textbf{M}arkup
\textbf{L}anguage (\textbf{XML}) format.
%
Thus, in order to activate the ``\textit{OutStream}'' system, the input needs to
contain a block identified by the ``\textbf{\xmlNode{OutStreamManager}}'' tag (as
shown below).

\begin{lstlisting}[style=XML]
-----------------------------------------------------------
<OutStreamManager>
    <!-- "OutStream" objects that need to be created-->
</OutStreamManager>
-----------------------------------------------------------
\end{lstlisting}
In the ``OutStreamManager'' block an unlimited number of ``Plot'' and ``Print''
sub-blocks can be inputted.
%
The input specifications and the main capabilities for both types are reported
in the following sections.
%
%%%%%%%%%
% PRINTING SYSTEM
%
%%%%%%%%%
\subsection{Printing system \label{sec:printing}}
The Printing system has been created in order to let the user dump the data,
contained in the internal data objects (see [reference to Data(s) section]), out
at anytime during the calculation.
%
Currently, the only available output is a \textbf{C}omma \textbf{S}eparated
\textbf{V}alue (\textbf{CSV}) file.
%
In the near future, an XML formatted file option will be available.
%
This will facilitate the exchanging of results and provide the possibility to
dump the solution of an analysis and "restart" another one constructing a
\textit{Data} from scratch.
%
The XML code, that is reported below, shows different ways to request a
\textit{Print} OutStream.
%
The user needs to provide a name for each sub-block (XML attribute).
%
These names are then used in the \textit{Steps'} blocks in order to activate the
Printing options at anytime.
%
As shown in the examples below, every \textit{Print} block must contain, at
least, the two required tags:
\vspace{-5mm}
\begin{itemize}
  \itemsep0em
  \item \xmlNode{type}, the output file type (csv or xml).
  %
  \textit{Note, only \textbf{csv} is currently available}
  \item \xmlNode{source}, the \textit{Data} name (one of the \textit{Data} defined in
  the ``\textit{Datas}'' block)
\end{itemize}
\vspace{-5mm}
If only these two tags are provided (as in the ``first-example'' below), the
output file will be filled with the whole content of the ``d-name''
\textit{Data}.
%
\begin{lstlisting}[style=XML]
-----------------------------------------------------------
<OutStreamManager>
  <Print name='first_example'>
    <type>csv</type>
    <source>d-name</source>
  </Print>
  <Print name='second-example'>
    <type>csv</type>
    <source>d-name</source>
    <variables>Output</variables>
  </Print>
  <Print name='third-example'>
    <type>csv</type>
    <source>d-name</source>
    <variables>Input</variables>
  </Print>
  <Print name='forth-example'>
    <type>csv</type>
    <source>d-name</source>
    <variables>Input|var-name-in,Output|var-name-out</variables>
  </Print>
</OutStreamManager>
-----------------------------------------------------------
\end{lstlisting}
If just few parts of the \xmlNode{source} are important for a particular analysis, the
additional XML tag \xmlNode{variables} can be provided.
%
In this block, the variables that need to be dumped must be inputted, in a comma
separated format.
%
The available options, for the \xmlNode{variables} sub-block, are listed below:
\vspace{-5mm}
\begin{itemize}
  \itemsep0em
  \item \textbf{Output}, the output space will be dumped out (see
  ``second-example'')
  \item \textbf{Input}, the input space will be dumped out (see
  ``third-example'')
  \item \textbf{Input|var-name-in/Output|var-name-out}, only the particular
  variables ``var-name-in'' and ``var-name-out'' will be reported in the output
  file (see ``forth-example'')
\end{itemize}
\vspace{-5mm}
Note that all the XML tags are case-sensitive but not their content.
%
%%%%%%%%%
% PLOTTING SYSTEM
%
%%%%%%%%%
\subsection{Plotting system \label{sec:plotting}}
The Plotting system provides all the capabilities to visualize the analysis
outcomes, in real-time or at the post-processing stage.
%
The system is based on the Python library MatPlotLib~\cite{MatPlotLib}.
%
MatPlotLib is a 2D/3D plotting library which produces publication quality
figures in a variety of hardcopy formats and interactive environments across
platforms.
%
This external tool has been wrapped in the RAVEN framework, and is usable by the
user.
%
Since it was unfeasible to support, in the source code, all the interfaces for
all the available plot types, the RAVEN Plotting system directly provide a
formatted input structure for 11 different plot types (2D/3D).
%
The user may request a plot not present among the supported ones, since the
RAVEN Plotting system has the capability to construct on the fly the interface
for a Plot, based on XML instructions.
%
This capability will be discussed in the
sub-section~\ref{sec:Interpretedplotting}.
%
%%%%%%%%%%%%%
% Plot Input Strucutre sub-sub-section
%%%%%%%%%%%%%
\subsubsection{Plot input structure \label{sec:PlotInputStructure}}
In order to create a plot, the user needs to add, within the
\xmlNode{OutStreamManager} block, a \xmlNode{Plot} sub-block.
%
As for the \textit{Print} OutStream, the user needs to specify a name as
attribute of the plot.
%
This name will then be used to request the plot in the \textit{Steps'} block.
%
In addition, the Plot block may need the following attributes:
\vspace{-5mm}
\begin{itemize}
  \itemsep0em
  \item \textbf{dim}, \textit{required integer attribute}, define the
  dimensionality of the plot: 2 (2D) or 3 (3D)
  \item \textbf{interactive}, \textit{optional bool attribute (default=False)'},
  specify if the Plot needs to be interactively created (real-time screen
  visualization)
  \item \textbf{overwrite}, \textit{optional bool attribute (default=False)'},
  if the plot needs to be dumped into picture file/s, does the code need to
  overwrite them every time a new plot (with the same name) is requested?
\end{itemize}
\vspace{-5mm}

As shown, in the XML input example below, the body of the Plot XML input
contains two main sub-nodes:
\vspace{-5mm}
\begin{itemize}
  \itemsep0em
  \item\xmlNode{actions}, where general control options for the figure layout are
  defined (see [])
  \item \xmlNode{plot\_settings}, where the actual plot options are provided
  \vspace{-5mm}
\end{itemize}
These two main sub-block are discussed in the following paragraphs.
%
%%%%%%%%%%%%%
% Actions' block sub-sub-sub section
%%%%%%%%%%%%%
\paragraph{``Actions'' input block \label{sec:actionsBlock}}
The input in the \xmlNode{actions} sub-node is common to all the Plot types, since, in
it, the user specifies all the controls that need to be applied to the figure
style.
%
This block must be unique in the definition of the \xmlNode{Plot} main block.
%
In the following list, all the predefined ``actions'' are reported:
\vspace{-5mm}
\begin{itemize}
  \itemsep0em
  \item \xmlNode{how}, comma separated list of output types:
     \begin{itemize}
    \item \textit{screen}, show the figure on the screen in interactive mode
    \item \textit{pdf}, save the figure as a Portable Document Format file (PDF)
    \item \textit{png}, save the figure as a Portable Network Graphics file
    (PNG)
    \item \textit{eps}, save the figure as a Encapsulated Postscript file (EPS)
    \item \textit{pgf}, save the figure as a LaTeX PGF Figure file (PGF)
    \item \textit{ps}, save the figure as a Postscript file (PS)
    \item \textit{gif}, save the figure as a Graphics Interchange Format (GIF)
    \item \textit{svg}, save the figure as a Scalable Vector Graphics file (SVG)
    \item \textit{jpeg}, save the figure as a jpeg file (JPEG)
    \item \textit{raw}, save the figure as a Raw RGBA bitmap file (RAW)
    \item \textit{bmp}, save the figure as a Windows bitmap file (BMP)
    \item \textit{tiff}, save the figure as a Tagged Image Format file (TIFF)
    \item \textit{svgz}, save the figure as a Scalable Vector Graphics file
    (SVGZ)
      \end{itemize}
  \item \xmlNode{title}, as the name suggests , within this block the user can specify
  the title of the figure.
  %
  In the body, few other keywords (required and not) are present:

% TITLE
 \begin{itemize}
    \item \textit{\xmlNode{text}}, string type, title of the figure
    \item \textit{\xmlNode{kwargs}}, within this block the user can specify optional
    parameters with the following format:
        \begin{lstlisting}[style=XML]
        --------------------------
         <kwargs>
           <param1>value1</param1>
           <param2>value2</param2>
         </kwargs>
        -------------------------
       \end{lstlisting}
    The kwargs block is able to convert whatever string into a python type (for
    example \xmlNode{param1>{'1stKeyword':45}</param1} will be converted into a
    dictionary, \xmlNode{param2>[56,67]</param2} into a list, etc.).
    %
    For reference regarding the available kwargs, see
    ``matplotlib.pyplot.title'' method in~\cite{MatPlotLib}.
    %
      \end{itemize}
% LABEL FORMAT
  \item \xmlNode{label\_format}, within this block the default scale formating can be
  modified.
  %
  In the body, few keywords can be specified (all optional):
 \begin{itemize}
    \item \textit{\xmlNode{style}}, string, the style of the number notation, 'sci' or
    'scientific' for scientific, 'plain' for plain notation.
    %
    Default = scientific
    \item \textit{\xmlNode{scilimits}}, tuple, (m, n), pair of integers; if style is
    ‘sci’, scientific notation will be used for numbers outside the range
    10`m`:sup: to 10`n`:sup:.
    %
    Use (0,0) to include all numbers.
    %
    NB.
    %
    The value for this keyword, needs to be inputted between brackets [for
    example, (5,6)].
    %
    Default = (0,0)
    \item \textit{\xmlNode{useOffset}}, bool or double, if True, the offset will be
    calculated as needed; if False, no offset will be used; if a numeric offset
    is specified, it will be used.
    %
    Default = False
    \item \textit{\xmlNode{axis}}, string, the axis where to apply the defined format,
    'x','y' or 'both'.
    %
    Default = 'both'.
    %
    NB.
    %
    If this action will be used in a 3-D plot, the user can input 'z' as well
    and 'both' will apply this format to all three axis.
    %
      \end{itemize}
% FIGURE PROPERTIES
  \item \xmlNode{figure\_properties}, within this block the user specifies how to
  customize the figure style/quality.
  %
  Thus, through this ``action'' the user has got full control on the quality of
  the figure, its dimensions, etc.
  %
  This control is performed by the following keywords:
 \begin{itemize}
    \item \textit{\xmlNode{figsize}}, tuple (optional), (width, hight), in inches
    \item \textit{\xmlNode{dpi}}, integer, dots per inch
    \item \textit{\xmlNode{facecolor}}, string, set the figure background color
    (please refer to ``matplotlib.figure.Figure'' in~\cite{MatPlotLib} for a
    list of all the colors available)
    \item \textit{\xmlNode{edgecolor}}, string, the figure edge background color
    (please refer to ``matplotlib.figure.Figure'' in~\cite{MatPlotLib} for a
    list of all the colors available)
    \item \textit{\xmlNode{linewidth}}, self explainable keyword
    \item \textit{\xmlNode{frameon}}, bool, if False, suppress drawing the figure
    frame
      \end{itemize}
% RANGE
  \item \xmlNode{range}, the range ``action'' allows to specify the ranges of all the
  axis.
  %
  All the keywords in the body of this block are optional:
     \begin{itemize}
    \item \textit{\xmlNode{ymin}}, double (optional), lower boundary for y axis
    \item \textit{\xmlNode{ymax}}, double (optional), upper boundary for y axis
    \item \textit{\xmlNode{xmin}}, double (optional), lower boundary for x axis
    \item \textit{\xmlNode{xmax}}, double (optional), upper boundary for x axis
    \item \textit{\xmlNode{zmin}}, double (optional), lower boundary for z axis.
    %
    NB.
    %
    Obviously, this keyword is effective in 3-D plots only
    \item \textit{\xmlNode{zmax}}, double (optional), upper boundary for z axis.
    %
    NB.
    %
    Obviously, this keyword is effective in 3-D plots only
      \end{itemize}
% CAMERA
  \item \xmlNode{camera}, the camera item is available in 3-D plots only.
  %
  Through this ``action'', it is possible to orientate the plot as wished.
  %
  The controls are:
     \begin{itemize}
    \item \textit{\xmlNode{elevation}}, double (optional), stores the elevation angle
    in the z plane
    \item \textit{\xmlNode{azimuth}}, double (optional), stores the azimuth angle in
    the x,y plane
      \end{itemize}
% SCALE
  \item \xmlNode{scale}, the scale block allows the specification of the axis scales:
     \begin{itemize}
    \item \textit{\xmlNode{xscale}}, string (optional), scale of the x axis.
    %
    Three options are available: ``linear'',``log'',``symlog''.
    %
    Default = linear
    \item \textit{\xmlNode{yscale}}, string (optional), scale of the y axis.
    %
    Three options are available: ``linear'',``log'',``symlog''.
    %
    Default = linear
    \item \textit{\xmlNode{zscale}}, string (optional), scale of the z axis.
    %
    Three options are available: ``linear'',``log'',``symlog''.
    %
    Default = linear.
    %
    NB.
    %
    Obviously, this keyword is effective in 3-D plots only
      \end{itemize}
% ADD_TEXT
  \item \xmlNode{add\_text}, same as title
% AUTOSCALE
  \item \xmlNode{autoscale}, the autoscale block is a convenience method for simple
  axis view autoscaling.
  %
  It turns autoscaling on or off, and then, if autoscaling for either axis is
  on, it performs the autoscaling on the specified axis or axes.
  %
  The following keywords are available:
     \begin{itemize}
    \item \textit{\xmlNode{enable}}, bool (optional), True turns autoscaling on, False
    turns it off.
    %
    None leaves the autoscaling state unchanged.
    %
    Default = True
    \item \textit{\xmlNode{axis}}, string (optional), string, the axis where to apply
    the defined format, 'x','y' or 'both'.
    %
    Default = 'both'.
    %
    NB.
    %
    If this action will be used in a 3-D plot, the user can input 'z' as well
    and 'both' will apply this format to all three axis.
    %
    \item \textit{\xmlNode{tight}}, bool (optional), if True, set view limits to data
    limits; if False, let the locator and margins expand the view limits; if
    None, use tight scaling if the only artist is an image, otherwise treat
    tight as False.
    %
      \end{itemize}
%HORIZONTAL_LINE
  \item \xmlNode{horizontal\_line}, this ``action'' provides the ability to draw a
  horizontal line in the current figure.
  %
  This capability might be useful, for example, if the user wants to highlight a
  trigger, function of a variable.
  %
  The following keywords are settable:
    \begin{itemize}
    \item \textit{\xmlNode{y}}, double (optional), y coordinate.
    %
    Default = 0
    \item \textit{\xmlNode{xmin}}, double (optional), starting coordinate on the x
    axis.
    %
    Default = 0
    \item \textit{\xmlNode{xmax}}, double (optional), ending coordinate on the x axis.
    %
    Default = 1
    \item \textit{\xmlNode{kwargs}}, within this block the user can specify optional
    parameters with the following format:
        \begin{lstlisting}[style=XML]
        --------------------------
         <kwargs>
           <param1>value1</param1>
           <param2>value2</param2>
         </kwargs>
        -------------------------
       \end{lstlisting}
    The kwargs block is able to convert whatever string into a python type (for
    example \xmlNode{param1>{'1stKeyword':45}</param1} will be converted into a
    dictionary, \xmlNode{param2>[56,67]</param2} into a list, etc.).
    %
    For reference regarding the available kwargs, see
    ``matplotlib.pyplot.axhline'' method in~\cite{MatPlotLib}.
    %
      \end{itemize}
  NB.
  %
  This capability is not available for 3-D plots.
  %
%VERTICAL_LINE
  \item \xmlNode{vertical\_line}, similarly to the ``horizontal\_line'' action, this
  block provides the ability to draw a vertical line in the current figure.
  %
  This capability might be useful, for example, if the user wants to highlight a
  trigger, function of a variable.
  %
  The following keywords are settable:
    \begin{itemize}
    \item \textit{\xmlNode{x}}, double (optional), x coordinate.
    %
    Default = 0
    \item \textit{\xmlNode{ymin}}, double (optional), starting coordinate on the y
    axis.
    %
    Default = 0
    \item \textit{\xmlNode{ymax}}, double (optional), ending coordinate on the y axis.
    %
    Default = 1
    \item \textit{\xmlNode{kwargs}}, within this block the user can specify optional
    parameters with the following format:
        \begin{lstlisting}[style=XML]
        --------------------------
         <kwargs>
           <param1>value1</param1>
           <param2>value2</param2>
         </kwargs>
        -------------------------
       \end{lstlisting}
    The kwargs block is able to convert whatever string into a python type (for
    example \xmlNode{param1>{'1stKeyword':45}</param1} will be converted into a
    dictionary, \xmlNode{param2>[56,67]</param2} into a list, etc.).
    %
    For reference regarding the available kwargs, see
    ``matplotlib.pyplot.axvline'' method in~\cite{MatPlotLib}.
    %
      \end{itemize}
  NB.
  %
  This capability is not available for 3-D plots.
  %
%HORIZONTAL_RECTANGLE
  \item \xmlNode{horizontal\_rectangle}, this ``action'' provides the possibility to
  draw, in the current figure, a horizontally orientated rectangle .
  %
  This capability might be useful, for example, if the user wants to highlight a
  zone in the plot.
  %
  The following keywords are settable:
    \begin{itemize}
    \item \textit{\xmlNode{ymin}}, double (required), starting coordinate on the y
    axis
    \item \textit{\xmlNode{ymax}}, double (required), ending coordinate on the y axis
    \item \textit{\xmlNode{xmin}}, double (optional), starting coordinate on the x
    axis.
    %
    Default = 0
    \item \textit{\xmlNode{xmax}}, double (optional), ending coordinate on the x axis.
    %
    Default = 1
    \item \textit{\xmlNode{kwargs}}, within this block the user can specify optional
    parameters with the following format:
        \begin{lstlisting}[style=XML]
        --------------------------
         <kwargs>
           <param1>value1</param1>
           <param2>value2</param2>
         </kwargs>
        -------------------------
       \end{lstlisting}
    The kwargs block is able to convert whatever string into a python type (for
    example \xmlNode{param1>{'1stKeyword':45}</param1} will be converted into a
    dictionary, \xmlNode{param2>[56,67]</param2} into a list, etc.).
    %
    For reference regarding the available kwargs, see
    ``matplotlib.pyplot.axhspan'' method in~\cite{MatPlotLib}.
    %
      \end{itemize}
  NB.
  %
  This capability is not available for 3-D plots.
  %
%VERTICAL_RECTANGLE
  \item \xmlNode{vertical\_rectangle}, this ``action'' provides the possibility to
  draw, in the current figure, a vertically orientated rectangle .
  %
  This capability might be useful, for example, if the user wants to highlight a
  zone in the plot.
  %
  The following keywords are settable:
    \begin{itemize}
    \item \textit{\xmlNode{xmin}}, double (required), starting coordinate on the x
    axis
    \item \textit{\xmlNode{xmax}}, double (required), ending coordinate on the x axis
    \item \textit{\xmlNode{ymin}}, double (optional), starting coordinate on the y
    axis.
    %
    Default = 0
    \item \textit{\xmlNode{ymax}}, double (optional), ending coordinate on the y axis.
    %
    Default = 1
    \item \textit{\xmlNode{kwargs}}, within this block the user can specify optional
    parameters with the following format:
        \begin{lstlisting}[style=XML]
        --------------------------
         <kwargs>
           <param1>value1</param1>
           <param2>value2</param2>
         </kwargs>
        -------------------------
       \end{lstlisting}
    The kwargs block is able to convert whatever string into a python type (for
    example \xmlNode{param1>{'1stKeyword':45}</param1} will be converted into a
    dictionary, \xmlNode{param2>[56,67]</param2} into a list, etc.).
    %
    For reference regarding the available kwargs, see
    ``matplotlib.pyplot.axvspan'' method in~\cite{MatPlotLib}.
    %
      \end{itemize}
  NB.
  %
  This capability is not available for 3-D plots.
  %
%AXES_BOX
  \item \xmlNode{axes\_box}, this keyword controls the axes' box.
  %
  No body and its value can be 'on' or 'off'.
  %
  \item \xmlNode{axis\_properties}, this block is used to set axis properties.
  %
  There are not fixed keywords.
  %
  If only a single property needs to be set, it can be specified as body of this
  block, otherwise a dictionary-like string needs to be provided.
  %
  For reference regarding the available keys, refer to
  ``matplotlib.pyplot.axis'' method in~\cite{MatPlotLib}.
  %
  \item \xmlNode{grid}, this block is used to define a grid that needs to be added in
  the plot.
  %
  The following keywords can be inputted:
    \begin{itemize}
    \item \textit{\xmlNode{b}}, double (required), starting coordinate on the x axis
    \item \textit{\xmlNode{which}}, double (required), ending coordinate on the x axis
    \item \textit{\xmlNode{axis}}, double (optional), starting coordinate on the y
    axis.
    %
    Default = 0
    \item \textit{\xmlNode{kwargs}}, within this block the user can specify optional
    parameters with the following format:
        \begin{lstlisting}[style=XML]
        --------------------------
         <kwargs>
           <param1>value1</param1>
           <param2>value2</param2>
         </kwargs>
        -------------------------
       \end{lstlisting}
    The kwargs block is able to convert whatever string into a python type (for
    example \xmlNode{param1>{'1stKeyword':45}</param1} will be converted into a
    dictionary, \xmlNode{param2>[56,67]</param2} into a list, etc.).
    %
    For reference regarding the available kwargs, see ``matplotlib.pyplot.grid''
    method in~\cite{MatPlotLib}.
    %
      \end{itemize}
  \vspace{-5mm}
\end{itemize}
%%%%%%%%%%%%%
%Plot block sub-sub-sub section
%%%%%%%%%%%%%
\paragraph{``plot\_settings'' input block \label{sec:plotSettings}}
The sub-block identified by the keyword \xmlNode{plot\_settings} is used to define the
plot characteristics.Within this sub-section at least a \xmlNode{plot} block must be
present.
%
the \xmlNode{plot} sub-section may not be unique within the \xmlNode{plot\_settings}
definition; the number of \xmlNode{plot} sub-blocks is equal to the number of plots
that need to be placed in the same figure.
%
For example, in the following XML cut, a ``line'' and a ``scatter'' type are
combined in the same figure.
%
\begin{lstlisting}[style=XML]
-----------------------------------------------------------
<OutStreamManager>
  <Plot name='example2PlotsCombined' dim='2'>
    <actions>
      <!-- Actions -->
    </actions>
    <plot\_settings>
       <plot>
        <type>line</type>
        <x>d-type|Output|x1</x>
        <y>d-type|Output|y1</y>
      </plot>
       <plot>
        <type>scatter</type>
        <x>d-type|Output|x2</x>
        <y>d-type|Output|y2</y>
      </plot>
      <xlabel>label X</xlabel>
      <ylabel>label Y</ylabel>
    </plot\_settings>
  </Plot>
</OutStreamManager>
-----------------------------------------------------------
\end{lstlisting}
As already mentioned, within the \xmlNode{plot\_settings} block, at least a \xmlNode{plot}
sub-block needs to be inputted.
%
Independently from the plot type, some keywords are mandatory:
\begin{itemize}
  \item \textit{\xmlNode{type}}, string, required parameter, the plot type (for
  example, line, scatter, wireframe, etc.);
  \item \textit{\xmlNode{x}}, string, required parameter, the parameter needs to be
  considered as x coordinate;
  \item \textit{\xmlNode{y}}, string, required parameter, the parameter needs to be
  considered as y coordinate;
  \item \textit{\xmlNode{z}}, string required parameter (3D plots only), the parameter
  needs to be considered as z coordinate.
  %
\end{itemize}
In addition other plot-dependent keywords, reported in
section~\ref{sec:23Dplotting}, can be provided.
%
\\Under the \xmlNode{plot\_settings} block other keywords, common to all the plots the
user decided to combine in the figure, can be inputted, such as:
\begin{itemize}
  \item \textit{\xmlNode{xlabel}}, string, optional parameter, x axis label;
  \item \textit{\xmlNode{ylabel}}, string, optional parameter, y axis laber;
  \item \textit{\xmlNode{zlabel}}, string, optional parameter (3D plots only), z axis
  label;
  \item \textit{\xmlNode{colorMap}}, string, the parameter needs to be used to define
  a color map.
  %
\end{itemize}
As already mentioned, the Plot system accepts as parameter (i.e., x, y, z,
colorMap) the \textbf{Datas} object only.
%
Considering the structure of ``Datas'', the parameters are inputted as follow:
\\ $DataObjectName|Input(or)Output|variableName$.
%
\\ If the ``variableName'' contains the symbol $|$, it must be surrounded by
brackets:
\\ $DataObjectName|Input(or)Output|(whatever|variableName)$.

%%%%%%%%%%%%%
%Predefined Plotting System block sub-sub-sub section
%%%%%%%%%%%%%
\paragraph{Predefined Plotting System: 2D/3D \label{sec:23Dplotting}}
As already mentioned above, the Plotting system provides specialized input
structure for several different kind of plots:
 \begin{itemize}
  \item \textit{2 Dimensional plots}:
       \begin{itemize}
    \item \textit{scatter}.
    %
    2 dimensional scatter plot.
    %
    It used to create a scatter plot of x vs y, where x and y are sequences of
    numbers of the same length;
    \item \textit{line}.
    %
    2 dimensional line plot.
    %
    It used to create a line plot of x vs y, where x and y are sequences of
    numbers of the same length;
    \item \textit{histogram}.
    %
    2 dimensional histogram plot.
    %
    Compute and draw the histogram of x.
    %
    It must be noticed that this plot accepts only the XML node \xmlNode{x} even if it
    is considered as 2D plot type;
    \item \textit{stem}.
    %
    2 dimensional stem plot.
    %
    A stem plot plots vertical lines at each x location from the baseline to y,
    and places a marker there;
    \item \textit{step}.
    %
    2 dimensional step plot;
    \item \textit{pseudocolor}.
    %
    2 dimensional scatter plot.
    %
    It creates a pseudocolor plot of a two dimensional array.
    %
    The two dimensional array is built creating a mesh from \xmlNode{x> and <y} data,
    in conjunction with the data inputted in \xmlNode{colorMap};
    \item \textit{contour}.
    %
    2 dimensional contour plot.
    %
    It plots the contour lines.
    %
    Contour plot is built creating a plot from \xmlNode{x> and <y} data, in
    conjunction with the data inputted in \xmlNode{colorMap};
    \item \textit{filledContour}.
    %
    2 dimensional contour plot.
    %
    It plots the filled contour leveles.
    %
    Filled contour plot is built creating a plot from \xmlNode{x> and <y} data, in
    conjunction with the data inputted in \xmlNode{colorMap};
      \end{itemize}
  \item \textit{3 Dimensional plots}:
       \begin{itemize}
    \item \textit{scatter}.
    %
    3 dimensional scatter plot.
    %
    It is used to create a scatter plot of (x,y) vs z, where x, y, z are
    sequences of numbers of the same length;
    \item \textit{line}.
    %
    3 dimensional line plot.
    %
    It used to create a line plot of (x,y) vs z, where x, y, z are sequences of
    numbers of the same length;
    \item \textit{stem}.
    %
    3 dimensional stem plot.
    %
    It creates a 3 Dimensional stem plot of (x,y) vs z;
    \item \textit{surface}.
    %
    3 dimensional surface plot.
    %
    Create a surface plot of (x,y) vs z.
    %
    By default it will be colored in shades of a solid color, but it also
    supports color mapping;
    \item \textit{wireframe}.
    %
    3 dimensional wire-frame plot.
    %
    No color mapping is supported;
    \item \textit{tri-surface}.
    %
    3 dimensional tri-surface plot.
    %
    It is a surface plot with automatic triangulation;
    \item \textit{contour3D}.
    %
    3 dimensional contour plot.
    %
    It plots the contour lines.
    %
    Contour plot is built creating a plot from \xmlNode{x>, <y> and <z} data, in
    conjunction with the data inputted in \xmlNode{colorMap}
    \item \textit{filledContour3D}.
    %
    3 dimensional filled contour plot.
    %
    It plots the filled contour leveles.
    %
    Filled contour plot is built creating a plot from \xmlNode{x>, <y> and <z} data,
    in conjunction with the data inputted in \xmlNode{colorMap};
    \item \textit{histogram}.
    %
    3 dimensional histogram plot.Compute and draw the histogram of x and y.
    %
    It must be noticed that this plot accepts only the XML nodes \xmlNode{x> and <y}
    even if it is considered as 3D plot type
       \end{itemize}
\end{itemize}
As already mentioned, the settings for each plot type are inputted within the
XML block called \xmlNode{plot}.
%
The sub-nodes that can be inputted depends on the plot type: each plot type has
its own set of parameters that can be specified.
%
\\In the following sub-sections all the options for the plot types listed above
are reported.

\subsubsection{2D \& 3D Scatter plot.}
In order to create a ``scatter'' plot, either 2D or 3D, the user needs to write
in the \xmlNode{type} body the keyword ``scatter''.
%
In order to customize the plot, the user can define the following XML sub-nodes:
  \begin{itemize}
  \item \xmlNode{s}\textbf{\textit{, integer, optional field.}}.
  %
  Size in points\^2.
  %
  \default{20};
  \item \xmlNode{c}\textbf{\textit{, string, optional field.}}.
  %
  color or sequence of color.
  %
  \xmlNode{c} can be a single color format string, or a sequence of color
  specifications of length N, or a sequence of N numbers to be mapped to colors
  using the cmap and norm specified via kwargs.
  %
  Note that \xmlNode{c} should not be a single numeric RGB or RGBA sequence because
  that is indistinguishable from an array of values to be colormapped.
  %
  \xmlNode{c} can be a 2-D array in which the rows are RGB or RGBA;
  \item \xmlNode{marker}\textbf{\textit{, string, optional field.}}.
  %
  Marker type.
  %
  \default{o};
  \item \xmlNode{alpha}\textbf{\textit{, string, optional field.}}.
  %
  The alpha blending value, between 0 (transparent) and 1 (opaque).
  %
  \default{None} ;
  \item \xmlNode{linewidths}\textbf{\textit{, string, optional field.}}.
  %
  Widths of lines.
  %
  Note that this is a tuple, and if you set the linewidths argument you must set
  it as a sequence of floats.
  %
  \default{None};
  \item \textit{\xmlNode{kwargs}}, within this block the user can specify optional
  parameters with the following format:
        \begin{lstlisting}[style=XML]
        --------------------------
         <kwargs>
           <param1>value1</param1>
           <param2>value2</param2>
         </kwargs>
        -------------------------
       \end{lstlisting}
  The kwargs block is able to convert whatever string into a python type (for
  example \xmlNode{param1>{'1stKeyword':45}</param1} will be converted into a
  dictionary, \xmlNode{param2>[56,67]</param2} into a list, etc.).
  %
  For reference regarding the available kwargs, see
  ``matplotlib.pyplot.scatter'' method in~\cite{MatPlotLib}.
  %
    \end{itemize}

\subsubsection{2D \& 3D Line plot.}
In order to create a ``line'' plot, either 2D or 3D, the user needs to write in
the \xmlNode{type} body the keyword ``line''.
%
In order to customize the plot, the user can define the following XML sub-nodes:
  \begin{itemize}
  \item \xmlNode{interpolationType}\textbf{\textit{, string, optional field.}}.
  %
  Type of interpolation algorithm to use for the data.
  %
  Available are ``nearest'', ``linear'', ``cubic'', ``multiquadric'',
  ``inverse'', ``gaussian'', ``Rbflinear'', ``Rbfcubic'', ``quintic'',
  ``thin\_plate''.
  %
  \default{linear};
  \item \xmlNode{interpPointsX}\textbf{\textit{, integer, optional field.}}.
  %
  Number of points need to be used for interpolation of x axis;
  \item \xmlNode{interpPointsY}\textbf{\textit{, integer, optional field.}}.
  %
  Number of points need to be used for interpolation of y axis (only 3D line
  plot);
    \end{itemize}

\subsubsection{2D \& 3D Histogram plot.}
In order to create a ``histogram'' plot, either 2D or 3D, the user needs to
write in the \xmlNode{type} body the keyword ``histogram''.
%
In order to customize the plot, the user can define the following XML sub-nodes:
  \begin{itemize}
  \item \xmlNode{bins}\textbf{\textit{, integer or array\_like, optional field.}}.
  %
  Number of bins if an integer is inputted or a sequence of edges if a python
  list is defined.
  %
  \default{10};
  \item \xmlNode{normed}\textbf{\textit{, boolean, optional field.}}.
  %
  if True normalize the histogram to 1.
  %
  \default{False};
  \item \xmlNode{weights}\textbf{\textit{, sequence, optional field.}}.
  %
  An array of weights, of the same shape as x.
  %
  Each value in x only contributes its associated weight towards the bin count
  (instead of 1).
  %
  If normed is True, the weights are normalized, so that the integral of the
  density over the range remains 1.
  %
  \default{None};
  \item \xmlNode{cumulative}\textbf{\textit{, boolean, optional field.}}.
  %
  If True, then a histogram is computed where each bin gives the counts in that
  bin plus all bins for smaller values.
  %
  The last bin gives the total number of datapoints.
  %
  If normed is also True then the histogram is normalized such that the last bin
  equals 1.
  %
  If cumulative evaluates to less than 0 (e.g., -1), the direction of
  accumulation is reversed.
  %
  In this case, if normed is also True, then the histogram is normalized such
  that the first bin equals 1.
  %
  \default{False} ;
  \item \xmlNode{histtype}\textbf{\textit{, string, optional field.}}.
  %
  The type of histogram to draw:
         \begin{itemize}
    \item \textbf{bar} is a traditional bar-type histogram.
    %
    If multiple data are given the bars are aranged side by side;
    \item \textbf{barstacked} is a bar-type histogram where multiple data are
    stacked on top of each other;
    \item \textbf{step} generates a lineplot that is by default unfilled;
    \item \textbf{stepfilled} generates a lineplot that is by default filled;
         \end{itemize}
  \default{bar};
  \item \xmlNode{align}\textbf{\textit{, string, optional field.}}.
  %
  Controls how the histogram is plotted.
  %
         \begin{itemize}
    \item \textbf{left} bars are centered on the left bin edge;
    \item \textbf{mid} bars are centered between the bin edges;
    \item \textbf{right} bars are centered on the right bin edges;
         \end{itemize}
  \default{mid};
  \item \xmlNode{orientation}\textbf{\textit{, string, optional field.}}.
  %
  Orientation of the histogram:
         \begin{itemize}
    \item \textbf{horizontal};
    \item \textbf{vertical};
         \end{itemize}
  \default{vertical};
  \item \xmlNode{rwidth}\textbf{\textit{, float, optional field.}}.
  %
  The relative width of the bars as a fraction of the bin width.
  %
  \default{None};
  \item \xmlNode{log}\textbf{\textit{, boolean, optional field.}}.
  %
  Set a log scale.
  %
  \default{False};
  \item \xmlNode{color}\textbf{\textit{, string, optional field.}}.
  %
  Color of the histogram.
  %
  \default{blue};
  \item \xmlNode{stacked}\textbf{\textit{, boolean, optional field.}}.
  %
  If True, multiple data are stacked on top of each other If False multiple data
  are aranged side by side if histtype is ‘bar’ or on top of each other if
  histtype is ‘step’.
  %
  \default{False};
  \item \textit{\xmlNode{kwargs}}, within this block the user can specify optional
  parameters with the following format:
        \begin{lstlisting}[style=XML]
        --------------------------
         <kwargs>
           <param1>value1</param1>
           <param2>value2</param2>
         </kwargs>
        -------------------------
       \end{lstlisting}
  The kwargs block is able to convert whatever string into a python type (for
  example \xmlNode{param1>{'1stKeyword':45}</param1} will be converted into a
  dictionary, \xmlNode{param2>[56,67]</param2} into a list, etc.).
  %
  For reference regarding the available kwargs, see ``matplotlib.pyplot.hist''
  method in~\cite{MatPlotLib}.
  %
    \end{itemize}

\subsubsection{2D \& 3D Stem plot.}
In order to create a ``stem'' plot, either 2D or 3D, the user needs to write in
the \xmlNode{type} body the keyword ``stem''.
%
In order to customize the plot, the user can define the following XML sub-nodes:
  \begin{itemize}
  \item \xmlNode{linefmt}\textbf{\textit{, string, optional field.}}.
  %
  Line style.
  %
  \default{b-};
  \item \xmlNode{markerfmt}\textbf{\textit{, string, optional field.}}.
  %
  Marker format.
  %
  \default{bo};
  \item \xmlNode{basefmt}\textbf{\textit{, string, optional field.}}.
  %
  Base format.
  %
  \default{r-};
  \item \textit{\xmlNode{kwargs}}, within this block the user can specify optional
  parameters with the following format:
        \begin{lstlisting}[style=XML]
        --------------------------
         <kwargs>
           <param1>value1</param1>
           <param2>value2</param2>
         </kwargs>
        -------------------------
       \end{lstlisting}
  The kwargs block is able to convert whatever string into a python type (for
  example \xmlNode{param1>{'1stKeyword':45}</param1} will be converted into a
  dictionary, \xmlNode{param2>[56,67]</param2} into a list, etc.).
  %
  For reference regarding the available kwargs, see ``matplotlib.pyplot.stem''
  method in~\cite{MatPlotLib}.
  %
    \end{itemize}

\subsubsection{2D Step plot}
In order to create a 2D ``step'' plot, the user needs to write in the \xmlNode{type}
body the keyword ``step''.
%
In order to customize the plot, the user can define the following XML sub-nodes:
  \begin{itemize}
  \item \xmlNode{where}\textbf{\textit{, string, optional field.}}.
  %
  Positioning:
     \begin{itemize}
    \item \textbf{pre}, the interval from x[i] to x[i+1] has level y[i+1];
    \item \textbf{post}, that interval has level y[i];
    \item \textbf{mid}, the jumps in y occur half-way between the x-values;
     \end{itemize}
  \default{mid};
  \item \textit{\xmlNode{kwargs}}, within this block the user can specify optional
  parameters with the following format:
        \begin{lstlisting}[style=XML]
        --------------------------
         <kwargs>
           <param1>value1</param1>
           <param2>value2</param2>
         </kwargs>
        -------------------------
       \end{lstlisting}
  The kwargs block is able to convert whatever string into a python type (for
  example \xmlNode{param1>{'1stKeyword':45}</param1} will be converted into a
  dictionary, \xmlNode{param2>[56,67]</param2} into a list, etc.).
  %
  For reference regarding the available kwargs, see ``matplotlib.pyplot.step''
  method in~\cite{MatPlotLib}.
  %
    \end{itemize}

\subsubsection{2D Pseudocolor plot}
In order to create a 2D ``pseudocolor'' plot, the user needs to write in the
\xmlNode{type} body the keyword ``pseudocolor''.
%
In order to customize the plot, the user can define the following XML sub-nodes:
  \begin{itemize}
  \item \xmlNode{interpolationType}\textbf{\textit{, string, optional field.}}.
  %
  Type of interpolation algorithm to use for the data.
  %
  Available are ``nearest'', ``linear'', ``cubic'', ``multiquadric'',
  ``inverse'', ``gaussian'', ``Rbflinear'', ``Rbfcubic'', ``quintic'',
  ``thin\_plate''.
  %
  \default{linear};
  \item \xmlNode{interpPointsX}\textbf{\textit{, integer, optional field.}}.
  %
  Number of points need to be used for interpolation of x axis;
  \item \textit{\xmlNode{kwargs}}, within this block the user can specify optional
  parameters with the following format:
        \begin{lstlisting}[style=XML]
        --------------------------
         <kwargs>
           <param1>value1</param1>
           <param2>value2</param2>
         </kwargs>
        -------------------------
       \end{lstlisting}
  The kwargs block is able to convert whatever string into a python type (for
  example \xmlNode{param1>{'1stKeyword':45}</param1} will be converted into a
  dictionary, \xmlNode{param2>[56,67]</param2} into a list, etc.).
  %
  For reference regarding the available kwargs, see ``matplotlib.pyplot.pcolor''
  method in~\cite{MatPlotLib}.
  %
    \end{itemize}

\subsubsection{2D Contour or filledContour plots.}
In order to create a 2D ``Contour'' or ``filledContour'' plot, the user needs to
write in the \xmlNode{type} body the keyword ``contour'' or ``filledContour'',
respectively.
%
In order to customize the plot, the user can define the following XML sub-nodes:
  \begin{itemize}
  \item \xmlNode{number\_bins}\textbf{\textit{, integer, optional field.}}.
  %
  Number of bins.
  %
  \default{5};
  \item \xmlNode{interpolationType}\textbf{\textit{, string, optional field.}}.
  %
  Type of interpolation algorithm to use for the data.
  %
  Available are ``nearest'', ``linear'', ``cubic'', ``multiquadric'',
  ``inverse'', ``gaussian'', ``Rbflinear'', ``Rbfcubic'', ``quintic'',
  ``thin\_plate''.
  %
  \default{linear};
  \item \xmlNode{interpPointsX}\textbf{\textit{, integer, optional field.}}.
  %
  Number of points need to be used for interpolation of x axis;
  \item \textit{\xmlNode{kwargs}}, within this block the user can specify optional
  parameters with the following format:
        \begin{lstlisting}[style=XML]
        --------------------------
         <kwargs>
           <param1>value1</param1>
           <param2>value2</param2>
         </kwargs>
        -------------------------
       \end{lstlisting}
  The kwargs block is able to convert whatever string into a python type (for
  example \xmlNode{param1>{'1stKeyword':45}</param1} will be converted into a
  dictionary, \xmlNode{param2>[56,67]</param2} into a list, etc.).
  %
  For reference regarding the available kwargs, see
  ``matplotlib.pyplot.contour'' method in~\cite{MatPlotLib}.
  %
    \end{itemize}

\subsubsection{3D Surface Plot.}
In order to create a 3D ``Surface'' plot, the user needs to write in the
\xmlNode{type} body the keyword ``surface''.
%
In order to customize the plot, the user can define the following XML sub-nodes:
  \begin{itemize}
  \item \xmlNode{rstride}\textbf{\textit{, integer, optional field.}}.
  %
  Array row stride (step size).
  %
  \default{1};
  \item \xmlNode{cstride}\textbf{\textit{, integer, optional field.}}.
  %
  Array column stride (step size).
  %
  \default{1};
  \item \xmlNode{cmap}\textbf{\textit{, string, optional field.}}.
  %
  Color map.
  %
  \default{jet};
  \item \xmlNode{antialiased}\textbf{\textit{, boolean, optional field.}}.
  %
  Antialiased rendering.
  %
  \default{False};
  \item \xmlNode{linewidth}\textbf{\textit{, integer, optional field.}}.
  %
  Widths of lines.
  %
  Note that this is a tuple, and if you set the linewidths argument you must set
  it as a sequence of floats.
  %
  \default{0};
  \item \xmlNode{interpolationType}\textbf{\textit{, string, optional field.}}.
  %
  Type of interpolation algorithm to use for the data.
  %
  Available are ``nearest'', ``linear'', ``cubic'', ``multiquadric'',
  ``inverse'', ``gaussian'', ``Rbflinear'', ``Rbfcubic'', ``quintic'',
  ``thin\_plate''.
  %
  \default{linear};
  \item \xmlNode{interpPointsX}\textbf{\textit{, integer, optional field.}}.
  %
  Number of points need to be used for interpolation of x axis;
  \item \xmlNode{interpPointsY}\textbf{\textit{, integer, optional field.}}.
  %
  Number of points need to be used for interpolation of y axis;
  \item \textit{\xmlNode{kwargs}}, within this block the user can specify optional
  parameters with the following format:
        \begin{lstlisting}[style=XML]
        --------------------------
         <kwargs>
           <param1>value1</param1>
           <param2>value2</param2>
         </kwargs>
        -------------------------
       \end{lstlisting}
  The kwargs block is able to convert whatever string into a python type (for
  example \xmlNode{param1>{'1stKeyword':45}</param1} will be converted into a
  dictionary, \xmlNode{param2>[56,67]</param2} into a list, etc.).
  %
  For reference regarding the available kwargs, see
  ``matplotlib.pyplot.contour'' method in~\cite{MatPlotLib}.
  %
    \end{itemize}

\subsubsection{3D Wireframe Plot.}
In order to create a 3D ``Wireframe'' plot, the user needs to write in the
\xmlNode{type} body the keyword ``wireframe''.
%
In order to customize the plot, the user can define the following XML sub-nodes:
  \begin{itemize}
  \item \xmlNode{rstride}\textbf{\textit{, integer, optional field.}}.
  %
  Array row stride (step size).
  %
  \default{1};
  \item \xmlNode{cstride}\textbf{\textit{, integer, optional field.}}.
  %
  Array column stride (step size).
  %
  \default{1};
  \item \xmlNode{cmap}\textbf{\textit{, string, optional field.}}.
  %
  Color map.
  %
  \default{jet};
  \item \xmlNode{interpolationType}\textbf{\textit{, string, optional field.}}.
  %
  Type of interpolation algorithm to use for the data.
  %
  Available are ``nearest'', ``linear'', ``cubic'', ``multiquadric'',
  ``inverse'', ``gaussian'', ``Rbflinear'', ``Rbfcubic'', ``quintic'',
  ``thin\_plate''.
  %
  \default{linear};
  \item \xmlNode{interpPointsX}\textbf{\textit{, integer, optional field.}}.
  %
  Number of points need to be used for interpolation of x axis;
  \item \xmlNode{interpPointsY}\textbf{\textit{, integer, optional field.}}.
  %
  Number of points need to be used for interpolation of y axis;
  \item \textit{\xmlNode{kwargs}}, within this block the user can specify optional
  parameters with the following format:
        \begin{lstlisting}[style=XML]
        --------------------------
         <kwargs>
           <param1>value1</param1>
           <param2>value2</param2>
         </kwargs>
        -------------------------
       \end{lstlisting}
  The kwargs block is able to convert whatever string into a python type (for
  example \xmlNode{param1>{'1stKeyword':45}</param1} will be converted into a
  dictionary, \xmlNode{param2>[56,67]</param2} into a list, etc.).
  %
  For reference regarding the available kwargs, see
  ``matplotlib.pyplot.contour'' method in~\cite{MatPlotLib}.
  %
    \end{itemize}

\subsubsection{3D Tri-surface Plot.}
In order to create a 3D ``Tri-surface'' plot, the user needs to write in the
\xmlNode{type} body the keyword ``tri-surface''.
%
In order to customize the plot, the user can define the following XML sub-nodes:
  \begin{itemize}
  \item \xmlNode{color}\textbf{\textit{, string, optional field.}}.
  %
  Color of the surface patches.
  %
  \default{b};
  \item \xmlNode{shade}\textbf{\textit{, boolean, optional field.}}.
  %
  Apply shading.
  %
  \default{False};
  \item \xmlNode{cmap}\textbf{\textit{, string, optional field.}}.
  %
  Color map.
  %
  \default{jet};
  \item \textit{\xmlNode{kwargs}}, within this block the user can specify optional
  parameters with the following format:
        \begin{lstlisting}[style=XML]
        --------------------------
         <kwargs>
           <param1>value1</param1>
           <param2>value2</param2>
         </kwargs>
        -------------------------
       \end{lstlisting}
  The kwargs block is able to convert whatever string into a python type (for
  example \xmlNode{param1>{'1stKeyword':45}</param1} will be converted into a
  dictionary, \xmlNode{param2>[56,67]</param2} into a list, etc.).
  %
  For reference regarding the available kwargs, see
  ``matplotlib.pyplot.contour'' method in~\cite{MatPlotLib}.
  %
    \end{itemize}

\subsubsection{3D Contour or filledContour plots.}
In order to create a 3D ``Contour'' or ``filledContour'' plot, the user needs to
write in the \xmlNode{type} body the keyword ``contour3D'' or ``filledContour3D'',
respectively.
%
In order to customize the plot, the user can define the following XML sub-nodes:
  \begin{itemize}
  \item \xmlNode{number\_bins}\textbf{\textit{, integer, optional field.}}.
  %
  Number of bins.
  %
  \default{5};
  \item \xmlNode{interpolationType}\textbf{\textit{, string, optional field.}}.
  %
  Type of interpolation algorithm to use for the data.
  %
  Available are ``nearest'', ``linear'', ``cubic'', ``multiquadric'',
  ``inverse'', ``gaussian'', ``Rbflinear'', ``Rbfcubic'', ``quintic'',
  ``thin\_plate''.
  %
  \default{linear};
  \item \xmlNode{interpPointsX}\textbf{\textit{, integer, optional field.}}.
  %
  Number of points need to be used for interpolation of x axis;
  \item \xmlNode{interpPointsY}\textbf{\textit{, integer, optional field.}}.
  %
  Number of points need to be used for interpolation of y axis;
  \item \textit{\xmlNode{kwargs}}, within this block the user can specify optional
  parameters with the following format:
        \begin{lstlisting}[style=XML]
        --------------------------
         <kwargs>
           <param1>value1</param1>
           <param2>value2</param2>
         </kwargs>
        -------------------------
       \end{lstlisting}
  The kwargs block is able to convert whatever string into a python type (for
  example \xmlNode{param1}\{'1stKeyword':45\}\xmlNode{/param1} will be converted into a
  dictionary, \xmlNode{param2}\[56,67\]\xmlNode{/param2} into a list, etc.).
  %
  For reference regarding the available kwargs, see
  ``matplotlib.pyplot.contour'' method in~\cite{MatPlotLib}.
  %
    \end{itemize}

%\subsubsection{Interpreted Plotting instruction \label{sec:Interpretedplotting}}

\subsubsection{Example XML input.}
-----------------------------------------------------------
\begin{lstlisting}[style=XML]
<OutStreamManager>
  <Plot name='2DHistoryPlot' dim='2' interactive='False' overwrite='False'>
    <actions>
      <how>pdf,png,eps</how>
      <title>
        <text>***</text>
      </title>
    </actions>
    <plot_settings>
       <plot>
        <type>line</type>
        <x>stories|Output|time</x>
        <y>stories|Output|pipe1_Hw</y>
        <kwargs>
         <color>green</color>
         <label>pipe1-Hw</label>
        </kwargs>
      </plot>
       <plot>
        <type>line</type>
        <x>stories|Output|time</x>
        <y>stories|Output|pipe1_aw</y>
        <kwargs>
         <color>blue</color>
         <label>pipe1-aw</label>
        </kwargs>
      </plot>
      <xlabel>time [s]</xlabel>
      <ylabel>evolution</ylabel>
    </plot_settings>
  </Plot>
</OutStreamManager>
\end{lstlisting}
-----------------------------------------------------------


\section{Existing Interfaces}
\label{sec:existingInterface}
%%%%%%%%%%%%%%%%%%%%%%%%%%%
%%%%%% Generic  INTERFACE  %%%%%%
%%%%%%%%%%%%%%%%%%%%%%%%%%%
\subsection{Generic Interface}
\label{subsec:genericInterface}
The GenericCode interface is meant to handle a wide variety of generic codes
that take straightforward input files and produce output CSV files.  There are
some limitations for this interface.
If a code: \vspace{-20pt}
\begin{itemize}
\item accepts a keyword-based input file with no cross-dependent inputs,
\item has no more than one filetype extension per command line flag,
\item and returns a CSV with the input parameters and output parameters,
\end{itemize}\vspace{-20pt}
the GenericCode interface should cover the code for RAVEN.

The GenericCode interface leverages a wildcard-based approach to editing input files. Using the
special wildcard format \texttt{\$RAVEN-\$}, RAVEN parses text-based inputs and replaces the
wildcards with sampled values. For example, consider RAVEN sampling variables named
\texttt{initial\_velocity} and \texttt{initial\_angle}. Assume we're using a projectile tracking model
with keyword based entry input files; for example,
\begin{lstlisting}[language=python]
  initial_height = 0     # starting height, m
  initial_angle = 35     # starting angle, degrees
  initial_velocity = 40  # starting velocity, m/s
  gravity = 9.8          # accel due to grav, m/s/s
  auxfile = gen.two      # additional properties file
  case = myOut           # output name (adds .csv)
 \end{lstlisting}
Since we want to sample \texttt{initial\_velocity} and \texttt{initial\_angle}, we create a new
template input and replace the values where samples should go with the wildcard and the variable
name:
\begin{lstlisting}[language=python]
  initial_height = 0     # starting height, m
  initial_angle = $RAVEN-initial_angle$ # starting angle, degrees
  initial_velocity = $RAVEN-initial_velocity$ # starting velocity, m/s
  gravity = 9.8          # accel due to grav, m/s/s
  auxfile = gen.two      # additional properties file
  case = myOut           # output name (adds .csv)
\end{lstlisting}
See more discussion of replacing the output case and auxiliary file names below. When RAVEN samples
values for the initial height and velocity, it will generate a new input file with those values in
place, for example,
\begin{lstlisting}[language=python]
  initial_height = 0     # starting height, m
  initial_angle = 22.7589 # starting angle, degrees
  initial_velocity = 47.2076 # starting velocity, m/s
  gravity = 9.8          # accel due to grav, m/s/s
  auxfile = gen.two      # additional properties file
  case = myOut           # output name (adds .csv)
\end{lstlisting}

If a code contains cross-dependent data, the generic interface is not able to
edit the correct values.  For example, if a geometry-building script specifies
inner\_radius, outer\_radius, and thickness, the generic interface cannot
calculate the thickness given the outer and inner radius, or vice versa.
In this case, the \textit{function} method explained in the Samplers (see \ref{sec:Samplers})
and Optimizers (see \ref{sec:Optimizers}) sections can be used.

 An example of the code interface is shown here.  The input parameters are read
 from the input files \texttt{gen.one} and \texttt{gen.two} respectively.
 The code is run using \texttt{python}, so that is part of the \xmlNode{clargs} node with the \xmlAttr{type} equal \xmlString{prepend}.
 The command line entry to normally run the code is
\begin{lstlisting}[language=bash]
python poly_inp.py -i gen.one -a gen.two -o myOut
\end{lstlisting}
and produces the output \texttt{myOut.csv}.

Example:
\begin{lstlisting}[style=XML]
    <Code name="poly" subType="GenericCode">
      <executable>GenericInterface/poly_inp.py</executable>
      <inputExtentions>.one,.two</inputExtentions>
      <clargs type='prepend' arg='python'/>
      <clargs type='input'   arg='-i' extension='.one'/>
      <clargs type='input'   arg='-a' extension='.two'/>
      <clargs type='output'  arg='-o'/>
    </Code>
\end{lstlisting}

If a code doesn't accept necessary Raven-editable auxiliary input files
or output filenames through the command line, the GenericCode interface
can also edit the input files and insert the filenames there.  For example,
in the previous example, say instead of \texttt{-a gen.two} and \texttt{-o myOut}
in the command line, \texttt{gen.one} has the following lines:
\begin{lstlisting}[language=python]
...
auxfile = gen.two
case = myOut
...
\end{lstlisting}
Then, our example XML for the code would be

Example:
\begin{lstlisting}[style=XML]
    <Code name="poly" subType="GenericCode">
      <executable>GenericInterface/poly_inp.py</executable>
      <inputExtentions>.one,.two</inputExtentions>
      <clargs   type='prepend' arg='python'/>
      <clargs   type='input'   arg='-i'  extension='.one'/>
      <fileargs type='input'   arg='two' extension='.two'/>
      <fileargs type='output'  arg='out'/>
    </Code>
\end{lstlisting}
and the corresponding template input file lines would be changed to read
\begin{lstlisting}[language=python]
...
auxfile = $RAVEN-two$
case = $RAVEN-out$
...
\end{lstlisting}


%%%%
If a code has hard-coded output file names that are not changeable,
the GenericCode interface can be invoked using the \xmlNode{outputFile}
node in which the output file name (CSV only) must be specified.
For example, in the previous example, say instead of \texttt{-a gen.two} and \texttt{-o myOut}
in the command line, the code always produce a CSV file named ``fixed\_output.csv'';

Then, our example XML for the code would be

Example:
\begin{lstlisting}[style=XML]
    <Code name="poly" subType="GenericCode">
      <executable>GenericInterface/poly_inp.py</executable>
      <inputExtentions>.one,.two</inputExtentions>
      <clargs   type='prepend' arg='python'/>
      <clargs   type='input'   arg='-i'  extension='.one'/>
      <fileargs type='input'   arg='two' extension='.two'/>
      <outputFile>fixed_output.csv</outputFile>
    </Code>
\end{lstlisting}

In addition, the ``wild-cards'' above can contain two special and optional symbols:
\begin{itemize}
  \item  \texttt{:}, that defines an eventual default value;
  \item  \texttt{|}, that defines the format of the value. The  Generic Interface currently supports the following formatting options (* in the examples means blank space):
    \begin{itemize}
       \item \textbf{plain integer}, in this case  the value that is going to be replaced by the Generic Interface, will be left-justified with a string length equal to the integer value specified here (e.g. ``\texttt{|}6'', the value is left-justified with a string length of 6);
      \item \textbf{d}, signed integer decimal, the value is going to be formatted as an integer (e.g.  if the value is 9 and the format ``\texttt{|}10d'', the replaced value will be formatted as follows: ``*********9'');
      \item \textbf{e}, floating point exponential format (lowercase), the value is going to be formatted as a float in scientific notation (e.g. if the value is 9.1234 and the format ``\texttt{|}10.3e'', the replaced value will be formatted as follows: ``*9.123e+00'' );
      \item \textbf{E}, floating point exponential format (uppercase), the value is going to be formatted as a float in scientific notation (e.g. if the value is 9.1234 and the format ``\texttt{|}10.3E'', the replaced value will be formatted as follows: ``*9.123E+00'' );
      \item \textbf{f or F}, floating point decimal format, the value is going to be formatted as a float in decimal notation (e.g. if the value is 9.1234 and the format ``\texttt{|}10.3f'', the replaced value will be formatted as follows: ``*****9.123'' );
      \item \textbf{g}, floating point format. Uses lowercase exponential format if exponent is less than -4 or not less than precision, decimal format otherwise (e.g. if the value is 9.1234 and the format ``\texttt{|}10.3g'', the replaced value will be formatted as follows: ``******9.12'' );
      \item \textbf{G}, floating point format. Uses uppercase exponential format if exponent is less than -4 or not less than precision, decimal format otherwise (e.g. if the value is 0.000009 and the format ``\texttt{|}10.3G'', the replaced value will be formatted as follows: ``*****9E-06'' ).
    \end{itemize}|
\end{itemize}
For example:
\begin{lstlisting}[language=python]
...
auxfile = $RAVEN-two:3$
case = $RAVEN-out:5|10$
...
\end{lstlisting}
Where,
\begin{itemize}
  \item  \texttt{:}, in case the variable ``two'' is not defined in the RAVEN XML input file, the Parser, will replace it with the value ``3''.;
  \item  \texttt{|}, the value that is going to be replaced by the Generic Interface, will be left- justified with a string length of ``10'';
\end{itemize}

%%%%%%%%%%%%%%%%%%%%%%%%%%%%%%%%%%%%%
%%%%%% RAVEN  INTERFACE  (RAVEN running RAVEN) %%%%%%
%%%%%%%%%%%%%%%%%%%%%%%%%%%%%%%%%%%%%
\subsection{RAVEN Interface}
\label{subsec:RAVENInterface}
The RAVEN interface is meant to provide the possibility to execute a RAVEN input file
driving a set of SLAVE RAVEN calculations. For example, if the user wants to optimize the parameters
of a surrogate model (e.g. minimizing the distance between the surrogate predictions and the real data), he
can achieve this task by setting up  a RAVEN input file (master) that performs an optimization on the feature
space characterized by the surrogate model parameters, whose training and validation assessment  is performed in the SLAVE
RAVEN runs.
\\ There are some limitations for this interface:
\begin{itemize}
\item only one  sub-level of RAVEN can be executed (i.e. if the SLAVE RAVEN input file contains the run of another RAVEN SLAVE, the MASTER RAVEN will error out)
\item only data from Outstreams of type Print can be collected by the MASTER RAVEN
\item only a maximum of two Outstreams can be collected (1 PointSet and 1 HistorySet)
\end{itemize}


Like for every other interface, most of the RAVEN workflow stays the same independently of which type of Model (i.e. Code) is used.
\\ Similarly to any other code interface, the user provides paths to executables and aliases for sampled variables within the
\xmlNode{Models} block.  The \xmlNode{Code} block will contain attributes \xmlAttr{name} and
\xmlAttr{subType}.  \xmlAttr{name} identifies that particular \xmlNode{Code} model within RAVEN, and
\xmlAttr{subType} specifies which code interface the model will use (In this case \xmlAttr{subType}=``RAVEN'').
The \xmlNode{executable}
block should contain the absolute or relative (with respect to the current working
directory) path to the RAVEN framework script (\textbf{raven\_framework}).
\\ In addition to the attributes and xml nodes reported above, the RAVEN accepts the following XML nodes (required and optional):
\begin{itemize} % nodes for code
  \item  \xmlNode{outputDatabase}, \xmlDesc{string, required parameter}
    will specify the \xmlNode{Database} that will be loaded as outputs of the INNER RAVEN. If this node
    is not specifed, \xmlNode{outputExportOutStreams} may be used instead.
  \item  \xmlNode{outputExportOutStreams}, \xmlDesc{comma separated list,
    required parameter} will specify the  \xmlNode{OutStreams} that will be loaded as outputs of the SLAVE RAVEN.
    Maximum two  \xmlNode{OutStreams} can be listed here (1 for PointSet and/or 1 for HistorySet).
  \item  \xmlNode{conversion}, \xmlDesc{Node,optional parameter} will specify details of conversion scripts to
    be used in creating the inner RAVEN input file.  This node contains the following nodes:
    \begin{itemize} % nodes for conversion
      \item \xmlNode{module}, \xmlDesc{Node, optional parameter} a module for directly manipulating the xml
        structure of perturbed input files. This can be used to modify the template input file in arbitrary
        ways; however, it should be used with caution, and is considered an advanced method.
        This node has the following attribute:
        \begin{itemize} % attributes for module
          \item \xmlAttr{source}, \xmlDesc{string, required} provides the path to the manipulation module including
              the module file itself.  The following method should be defined in order to perform the input
              manipulation:
              \begin{itemize} % functions for module
                 \item \textbf{\textit{modifyInput}}, manipulates the input in arbitrary ways. This method
                   takes two arguments. The first is the root \xmlNode{Simulation} node of the template input
                   file that has already been modified with the perturbed samples (the object is a Python
                   \texttt{xml.etree.ElementTree.Element} object). The second input is a dictionary with all
                   the modification information used to previously modify the template xml. The method should
                   return the modified root.
                 Example:
                  \begin{lstlisting}[language=python]
import xml.etree.ElementTree as ET
def modifyInput(root, modDict):
  """
    Manipulate the inner RAVEN xml input.
    @ In, root, ET.Element, perturbed RAVEN input
    @ In, modDict, dictionary, modifications made to the input
    @ Out, root, ET.Element, modified RAVEN input
  """
  # adds the file <Input name='aux_inp'>auxfile.txt</Input> to the <Files> node
  filesNode = root.find('Files')
  newNode = ET.Element('Input')
  newNode.text = 'auxfile.txt'
  newNode.attrib['name'] = 'aux_inp'
  filesNode.append(newNode)
  return root
                   \end{lstlisting}
              \end{itemize} % end functions for module
        \end{itemize} % end attributes for module

      \item \xmlNode{module}, \xmlDesc{Node, optional parameter} contains the information about a specific
        conversion module (python file).  This node can be repeated multiple times.
        This node has the following attribute:
        \begin{itemize} % attributes for module
          \item \xmlAttr{source}, \xmlDesc{string, required} provides the path to the conversion module including
              the module file itself.  There are two methods that can be placed in the conversion module:
              \begin{itemize} % functions for module
                 \item \textbf{\textit{manipulateScalarSampledVariables}}, a method that is aimed to manipulate sampled variables and to create more in case needed.
                 Example:
                  \begin{lstlisting}[language=python]
def manipulateScalarSampledVariables(sampledVariables):
  """
  This method is aimed to manipulate scalar variables.
  The user can create new variables based on the
  variables sampled by RAVEN
   @ In, sampledVariables, dict, dictionary of
       sampled variables ({"var1":value1,"var2":value2})
   @ Out, None, the new variables should be
                 added in the "sampledVariables" dictionary
  """
  newVariableValue =
    sampledVariables['Distributions|Uniform@name:a_dist|lowerBound']
    + 1.0
  sampledVariables['Distributions|Uniform@name:a_dist|upperBound'] =
    newVariableValue
  return
                   \end{lstlisting}

                 \item \textbf{\textit{convertNotScalarSampledVariables}}, a method that is aimed to convert not scalar variables (e.g. 1D arrays) into multiple scalar variables
                 (e.g.  \xmlNode{constant}(s) in a sampling strategy).
                  This method is going to be required in case not scalar variables are detected by the interface.
                  Example:
                  \begin{lstlisting}[language=python]
 def convertNotScalarSampledVariables(noScalarVariables):
   """
   This method is aimed to convert not scalar
   variables into multiple scalar variables. The user MUST
    create new variables based on the not Scalar Variables
     sampled (and passed in) by RAVEN
   @ In, noScalarVariables, dict, dictionary of sampled
        variables that are not scalar ({"var1":1Darray1,"var2":1Darray2})
   @ Out, newVars, dict,  the new variables that have
        been created based on the not scalar variables
        contained in "noScalarVariables" dictionary
   """
   oneDimensionalArray =
       noScalarVariables['temperatureHistory']
   newVars = {}
   for cnt, value in enumerate(oneDimensionalArray):
     newVars['Samplers|MonteCarlo@name:myMC|constant'+
                '@name=temperatureHistory'+str(cnt)] =
                oneDimensionalArray[cnt]
   return newVars
                  \end{lstlisting}
              \end{itemize} % end functions for module
        \end{itemize} % end attributes for module
        The \xmlNode{module} node also takes the following node:
        \begin{itemize} % nodes of module
          \item \xmlNode{variables}, \xmlDesc{comma-separated list, required} provides a comma-separated list of
            the variables from the MASTER RAVEN that need to be accessed by the conversion script module.  The
            variables listed here use the pipe naming system (un-aliased names).
        \end{itemize} % end nodes of module
    \end{itemize} % end nodes for conversion
\end{itemize} % end nodes for code

Code input example:
\begin{lstlisting}[style=XML]
<Code name="RAVENrunningRAVEN" subType="RAVEN">
  <executable>../../../raven_framework</executable>
  <outputExportOutStreams>
     HistorySetOutStream,PointSetOutStream
  </outputExportOutStreams>
  <conversion>
    <module source=/Users/username/whateverConversionModule.py>
      <variables>a,b,x,y</variables>
    </module>
  </conversion>
</Code>
\end{lstlisting}

Like for every other interface,  the syntax of the variable names is important to make the parser understand how to perturb an input file.
\\ For the RAVEN interface, a syntax inspired by the XPath nomenclature is used.
\begin{lstlisting}[style=XML]
<Samplers>
  <MonteCarlo name="MC_external">
     ...
    <variable name="Models|ROM@subType:SciKitLearn@name:ROM1|C">
      <distribution>C_distrib</distribution>
    </variable>
    <variable name="Models|ROM@subType:SciKitLearn@name:ROM1|tol">
      <distribution>toll_distrib</distribution>
    </variable>
    <variable name="Samplers|Grid@name:'+
          'GridName|variable@name:var1|grid@construction:equal@type:value@steps">
      <distribution>categorical_step_distrib</distribution>
    </variable>
    ...
  </MonteCarlo>
</Samplers>
\end{lstlisting}
In the above example, it can be inferred that each XML node (subnode) needs to be separated by a ``|'' separator. In addition,
every time an XML node has attributes, the user can specify them using the ``@'' separator to specify a value for them.
The first variable above will be pointing to the following XML sub-node ( \xmlNode{C}):
\begin{lstlisting}[style=XML]
<Models>
  <ROM name="ROM1" subType="SciKitLearn">
     ...
     <C>10.0</C>
    ...
 </ROM>
</Models>
\end{lstlisting}
The second variable above will be pointing to the following XML sub-node ( \xmlNode{tol}):
\begin{lstlisting}[style=XML]
<Models>
  <ROM name="ROM1" subType="SciKitLearn">
     ...
     <tol>0.0001</tol>
    ...
 </ROM>
</Models>
\end{lstlisting}
The third variable above will be pointing to the following XML attribute ( \xmlAttr{steps}):
\begin{lstlisting}[style=XML]
<Samplers>
  <Grid name="GridName">
     ...
    <variable name="var1">
       ...
      <grid construction="equal" type="value" steps="1">0 1</grid>
       ...
    </variable>

    ...
  </MonteCarlo>
</Samplers>
\end{lstlisting}

The above nomenclature must be used for all the variables to be sampled and for the variables generated by the two methods contained, in case, in
the module that gets specified by the \xmlNode{conversionModule} in the \xmlNode{Code} section.
\\ Finally the SLAVE RAVEN input file (s) must be ``tagged'' with the attribute  \xmlAttr{type="raven"} in the Files section. For example,

\begin{lstlisting}[style=XML]
<Files>
    <Input name="slaveRavenInputFile" type="raven" >
      test_rom_trainer.xml
    </Input>
</Files>
\end{lstlisting}

\subsubsection{ExternalXML and RAVEN interface}
Care must be taken if the SLAVE RAVEN uses \xmlNode{ExternalXML} nodes.  In this case, each file containing
external XML nodes must be added in the \xmlNode{Step} as an \xmlNode{Input} class \xmlAttr{Files} to make sure it gets copied to
the individual run directory.  The type for these files can be anything, with the exception of type
\xmlString{raven}.

%%%%%%%%%%%%%%%%%%%%%%%%%%%%
%%%%%% RELAP5  INTERFACE  %%%%%%
%%%%%%%%%%%%%%%%%%%%%%%%%%%%
\subsection{RELAP5 Interface}
\label{subsec:RELAP5Interface}

\subsubsection{Sequence}
In the \xmlNode{Sequence} section, the names of the steps declared in the
\xmlNode{Steps} block should be specified.
%
As an example, if we called the first multirun ``Grid\_Sampler'' and the second
multirun ``MC\_Sampler'' in the sequence section we should see this:
\begin{lstlisting}[style=XML]
<Sequence>Grid_Sampler,MC_Sampler</Sequence>
\end{lstlisting}
%%%%%%%%%%%%%%%%%%%%%%%%%%%%%%%%%%%%%%%%%%%%%%%%%%%

\subsubsection{batchSize and mode}
For the \xmlNode{batchSize} and \xmlNode{mode} sections please refer to the
\xmlNode{RunInfo} block in the previous chapters.
%
%%%%%%%%%%%%%%%%%%%%%%%%%%%%%%%%%%%%%%%%%%%%%%%%%%%%
\subsubsection{RunInfo}
After all of these blocks are filled out, a standard example RunInfo block may
look like the example below:
\begin{lstlisting}[style=XML]
<RunInfo>
  <WorkingDir>~/workingDir</WorkingDir>
  <Sequence>Grid_Sampler,MC_Sampler</Sequence>
  <batchSize>1</batchSize>
  <mode>mpi</mode>
  <expectedTime>1:00:00</expectedTime>
  <ParallelProcNumb>1</ParallelProcNumb>
</RunInfo>
\end{lstlisting}
%%%%%%%%%%%%%%%%%%%%%%%%%%%%%%%%%%%%%%%%%%%%%%%%%%%%%%%%%%%
\subsubsection{Files}
In the \xmlNode{Files} section, as specified before, all of the files needed for
the code to run should be specified.
%
In the case of RELAP5, the files typically needed are:
\begin{itemize}
  \item RELAP5 Input file
  \item Table file or files that RELAP needs to run
\end{itemize}
Example:
\begin{lstlisting}[style=XML]
<Files>
  <Input name='tpfh2o' type=''>tpfh2o</Input>
  <Input name='inputrelap.i' type=''>X10.i</Input>
</Files>
\end{lstlisting}

It is a good practice to put inside the working directory all of these files and
also:
\begin{itemize}
  \item the RAVEN input file
  \item the license for the executable of RELAP5
\end{itemize}
\textcolor{red}{
\textbf{It is important to notice that the interface output collection relies on the MINOR EDITS. The user must specify the MINOR
EDITS block and those variables are the only one the INTERFACE will read and make available to RAVEN. In addition, it is important to notice that:}
\begin{itemize}
  \item \textbf{the simulation time is stored in a variable called \textit{``time''}};
  \item \textbf{all the variables specified in the MINOR EDIT block are going to be converted using underscores (e.g.  an edit such as
  $301 \:\:\: p \:\:\: 345010000$ will be named in the converted CSVs as $p\_345010000$).In addition, if a variable contains spaces, the trailing spaces
   are going to be removed and internal spaces are replaced with underscores (e.g. $HTTEMP 1131008 12$ will become $HTTEMP\_1131008\_12$}.
\end{itemize}
}

Remember also that a RELAP5 simulation run is considered successful (i.e., the simulation did not crash) if it terminates with the following
message:
\textcolor{red}{Transient terminated by end of time step cards}
or
\textcolor{red}{Transient terminated by trip}

If the a RELAP5 simulation run stops with messages other than this one (e.g., `` Transient terminated by failure.'') than the simulation is considered as
crashed, i.e., it will not be saved.
Hence, it is strongly recommended to set up the RELAP5 input file so that the simulation exiting conditions are set through control logic trip variables
(e.g., simulation mission time and clad temperature equal to clad failure temperature).

%%%%%%%%%%%%%%%%%%%%%%%%%%%%%%%%%%%%%%%%%%%%%%%%%%%%
\subsubsection{Models}
\label{subsubsection:Relap5Models}
For the \xmlNode{Models} block here is a standard example of how it would look
when using RELAP5 as the external model:
\begin{lstlisting}[style=XML]
<Models>
  <Code name='MyRELAP' subType='Relap5'>
    <executable>~/path_to_the_executable</executable>
  </Code>
</Models>
\end{lstlisting}
In case the \textbf{multi-deck} approach is used in RELAP5, the interface is going to load all the outputs in one CSV RAVEN is
going to read. This means that all the decks' outputs are going to be loaded in one of the Output of RAVEN. In case the user
wants to select the outputs coming from only one deck, the following XML node needs to be specified:
\begin{itemize}
   \item \xmlNode{outputDeckNumber}, \xmlDesc{integer, optional parameter}, the deck number from
   which the results needs to be retrieved. \default{all}.
\end{itemize}
In addition, if some command line parameters need to be passed to RELAP5 \\(e.g. ``-r
$\: restartFileWithCustomName.r$''), the user might use (optionally) the \xmlNode{clargs} XML nodes.
\begin{lstlisting}[style=XML]
<Models>
  <Code name='MyRELAP' subType='Relap5'>
    <executable>~/path_to_the_executable</executable>
    <outputDeckNumber>1</outputDeckNumber>
    <clargs type="text" arg="-r restartFileWithCustomName.r"/>
  </Code>
</Models>
\end{lstlisting}

An additional feature of the RELAP5 Code interface is the possibility to specify operation based on the value of user-inputted
cards. For example, let's assume the values in cards 1180801:2 and 1180802:2 must come from a calculation based on sampled variables (e.g. 20100154:2 and 20100155:2), the user can specify the following XML node:

\begin{itemize}
   \item \xmlNode{operator}, \xmlDesc{XML node, optional parameter}, The operator block.
    This XML node must contain the following attribute:
     \begin{itemize}
      \item \textit{variables}, \xmlDesc{comma separated list, required parameter}, The list of variables
       (coming from a Sampler) that will be used in the  \xmlNode{expression} XML node.
      \end{itemize}
     Within the  \xmlNode{operator} the following XML sub-nodes must be specified:
     \begin{itemize}
        \item \xmlNode{expression}, \xmlDesc{string, required parameter}, The string representing the expression to be
        performed. The ``card'' (if needed to be used) must be identified with the token \%card\% and it
         will be replaced with the values of the cards
        (specified in the XML node \xmlNode{cards} ) from the original input file. In this expression, all the functions available in
        the Python $math$ module can be used (e.g. $sqrt$, $exp$, $sin$, etc.).
         \item \xmlNode{cards}, \xmlDesc{comma separated list, required parameter},
          The list of cards in the original input file whose values need to be replaced by the value resulting from the expression
          contained in \xmlNode{expression}.
     \end{itemize}
    \nb the user can specify as many \xmlNode{operator} nodes as needed.
\end{itemize}
An example is reported below:
\begin{lstlisting}[style=XML]
<Models>
  <Code name='MyRELAP' subType='Relap5'>
    <executable>~/path_to_the_executable</executable>
    ...
    <operator variables="20100154:2,20100155:2">
      <expression> %card%*20100155:2*2./20100155:2</expression>
      <cards>1180801:2,1180802:2,1180901:3</cards>
    </operator>
    ...
  </Code>
</Models>
\end{lstlisting}


%%%%%%%%%%%%%%%%%%%%%%%%%%%%%%%%%%%%%%%%%%%%%%%%%%%%%%%%%
\subsubsection{Distributions}
The \xmlNode{Distribution} block defines the distributions that are going
to be used for the sampling of the variables defined in the \xmlNode{Samplers}
block.
%
For all the possibile distributions and all their possible inputs please see the
chapter about Distributions (see~\ref{sec:distributions}).
%
Here we give a general example of three different distributions:
\begin{lstlisting}[style=XML,morekeywords={name,debug}]
<Distributions verbosity='debug'>
  <Triangular name='BPfailtime'>
    <apex>5.0</apex>
    <min>4.0</min>
    <max>6.0</max>
  </Triangular>
  <LogNormal name='BPrepairtime'>
    <mean>0.75</mean>
    <sigma>0.25</sigma>
  </LogNormal>
  <Uniform name='ScalFactPower'>
    <lowerBound>1.0</lowerBound>
    <upperBound>1.2</upperBound>
  </Uniform>
 </Distributions>
\end{lstlisting}

It is good practice to name the distribution something similar to what kind of
variable is going to be sampled, since there might be many variables with the
same kind of distributions but different input parameters.
%
%%%%%%%%%%%%%%%%%%%%%%%%%%%%%%%%%%%%%%%%%%%%%%%%%%%%%%%%%
\subsubsection{Samplers}
In the \xmlNode{Samplers} block we want to define the variables that are going
to be sampled.
%
\textbf{Example}:
We want to do the sampling of 3 variables:
\begin{itemize}
  \item Battery Fail Time
  \item Battery Repair Time
  \item Scaling Factor Power Rate
\end{itemize}

We are going to sample these 3 variables using two different sampling methods:
grid and MonteCarlo.

In RELAP5, the sampler reads the variable as, given the name, the first number
is the card number and the second number is the word number.
%
In this example we are sampling:
\begin{itemize}
  \item For card 0000588 (trip) the word 6 (battery failure time)
  \item For card 0000575 (trip) the word 6 (battery repair time)
  \item For card 20210000 (reactor power) the word 4 (reactor scaling factor)
\end{itemize}

We proceed to do so for both the Grid sampling and the MonteCarlo sampling.

\begin{lstlisting}[style=XML,morekeywords={name,type,construction,lowerBound,steps,limit,initialSeed}]
<Samplers verbosity='debug'>
  <Grid name='Grid_Sampler' >
    <variable name='0000588:6'>
      <distribution>BPfailtime</distribution>
      <grid type='value' construction='equal'  steps='10'>0.0 28800</grid>
    </variable>
    <variable name='0000575:6'>
      <distribution>BPrepairtime</distribution>
      <grid type='value' construction='equal' steps='10'>0.0 28800</grid>
    </variable>
    <variable name='20210000:4'>
      <distribution>ScalFactPower</distribution>
      <grid type='value' construction='equal' steps='10'>1.0 1.2</grid>
    </variable>
  </Grid>
  <MonteCarlo name='MC_Sampler'>
     <samplerInit>
       <limit>1000</limit>
     </samplerInit>
    <variable name='0000588:6'>
      <distribution>BPfailtime</distribution>
    </variable>
    <variable name='0000575:6'>
      <distribution>BPrepairtime</distribution>
    </variable>
    <variable name='20210000:4'>
      <distribution>ScalFactPower</distribution>
    </variable>
  </MonteCarlo>
</Samplers>
\end{lstlisting}

In case the RELAP5 input file is a multi-deck, the user can specify the deck to which each sampled variable
corresponds to. As an example, the following sampling strategy:

\begin{lstlisting}[style=XML,morekeywords={name,type,construction,lowerBound,steps,limit,initialSeed}]
<MonteCarlo name='MC_Sampler'>
   <samplerInit>
     <limit>1000</limit>
   </samplerInit>
  <variable name='1|0000588:6'>
    <distribution>BPfailtime</distribution>
  </variable>
  <variable name='2|0000575:6'>
    <distribution>BPrepairtime</distribution>
  </variable>
</MonteCarlo>
</Samplers>
\end{lstlisting}
performs:
\begin{itemize}
  \item the sampling of the distribution \\\xmlNode{BPfailtime} and it provides the sampled value
        to the 6th word of card 0000588 for the first deck
  \item the sampling of the distribution \\\xmlNode{BPrepairtime} and it provides the sampled value
        to the 6th word of card 0000575 for the second deck
\end{itemize}

It can be seen that each variable is connected with a proper distribution
defined in the \\\xmlNode{Distributions} block (from the previous example).
%
The following demonstrates how the input for the first variable is read.

We are sampling a a variable situated in word 6 of the card 0000588 using a Grid
sampling method.
%
The distribution that this variable is following is a Triangular distribution
(see section above).
%
We are sampling this variable beginning from 0.0 in 10 \textit{equal} steps of
2880.
%
In case of Dynamic Event Tree-based sampling, the input is very similar to the other
sampling strategies with the only ``limitation'' that the sampled variables
directly linked to the Dynamic Event Tree must be part of a RELAP5 trip.
In case, other variables must be sampled, they are considered ``epistemic'' variables and
should be sampled using the Hybrid Dynamic Event Tree approach.


For example, the Dynamic Event Tree sampling and the  Hybrid Dynamic Event Tree sampling would look
like the following:

\begin{lstlisting}[style=XML,morekeywords={name,type,construction,lowerBound,steps,limit,initialSeed}]
<Samplers verbosity='debug'>
  <DynamicEventTree name='DET'>
    <variable name='414:6'>
      <distribution>endtimedist</distribution>
      <grid type='CDF' construction='custom'>0.1 0.3 0.99</grid>
    </variable>
    <variable name='454:6'>
      <distribution>endtime2dist</distribution>
      <grid type='CDF' construction='custom'>0.11 0.5 0.99</grid>
    </variable>
  </DynamicEventTree>

  <DynamicEventTree name='HDET'>
    <HybridSampler type="MonteCarlo">
      <!-- in here we specify the epistemic like variables -->
      <samplerInit>
        <limit>10</limit>
      </samplerInit>
      <variable name="200:1">
        <distribution>missionTimeDist</distribution>
      </variable>
    </HybridSampler>
    <variable name='414:6'>
      <distribution>endtimedist</distribution>
      <grid type='CDF' construction='custom'>0.1 0.3 0.99</grid>
    </variable>
    <variable name='454:6'>
      <distribution>endtime2dist</distribution>
      <grid type='CDF' construction='custom'>0.11 0.5 0.99</grid>
    </variable>
  </DynamicEventTree>
</Samplers>
\end{lstlisting}

%%%%%%%%%%%%%%%%%%%%%%%%%%%%%%%%%%%%%%%%%%%%%%%%%%%%%%%%%%%
\subsubsection{Steps}
For a RELAP interface, the \xmlNode{MultiRun} step type will most likely be
used.
%
First, the step needs to be named: this name will be one of the names used in
the \xmlNode{Sequence} block.
%
In our example, \texttt{Grid\_Sampler} and \texttt{MC\_Sampler}.
%
\begin{lstlisting}[style=XML,morekeywords={name,debug,re-seeding}]
     <MultiRun name='Grid_Sampler' verbosity='debug'>
\end{lstlisting}

With this step, we need to import all the files needed for the simulation:
\begin{itemize}
  \item RELAP input file
  \item element tables -- tpfh2o
\end{itemize}
\begin{lstlisting}[style=XML,morekeywords={name,class,type}]
    <Input   class='Files' type=''>inputrelap.i</Input>
    <Input   class='Files' type=''>tpfh2o</Input>
\end{lstlisting}
We then need to define which model will be used:
\begin{lstlisting}[style=XML]
    <Model  class='Models' type='Code'>MyRELAP</Model>
\end{lstlisting}
We then need to specify which Sampler is used, and this can be done as follows:
\begin{lstlisting}[style=XML]
    <Sampler class='Samplers' type='Grid'>Grid_Sampler</Sampler>
\end{lstlisting}
And lastly, we need to specify what kind of output the user wants.
%
For example the user might want to make a database (in RAVEN the database
created is an HDF5 file).
%
Here is a classical example:
\begin{lstlisting}[style=XML,morekeywords={class,type}]
    <Output  class='Databases' type='HDF5'>Grid_out</Output>
\end{lstlisting}
Following is the example of two MultiRun steps which use different sampling
methods (grid and Monte Carlo), and creating two different databases for each
one:
\begin{lstlisting}[style=XML]
<Steps verbosity='debug'>
  <MultiRun name='Grid_Sampler' verbosity='debug'>
    <Input   class='Files' type=''>inputrelap.i</Input>
    <Input   class='Files'     type=''    >tpfh2o</Input>
    <Model   class='Models'    type='Code'>MyRELAP</Model>
    <Sampler class='Samplers'  type='Grid'>Grid_Sampler</Sampler>
    <Output  class='Databases' type='HDF5'>Grid_out</Output>
  </MultiRun>
  <MultiRun name='MC_Sampler' verbosity='debug' re-seeding='210491'>
    <Input   class='Files' type=''>inputrelap.i</Input>
    <Input   class='Files'     type=''          >tpfh2o</Input>
    <Model   class='Models'    type='Code'      >MyRELAP</Model>
    <Sampler class='Samplers'  type='MonteCarlo'>MC_Sampler</Sampler>
    <Output  class='Databases' type='HDF5'      >MC_out</Output>
  </MultiRun>
</Steps>
\end{lstlisting}
%%%%%%%%%%%%%%%%%%%%%%%%%%%%%%%%%%%%%%%%%%%%%%%%%%%%%%
\subsubsection{Databases}
As shown in the \xmlNode{Steps} block, the code is creating two database objects
called \texttt{Grid\_out} and \texttt{MC\_out}.
%
So the user needs to input the following:
\begin{lstlisting}[style=XML]
<Databases>
  <HDF5 name="Grid_out" readMode="overwrite"/>
  <HDF5 name="MC_out" readMode="overwrite"/>
</Databases>
\end{lstlisting}
As listed before, this will create two databases.
%
The files will have names corresponding to their \xmlAttr{name} appended with
the .h5 extension (i.e. \texttt{Grid\_out.h5} and \texttt{MC\_out.h5}).

\subsubsection{Modified Version of the Institute of Nuclear Safety System Incorporated (Japan)}
The Institute of Nuclear Safety System Incorporated (Japan) has modified the \textbf{RELAP5}  source code
in order to be able to control some additional parameters from an auxiliary input file (\textbf{modelPar.inp}).
\\In order to use this interface, the user needs to input the $subType$ attribute\textbf{Relap5inssJp}:
\begin{lstlisting}[style=XML]
<Models>
  <Code name='MyRELAP' subType='Relap5'>
    <executable>~/path_to_the_executable</executable>
    <!-- here is taking the output from the first deck only -->
    <outputDeckNumber>1</outputDeckNumber>
  </Code>
</Models>
\end{lstlisting}
For perturbing such input file, the approach presented in section \ref{subsec:genericInterface} (Generic Interface)
has been employed. For the standard \textbf{RELAP5} input, the same approach previously in this section is used.
\\For example, in the following Sampler block, the card $9100101$ is perturbed with the same approach used in standard \textbf{RELAP5}; in addition, the variable $modelParTest$  is going to be perturbed in the \textbf{modelPar.inp} input file.
\begin{lstlisting}[style=XML]
    <MonteCarlo name="mc_loca">
      <samplerInit>
        <limit>1</limit>
      </samplerInit>
      <variable name="9100101:3">
        <distribution>break_size</distribution>
      </variable>
      <variable name="modelParTest">
          <distribution>break_size</distribution>
      </variable>
    </MonteCarlo>
\end{lstlisting}

%%%%%%%%%%%%%%%%%%%%%%%%%%%
%%%%%% RELAP7 INTERFACE  %%%%%%
%%%%%%%%%%%%%%%%%%%%%%%%%%%
\subsection{RELAP7 Interface}
This section covers the input specifications for running RELAP7 through RAVEN.
It is important to notice that this short explanation assumes that the reader already knows
how to use the control logic system in RELAP7.
Since the presence of the control logic system in RELAP7, this code interface is different with respect to the others
and uses some special keyword available in RAVEN (see the following).

\subsubsection{Files}
In the \xmlNode{Files} section, as specified before, all of the files needed for
the code to run should be specified.
%
In the case of RELAP7, the files typically needed are the following:
\begin{itemize}
  \item RELAP7 Input file
  \item Control Logic file
\end{itemize}
Example:
\begin{lstlisting}[style=XML]
<Files>
  <Input name='nat_circ.i' type=''>nat_circ.i</Input>
  <Input name='control_logic.py' type=''>control_logic.py</Input>
</Files>
\end{lstlisting}
The RAVEN/RELAP7 interface recognizes as RELAP7 inputs the files with the extensions  ``*.i'', ``*.inp'' and ``*.in''.

%%%%%%%%%%%%%%%%%%%%%%%%%%%%%%%%%%%%%%%%%%%%%%%%%%%%%%%%
\subsubsection{Models}
For the \xmlNode{Models} block RELAP7 uses the RAVEN executable, since through this executable the stochastic
environment gets activated (possibility to sample parameters directly in the control logic system)
%
Here is a standard example of what can be used to use RELAP7 as the model:
\begin{lstlisting}[style=XML]
<Models>
    <Code name='MyRAVEN' subType='RAVEN'><executable>~path/to/RAVEN-opt</executable></Code>
</Models>
\end{lstlisting}
%%%%%%%%%%%%%%%%%%%%%%%%%%%%%%%%%%%%%%%%%%%%%%%%%%%%%%%%
\subsubsection{Distributions}
As for all the other codes interfaces  the \xmlNode{Distributions} block needs to be specified in order to employ
as sampling strategy (e.g. MonteCarlo, Stratified, etc.). In this block, the user specifies the distributions that need to be used.
Once the user defines the distributions in this block, RAVEN activates the Distribution environment in the RAVEN/RELAP7 control logic
system. The sampling of the parameters is then performed directly in the control logic input file.

%
For example, let's consider the sampling of a normal distribution for the primary pressure in
RELAP7:
%
\begin{lstlisting}[style=XML]
<Distributions>
 <Normal name="Prim_Pres">
 <mean>1000000</mean>
 <sigma>100<sigma/>
 </Normal>
</Distributions>
\end{lstlisting}
In order to change a parameter (independently on the sampling strategy), the control logic input file should be modified as follows:
%\lstset{margin=1.5cm}
\begin{lstlisting}[language=Python]
def initial_function(monitored, controlled, auxiliary)
    print("monitored",monitored,"controlled",
    controlled,"auxiliary",auxiliary)

    controlled.pressureInPressurizer =
     distributions.Prim_Pres.getDistributionRandom()
    return
\end{lstlisting}

%%%%%%%%%%%%%%%%%%%%%%%%%%%%%%%%%%%%%%%%%%%%%%%
\subsubsection{Samplers}
In the \xmlNode{Samplers} block, all the variables that needs to be sampled must be specified.
In case some of these variables are directly sampled in the Control Logic system, the
\xmlNode{variable} needs to be replaced with \xmlNode{Distribution}. In this way, RAVEN is able
to understand which variables needs to be directly modified through input file (i.e. modifying the original
input file *.i)  and which variables are going to be ``sampled'' through the control logic system.
%
For the example, we are performing Grid Sampling.
%
The global initial pressure wasn't specified in the control logic so it is going to be specified
using the node \xmlNode{variable}. The ``pressureInPressurizer'' variable is instead sampled in the
control logic system; for this reason, it is going to be specified using the node  \xmlNode{Distribution}.
%
For example,
%
\begin{lstlisting}[style=XML]
<Samplers>
 <Grid name="MC_samp">
   <samplerInit> <limit>500</limit> </samplerInit>
   <variable name="GlobalParams|global_init_P">
      <distribution>Prim_Pres</distribution>
      <grid construction="equal" steps="10" type="CDF">0.0 1.0</grid>
   </variable>
   <Distribution name="pressureInPressurizer">
      <distribution>Prim_Pres</distribution>
      <grid construction="equal" steps="10" type="CDF">0.0 1.0</grid>
   </Distribution>
 </Grid>
</Samplers>
\end{lstlisting}


%%%%%%%%%%%%%%%%%%%%%%%%%%%%%%%%%%
%%%%%% MooseBasedApp INTERFACE  %%%%%%
%%%%%%%%%%%%%%%%%%%%%%%%%%%%%%%%%%
\subsection{MooseBasedApp Interface}
\subsubsection{Files}
In the \xmlNode{Files} section, as specified before, all of the files needed for
the code to run should be specified.
%
In the case of any MooseBasedApp, the files typically needed are the following:
\begin{itemize}
  \item MooseBasedApp GetPot input file
  \item Restart Files (if the calculation is instantiated from a restart point)
  \item Mesh Files (in case the mesh is externally specified)
  \item Any other generic input file (CSVs with Power histories, boundary conditions files, etc.)
\end{itemize}
Example:
\begin{lstlisting}[style=XML]
<Files>
  <Input name='mooseBasedApp.i' type=''>mooseBasedApp.i</Input>
  <Input name='0020_mesh.cpr' type=''>0020_mesh.cpr</Input>
  <Input name='0020.xdr.0000' type="">0020.xdr.0000</Input>
  <Input name='0020.rd-0' type="">0020.rd-0</Input>
  <Input name='exodus_mesh.e' type="">exodus_mesh.e</Input>
  <Input name='a_generic_additional_input_file.csv' type="Generic">a_generic_additional_input_file.csv</Input>
</Files>
\end{lstlisting}
If any file is tagged with the type \texttt{Generic}, it will be perturbable with the approach (wildcards)
 explained in the generic code interface (see  \ref{subsec:genericInterface}).



%%%%%%%%%%%%%%%%%%%%%%%%%%%%%%%%%%%%%%%%%%%%%%%%%%%%%%%%
\subsubsection{Models}
In the \xmlNode{Models} block particular MooseBasedApp executable needs to be specified.
%
Here is a standard example of what can be used to use with a typical MooseBasedApp (Bison) as the model:
\begin{lstlisting}[style=XML]
<Models>
    <Code name='MyMooseBasedApp' subType='MooseBasedApp'><executable>~path/to/Bison-opt</executable></Code>
</Models>
\end{lstlisting}
%%%%%%%%%%%%%%%%%%%%%%%%%%%%%%%%%%%%%%%%%%%%%%%%%%%%%%%%
%%%%%%%%%%%%%%%%%%%%%%%%%%%%%%%%%%%%%%%%%%%%%%%%%%%%%%%%%
\subsubsection{Distributions}
The \xmlNode{Distributions} block defines the distributions that are going
to be used for the sampling of the variables defined in the \xmlNode{Samplers}
block.
%
For all the possible distributions and all their possible inputs please see the
chapter about Distributions (see~\ref{sec:distributions}).
%
Here we give a general example of three different distributions:
\begin{lstlisting}[style=XML,morekeywords={name,debug}]
<Distributions>
    <Normal name='ThermalConductivity1'>
        <mean>1</mean>
        <sigma>0.001</sigma>
        <lowerBound>0.5</lowerBound>
        <upperBound>1.5</upperBound>
    </Normal>
    <Normal name='SpecificHeat'>
        <mean>1</mean>
        <sigma>0.4</sigma>
        <lowerBound>0.5</lowerBound>
        <upperBound>1.5</upperBound>
    </Normal>
    <Triangular name='ThermalConductivity2'>
        <apex>1</apex>
        <min>0.1</min>
        <max>4</max>
    </Triangular>
</Distributions>
\end{lstlisting}

It is good practice to name the distribution something similar to what kind of
variable is going to be sampled, since there might be many variables with the
same kind of distributions but different input parameters.
%
%%%%%%%%%%%%%%%%%%%%%%%%%%%%%%%%%%%%%%%%%%%%%%%%%%%%%%%%%
\subsubsection{Samplers}
In the \xmlNode{Samplers} block we want to define the variables that are going
to be sampled.
%
\textbf{Example}:
We want to do the sampling of 3 variables:
\begin{itemize}
  \item Thermal Conductivity of the Fuel;
  \item Specific Heat Transfer Ratio of the Cladding;
  \item Thermal Conductivity of the Cladding.
\end{itemize}

We are going to sample these 3 variables using two different sampling methods:
Grid and Monte-Carlo.

In order to perturb any MooseBasedApp, the user needs to specify the variables to be
sampled indicating the path to the value separated with the symbol ``$|$''. For example,
if the variable that we want to perturb is specified in the input as follows:
\begin{lstlisting}[style=XML]
[Materials]
  ...
  [./heatStructure]
     ...
     thermal_conductivity = 1.0
     ...
  [../]
  ...
[]
\end{lstlisting}
the variable name in the Sampler input block needs to be named as follows:
\begin{lstlisting}[style=XML]
...
<Samplers>
  <aSampler name='aUserDefinedName' >
    <variable name='Materials|heatStructure|thermal_conductivity'>
      ...
    </variable>
  </aSampler>
</Samplers>
...
\end{lstlisting}
%
In case some variables in external (\texttt{Generic} input files) need to be perturbed,
the wildcard approach can be used (for those variables):
\begin{lstlisting}[style=XML]
...
<Samplers>
  <aSampler name='aUserDefinedName' >
    <variable name='aWildCard1'>
      ...
    </variable>
        <variable name='aWildCard2'>
      ...
    </variable>
    <variable name='Materials|heatStructure|thermal_conductivity'>
      ...
    </variable>
  </aSampler>
</Samplers>
...
\end{lstlisting}
In this case the tagged file (\texttt{Generic}) will be parsed to find the variables
\texttt{\$RAVEN-aWildCard1\$} and \texttt{\$RAVEN-aWildCard1\$} and to replace their values
with the corresponding sampled variables (for more details, see  \ref{subsec:genericInterface})

In this example, we proceed to do so for both the Grid sampling and the Monte-Carlo sampling.
\begin{lstlisting}[style=XML,morekeywords={name,type,construction,lowerBound,steps,limit,initialSeed}]
<Samplers verbosity='debug'>
    <Grid name='myGrid'>
      <variable name='Materials|heatStructure1|thermal_conductivity' >
        <distribution>ThermalConductivity1</distribution>
        <grid         type='value' construction='custom' >0.6 0.7 0.8</grid>
      </variable>
      <variable name='Materials|heatStructure1|specific_heat' >
        <distribution >SpecificHeat</distribution>
        <grid         type='CDF'    construction='custom'>0.5 1.0 0.0</grid>
      </variable>
      <variable name='Materials|heatStructure2|thermal_conductivity'>
        <distribution  >ThermalConductivity2</distribution>
        <grid type='value' upperBound='4' construction='equal' steps='1'>0.5</grid>
      </variable>
      <variable name='aWildCard1'>
        <distribution  >ThermalConductivity2</distribution>
        <grid type='value' upperBound='4' construction='equal' steps='1'>0.5</grid>
      </variable>
    </Grid>
  <MonteCarlo name='MC_Sampler' limit='1000'>
      <variable name='Materials|heatStructure1|thermal_conductivity' >
        <distribution>ThermalConductivity1</distribution>
      </variable>
      <variable name='Materials|heatStructure1|specific_heat' >
        <distribution >SpecificHeat</distribution>
      </variable>
      <variable name='Materials|heatStructure2|thermal_conductivity'>
        <distribution  >ThermalConductivity2</distribution>
      </variable>
      <variable name='aWildCard1'>
        <distribution  >ThermalConductivity2</distribution>
      </variable>
  </MonteCarlo>
</Samplers>
\end{lstlisting}
%%%%%%%%%%%%%%%%%%%%%%%%%%%%%%%%%%%%%%%%%%%%%%%%%%%%%%%%%%%
\subsubsection{Steps}
For a MooseBasedApp, the \xmlNode{MultiRun} step type will most likely be
used, as first step.
%
First, the step needs to be named: this name will be one of the names used in
the \xmlNode{Sequence} block.
%
In our example, \texttt{Grid\_Sampler} and \texttt{MC\_Sampler}.
%
\begin{lstlisting}[style=XML,morekeywords={name,debug,re-seeding}]
     <MultiRun name='Grid_Sampler' >
\end{lstlisting}

With this step, we need to import all the files needed for the simulation:
\begin{itemize}
  \item MooseBasedApp YAML input file;
  \item eventual restart files (optional);
  \item other auxiliary files (e.g., powerHistory tables, etc.).
\end{itemize}
\begin{lstlisting}[style=XML,morekeywords={name,class,type}]
    <Input   class='Files' type=''>mooseBasedApp.i</Input>
    <Input   class='Files' type=''>0020_mesh.cpr</Input>
    <Input   class='Files' type=''>0020.xdr.0000</Input>
    <Input   class='Files' type=''>0020.rd-0</Input>
\end{lstlisting}
We then need to define which model will be used:
\begin{lstlisting}[style=XML]
    <Model  class='Models' type='Code'>MyMooseBasedApp</Model>
\end{lstlisting}
We then need to specify which Sampler is used, and this can be done as follows:
\begin{lstlisting}[style=XML]
    <Sampler class='Samplers' type='Grid'>Grid_Sampler</Sampler>
\end{lstlisting}
And lastly, we need to specify what kind of output the user wants.
%
For example the user might want to make a database (in RAVEN the database
created is an HDF5 file) and a DataObject of type PointSet, to use in sub-sequential
post-processing.
%
Here is a classical example:
\begin{lstlisting}[style=XML,morekeywords={class,type}]
    <Output  class='Databases' type='HDF5'>MC_out</Output>
    <Output  class='DataObjects' type='PointSet'>MCOutData</Output>
\end{lstlisting}

Following is the example of two MultiRun steps which use different sampling
methods (grid and Monte Carlo), and creating two different databases for each
one:
\begin{lstlisting}[style=XML]
<Steps verbosity='debug'>
  <MultiRun name='Grid_Sampler' verbosity='debug'>
    <Input  class='Files' type=''>mooseBasedApp.i</Input>
    <Input  class='Files' type=''>0020_mesh.cpr</Input>
    <Input  class='Files' type='' >0020.xdr.0000</Input>
    <Input  class='Files' type=''>0020.rd-0</Input>
    <Model  class='Models'    type='Code'>MyMooseBasedApp</Model>
    <Sampler class='Samplers'  type='Grid'>Grid_Sampler</Sampler>
    <Output  class='Databases' type='HDF5'>Grid_out</Output>
    <Output  class='DataObjects' type='PointSet'>gridOutData</Output>
  </MultiRun>
  <MultiRun name='MC_Sampler' verbosity='debug' re-seeding='210491'>
    <Input  class='Files' type=''>mooseBasedApp.i</Input>
    <Input  class='Files' type=''>0020_mesh.cpr</Input>
    <Input  class='Files' type='' >0020.xdr.0000</Input>
    <Input  class='Files' type=''>0020.rd-0</Input>
    <Model  class='Models'    type='Code'>MyMooseBasedApp</Model>
    <Sampler class='Samplers'  type='MonteCarlo' >MC_Sampler</Sampler>
    <Output  class='Databases' type='HDF5' >MC_out</Output>
    <Output  class='DataObjects' type='PointSet'>MCOutData</Output>
  </MultiRun>
</Steps>
\end{lstlisting}
%%%%%%%%%%%%%%%%%%%%%%%%%%%%%%%%%%%%%%%%%%%%%%%%%%%%%%
\subsubsection{Databases}
As shown in the \xmlNode{Steps} block, the code is creating two database objects
called \texttt{Grid\_out} and \texttt{MC\_out}.
%
So the user needs to input the following:
\begin{lstlisting}[style=XML]
<Databases>
  <HDF5 name="Grid_out" readMode="overwrite"/>
  <HDF5 name="MC_out" readMode="overwrite"/>
</Databases>
\end{lstlisting}
As listed before, this will create two databases.
%
The files will have names corresponding to their \xmlAttr{name} appended with
the .h5 extension (i.e. \texttt{Grid\_out.h5} and \texttt{MC\_out.h5}).
%%%%%%%%%%%%%%%%%%%%%%%%%%%%%%%%%%%%%%%%%%%%%%%%%%%%%%
\subsubsection{DataObjects}
As shown in the \xmlNode{Steps} block, the code is creating two DataObjects of type PointSet
called \texttt{gridOutData} and \texttt{MCOutData}.
%
So the user needs to input the following:
\begin{lstlisting}[style=XML]
<DataObjects>
    <PointSet name='gridOutData'>
      <Input>
          Materials|heatStructure2|thermal_conductivity,
          Materials|heatStructure1|specific_heat,
          Materials|heatStructure2|thermal_conductivity
      </Input>
      <Output>aveTempLeft</Output>
    </PointSet>
    <PointSet name='MCOutData'>
      <Input>
          Materials|heatStructure2|thermal_conductivity,
          Materials|heatStructure1|specific_heat,
          Materials|heatStructure2|thermal_conductivity
      </Input>
      <Output>aveTempLeft</Output>
    </PointSet>
</DataObjects>
\end{lstlisting}
As listed before, this will create two DataObjects that can be used in sub-sequential post-processing.
%%%%%%%%%%%%%%%%%%%%%%%%%%%%%%%%%%%%%%%%%%%%%%%%%%%%%%
\subsubsection{OutStreams}
As fully explained in section~\ref{sec:outstream}, if the user want to print out or plot the content of a \textbf{DataObjects},
he needs to create an \textbf{OutStream} in the \xmlNode{OutStreams} XML block.
\\As it shown in the example below, for MooseBasedApp (and any other Code interface that might use the symbol $|$
for the Sampler's variable syntax), in the Plot \xmlNode{x} and \xmlNode{y} specification, the user needs to
utilize curly brackets.
\begin{lstlisting}[style=XML]
<OutStreams>
  <Print name='gridOutDataDumpCSV'>
    <type>csv</type>
    <source>gridOutData</source>
  </Print>
   <Plot verbosity='debug' name='test'   overwrite='False'>
    <plotSettings>
       <plot>
        <type>line</type>
        <x>MCOutData|Input|{Materials|heatStructure2|thermal_conductivity}</x>
        <y>MCOutData|Output|aveTempLeft</y>
        <kwargs><color>blue</color></kwargs>
      </plot>
    </plotSettings>
    <actions><how>screen,png</how></actions>
  </Plot>
</OutStreams>
\end{lstlisting}

%%%%%%%%%%%%%%%%%%%%%%%%%%%%%%%%%%
%%%%%% Moose VectorPostProcessor INTERFACE  %%%%%%
%%%%%%%%%%%%%%%%%%%%%%%%%%%%%%%%%%
\subsection{MooseVPP Interface}

The Moose Vector Post Processor is used mainly in the solid mechanics analysis.
This interface loads the values of the vector ouput processor to a \xmlNode{DataObjects} object.

To use this interface the [DomainIntegral] needs to be present in the MooseBasedApp's
 input file and the subnode \xmlNode{fileargs} should be defined in the subnode \xmlNode{Code} in
 the \xmlNode{Models} block of the RAVEN input file. The \xmlNode{fileargs} is required to have attributes with
the below specified values:

\begin{itemize}
    \item \xmlAttr{type}, \xmlDesc{string, required field}, must be "MooseVPP"
    \item \xmlAttr{arg}, \xmlDesc{string, required field}, the string value attached to the vector post processor action
         while creating the output files.
\end{itemize}

This interface is actually identical to the MooseBasedApp interface, however there is
 few constraints on defining the output values of the post processor.
The definition of these outputs in the \xmlNode{DataObjects} depends on the definition of
 the [DomainIntegral].

The location of the value outputted is defined as \textit{ID\#} and the value is as
\textit{value\#}. The ''\#'' defines
the number of the location. The example below contains 3 locations in the [DomainIntegral]
where the values are outputted.

%
Example:
\begin{lstlisting}[style=XML]
 ...
  <Models>
    <Code name="MOOSETestApp" subType="MooseBasedApp">
      <executable>%FRAMEWORK_DIR%/../../moose/
        modules/combined/modules-%METHOD%</executable>
      <fileargs type = "MooseVPP" arg = "_J_1_" />
      <alias variable = "poissonsRatio" >
        Materials|stiffStuff|poissons_ratio</alias>
      <alias variable = "youngModulus"  >
        Materials|stiffStuff|youngs_modulus</alias>
    </Code>
  </Models>
 ...
  <DataObjects>
    <PointSet name="collset">
      <Input>youngModulus,poissonsRatio</Input>
      <Output>ID1,ID2,ID3,value1,value2,value3</Output>
    </PointSet>
  </DataObjects>
 ...
\end{lstlisting}




%%%%%%%%%%%%%%%%%%%%%%%%%%%%%%%%%%%%%
%%%%%% OPENMODELICA INTERFACE  %%%%%%
%%%%%%%%%%%%%%%%%%%%%%%%%%%%%%%%%%%%%
\subsection{OpenModelica Interface}
OpenModelica (\url{http://www.openmodelica.org}) is an open souce implementation of the Modelica simulation language.  Modelica is "a non-proprietary,
object-oriented, equation based language to conveniently model complex physical systems containing, e.g., mechanical, electrical, electronic, hydraulic,
thermal, control, electric power or process-oriented subcomponents."\footnote{\url{http://www.modelica.org}}.  Modelica models are specified in text files
with a file extension of .mo.  A standard Modelica example called BouncingBall which simulates the trajectory of an object falling in one dimension from a
height is shown as an example:
\begin{lstlisting}
model BouncingBall
  parameter Real e=0.7 "coefficient of restitution";
  parameter Real g=9.81 "gravity acceleration";
  Real h(start=1) "height of ball";
  Real v "velocity of ball";
  Boolean flying(start=true) "true, if ball is flying";
  Boolean impact;
  Real v_new;
  Integer foo;

equation
  impact = h <= 0.0;
  foo = if impact then 1 else 2;
  der(v) = if flying then -g else 0;
  der(h) = v;

  when {h <= 0.0 and v <= 0.0,impact} then
    v_new = if edge(impact) then -e*pre(v) else 0;
    flying = v_new > 0;
    reinit(v, v_new);
  end when;

end BouncingBall;
\end{lstlisting}

\subsubsection{Files}
An OpenModelica installation specific to the operating system is used to create a stand-alone executable program that performs the model calculations.
A separate XML file containing model parameters and initial conditions is also generated as part of the build process.  The RAVEN OpenModelica interface
modifies input parameters by changing copies of this file.  Both the executable and XML parameter file names must be provided to RAVEN.  In the case of
the BouncingBall model previously mentioned on the Windows operating system, the \textless Files\textgreater  specification would look like:
\begin{lstlisting}[style=XML]
<Files>
  <Input name='BouncingBall_init.xml' type=''>BouncingBall_init.xml</Input>
  <Input name='BouncingBall.exe' type=''>BouncingBall.exe</Input>
</Files>
\end{lstlisting}
\subsubsection{Models}
OpenModelica models may provide simulation output in a number of formats.  The particular format used is specified during the model generation
process.  RAVEN works best with Comma-Separated Value (CSV) files, which is one of the possible output format options.  Models are generated
using the OpenModelica Shell (OMS) command-line interface, which is part of the OpenModelica installation.  To generate an executable that provides
CSV-formatted output, use OMSl commands as follows:
\lstset{
    frame=single,
    breaklines=true,
    postbreak=\raisebox{0ex}[0ex][0ex]{\ensuremath{\color{red}\hookrightarrow\space}}
}
 \begin{enumerate}
\item Change to the directory containing the .mo file to generate an executable for:
\begin{lstlisting}
>> cd("C:/MinGW/msys/1.0/home/bobk/projects/raven/framework/CodeInterfaces/OpenModelica")
"C:/MinGW/msys/1.0/home/bobk/projects/raven/framework/CodeInterfaces/OpenModelica"
\end{lstlisting}
\item Load the model file into memory:
\begin{lstlisting}
>> loadFile("BouncingBall.mo")
true
\end{lstlisting}
\item Create the model executable, specifying CSV output format:
\begin{lstlisting}
>> buildModel(BouncingBall, outputFormat="csv")
{"C:/MinGW/msys/1.0/home/bobk/projects/raven/framework/CodeInterfaces/OpenModelica/BouncingBall","BouncingBall_init.xml"}
Warning: The initial conditions are not fully specified. Use +d=initialization for more information.
\end{lstlisting}
At this point the model executable and XML initialization file should have been created in the same directory as the original model file.
\end{enumerate}
The model executable is specified to RAVEN using the \textless Models\textgreater  section of the input file as follows:
\begin{lstlisting}[style=XML]
<Simulation>
    ...
  <Models>
    <Code name="BouncingBall" subType = "OpenModelica">
      <executable>BouncingBall.exe</executable>
    </Code>
  </Models>
    ...
</Simulation>
\end{lstlisting}
\subsubsection{CSV Output}
The CSV files produced by OpenModelica model executables require adjustment before it may be read by RAVEN.
The first few lines of original CSV output from the
BouncingBall example is shown below:
\begin{lstlisting}
"time","h","v","der(h)","der(v)","v_new","foo","flying","impact",
0,1,0,0,-9.810000000000001,0,2,1,0,
  ...
\end{lstlisting}
RAVEN will not properly read this file as-generated for two reasons:
\begin{itemize}
  \item The variable names in the first line are each enclosed in double-quotes.
  \item Each line has a trailing comma.
\end{itemize}
 The OpenModelica inteface will automatically remove the double-quotes and trailing commas through its implementation of the
finalizeCodeOutput function.


%%%%%%%%%%%%%%%%%%%%%%%%%%%%%%%%%%%%%
%%%%%% DYMOLA INTERFACE  %%%%%%%%%%%%
%%%%%%%%%%%%%%%%%%%%%%%%%%%%%%%%%%%%%
\subsection{Dymola Interface}
Modelica is "a non-proprietary, object-oriented, equation-based language to conveniently model complex physical systems containing, e.g., mechanical, electrical, electronic, hydraulic,
thermal, control, electric power or process-oriented subcomponents."\footnote{\url{http://www.modelica.org}}.  Modelica models (with a file extension of .mo) are built, translated (compiled), and simulated in Dymola (http://www.modelon.com/p-
roducts/dymola/), which is a commercial modeling and simulation environment based on the Modelica modeling language.
A standard Modelica example called BouncingBall, which simulates the trajectory of an object falling in one dimension from a height, is shown as an example:
\begin{lstlisting}
model BouncingBall
  parameter Real e=0.7 "coefficient of restitution";
  parameter Real g=9.81 "gravity acceleration";
  parameter Real hstart = 10 "height of ball at time zero";
  parameter Real vstart = 0 "velocity of ball at time zero";
  Real h(start=hstart,fixed=true) "height of ball";
  Real v(start=vstart,fixed=true) "velocity of ball";
  Boolean flying(start=true) "true, if ball is flying";
  Boolean impact;
  Real v_new;
  Integer foo;

equation
  impact = h <= 0.0;
  foo = if impact then 1 else 2;
  der(v) = if flying then -g else 0;
  der(h) = v;

  when {h <= 0.0 and v <= 0.0,impact} then
    v_new = if edge(impact) then -e*pre(v) else 0;
    flying = v_new > 0;
    reinit(v, v_new);
  end when;

  annotation (uses(Modelica(version="3.2.1")),
    experiment(StopTime=10, Interval=0.1),
    __Dymola_experimentSetupOutput);

end BouncingBall;
\end{lstlisting}

\subsubsection{Files}
When a modelica model, e.g., BouncingBall model, is implemented in Dymola, the platform dependent C-code from a Modelica model and the corresponding executable code
(i.e., by default dymosim.exe on the Windows operating system) are generated for simulation.  After the executable is generated, it may be run multiple times (with Dymola license).
A separate TEXT file (by default dsin.txt) containing model parameters and initial conditions are also generated as part of the build process.  The RAVEN Dymola interface
modifies input parameters by changing copies of this file.  Both the executable and TEXT parameter file (or simulation initialization file) names must be provided to RAVEN. The TEXT parameter file must be of type 'DymolaInitialisation'.  In the case of
the BouncingBall model previously mentioned on the Windows operating system, the \textless Files\textgreater  specification would look like:
\begin{lstlisting}[style=XML]
<Files>
  <Input name='dsin.txt' type='DymolaInitialisation'>dsin.txt</Input>
</Files>
\end{lstlisting}

The Dymola interface can only pass scalar values into the TEXT parameter file. If the user wants to pass vector information to Dymola, he can do so by providing an optional TEXT vector file to Dymola. This file must have the type 'DymolaVectors'. This additional file can then be read by the Dymola model. If vecor data is passed from RAVEN to the Dymola interface and the TEXT vector file is not specified, the interface will display an error and stop the Dymola execution. If the TEXT vector file is specified (and vector data is passed to the interface), the interface will write the datd into the specified file, but also display a warning, saying that the Dymola interface found vector data to be passed and if this data is supposed to go into the simulation initialisation file of type 'DymolaInitialisation' the array must be split into scalars. The \textless Files\textgreater specification for the vector data look as follows:
\begin{lstlisting}[style=XML]
<Files>
  <Input name='timeSeriesData.txt' type='DymolaVectors'>timeSeriesData.txt</Input>
</Files>
\end{lstlisting}

\subsubsection{Models}
An executable (dymosim.exe) and a simulation initialization file (dsin.txt) can be generated after either translating or simulating the
Modelica model (BouncingBall.mo) using the Dymola Graphical User Interface (GUI) or Dymola Application Programming Interface (API)-routines.
To generate an executable and a simulation initialization file, use the Dymola API-routines (or Dymola GUI) to translate the model as follows:
\lstset{
    frame=single,
    breaklines=true,
    postbreak=\raisebox{0ex}[0ex][0ex]{\ensuremath{\color{red}\hookrightarrow\space}}
}
\begin{enumerate}
\item Change to the directory containing the .mo file to generate an executable.  In Dymola GUI, this corresponds to File/Change Directory in menus:
\begin{lstlisting}
>> cd("C:/msys64/home/KIMJ/projects/raven/framework/CodeInterfaces/Dymola");
C:/msys64/home/KIMJ/projects/raven/framework/CodeInterfaces/Dymola
 = true
\end{lstlisting}
\item Reads the specified file and displays its window.  In Dymola GUI, this corresponds to File/Open in the menus:
\begin{lstlisting}
>> openModel("BouncingBall.mo")
 = true
\end{lstlisting}
\item Compile the model (with current settings), and create the model executable and the corresponding simulation initialization file.  In Dymola GUI, this corresponds to Translate Model in the menus:
\begin{lstlisting}
>> translateModel("BouncingBall");
 = true
\end{lstlisting}
At this point the model executable and the simulation initialization file should have been created in the same directory as the original model file.
Additionally, they could be created by simulating the model.  The following command corresponds to Simulate in the menus in Dymola GUI:
\begin{lstlisting}
>> simulateModel("BouncingBall", stopTime=10, numberOfIntervals=0, outputInterval=0.1, method="dassl", resultFile="BouncingBall");
 = true
\end{lstlisting}
The file extension (.mat) is automatically added to a output file (resultFile), e.g., BouncingBall.mat.  If the generated executable code is triggered directly from a
command prompt, the output file is always named as "dsres.mat".
\end{enumerate}
The model executable is specified to RAVEN using the \textless Models\textgreater  section of the input file as follows:
\begin{lstlisting}[style=XML]
<Simulation>
    ...
  <Models>
    <Code name="BouncingBall" subType = "Dymola">
      <executable>dymosim.exe</executable>
    </Code>
  </Models>
    ...
</Simulation>
\end{lstlisting}
RAVEN works best with Comma-Separated Value (CSV) files.  Therefore, the default
.mat output type needs to be converted to .csv output.
The Dymola interface will automatically convert the .mat output to human-readable
forms, i.e., .csv output, through its implementation of the finalizeCodeOutput function.
\\In order to speed up the reading and conversion of the .mat file, the user can specify
the list of variables (in addition to the Time variable) that need to be imported and
converted into a csv file minimizing
the IO memory usage as much as possible. Within the \xmlNode{Code} the following
XML
node (in addition ot the \xmlNode{executable} one) can be inputted:

\begin{itemize}
   \item \xmlNode{outputVariablesToLoad}, \xmlDesc{space separated list, optional
   parameter}, a space separated list of variables that need be exported from the .mat
   file (in addition to the Time variable). \default{all the variables in the .mat file}.
\end{itemize}
For example:
\begin{lstlisting}[style=XML]
<Simulation>
    ...
  <Models>
    <Code name="BouncingBall" subType = "Dymola">
      <executable>dymosim.exe</executable>
      <outputVariablesToLoad>var1 var2 var3</outputVariablesToLoad>
    </Code>
  </Models>
    ...
</Simulation>
\end{lstlisting}


%%%%%%%%%%%%%%%%%%%%%%%%%%%%%%%%%%%%%%%%%%%%%%%%%
%%%%%% MESH GENERATION COUPLED INTERFACES %%%%%%%
%%%%%%%%%%%%%%%%%%%%%%%%%%%%%%%%%%%%%%%%%%%%%%%%%
\subsection{Mesh Generation Coupled Interfaces}
Some software requires a provided mesh that requires a separate code run to generate.
In these cases, we use sampled geometric
variables to generate a new mesh for each perturbation of the original problem, then run the input with
the remainder of the perturbed parameters and the perturbed mesh.  RAVEN currently provides two interfaces for
this type of calculation, listed below.

%%%%%%%%%% CUBIT MOOSE INTERFACE %%%%%%%%%%
\subsubsection{MooseBasedApp and Cubit Interface}
Many MOOSE-based applications use Cubit (\url{https://cubit.sandia.gov}) to generate Exodus II files as
geometry and meshing for calculations.  To use the developed interface, Cubit's
bin directory must be added to the user's PYTHONPATH.  Input parameters for Cubit can be listed in a journal
(\texttt{.jou}) file.  Parameter values
are typically hardcoded into the Cubit command syntax, but variables may be
predefined in a journal file through Aprepro syntax.  This is an example of a journal
file that generates a rectangle of given height and width, meshes it, defines its
volume and sidesets, lists its element type, and writes it as an Exodus file:

%Cubit (\url{https://cubit.sandia.gov}) is a toolkit developed at Sandia National
%Laboratory used to create two- and three-dimensional finite element meshes with
%various options for defining geometric properties as a part of the grid. It is
%capable of reading and writing a variety of standard mesh file types, including
%Genesis or Exodus II (*.e) files.  As MOOSE applications use Exodus II files for
%meshes and results, Cubit is commonly used to generate meshes for problems of
%interest.  Cubit commands are used to create the geometry, mesh the object, and
%identify volumes, sidesets, and nodesets for a mesh.  These commands may be
%placed in journal files (*.jou) to be used as input to Cubit.  Parameter values
%are typically hardcoded into the Cubit command syntax, but variables may be
%predefined in a journal file through Aprepro syntax.  An example of a journal
%file that generates a rectangle of given height and width, meshes it, defines its
%volume and sidesets, lists its element type, and writes it as an Exodus file is given:

\begin{lstlisting}
#{x = 3}
#{y = 3}
#{out_name = "'out_mesh.e'"}
create surface rectangle width {x} height {y} zplane
mesh surface 1
set duplicate block elements off
block 1 surface 1
Sideset 1 curve 3
Sideset 2 curve 4
Sideset 3 curve 1
Sideset 4 curve 2
Block all element type QUAD4
export genesis {out_name} overwrite
\end{lstlisting}

The first three lines are the Aprepro variable definitions that RAVEN requires to
insert sampled variables.  All variables that RAVEN samples
need to be defined as Aprepro variables in the journal
file.
%These are typically geometric parameters, though almost anything that Cubit
%quantifies in a command may be defined as a variable and sampled through RAVEN
%such as internal mesh refinement values.
One essential caveat to running
this interface is that an Aprepro variable MUST be defined with the name "out\_name".
In order to run this script without RAVEN inserting the correct syntax for the
output file name and properly generate the Exodus file for a mesh, the output file
name is REQUIRED to be in both single and double quotation marks with the file
extension appended to the end of the file base name (e.g. '"output\_file.e"').

%%%%%%%%%%%%%%%%%%%%%%%%%%%%%%%%%%%%%%%%%%%%%%%%%%
\paragraph{Files}
\xmlNode{Files} works the same as in other interfaces with name and type
attributes for each node entry.  The \xmlAttr{name} attribute is a user-chosen internal
name for the file contained in the node, and \xmlAttr{type} identifies which base-level
interface the file is used within.  \xmlNode{type} should only be specified for inputs
that RAVEN will perturb.  For Moose input files, \xmlNode{type} should be \xmlString{MooseInput} and for
Cubit journal files, the \xmlNode{type} should be \xmlString{CubitInput}.  The node should contain the
path to the file from the working directory.  The following is an example
of a typical \xmlNode{Files} block.

\begin{lstlisting}[style=XML]
<Files>
  <Input name='moose_test' type='MooseInput'>simple_diffusion.i</Input>
  <Input name='mesh_in'    type='CubitInput'>rectangle.jou</Input>
  <Input name='other_file' type=''          >some_file_moose_input_needs.ext</Input>
</Files>
\end{lstlisting}

%%%%%%%%%%%%%%%%%%%%%%%%%%%%%%%%%%%%%%%%%%%%%%%%%%
\paragraph{Models}
A user provides paths to executables and aliases for sampled variables within the
\xmlNode{Models} block.  The \xmlNode{Code} block will contain attributes name and
subType.  Name identifies that particular \xmlNode{Code} model within RAVEN, and
subType specifies which code interface the model will use. The \xmlNode{executable}
block should contain the absolute or relative (with respect to the current working
directory) path to the MooseBasedApp that RAVEN will use to run generated input
files.  The absolute or relative path to the Cubit executable is specified within
\xmlNode{preexec}.  If the \xmlNode{preexec} block is not needed, the
MooseBasedApp interface is probably preferable to the Cubit-Moose interface.

Aliases are defined by specifying the variable attribute in an \xmlNode{alias} node with
the internal RAVEN variable name chosen with the node containing the model
variable name.  The Cubit-Moose interface uses the same syntax as the
MooseBasedApp to refer to model variables, with pipes separating terms starting
with the highest YAML block going down to the individual parameter that RAVEN
will change.  To specify variables that are going to be used in the Cubit
journal file, the syntax is "Cubit|aprepro\_var".  The Cubit-Moose interface
will look for the Cubit tag in all variables passed to it and upon finding it,
send it to the Cubit interface.  If the model variable does not begin with \xmlString{Cubit},
the variable MUST be specified in the MooseBasedApp input file.  While the model
variable names are not required to have aliases defined (the \xmlNode{alias}
blocks are optional), it is highly suggested to do so not only to ensure brevity
throughout the RAVEN input, but to easily identify where variables are being sent
in the interface.

An example \xmlNode{Models} block follows.

\begin{lstlisting}[style=XML]
<Models>
  <Code name="moose-modules" subType="CubitMoose">
    <executable>%FRAMEWORK_DIR%/../../moose/modules/combined/...
      modules-%METHOD%</executable>
    <preexec>/hpc-common/apps/local/cubit/13.2/bin/cubit</preexec>
    <alias variable="length">Cubit@y</alias>
    <alias variable="bot_BC">BCs|bottom|value</alias>
  </Code>
</Models>
\end{lstlisting}

%%%%%%%%%%%%%%%%%%%%%%%%%%%%%%%%%%%%%%%%%%%%%%%%%%
\paragraph{Distributions}
The \xmlNode{Distributions} block defines all distributions used to
sample variables in the current RAVEN run.

For all the possible distributions and their possible inputs please
refer to the Distributions chapter (see~\ref{sec:distributions}).
%
It is good practice to name the distribution something similar to what kind of
variable is going to be sampled, since there might be many variables with the
same kind of distributions but different input parameters.

%%%%%%%%%%%%%%%%%%%%%%%%%%%%%%%%%%%%%%%%%%%%%%%%%%
\paragraph{Samplers}
The \xmlNode{Samplers} block defines the variables to be sampled.

After defining a sampling scheme, the variables to be sampled and
their distributions are identified in the \xmlNode{variable} blocks.
The name attribute in the \xmlNode{variable} block must either be the
full MooseBasedApp model variable name or the alias name specifed in
\xmlNode{Models}.  If the sampled variable is a geometric property
that will be used to generate a mesh with Cubit, remember the syntax for
variables being passed to journal files (Cubit|aprepro\_var).

For listings of available samplers
refer to the Samplers chapter (see~\ref{sec:Samplers}).

See the following for an example of a grid based sampler for
length and the bottom boundary condition (both of which have aliases
defined in \xmlNode{Models}).

\begin{lstlisting}[style=XML]
<Samplers>
  <Grid name="Grid_sampling">
    <variable name="length" >
      <distribution>length_dist</distribution>
      <grid type="value" construction="custom">1.0 2.0</grid>
    </variable>
    <variable name="bot_BC">
      <distribution>bot_BC_dist</distribution>
      <grid type="value" construction="custom">3.0 6.0</grid>
    </variable>
  </Grid>
</Samplers>
\end{lstlisting}

%%%%%%%%%%%%%%%%%%%%%%%%%%%%%%%%%%%%%%%%%%%%%%%%%%
\paragraph{Steps,OutStreams,DataObjects}
This interface's \xmlNode{Steps}, \xmlNode{OutStreams}, and
\xmlNode{DataObjects} blocks do not deviate significantly from
other interfaces' respective nodes.  Please refer to previous
entries for these blocks if needed.

%%%%%%%%%%%%%%%%%%%%%%%%%%%%%%%%%%%%%%%%%%%%%%%%%%
\paragraph{File Cleanup}
The Cubit-Moose interface automatically removes files that are commonly
unwanted after the RAVEN run reaches completion. Cubit has been described as
"talkative" due to additional journal files with execution information
being generated by the program after every completed journal file run.
The quantity of these files can quickly become unwieldly if the working
directory is not kept clean; thus these files are removed.  In addition, some users
may wish to remove Exodus files after the RAVEN run is complete as
the typical size of each file is quite large and it is assumed that any
output quantities of interest will be collected by appropriate postprocessors
and the OutStreams.  Exodus files are not automatically removed,
but by using the \xmlNode{deleteOutExtension} node in \xmlNode{RunInfo}, one
may specify the Exodus extension to save a fair amount of storage space
after RAVEN completes a sequence. For example:

\begin{lstlisting}[style=XML]
<RunInfo>
  ...
  <deleteOutExtension>e</deleteOutExtension>
  ...
</RunInfo>
\end{lstlisting}

%%%%%%%%%% BISON MESH SCRIPT MOOSE INTERFACE %%%%%%%%%%
\subsubsection{MooseBasedApp and Bison Mesh Script Interface}
For BISON users, a Python mesh generation script is included in
the \%BISON\_DIR\%/tools/UO2/ directory.  This script generates
3D or 2D (RZ) meshes for nuclear fuel rods using Cubit with
templated commands.  The BISON Mesh Script (BMS) is capable of
generating rods with discrete fuel pellets of various size in
assorted configurations.  To use this interface, Cubit's bin
directory must be added to the user's PYTHONPATH.

%%%%%%%%%%%%%%%%%%%%%%%%%%%%%%%%%%%%%%%%%%%%%%%%%%
\paragraph{Files}
Similar to the Cubit-Moose interface, the BisonAndMesh interface
requires users to specify all files required to run their input
so that these file may be copied into the respective sequence's
working directory.  The user will give each file an internal
RAVEN designation with the name attribute, and the MooseBasedApp
and BISON Mesh Script inputs must be assigned their respective types
in another attribute of the \xmlNode{Input} node.  An example follows.

\begin{lstlisting}[style=XML]
<Files>
  <Input name='bison_test' type='MooseInput'>simple_bison_test.i</Input>
  <Input name='mesh_in'    type='BisonMeshInput'>coarse_input.py</Input>
  <Input name='other_file' type=''>some_file_moose_input_needs.ext</Input>
</Files>
\end{lstlisting}

%%%%%%%%%%%%%%%%%%%%%%%%%%%%%%%%%%%%%%%%%%%%%%%%%%
\paragraph{Models}
A user provides paths to executables and aliases for sampled variables within the
\xmlNode{Models} block.  The \xmlNode{Code} block will contain attributes \xmlAttr{name} and
\xmlAttr{subType}.  \xmlAttr{name} identifies that particular \xmlNode{Code} model within RAVEN, and
\xmlAttr{subType} specifies which code interface the model will use. The \xmlNode{executable}
block should contain the absolute or relative (with respect to the current working
directory) path to the MooseBasedApp that RAVEN will use to run generated input
files.  The absolute or relative path to the mesh script python file is specified within
\xmlNode{preexec}.  If the \xmlNode{preexec} block is not needed, use the
MooseBasedApp interface.

Aliases are defined by specifying the variable attribute in an \xmlNode{alias} node with
the internal RAVEN variable name chosen with the node containing the model
variable name.  The BisonAndMesh interface uses the same syntax as the
MooseBasedApp to refer to model variables, with pipes separating terms starting
with the highest YAML block going down to the individual parameter that RAVEN
will change.  To specify variables that are going to be used in the BISON Mesh Script
python input, the syntax is "Cubit|dict\_name|var\_name".  The interface
will look for the Cubit tag in all variables passed to it and upon finding the tag,
send it to the BISON Mesh Script interface.  If the model variable does not begin with Cubit,
the variable MUST be specified in the MooseBasedApp input file.  While the model
variable names are not required to have aliases defined (the \xmlNode{alias}
blocks are optional), it is highly suggested to do so not only to ensure brevity
throughout the RAVEN input, but to easily identify where variables are being sent
in the interface.

An example \xmlNode{Models} block follows.

\begin{lstlisting}[style=XML]
<Models>
  <Code name="Bison-opt" subType="BisonAndMesh">
    <executable>%FRAMEWORK_DIR%/../../bison/bison-%METHOD%</executable>
    <preexec>%FRAMEWORK_DIR%/../../bison/tools/UO2/mesh_script.py</preexec>
    <alias variable="pellet_radius" >Cubit@Pellet1|outer_radius</alias>
    <alias variable="clad_thickness">Cubit@clad|clad_thickness</alias>
    <alias variable="fuel_k"        >Materials|fuel_thermal|thermal_conductivity</alias>
    <alias variable="clad_k"        >Materials|clad_thermal|thermal_conductivity</alias>
  </Code>
</Models>
\end{lstlisting}

%%%%%%%%%%%%%%%%%%%%%%%%%%%%%%%%%%%%%%%%%%%%%%%%%%
\paragraph{Distributions}
The \xmlNode{Distributions} block defines all distributions used to
sample variables in the current RAVEN run.

For all the possible distributions and their possible inputs please
refer to the Distributions chapter (see~\ref{sec:distributions}).
%
It is good practice to name the distribution something similar to what kind of
variable is going to be sampled, since there might be many variables with the
same kind of distributions but different input parameters.

%%%%%%%%%%%%%%%%%%%%%%%%%%%%%%%%%%%%%%%%%%%%%%%%%%
\paragraph{Samplers}
The \xmlNode{Samplers} block defines the variables to be sampled.

After defining a sampling scheme, the variables to be sampled and
their distributions are identified in the \xmlNode{variable} blocks.
The name attribute in the \xmlNode{variable} block must either be the
full MooseBasedApp model variable name or the alias name specified in
\xmlNode{Models}.  If the sampled variable is a geometric property
that will be used to generate a mesh with Cubit, remember the syntax for
variables being passed to journal files (Cubit|aprepro\_var).

For listings of available samplers
refer to the Samplers chapter (see~\ref{sec:Samplers}).

See the following for an example of a grid based sampler for
length and the bottom boundary condition (both of which have aliases
defined in \xmlNode{Models}).

\begin{lstlisting}[style=XML]
<Samplers>
  <Grid name="Grid_sampling">
    <variable name="length" >
      <distribution>length_dist</distribution>
      <grid type="value" construction="custom">1.0 2.0</grid>
    </variable>
    <variable name="bot_BC">
      <distribution>bot_BC_dist</distribution>
      <grid type="value" construction="custom">3.0 6.0</grid>
    </variable>
  </Grid>
</Samplers>
\end{lstlisting}

%%%%%%%%%%%%%%%%%%%%%%%%%%%%%%%%%%%%%%%%%%%%%%%%%%
\paragraph{Steps,OutStreams,DataObjects}
This interface's \xmlNode{Steps}, \xmlNode{OutStreams}, and
\xmlNode{DataObjects} blocks do not deviate significantly from
other interfaces' respective nodes.  Please refer to previous
entries for these blocks if needed.

%%%%%%%%%%%%%%%%%%%%%%%%%%%%%%%%%%%%%%%%%%%%%%%%%%
\paragraph{File Cleanup}
The BisonAndMesh interface automatically removes files that are commonly
unwanted after the RAVEN run reaches completion. Cubit has been described as
"talkative" due to additional journal files with execution information
being generated by the program after every completed journal file run.
The BISON Mesh Script creates a journal file to run with cubit after reading input parameters;
so Cubit will generate its "redundant" journal files, and .pyc files will
litter the working directory as artifacts of the python mesh script
reading from the .py input files.  The quantity of these files can quickly
become unwieldly if the working directory is not kept clean, thus these
files are removed.  Some users
may wish to remove Exodus files after the RAVEN run is complete as
the typical size of each file is quite large and it is assumed that any
output quantities of interest will be collected by appropriate postprocessors
and the OutStreams.  Exodus files are not automatically removed,
but by using the \xmlNode{deleteOutExtension} node in \xmlNode{RunInfo}, one
may specify the Exodus extension (*.e) to save a fair amount of storage space
after RAVEN completes a sequence. For example:

\begin{lstlisting}[style=XML]
<RunInfo>
  ...
  <deleteOutExtension>e</deleteOutExtension>
  ...
</RunInfo>
\end{lstlisting}


%%%%%%%%%%%%%%%%%%%%%%%%%%%%%%%%%%%%%%%%%%%%%%%%%

%%%%%%%%%%%%%%%%%%%%%%%%%%%%%%%%%%%%%%%%%%%%%%%%%
%%%%%%%%%%%%% RATTLESNAKE INTERFACE %%%%%%%%%%%%%
%%%%%%%%%%%%%%%%%%%%%%%%%%%%%%%%%%%%%%%%%%%%%%%%%
\subsection{Rattlesnake Interfaces} \label{RattlesnakeInterfaces}
%
This section covers the input specification for running Rattlesnake through RAVEN. It is important
to notice that this short explanation assumes that the reader already knows how to use Rattlesnake.
The interface can be used to perturb the Rattlesnake MOOSE-based input file as well as the Yak
cross section libraries XML input files (e.g. multigroup cross section libraries) and Instant format
cross section libraries.
%
%%%%%%%%%%%%%%%%%%%%%%%%%%%%%%%%%%%%%%%%%%%%%%%%%%
\subsubsection{Files}
\xmlNode{Files} works the same as in other interfaces with name and type
attributes for each node entry.  The \xmlAttr{name} attribute is a user-chosen internal
name for the file contained in the node, and \xmlAttr{type} identifies which base-level
interface the file is used within.  \xmlAttr{type} should only be specified for inputs
that RAVEN will perturb. Take Rattlesnake input files for example, \xmlAttr{type} should
be \xmlString{RattlesnakeInput}.

\paragraph{Perturb Yak Multigroup Cross Section Libraries}
If the user would like to perturb the Yak multigroup cross section libraries, the user need to use the
\xmlString{YakXSInput} for the \xmlAttr{type} of the libaries. In addition, the \xmlAttr{type} of the
alias files that are used to perturb the Yak multigroup cross section libraries should be
\xmlString{YakXSAliasInput}. The following is an example of a typical \xmlNode{Files} block.
%
\begin{lstlisting}[style=XML]
<Files>
  <Input name='rattlesnakeInput' type='RattlesnakeInput'>simple_diffusion.i</Input>
  <Input name='crossSection'    type='YakXSInput'>xs.xml</Input>
  <Input name='alias' type='YakXSAliasInput'>alias.xml</Input>
</Files>
\end{lstlisting}
%
The alias files are employed to define the variables that will be used to perturb Yak multigroup cross section
libraries. The following is an example of a typical alias file:
%
\begin{lstlisting}[style=XML]
<Multigroup_Cross_Section_Libraries Name="twigl" NGroup="2" Type="rel">
    <Multigroup_Cross_Section_Library ID="1">
        <Fission gridIndex="1" mat="pseudo-seed1" gIndex="1">f11</Fission>
        <Capture gridIndex="1" mat="pseudo-seed1" gIndex="1">c11</Capture>
        <TotalScattering gridIndex="1" mat="pseudo-seed1" gIndex="1">t11</TotalScattering>
        <Nu gridIndex="1" mat="pseudo-seed1" gIndex="1">n11</Nu>
        <Fission gridIndex="1" mat="pseudo-seed2" gIndex="2">f22</Fission>
        <Capture gridIndex="1" mat="pseudo-seed2" gIndex="2">c22</Capture>
        <TotalScattering gridIndex="1" mat="pseudo-seed2" gIndex="2">t22</TotalScattering>
        <Nu gridIndex="1" mat="pseudo-seed2" gIndex="2">n22</Nu>
        <Fission gridIndex="1" mat="pseudo-seed1-dup" gIndex="1">f11</Fission>
        <Capture gridIndex="1" mat="pseudo-seed1-dup" gIndex="1">c11</Capture>
        <TotalScattering gridIndex="1" mat="pseudo-seed1-dup" gIndex="1">t11</TotalScattering>
        <Nu gridIndex="1" mat="pseudo-seed1-dup" gIndex="1">n11</Nu>
        <Transport gridIndex="1" mat="pseudo-seed1-dup" gIndex="1">d11</Transport>
    </Multigroup_Cross_Section_Library>
</Multigroup_Cross_Section_Libraries>
\end{lstlisting}
%
In the above alias file, the \xmlAttr{Name} of \xmlNode{Multigroup\_Cross\_Section\_Libraries} are used to indicate
which Yak multigroup cross section library input file will be perturbed.
The \xmlAttr{NGroup}, \xmlAttr{ID}, and \xmlNode{Multigroup\_Cross\_Section\_Library}
should be consistent with the Yak multigroup cross section library input files.
The \xmlNode{Fission}, \xmlNode{Capture}, \xmlNode{TotalScattering}, \xmlNode{Nu}, \xmlAttr{gridIndex},
\xmlAttr{mat}, and \xmlAttr{gIndex} are used to find the corresponding cross sections in the Yak multigroup cross
section library input files. For example:
%
\begin{lstlisting}[style=XML]
<Fission gridIndex="1" mat="pseudo-seed1" gIndex="1">f11</Fission>
\end{lstlisting}
%
This node defines an alias with name \xmlString{f11} used to represent the fission cross section at energy group \xmlString{1}
for material with name 'pseudo-seed1' at grid index \xmlString{1} in the Yak multigroup cross section library input files.

\nb The attribute \xmlAttr{Type="rel"} indicates that the cross sections will be perturbed relatively (i.e. perturbed by
percents). In this case, the user also needs to specify a relative covariance matrix for \xmlNode{covaraince \xmlAttr{type="rel"}} in
\xmlNode{MultivariateNormal} distribution, and the values for \xmlNode{mu} should be `ones'. In the other case, if
the user choose \xmlAttr{Type="abs"}, the cross sections will be perturbed absolutely (i.e. perturbed by values), and
the user needs to provide an absolute covariance matrix and specify `zeros' for \xmlNode{mu} in \xmlNode{MultivariateNormal}
distribution.

\nb Currently, only the following cross sections can be perturbed by the user: Fission, Capture, Nu, TotalScattering,
and Transport.

\paragraph{Perturb Instant format Cross Section Libraries}
If the user would like to perturb the Instant cross section libraries, the user need to use the
\xmlString{InstantXSInput} for the \xmlAttr{type} of the libaries. In addition, the \xmlAttr{type} of the
alias files that are used to perturb the Instant format cross section libraries should be
\xmlString{InstantXSAliasInput}. The following is an example of a typical \xmlNode{Files} block.
%
\begin{lstlisting}[style=XML]
<Files>
  <Input name='rattlesnakeInput' type='RattlesnakeInput'>iaea2d_ls_sn.i</Input>
  <Input name='crossSection'    type='InstantXSInput'>iaea2d_materials.xml</Input>
  <Input name='alias' type='InstantXSAliasInput'>alias.xml</Input>
</Files>
\end{lstlisting}
%
The alias files are employed to define the variables that will be used to perturb Instant format cross section
libraries. The following is an example of a typical alias file:
%
\begin{lstlisting}[style=XML]
<Materials>
  <Macros NG="2" Type="rel">
    <material ID="1">
      <FissionXS gIndex="1">f11</FissionXS>
      <CaptureXS gIndex="1">c11</CaptureXS>
      <TotalScatteringXS gIndex="1">t11<TotalScatteringXS>
      <Nu gIndex="1">n11</Nu>
      <DiffusionCoefficient gIndex="1">d11</DiffusionCoefficient>
    </material>
  </Macros>
</Materials>
\end{lstlisting}

%
In the above alias file, the \xmlAttr{NG} and \xmlAttr{ID} should be consistent with the Instant format cross
section library input files. The \xmlNode{FissionXS}, \xmlNode{CaptureXS}, \xmlNode{TotalScatteringXS}, \xmlNode{Nu}, \xmlAttr{gIndex},
are used to find the corresponding cross sections in the Instant format cross
section library input files. For example, the variable \xmlString{f11} used to represent the fission cross section at energy group \xmlString{1}
for material with \xmlString{ID} equal \xmlString{1} in the given cross section library.

\nb The attribute \xmlAttr{Type="rel"} indicates that the cross sections will be perturbed relatively (i.e. perturbed by
percents). In this case, the user also needs to specify a relative covariance matrix for \xmlNode{covaraince \xmlAttr{type="rel"}} in
\xmlNode{MultivariateNormal} distribution, and the values for \xmlNode{mu} should be `ones'. In the other case, if
the user choose \xmlAttr{Type="abs"}, the cross sections will be perturbed absolutely (i.e. perturbed by values), and
the user needs to provide an absolute covariance matrix and specify `zeros' for \xmlNode{mu} in \xmlNode{MultivariateNormal}
distribution.

\nb Currently, only the following cross sections can be perturbed by the user: FissionXS, CaptureXS, Nu, TotalScatteringXS,
and DiffusionCoefficient.

%%%%%%%%%%%%%%%%%%%%%%%%%%%%%%%%%%%%%%%%%%%%%%%%%%
\subsubsection{Models}
A user provides paths to executables and aliases for sampled variables within the
\xmlNode{Models} block.  The \xmlNode{Code} block will contain attributes \xmlNode{name} and
\xmlNode{subType}. The \xmlNode{name} identifies that particular \xmlNode{Code} model within RAVEN, and
\xmlNode{subType} specifies which code interface the model will use. The \xmlNode{executable}
block should contain the absolute or relative (with respect to the current working
directory) path to Rattlesnake that RAVEN will use to run generated input
files.

An example \xmlNode{Models} block follows.

\begin{lstlisting}[style=XML]
<Models>
  <Code name="Rattlesnake" subType="Rattlesnake">
    <executable>%FRAMEWORK_DIR%/../../rattlesnake/
     rattlesnake-%METHOD%</executable>
  </Code>
</Models>
\end{lstlisting}

%%%%%%%%%%%%%%%%%%%%%%%%%%%%%%%%%%%%%%%%%%%%%%%%%%
\subsubsection{Distributions}
The \xmlNode{Distributions} block defines all distributions used to
sample variables in the current RAVEN run.

For all the possible distributions and their possible inputs please
refer to the Distributions chapter (see~\ref{sec:distributions}).
%
It is good practice to name the distribution something similar to what kind of
variable is going to be sampled, since there might be many variables with the
same kind of distributions but different input parameters.

%%%%%%%%%%%%%%%%%%%%%%%%%%%%%%%%%%%%%%%%%%%%%%%%%%
\paragraph{Samplers}
The \xmlNode{Samplers} block defines the variables to be sampled.
After defining a sampling scheme, the variables to be sampled and
their distributions are identified in the \xmlNode{variable} blocks.
The name attribute in the \xmlNode{variable} block must either be the
full MooseBasedApp (Rattlesnake) model variable name, the alias name specifed in
\xmlNode{Models}, or the variable name specified in the provided alias files.

For listings of available samplers, please refer to the Samplers chapter (see~\ref{sec:Samplers}).
See the following for an example of a grid based sampler for
the first energy group fission and capture cross sections  (both of which have
defined in alias files provided in \xmlNode{Files}).

\begin{lstlisting}[style=XML]
<Samplers>
  <Grid name="Grid_sampling">
    <variable name="fission_group_1" >
      <distribution>fission_dist</distribution>
      <grid type="value" construction="custom">1.0 2.0</grid>
    </variable>
    <variable name="capture_group_1">
      <distribution>capture_dist</distribution>
      <grid type="value" construction="custom">3.0 6.0</grid>
    </variable>
  </Grid>
</Samplers>
\end{lstlisting}

%%%%%%%%%%%%%%%%%%%%%%%%%%%%%%%%%%%%%%%%%%%%%%%%%%
\subsubsection{Steps}
For a Rattlesnake interface, the \xmlNode{MultiRun} step type will most likely be used. First, the step needs
to be named: this name will be one of the names used in the \xmlNode{Sequence} block. In our example, \xmlString{Grid\_Rattlesnake}.
%
\begin{lstlisting}[style=XML]
<MultiRun name='Grid_Rattlesnake' verbosity='debug'>
    <Input   class='Files' type=''>RattlesnakeInput.i</Input>
    <Input   class='Files' type=''>xs.xml</Input>
    <Input   class='Files' type=''>alias.xml</Input>
    <Model   class='Models' type='Code'>Rattlesnake</Model>
    <Sampler class='Samplers' type='Grid'>Grid_Samplering</Sampler>
    <Output  class='DataObjects' type='PointSet'>solns</Ouput>
\end{lstlisting}
%
With this step, we need to import all the files needed for the simulation:
%
\begin{itemize}
  \item Rattlesnake MOOSE-based input file;
  \item Yak multigroup cross section libraries input files (XML);
  \item Yak alias files used to define the perturbed variables (XML).
\end{itemize}
We then need to define \xmlNode{Model}, \xmlNode{Sampler} and \xmlNode{Output}. The \xmlNode{Output} can be
\xmlNode{DataObjects} or \xmlNode{OutStreams}.

%%%%%%%%%%%%%%%%%%%%%%%%%%%%%%%%%%%%%%%%%%%%%%%%%%%%%%%%%%%%%%%%%%%%%%%%%%%%%%%%%%
%%%%%%%%%%%%%%%%%%%%%%%%%%%%%%%% MAAP5  INTERFACE %%%%%%%%%%%%%%%%%%%%%%%%%%%%%%%%
%%%%%%%%%%%%%%%%%%%%%%%%%%%%%%%%%%%%%%%%%%%%%%%%%%%%%%%%%%%%%%%%%%%%%%%%%%%%%%%%%%
\subsection{MAAP5 Interface}
This section presents the main aspects of the interface coupling RAVEN with MAAP5,
the consequent RAVEN input adjustments and the modifications of the MAAP5
files required to run the two coupled codes.
The interface works both for forward sampling and the DET,
however there are some differences depending on the selected sampling strategy.
%%%%%%%%%%%%%%%%%%%%%%%%%%%%%%%%%%%%%%%%%%%%%%%%%%%%%%%%%%%%%%%%%%%%%%%%%%%%%%%%%%
\subsubsection{RAVEN Input file}
%%%%%%%%%%%%%%%%%%%%%%%%%%%%%%%%%%%%%%%%%%%%%%%%%%%%%%%%%%%%%%%%%%%%%%%%%%%%%%%%%%
\paragraph{Files}
MAAP5 requires more than one file to run a simulation.
This means that, since the \xmlNode{Files} section has to contain all the files required by
the external model (MAAP5) to be run, all these files need to be included within this node.
This involves not only the input file (.inp) but also the include file, the parameter file, all the
files defining the different ``PLOTFILs'', if any, and the other files which could
result useful for the MAAP5 simulation run.

Example:
\begin{lstlisting}[style=XML]
<Files>
  <Input name="test.inp" type="">test.inp</Input>
  <Input name="include" type="">include</Input>
  <Input name="plot.txt" type="">plot.txt</Input>
  <Input name="plant.par" type="">plant.par</Input>
</Files>
\end{lstlisting}
The files mentioned in this section
 need, then, to be put into the working directory specified
by the \xmlNode{workingDir} node into the \xmlNode{RunInfo} block.
%%%%%%%%%%%%%%%%%%%%%%%%%%%%%%%%%%%%%%%%%%%%%%%%%%%%%%%%%%%%%%%%%%%%%%%%%%%%%%%%%%
\paragraph{Models}
The \xmlNode{Models} block contains the name of the executable file of MAAP5
(with the path, if necessary),
and the name of the interface (e.g. MAAP5\_GenericV7).
The block has also some required nodes:
\begin{itemize}
  \item \xmlNode{boolMaapOutputVariables}: containing the number of the MAAP5 IEVNT corresponding to the boolean events of interest;
  \item \xmlNode{contMaapOutputVariables}: containing the list of all the continuous variables we are interested at,
  and that we want to monitor;
  \item \xmlNode{stopSimulation}: this node is required only in case of DET sampling strategy.
The user needs to specify
  if the MAAP5 simulation run stops due to the reached END TIME, specifying ''mission\_time'',
or due to the occurrence of a specific event by
  inserting the number of the corresponding MAAP5 IEVNT (e.g IEVNT(691) for core uncovery)
 \item \xmlNode{includeForTimer}: also this node is required only in case of DET sampling
 strategy and it contains the name of the
 MAAP5 include file where the TIMERS for the different variables are defined
(see paragraph ''MAAP5 include file below'' for more information about timers).
\end{itemize}

A \xmlNode{Models} block is shown as an example below:
\begin{lstlisting}[style=XML]
<Models>
  <Code name="MyMAAP" subType="MAAP5\_GenericV7">
    <executable>MAAP5.exe</executable>
    <clargs type='input' extension='.inp'/>
    <boolMaapOutputVariables>691</boolMaapOutputVariables>
    <contMaapOutputVariables>PPS,PSGGEN(1),ZWDC2SG(1)
    </contMaapOutputVariables>
    <stopSimulation>mission_time</stopSimulation>
    <includeForTimer>include</includeForTimer>
  </Code>
</Models>
\end{lstlisting}
%%%%%%%%%%%%%%%%%%%%%%%%%%%%%%%%%%%%%%%%%%%%%%%%%%%%%%%%%%%%%%%%%%%%%%%%%%%%%%%%%%
\paragraph{Other blocks}
All the other blocks (e.g. \xmlNode{Distributions}, \xmlNode{Samplers}, \xmlNode{Steps},
 \xmlNode{Databases}, \xmlNode{OutStream}, etc.)
do not require any particular arrangements than already provided by a RAVEN input.
User can, therefore, refer to the corresponding sections of the User's Manual.
This is valid for both forward sampling and DET.
%%%%%%%%%%%%%%%%%%%%%%%%%%%%%%%%%%%%%%%%%%%%%%%%%%%%%%%%%%%%%%%%%%%%%%%%%%%%%%%%%%
\subsubsection{MAAP5 Input files}
%%%%%%%%%%%%%%%%%%%%%%%%%%%%%%%%%%%%%%%%%%%%%%%%%%%%%%%%%%%%%%%%%%%%%%%%%%%%%%%%%%
The coupling of RAVEN and MAAP5 requires modifications to some
MAAP5 files in order to work. This is particularly true when a DET analysis is performed.
The MAAP5 input files that need to be modified are:
\begin{itemize}
  \item MAAP5 include file
  \item MAAP5 input file (.inp)
  \item PLOTFIL blocks
\end{itemize}
%%%%%%%%%%%%%%%%%%%%%%%%%%%%%%%%%%%%%%%%%%%%%%%%%%%%%%%%%%%%%%%%%%%%%%%%%%%%%%%%%%
\paragraph{MAAP5 include file}
Usually MAAP5 simulation provides the presence of some include files, for example,
containing the user-defined variables, timers, definition of the plotfil, etc.
The adjustments explained in this section are required only in case of a DET analysis.
The user needs to modify the include file containing the set of the
timers used into the run, by adding the definition of the different timers,
one for each variable that causes a branching.
The include file to be modified should correspond to that one defined in the \xmlNode{includeForTimer}
block of the RAVEN xml input.

User is supposed to check that the numbers used for the different timers definition
are not already used in any of the other MAAP5 files.
These timers should be preceeded by a line reporting ''C Branching + name of the variable
sampled by RAVEN causing the branching''.

For example, we assume that DIESEL is the name of the variable corresponding to the failure time
of the Diesel generators (user defined). User has to firstly ensure that, for example,
''TIMER 100'' is not already used into the model, then the following lines
need to be added into the selected include file for the set of the timer corresponding
to the Diesel generators failure:
\begin{lstlisting}[style=XML]
C Branching DIESEL
WHEN (TIM>DIESEL)
   SET TIMER 100
END
\end{lstlisting}
It is worth mentioning that at this step a TIMER should be defined
also for the event IEVNT specified into the \xmlNode{stopSimulation},
 if this is the stop condition for the MAAP5 run:
\begin{lstlisting}[style=XML]
WHEN IEVNT(691) == 1.0
  SET TIMER 10
END
\end{lstlisting}
The interface will check that one timer is defined for each variable
of the DET. If not, an error arises suggesting to user the name of the variable having no
timer defined.
%%%%%%%%%%%%%%%%%%%%%%%%%%%%%%%%%%%%%%%%%%%%%%%%%%%%%%%%%%%%%%%%%%%%%%%%%%%%%%%%%%
\paragraph{MAAP5 input file}
In the ''parameter change'' section of the MAAP5 input file, the user should declare
the name of the variables sampled by RAVEN according to the following statement:
\begin{lstlisting}[style=XML]
 variable = $ RAVEN-variable:default$
\end{lstlisting}
where the dafault value is optional.

For example:
\begin{lstlisting}[style=XML]
DIESEL = $RAVEN-DIESEL:-1$
\end{lstlisting}
This is valid for both forward and DET sampled variables.
In particular, in case of DET analysis, the variables causing the occurrence of the branch should be
assigned within a block identified by the comment ''C DET Sampled variables'':
\begin{lstlisting}[style=XML]
C DET Sampled Variables
DIESEL = $RAVEN-DIESEL:-1$
C End DET Sampled Variables
\end{lstlisting}
If the sampled variables are user-defined, then the user shall ensure that they are initialized
(to the default value) and set within the user-defined variables section of one of the include
file.
As usual, a distribution and a sampling strategy should then correspond to each of these variables
into the RAVEN xml input file.

Only for the DET analysis, then, the occurence of a branch will be identified by a comment before. This comment is
 ''C BRANCHING + name of the variable determining the branch'' and acts as a sort of branching marker.
Looking for these markers, indeed, the interface (in case of DET sampler) verifies that at least
one branching exists, and furthermore, that one branching is defined for each of the
variables contained into ''DET sampled variables''.

Within the block, the occurrence of the branching leads the value of a variable (user-defined)
called ''TIM+number of the corresponding timer set into
the include file'' to switch to 1.0. The code, in fact, detects if a branch has occurred by monitoring
the value of these kind of variables. SInce these variables are user-defined, they need to be
initialized to a value (different from 1.0), into the ''user-defined variables'' section of one of the include
files.

Therefore following the previous example, if we want that, when the diesels failure occurs it leads
to the event ''Loss of AC Power'' (IEVNT(205) of MAAP5), we will have:
\begin{lstlisting}[style=XML]
C Branching TIMELOCA
WHEN TIM > DIESEL
 TIM100=1.0
 IEVNT(205)=1.0
END
\end{lstlisting}
It is worth noticing that no comments should be contained within the line of assignment
(i.e. IEVNT(205)=1.0 //LOSS OF AC POWER is not allowed).

Finally, only in case of DET analysis, a stop simulation condition (provided by the comment
''C Stop Simulation condition'') needs to be put into the input.
The original input should have all the timers (linked with the branching) separated by an OR
condition, even including that one of the event that stops the simulation (e.g. IEVNT(691)),
if any.
\begin{lstlisting}[style=XML]
C Stop Simulation condition
IF (TIMER 10 > 0) OR (TIMER 100 > 0) OR ... (TIMER N > 0)
 TILAST=TIM
END
\end{lstlisting}
This allows the simulation run to stop when a branch condition occurs, creating the restart file that will
be used by the two following branches.

For each branch, then, the interface will automatically update the name of the RESTART FILE to be used and
of the RESTART TIME that will be equal to the difference between the END TIME of the ''parent'' simulation
and the PRINT INTERVAL (which specifies the interval at which the restart output is written).
%%%%%%%%%%%%%%%%%%%%%%%%%%%%%%%%%%%%%%%%%%%%%%%%%%%%%%%%%%%%%%%%%%%%%%%%%%%%%%%%%%
\paragraph{MAAP5 PLOTFIL blocks}
This section refers to the ''PLOTFIL blocks'' used to modify the plot file (.csv) defined into the parameter file.
These blocks need to be modified in order to include some variables.
It is important, indeed, that the MAAP5 csv PLOTFIL files contain the evolution of:
\begin{itemize}
  \item RAVEN sampled variables (e.g. DIESEL) (both for Forward and DET sampling)
  \item the variables whose value is modified by the occurrence of one of the branches, either continuous or boolean (e.g. IEVNT(225))
  \item the variables of interest defined within \xmlNode{boolMaapOutputVariables}
  and \\ \xmlNode{contMaapOutputVariables} blocks (both for Forward and DET sampling)
\end{itemize}
If one of these variables is not contained into one of csv files, RAVEN will give an error.

%%%%%%%%%%%%%%%%%%%%%%%%%%%%%%%%%%%%%%%%%%%%%%%%%
%%%%%%%%%%%%% MAMMOTH INTERFACE %%%%%%%%%%%%%
%%%%%%%%%%%%%%%%%%%%%%%%%%%%%%%%%%%%%%%%%%%%%%%%%
\subsection{MAMMOTH Interface}
%
This section covers the input specification for running MAMMOTH through RAVEN.
It is important to notice that this short explanation assumes that the reader already knows how to use MAMMOTH.
The interface can be used to perturb Bison, Rattlesnake, RELAP-7, and general MOOSE input files that utilize
MOOSE's standard YAML input structure as well as Yak multigroup cross section library XML input files.
%

%%%%%%%%%%%%%%%%%%%%%%%%%%%%%%%%%%%%%%%%%%%%%%%%%%
\subsubsection{Files}
\xmlNode{Files} works the same as in other interfaces with name and type
attributes for each node entry.  The \xmlAttr{name} attribute is a user-chosen internal
name for the file contained in the node, and \xmlAttr{type} identifies which base-level
interface the file is used within.  \xmlAttr{type} should be specified for all inputs
used in RAVEN's MultiRun for MAMMOTH (including files not perturbed by RAVEN).
The MAMMOTH input file's \xmlAttr{type} should have \xmlString{MAMMOTHInput} prepended
to the driver app's input specification (e.g. \xmlString{MAMMOTHInput|appNameInput}).
Any other app's input file needs a \xmlAttr{type} with the app's name prepended to \xmlString{Input}
(e.g. \xmlString{BisonInput}, \xmlString{Relap7Input}, etc.).  In addition, the \xmlAttr{type} for any mesh
input is the app in which that mesh is utilized prepended to \xmlString{|Mesh}; so a Bison mesh would have
a \xmlAttr{type} of \xmlString{Bison|Mesh} and similarly a mesh for Rattlesnake would have \xmlString{Rattlesnake|Mesh}
as its \xmlAttr{type}. In cases where a file needs to be copied to each perturbed run
directory (to be used as function input, control logic, etc.), one can use the \xmlAttr{type}
\xmlString{AncillaryInput} to make it clear in the RAVEN input file that this is file
is required for the simulation to run but contains no perturbed parameters.
For Yak multigroup cross section libraries,
the \xmlAttr{type} should be \xmlString{YakXSInput}, and for the Yak
alias files that are used to perturb the Yak multigroup cross section libraries, the \xmlAttr{type} should be
\xmlString{YakXSAliasInput}.

The node should contain the path to the file from the working directory.
The following is an example of a typical \xmlNode{Files} block.
%
\begin{lstlisting}[style=XML]
<Files>
  <Input name='mammothInput' type='MAMMOTHInput|RattlesnakeInput'>test_mammoth.i</Input>
  <Input name='crossSection'    type='YakXSInput'>xs.xml</Input>
  <Input name='alias' type='YakXSAliasInput'>alias.xml</Input>
  <Input name='bisonInput'    type='BisonInput'>test_bison.xml</Input>
  <Input name='bisonMesh'    type='Bison|Mesh'>bisonMesh.e</Input>
  <Input name='fuelCTEfunct' type='AncillaryInput'>uo2_CTE.csv</Input>
  <Input name='rattlesnakeMesh'    type='Rattlesnake|Mesh'>rattlesnakeMesh.e</Input>
</Files>
\end{lstlisting}
%
The alias files are employed to define the variables that will be used to perturb Yak multigroup cross section
libraries. Please see the section \ref{RattlesnakeInterfaces} for the example.
%
%%%%%%%%%%%%%%%%%%%%%%%%%%%%%%%%%%%%%%%%%%%%%%%%%%
\subsubsection{Models}
A user provides paths to executables and aliases for sampled variables within the
\xmlNode{Models} block.  The \xmlNode{Code} block will contain \xmlAttr{name} and
\xmlAttr{subType}.  The attribute \xmlAttr{name} identifies that particular \xmlNode{Code} model within RAVEN, and
\xmlAttr{subType} specifies which code interface the model will use. The \xmlNode{executable}
block should contain the absolute or relative (with respect to the current working
directory) path to MAMMOTH that RAVEN will use to run generated input
files.

An example \xmlNode{Models} block follows.

\begin{lstlisting}[style=XML]
<Models>
  <Code name="Mammoth" subType="MAMMOTH">
    <executable>\%FRAMEWORK_DIR\%/../../mammoth/
     mammoth-%METHOD%</executable>
  </Code>
</Models>
\end{lstlisting}

%%%%%%%%%%%%%%%%%%%%%%%%%%%%%%%%%%%%%%%%%%%%%%%%%%
\subsubsection{Distributions}
The \xmlNode{Distributions} block defines all distributions used to
sample variables in the current RAVEN run.

For all the possible distributions and their possible inputs please
refer to the Distributions chapter (see~\ref{sec:distributions}).
%
It is good practice to name the distribution something similar to what kind of
variable is going to be sampled, since there might be many variables with the
same kind of distributions but different input parameters.

%%%%%%%%%%%%%%%%%%%%%%%%%%%%%%%%%%%%%%%%%%%%%%%%%%
\paragraph{Samplers}
The \xmlNode{Samplers} block defines the variables to be sampled.
After defining a sampling scheme, the variables to be sampled and
their distributions are identified in the \xmlNode{variable} blocks.
The \xmlAttr{name} attribute in the \xmlNode{variable} block must either be the
app's name prepended to the full MooseBasedApp model variable name, the alias name specifed in
\xmlNode{Models}, or the variable name specified in the provided alias files.

For listings of available samplers, please refer to the Samplers chapter (see~\ref{sec:Samplers}).
See the following for an example of a grid based sampler used to generate the samples for
the first energy group fission and capture cross sections (both of which have
defined in alias files provided in \xmlNode{Files}), the initial condition temperature defined in Rattlesnake
input file and the poissons ratio, clad thickness, and gap width defined in Bison input files with clad
and gap parameters calculated using an external function with sampled clad inner and outer diameters
as inputs.
%
\begin{lstlisting}[style=XML]
<Samplers>
  <Grid name="Grid_sampling">
    <variable name="Rattlesnake@fission_group_1" >
      <distribution>fission_dist</distribution>
      <grid type="value" construction="custom">1.0 2.0</grid>
    </variable>
    <variable name="Rattlesnake@capture_group_1">
      <distribution>capture_dist</distribution>
      <grid type="value" construction="custom">3.0 6.0</grid>
    </variable>
    <variable name="Rattlesnake@AuxVariables|Temp|initial_condition">
      <distribution>uniform</distribution>
      <grid type="value" construction="custom">3.0 6.0</grid>
    </variable>
    <variable name="Bison@Materials|fuel_solid_mechanics_elastic|poissons_ratio">
      <distribution>normal</distribution>
      <grid type="value" construction="custom">3.0 6.0</grid>
    </variable>
    <variable name='clad_outer_diam'>
      <distribution>clad_outer_diam_dist</distribution>
      <grid construction='equal' steps='144' type='CDF'>0.02275 0.97725</grid>
    </variable>
    <variable name='clad_inner_diam'>
      <distribution>clad_inner_diam_dist</distribution>
      <grid construction='equal' steps='144' type='CDF'>0.02275 0.97725</grid>
    </variable>
    <variable name='Bison@Mesh|clad_thickness'>
      <function>clad_thickness_calc</function>
    </variable>
    <variable name='Bison@Mesh|clad_gap_width'>
      <function>clad_gap_width_calc</function>
    </variable>
  </Grid>
</Samplers>
\end{lstlisting}
%
In order to make the input variables of one application distinct from input variables of another,
an app's name followed by the '@' symbol is prepended to the variable name (e.g. \xmlString{appName@varName}).
Each variable to be used in an app's input file and sampled in the MAMMOTH interface is required
to have a destination app specified. All variables utilizing Rattlesnake's executable (whether
they are in the Rattlesnake input file or not) are listed as Rattlesnake variables
as that application's interface will sort input file and cross section
variables itself.  Notice that the clad inner and outer diameter sampled parameters have no app
name specified.  These parameters are utilized to sample values used as inputs
for the clad thickness and gap width variables in BISON, so by not specifying a destination
app, these are passed through the interface having only been used in an external function
to calculate parameters usable in an app's input.
%%%%%%%%%%%%%%%%%%%%%%%%%%%%%%%%%%%%%%%%%%%%%%%%%%
\subsubsection{Steps}
For a MAMMOTH interface run, the \xmlNode{MultiRun} step type will most likely be used. First, the step needs
to be named: this name will be one of the names used in the \xmlNode{Sequence} block. In our example, \xmlString{Grid\_Mammoth}.
%
\begin{lstlisting}[style=XML]
<MultiRun name='Grid_Mammoth' verbosity='debug'>
    <Input   class='Files' type=''>mammothInput</Input>
    <Input   class='Files' type=''>crossSection</Input>
    <Input   class='Files' type=''>alias</Input>
    <Input   class='Files' type=''>bisonInput</Input>
    <Input   class='Files' type=''>bisonMesh</Input>
    <Input   class='Files' type=''>fuelCTEfunct</Input>
    <Input   class='Files' type=''>rattlesnakeMesh</Input>
    <Model   class='Models' type='Code'>Mammoth</Model>
    <Sampler class='Samplers' type='Grid'>Grid_Samplering</Sampler>
    <Output  class='DataObjects' type='PointSet'>solns</Output>
</MultiRun>
\end{lstlisting}
%
With this step, we need to import all the files needed for the simulation:
%
\begin{itemize}
  \item MAMMOTH|Rattlesnake YAML input file;
  \item Yak multigroup cross section libraries input files (XML);
  \item Yak alias files used to define the perturbed variables (XML);
  \item Bison YAML input file;
  \item Bison mesh file;
  \item Bison function file for the fuel's coefficient of thermal expansion as a function of temperature;
  \item Rattlesnake mesh file.
\end{itemize}
As well as \xmlNode{Model}, \xmlNode{Sampler} and outputs, such as \xmlNode{OutStreams} and \xmlNode{DataObjects}.

%%%%%%%%%%%%%%%%%%%%%%%%%%%%
%%%%%% MELCOR  INTERFACE  %%%%%%
%%%%%%%%%%%%%%%%%%%%%%%%%%%%
\subsection{MELCOR Interface}
\label{subsec:MELCORInterface}

The current implementation of MELCOR interface is valid for MELCOR 2.1/2.2; its validity for MELCOR
1.8 is \textbf{not been tested}.

\subsubsection{Sequence}
In the \xmlNode{Sequence} section, the names of the steps declared in the
\xmlNode{Steps} block should be specified.
%
As an example, if we called the first multirun ``Grid\_Sampler'' and the second
multirun ``MC\_Sampler'' in the sequence section we should see this:
\begin{lstlisting}[style=XML]
<Sequence>Grid_Sampler,MC_Sampler</Sequence>
\end{lstlisting}
%%%%%%%%%%%%%%%%%%%%%%%%%%%%%%%%%%%%%%%%%%%%%%%%%%%

\subsubsection{batchSize and mode}
For the \xmlNode{batchSize} and \xmlNode{mode} sections please refer to the
\xmlNode{RunInfo} block in the previous chapters.
%
%%%%%%%%%%%%%%%%%%%%%%%%%%%%%%%%%%%%%%%%%%%%%%%%%%%%
\subsubsection{RunInfo}
After all of these blocks are filled out, a standard example RunInfo block may
look like the example below:
\begin{lstlisting}[style=XML]
<RunInfo>
  <WorkingDir>~/workingDir</WorkingDir>
  <Sequence>Grid_Sampler,MC_Sampler</Sequence>
  <batchSize>8</batchSize>
</RunInfo>
\end{lstlisting}
In this example, the \xmlNode{batchSize} is set to $8$; this means that 8 simultaneous (parallel) instances
of MELCOR are going to be executed when a sampling strategy is employed.
%%%%%%%%%%%%%%%%%%%%%%%%%%%%%%%%%%%%%%%%%%%%%%%%%%%%%%%%%%%
\subsubsection{Files}
In the \xmlNode{Files} section, as specified before, all of the files needed for
the code to run should be specified.
%
In the case of MELCOR, the files typically needed are:
\begin{itemize}
  \item MELCOR Input file (file extension ``.i'' or ``.inp'')
  \item Restart file (if present)
\end{itemize}
Example:
\begin{lstlisting}[style=XML]
<Files>
  <Input name='melcorInputFile' type=''>inputFileMelcor.i</Input>
  <Input name='aRestart' type=''>restartFile</Input>
</Files>
\end{lstlisting}

It is a good practice to put inside the working directory (\xmlNode{WorkingDir}) all of these files.

\textcolor{red}{
\textbf{It is important to notice that the interface output collection  (i.e. the parser of the MELCOR output)
currently is able to extract \textit{CONTROL VOLUME HYDRODYNAMICS EDIT AND CONTROL FUNCTION EDIT} data only. Only those
variables are going to be exported and make available to RAVEN.
In addition, it is important to notice that:}
\begin{itemize}
  \item \textbf{the simulation time is stored in a variable called \textit{``time''}};
  \item \textbf{all the variables specified in the \textit{CONTROL VOLUME HYDRODYNAMICS EDIT}
   block are going to be converted using underscores. For example, the following EDITs:}
    \begin{table}[h]
    \centering
    \begin{tabular}{ccccc}
        VOLUME & PRESSURE & TLIQ   & TVAP   & MASS     \\
                & PA       & K      & K      & KG       \\
             1      & 1.00E+07 & 584.23 & 584.23 & 1.66E+03
     \end{tabular}
    \end{table}
    \\\textbf{will be converted in the following way (CSV):}
    \begin{table}[h]
    \centering
    \begin{tabular}{ccccc}
         $time$ & $volume\_1\_PRESSURE$& $volume\_1\_TLIQ$ & $volume\_1\_TVAP$   & $volume\_1\_MASS$     \\
             1.0   & 1.00E+07 & 584.23 & 584.23 & 1.66E+03
     \end{tabular}
    \end{table}
\end{itemize}
}

CONTROL FUNCTION EDIT data will not be converted in this manner. All data will be labeled using a label identical to what was entered in the MELCOR input file, with no changes.

Remember also that a MELCOR simulation run is considered successful (i.e., the simulation did not crash) if it terminates with the
following message:

\textcolor{red}{Normal termination}

If the a MELCOR simulation run stops with messages other than this one than the simulation is considered as
crashed, i.e., it will not be saved.
Hence, it is strongly recommended to set up the MELCOR input file so that the simulation exiting conditions are set through control logic
trip variables.

%%%%%%%%%%%%%%%%%%%%%%%%%%%%%%%%%%%%%%%%%%%%%%%%%%%%
\subsubsection{Models}
For the \xmlNode{Models} block here is a standard example of how it would look
when using MELCOR 2.1/2.2 as the external code:
\begin{lstlisting}[style=XML]
<Models>
  <Code name='MyMELCOR' subType='Melcor'>
    <executable>~/path_to_the_executable_of_melcor</executable>
    <preexec>~/path_to_the_executable_of_melgen</preexec>
  </Code>
</Models>
\end{lstlisting}
As it can be seen above, the \xmlNode{preexec} node must be specified, since MELCOR 2.1/2.2 must run the MELGEN utility
code before executing. Once the \xmlNode{preexec} node is inputted, the execution of MELGEN is performed automatically by the Interface.
\\In addition, if some command line parameters need to be passed to MELCOR, the user might use (optionally) the \xmlNode{clargs} XML nodes.
\begin{lstlisting}[style=XML]
<Models>
  <Code name='MyMELCOR' subType='Melcor'>
    <executable>~/path_to_the_executable_of_melcor</executable>
    <preexec>~/path_to_the_executable_of_melgen</preexec>
    <clargs type="text" arg="-r whatever command line instruction"/>
  </Code>
</Models>
\end{lstlisting}

%%%%%%%%%%%%%%%%%%%%%%%%%%%%%%%%%%%%%%%%%%%%%%%%%%%%%%%%%
\subsubsection{Distributions}
The \xmlNode{Distribution} block defines the distributions that are going
to be used for the sampling of the variables defined in the \xmlNode{Samplers}
block.
%
For all the possible distributions and all their possible inputs please see the
chapter about Distributions (see~\ref{sec:distributions}).
%
Here we report an example of a Normal distribution:
\begin{lstlisting}[style=XML,morekeywords={name,debug}]
<Distributions verbosity='debug'>
    <Normal name="temper">
      <mean>1.E+7</mean>
      <sigma>1.5</sigma>
      <upperBound>9.E+6</upperBound>
      <lowerBound>1.1E+7</lowerBound>
    </Normal>
 </Distributions>
\end{lstlisting}

It is good practice to name the distribution something similar to what kind of
variable is going to be sampled, since there might be many variables with the
same kind of distributions but different input parameters.
%
%%%%%%%%%%%%%%%%%%%%%%%%%%%%%%%%%%%%%%%%%%%%%%%%%%%%%%%%%
\subsubsection{Samplers}
In the \xmlNode{Samplers} block we want to define the variables that are going
to be sampled.
%
\textbf{Example}:
We want to do the sampling of 1 single variable:
\begin{itemize}
  \item The in pressure ($P\_in$) of a control volume regulated by a Tabular Function $TF\_TAB$
\end{itemize}

We are going to sample this variable using two different sampling methods:
Grid and MonteCarlo.

The interface of MELCOR uses the \textbf{\textit{GenericCode}} (see section \ref{subsec:genericInterface})
interface for the input perturbation; this means that the original input file (listed in the \xmlNode{Files} XML block)
needs to implement wild-cards.
%
In this example we are sampling the variable:
\begin{itemize}
  \item \textit{PRE}, which acts on the Tabular Function $TF\_TAB$ whose $TF\_ID $ is $P\_in$.
\end{itemize}

We proceed to do so for both the Grid sampling and the MonteCarlo sampling.

\begin{lstlisting}[style=XML,morekeywords={name,type,construction,lowerBound,steps,limit,initialSeed}]
<Samplers verbosity='debug'>
  <Grid name='Grid_Sampler' >
    <variable name='PRE'>
      <distribution>temper</distribution>
      <grid type='CDF' construction='equal'  steps='10'>0.001 0.999</grid>
    </variable>
  </Grid>
  <MonteCarlo name='MC_Sampler'>
     <samplerInit>
       <limit>1000</limit>
     </samplerInit>
    <variable name='PRE'>
      <distribution>temper</distribution>
  </MonteCarlo>
</Samplers>
\end{lstlisting}

It can be seen that each variable is connected with a proper distribution
defined in the \\\xmlNode{Distributions} block (from the previous example).
%
The following demonstrates how the input for the variable is read.

We are sampling a variable whose wild-card in the original input file is named $\$RAVEN-PRE\$$
using a Grid sampling method.
%
The distribution that this variable is following is a Normal distribution
(see section above).
%
We are sampling this variable beginning from 0.001 (CDF) in 10 \textit{equal} steps of
0.0998 (CDF).
%
%%%%%%%%%%%%%%%%%%%%%%%%%%%%%%%%%%%%%%%%%%%%%%%%%%%%%%%%%%%
\subsubsection{Steps}
For a MELCOR interface, the \xmlNode{MultiRun} step type will most likely be
used.
%
First, the step needs to be named: this name will be one of the names used in
the \xmlNode{sequence} block.
%
In our example, \texttt{Grid\_Sampler} and \texttt{MC\_Sampler}.
%
\begin{lstlisting}[style=XML,morekeywords={name,debug,re-seeding}]
     <MultiRun name='Grid_Sampler' verbosity='debug'>
\end{lstlisting}

With this step, we need to import all the files needed for the simulation:
\begin{itemize}
  \item MELCOR input file
  \item any other file needed by the calculation (e.g. restart file)
\end{itemize}
\begin{lstlisting}[style=XML,morekeywords={name,class,type}]
    <Input   class='Files' type=''>inputFileMelcor.i</Input>
    <Input   class='Files' type=''>restartFile</Input>
\end{lstlisting}
We then need to define which model will be used:
\begin{lstlisting}[style=XML]
    <Model  class='Models' type='Code'>MyMELCOR</Model>
\end{lstlisting}
We then need to specify which Sampler is used, and this can be done as follows:
\begin{lstlisting}[style=XML]
    <Sampler class='Samplers' type='Grid'>Grid_Sampler</Sampler>
\end{lstlisting}
And lastly, we need to specify what kind of output the user wants.
%
For example the user might want to make a database (in RAVEN the database
created is an HDF5 file).
%
Here is a classical example:
\begin{lstlisting}[style=XML,morekeywords={class,type}]
    <Output  class='Databases' type='HDF5'>Grid_out</Output>
\end{lstlisting}
Following is the example of two MultiRun steps which use different sampling
methods (Grid and Monte Carlo), and creating two different databases for each
one:
\begin{lstlisting}[style=XML]
<Steps verbosity='debug'>
  <MultiRun name='Grid_Sampler' verbosity='debug'>
    <Input   class='Files' type=''>inputFileMelcor.i</Input>
    <Input   class='Files' type=''>restartFile</Input>
    <Model   class='Models'    type='Code'>MyMELCOR</Model>
    <Sampler class='Samplers'  type='Grid'>Grid_Sampler</Sampler>
    <Output  class='Databases' type='HDF5'>Grid_out</Output>
    <Output  class='DataObjects' type='PointSet'   >GridMelcorPointSet</Output>
    <Output  class='DataObjects' type='HistorySet'>GridMelcorHistorySet</Output>
  </MultiRun>
  <MultiRun name='MC_Sampler' verbosity='debug' re-seeding='210491'>
    <Input   class='Files' type=''>inputFileMelcor.i</Input>
    <Input   class='Files' type=''>restartFile</Input>
    <Model   class='Models'    type='Code'>MyMELCOR</Model>
    <Sampler class='Samplers'  type='MonteCarlo'>MC_Sampler</Sampler>
    <Output  class='Databases' type='HDF5'      >MC_out</Output>
    <Output  class='DataObjects' type='PointSet'   >MonteCarloMelcorPointSet</Output>
    <Output  class='DataObjects' type='HistorySet'>MonteCarloMelcorHistorySet</Output>
  </MultiRun>
</Steps>
\end{lstlisting}
%%%%%%%%%%%%%%%%%%%%%%%%%%%%%%%%%%%%%%%%%%%%%%%%%%%%%%
\subsubsection{Databases}
As shown in the \xmlNode{Steps} block, the code is creating two database objects
called \texttt{Grid\_out} and \texttt{MC\_out}.
%
So the user needs to input the following:
\begin{lstlisting}[style=XML]
<Databases>
  <HDF5 name="Grid_out" readMode="overwrite"/>
  <HDF5 name="MC_out" readMode="overwrite"/>
</Databases>
\end{lstlisting}
As listed before, this will create two databases.
%
The files will have names corresponding to their \xmlAttr{name} appended with
the .h5 extension (i.e. \texttt{Grid\_out.h5} and \texttt{MC\_out.h5}).
%%%%%%%%%%%%%%%%%%%%%%%%%%%%%%%%%%%%%%%%%%%%%%%%%%%%%%
\subsubsection{DataObjects}
As shown in the \xmlNode{Steps} block, the code is creating $4$ data objects ($2$ HistorySet and $2$ PointSet)
called \texttt{GridMelcorPointSet} \texttt{GridMelcorHistorySet} \texttt{MonteCarloMelcorPointSet} and
 \texttt{MonteCarloMelcorHistorySet}.
%
So the user needs to input the following block as well, where the Input and Output variables are listed:
\begin{lstlisting}[style=XML]
  <DataObjects>
    <PointSet name="GridMelcorPointSet">
      <Input>PRE</Input>
      <Output>
        time,volume_1_PRESSURE,volume_1_TLIQ,
        volume_1_TVAP,volume_1_MASS
      </Output>
    </PointSet>
    <HistorySet name="GridMelcorHistorySet">
      <Input>PRE</Input>
      <Output>
        time,volume_1_PRESSURE,volume_1_TLIQ,
        volume_1_TVAP,volume_1_MASS
      </Output>
    </HistorySet>
    <PointSet name="MonteCarloMelcorPointSet">
      <Input>PRE</Input>
      <Output>
        time,volume_1_PRESSURE,volume_1_TLIQ,
        volume_1_TVAP,volume_1_MASS
      </Output>
    </PointSet>
    <HistorySet name="MonteCarloMelcorHistorySet">
      <Input>PRE</Input>
      <Output>
        time,volume_1_PRESSURE,volume_1_TLIQ,
        volume_1_TVAP,volume_1_MASS
      </Output>
    </HistorySet>
  </DataObjects>
\end{lstlisting}
As mentioned before, this will create $4$ DataObjects.
%
%%%%%%%%%%%%%%%%%%%%%%%
%%%%%% SCALE  INTERFACE %%%%%%
%%%%%%%%%%%%%%%%%%%%%%%
\subsection{SCALE Interface}
This section presents the main aspects of the interface between RAVEN and SCALE system,
the consequent RAVEN input adjustments and the modifications of the SCALE
files required to run the two coupled codes.
\\ \textcolor{red}{
\textbf{\textit{\nb Considering the large amount of SCALE sequences, this interface is
currently limited in driving the following SCALE calculation codes:}}
\begin{itemize}
  \item \textbf{\textit{ORIGEN}}
  \item \textbf{\textit{TRITON (using NEWT as transport solver)}}
\end{itemize}
}

In the following sections a short explanation on how to use RAVEN coupled with SCALE is reported.

%%%%%%%%%%%%%%%%%%%%%%%%%%%%%%%%%%%%%%%%%%%%%%%%%%%%%%%%%

\subsubsection{Models}
As for any other Code, in order to activate the SCALE interface, a  \xmlNode{Code} XML node needs to be inputted, within the
main XML node \xmlNode{Models}.
\\The  \xmlNode{Code} XML node contains the
information needed to execute the specific External Code.

\attrsIntro
%
\vspace{-5mm}
\begin{itemize}
  \itemsep0em
  \item \nameDescription
  \item \xmlAttr{subType}, \xmlDesc{required string attribute}, specifies the
  code that needs to be associated to this Model.
  %
  \nb See Section~\ref{sec:existingInterface} for a list of currently supported
  codes.
  %
\end{itemize}
\vspace{-5mm}

\subnodesIntro
%
\begin{itemize}
  \item \xmlNode{executable} \xmlDesc{string, required field} specifies the path
  of the executable to be used.
  %
  \nb Either an absolute or relative path can be used.
  \item \aliasSystemDescription{Code}
  %
\end{itemize}

In addition (and specifc for the SCALE interface), the  \xmlNode{Code} can contain the following optional nodes:

\begin{itemize}
  \item \xmlNode{sequence}, optional, comma separated list. In this node the user can specify a list of sequences that need to be
  executed in sequence. For example, if a TRITON calculation needs to be followed by an ORIGEN decay heat calculation the user
  would input here the sequence ``\textit{triton,origen}''. \default{triton}.
  \\\nb Currently only the following entries are supported:
    \begin{itemize}
     \item  ``\textit{triton}''
     \item  ``\textit{origen}''
     \item  ``\textit{triton,origen}''
    \end{itemize}
  \item \xmlNode{timeUOM}, optional, string. In this node the user can specify  the \textit{units} for the independent variable ``time''.
   If the outputs are exported by SCALE in a different unit (e.g days, years, etc.), the SCALE interface will convert all the different
   time scales into the unit here specified (in order to have a consistent  (and unique) time scale). Available are:
    \begin{itemize}
     \item ``\textit{s}'', seconds
     \item ``\textit{m}'', minutes
     \item ``\textit{h}'', hours
     \item ``\textit{d}'', days
     \item ``\textit{y}'', years
    \end{itemize}
    \default{s}
\end{itemize}

An example  is shown  below:
\begin{lstlisting}[style=XML]
<Models>
    <Code name="MyScale" subType="Scale">
      <executable>path/to/scalerte</executable>
      <sequence>triton,origen</sequence>
      <timeUOM>d</timeUOM>
    </Code>
</Models>
\end{lstlisting}

%%%%%%%%%%%%%%%%%%%%%%%%%%%%%%%%%%%%%%%%%%%%%%%%%%%%%%%%%%%%%%%%%%%%%%%%%%%%%%%%%%
\subsubsection{Files}
%%%%%%%%%%%%%%%%%%%%%%%%%%%%%%%%%%%%%%%%%%%%%%%%%%%%%%%%%%%%%%%%%%%%%%%%%%%%%%%%%%
The \xmlNode{Files} XML node has to contain all the files required by the particular
sequence (s) of the external code  (SCALE) to be run.
This involves not only the input file(s) (.inp) but also the auxiliary files that might be needed (e.g. binary initial compositions, etc.).
As mentioned, the current SCALE interface only supports TRITON and ORIGEN sequences. For this reason, depending on the
type of sequence (see previous section) to be run, the relative input files need to be marked with the sequence they are associated
with. This means that the type of the input file must be either ``triton'' or ``origen''. The auxiliary files that might be needed by
a particular sequence (e.g. binary initial compositions, etc.) should not be marked with any specific type (i.e. \textit{type=``''}).
Example:
\begin{lstlisting}[style=XML]
<Files>
  <Input name="triton_input" type="triton">pwr_depletion.inp</Input>
  <Input name="origen_input" type="origen">decay_calc.inp</Input>
  <Input name="binary_comp" type="">pwr_depletion.f71</Input>
</Files>
\end{lstlisting}
The files mentioned in this section
 need, then, to be placed into the working directory specified
by the \xmlNode{workingDir} node in the \xmlNode{RunInfo} XML node.

\paragraph{Output Files conversion}
Since RAVEN expects to receive a CSV file containing the outputs of the simulation, the results in the SCALE output
files need to be converted by the code interface.
\\As mentioned, the current interface \textcolor{red}{ is able to collect data from TRITON and ORIGEN sequences only}.
%% TRITON
\\The following information is collected from TRITON output:
\begin{itemize}
  \item \textit{\textbf{k-eff and k-inf time-dep information}}
  \begin{lstlisting}[basicstyle=\tiny]
  Outer   Eigenvalue Eigenvalue Max Flux   Max Flux     Max Fuel   Max Fuel     Wall   Elapsed   Iteration  CPU   Inners
 Iter. #              Delta      Delta   Location(r,g)   Delta   Location(r,g) Clock   CPU Time   CPU Time Usage Converged
 - - - - - - - - - - - - - - - - - - - - - - - - - - - - - - - - - - - - - - - - - - - - - - - - - - - - - - - - - - - -
     1    1.00000   0.000E+00 6.480E+09 (    4,252)   1.000E+00 (  614,  0) 14:16:42   89.9 s    89.9 s  92.7%    F
     2    0.35701   1.801E+00 4.149E+01 (  319,  4)   2.673E+00 ( 7035,  0) 14:18:16  182.8 s    92.9 s  98.8%    F
 k-eff =       0.94724509     Time=      0.00d Nominal conditions

   Four-Factor Estimate of k-infinity.  Fast/Thermal boundary:   0.6250 eV
      Fiss. neutrons/thermal abs. (eta):          1.279827
      Thermal utilization (f):                    0.960903
      Resonance Escape probability (p):           0.706209
      Fast-fission factor (epsilon):              1.091716
                                            --------------
      Infinite neutron multiplication             0.948143

\end{lstlisting}
   that will be converted in the following way (CSV):
   \begin{table}[h]
    \centering
    \caption{CSV transport info}
    \label{CSVkeff}
    \tabcolsep=0.11cm
    \tiny
    \begin{tabular}{|c|c|c|c|c|c|c|c|c|c|}
     time & keff       & iter\_number & keff\_delta & max\_flux\_delta & kinf     & kinf\_epsilon & kinf\_p  & kinf\_f  & kinf\_eta \\
     0.00 & 0.94724509 & 2            & 1.801E+00   & 4.149e+01        & 0.948143 & 1.091716      & 0.706209 & 0.960903 & 1.279827
    \end{tabular}
   \end{table}

  \item \textit{\textbf{material powers}}
  \begin{lstlisting}[basicstyle=\tiny]
  --- Material powers for depletion pass no.   1 (MW/MITHM) ---
       Time =     0.00 days (   0.000 y), Burnup =    0.000     GWd/MTIHM, Transport k=  0.9473

                    Total    Fractional  Mixture     Mixture       Mixture
         Mixture    Power      Power      Power    Thermal Flux  Total Flux
          Number (MW/MTIHM)    (---)   (MW/MTIHM)  n/(cm^2*sec)  n/(cm^2*sec)
            13      32.985    0.99054     32.985    5.3666e+13    1.2574e+14
             6       0.252    0.00757     N/A       2.7587e+13    9.1781e+13
         Total      33.300    1.00000
\end{lstlisting}
   that will be converted in the following way (CSV):
   \begin{table}[h]
     \centering
     \caption{CSV material powers}
     \label{CSVmatPowers}
     \tabcolsep=0.11cm
     \tiny
     \begin{tabular}{|c|c|c|c|c|c|c|c|c|c|l}
     \cline{1-10}
     time    & bu  & tot\_power\_mix\_13 & fract\_power\_mix\_13 & th\_flux\_mix\_13 & tot\_flux\_mix\_13 & tot\_power\_mix\_6 & fract\_power\_mix\_6 & th\_flux\_mix\_6 & tot\_flux\_mix\_6 &  \\ \cline{1-10}
     1.0E-06 & 0.0 & 32.985              & 0.99054               & 5.3666e+13        & 1.2574e+14         & 0.252              & 0.00757              & 2.7587e+13       & 9.1781e+13        &  \\ \cline{1-10}
     \end{tabular}
   \end{table}


 \item \textit{\textbf{nuclide/element tables}}
  \begin{lstlisting}[basicstyle=\tiny]
            | nuclide concentrations
            | time: days
      grams |    0.00e+00d
------------+--------------------
       u235 |   2.9619e+04
       u238 |   9.6993e+05
   subtotal |   1.0010e+06
      total |   1.1858e+06
\end{lstlisting}
   that will be converted in the following way (CSV):
   \begin{table}[h]
    \centering
    \caption{CSV Nuclide/element Tables}
    \label{CSVnuclideTables}
    \tabcolsep=0.11cm
    \tiny
    \begin{tabular}{|c|c|c|}
     time & u235\_conc       & u238\_conc   \\
     0.00 & 2.9619e+04  & 9.6993e+05
    \end{tabular}
   \end{table}
\end{itemize}
%% ORIGEN
The following information is collected from ORIGEN output:
\begin{itemize}
  \item \textit{\textbf{history overview}}
  \begin{lstlisting}[basicstyle=\tiny]
=========================================================================================================================
=   History overview for case 'decay' (#1/1)                                                                            =
-------------------------------------------------------------------------------------------------------------------------
   step          t0          t1          dt           t        flux     fluence       power      energy
    (-)       (sec)       (sec)         (s)         (s)   (n/cm2-s)     (n/cm2)        (MW)       (MWd)
      1  0.0000E+00  1.0000E-06  1.0000E-06  1.0000E-06  0.0000E+00  0.0000E+00  0.0000E+00  0.0000E+00
\end{lstlisting}
   that will be converted in the following way (CSV):
    \begin{table}[h]
    \centering
    \caption{CSV History Overview}
    \label{CSVhistoryOverview}
    \tabcolsep=0.11cm
    \tiny
    \begin{tabular}{|c|c|c|c|c|c|c|c|}
    \hline
     time    & t0  & t1      & dt      & flux & fluence & power & energy \\
     1.0E-06 & 0.0 & 1.0E-06 & 1.0E-06 & 0.0  & 0.0     & 0.0   & 0.0
    \end{tabular}
   \end{table}

   \item \textit{\textbf{concentration tables}}
  \begin{lstlisting}[basicstyle=\tiny]
=========================================================================================================================
=   Nuclide concentrations in watts, actinides for case 'decay' (#1/1)                                                  =
-------------------------------------------------------------------------------------------------------------------------
  (relative cutoff; integral of concentrations over time >   1.00E-04 % of integral of all concentrations over time)
.
                0.0E+00sec  1.0E-06sec
  th231       8.6167E-08  8.6167E-08
  th234       7.7763E-09  7.7763E-09
------------
  totals       4.6831E+03  4.6831E+03
=========================================================================================================================
.
.
=========================================================================================================================
=   Nuclide concentrations in watts, fission products for case 'decay' (#1/1)                                           =
-------------------------------------------------------------------------------------------------------------------------
  (relative cutoff; integral of concentrations over time >   1.00E-04 % of integral of all concentrations over time)
.
                0.0E+00sec  1.0E-06sec
  ga74        2.4264E-01  2.4264E-01
  ga75        1.8106E+00  1.8106E+00
------------
  totals       1.2266E+06  1.2266E+06
  \end{lstlisting}
  that will be converted in the following way (CSV):
   \begin{table}[h]
    \centering
    \caption{CSV Concentration Tables}
    \label{CSVconcentrationTables}
    \tabcolsep=0.11cm
    \tiny
    \begin{tabular}{|c|c|c|c|c|c|c|c|}
    \hline
     time    & ga74\_watts  & ga75\_watts      & subtotals\_fission\_products      & th231\_watts & th234\_watts & subtotals\_actinides & totals\_watts \\ \hline
     0.0E+00 & 2.4264E-01 & 1.8106E+00 & 1.2266E+06 & 8.6167E-08  & 7.7763E-09     & 4.6831E+03   & 1.2313E+06    \\
     1.0E-06 & 2.4264E-01 & 1.8106E+00 & 1.2266E+06 & 8.6167E-08  & 7.7763E-09     & 4.6831E+03  & 1.2313E+06
    \end{tabular}
   \end{table}
\end{itemize}

\textbf{Remember also that a SCALE simulation run is considered successful (i.e., the simulation did not crash) if it does not contain, in
the last 20 lines, the following message:}

\textcolor{red}{terminated due to errors}

\textbf{If the a SCALE simulation terminates with this message, the simulation is considered ``failed'', i.e., it will not be saved.}

%%%%%%%%%%%%%%%%%%%%%%%%%%%%%%%%%%%%%%%%%%%%%%%%%%%%%%%%%%%%%%%%%%%%%%%%%%%%%%%%%%
\subsubsection{Samplers or Optimizers}
In the \xmlNode{Samplers} or  \xmlNode{Optimizers} block we want to define the variables that are going
to be sampled or optimized.
%
\\The perturbation or optimization of the input of any SCALE sequence is performed using the approach detailed in the \textit{Generic Interface} section (see \ref{subsec:genericInterface}). Briefly, this approach uses
 ``wild-cards'' (placed in the original input files) for injecting the perturbed values.
 For example, if the original input file (that needs to be perturbed) is the following:
\begin{lstlisting}[language=python]
=origen
case(actual_mass){
  lib{ file="end7dec" }
  mat{ iso=[zr-95=1.0] units="moles" }
  time=[1.0] %1 day
}
end
\end{lstlisting}
and  the initial moles of ``zr-95'' need to be perturbed, a RAVEN ``wild-card'' will be defined:
\begin{lstlisting}[language=python]
=origen
case(actual_mass){
  lib{ file="end7dec" }
  mat{ iso=[zr-95=$RAVEN-zrMoles$] units="moles" }
  time=[1.0] %1 day
}
end
\end{lstlisting}

Finally, the variable \textbf{\textit{zrMoles}} needs to be specified in the specific Sampler or Optimizer that will be used:

\begin{lstlisting}[style=XML]
...
<Samplers>
  <aSampler name='aUserDefinedName' >
    <variable name='zrMoles'>
      ...
    </variable>
  </aSampler>
</Samplers>
...
<Optimizers>
  <anOptimizer name='aUserDefinedName' >
    <variable name='zrMoles'>
      ...
    </variable>
  </anOptimizer>
</Samplers>
...
\end{lstlisting}
%
%%%%%%%%%%%%%%%%%%%%%%%
%%%%%% CTF  INTERFACE %%%%%%
%%%%%%%%%%%%%%%%%%%%%%%
\subsection{CTF Interface}
This section presents the main aspects of the interface between RAVEN and CTF (COBRA-TF) system,
the consequent RAVEN input adjustments and the modifications of the CTF
files required to run the two coupled codes. \noindent

\noindent \textcolor{red}{
\textbf{\textit{\nb This interface is currently working only with the specific type of CTF output file (.ctf.out or deck.out (if input file name is deck.inp)) }}
}

\noindent In the following sections a short explanation on how to use RAVEN coupled with CTF is reported.
%%%%%%%%%%%%%%%%%%%%%%%%%%%%%%%%%%%%%%%%%%%%%%%%%%%
\subsubsection{Sequence}
%%%%%%%%%%%%%%%%%%%%%%%%%%%%%%%%%%%%%%%%%%%%%%%%%%%
In the \xmlNode{Sequence} section, the names of the steps declared in the
\xmlNode{Steps} block should be specified.
%
As an example, if we called the first MultiRun ``Grid\_Sampler'' and the second
MultiRun ``MC\_Sampler'' in the sequence section we should see this:

\begin{lstlisting}[style=XML]
<Sequence>Grid_Sampler, MC_Sampler</Sequence>
\end{lstlisting}

%%%%%%%%%%%%%%%%%%%%%%%%%%%%%%%%%%%%%%%%%%%%%%%%%%%
\subsubsection{batchSize and mode}
%%%%%%%%%%%%%%%%%%%%%%%%%%%%%%%%%%%%%%%%%%%%%%%%%%%
For the \xmlNode{batchSize} and \xmlNode{mode} sections please refer to the
\xmlNode{RunInfo} block in the previous chapters.

%%%%%%%%%%%%%%%%%%%%%%%%%%%%%%%%%%%%%%%%%%%%%%%%%%%%
\subsubsection{RunInfo}
%%%%%%%%%%%%%%%%%%%%%%%%%%%%%%%%%%%%%%%%%%%%%%%%%%%%
After all of these blocks are filled out, a standard example RunInfo block may
look like the example below: \\

\begin{lstlisting}[style=XML]
<RunInfo>
  <WorkingDir>~/workingDir</WorkingDir>
  <Sequence>Grid_Sampler,MC_Sampler</Sequence>
  <batchSize>8</batchSize>
</RunInfo>
\end{lstlisting}
In this example, the \xmlNode{batchSize} is set to $8$; this means that 8 simulatenous (parallel) instances
of CTF are going to be executed when a sampling strategy is employed.

%%%%%%%%%%%%%%%%%%%%%%%%%%%%%%%%%%%%%%%%%%%%%%%%%%%%%%%%%
\subsubsection{Models}
%%%%%%%%%%%%%%%%%%%%%%%%%%%%%%%%%%%%%%%%%%%%%%%%%%%%%%%%%
As any other Code, in order to activate the CTF interface, a \xmlNode{Code} XML node needs to be inputted, within the
main XML node \xmlNode{Models}.
\\The  \xmlNode{Code} XML node contains the
information needed to execute the specific External Code.

\attrsIntro
%
\vspace{-5mm}
\begin{itemize}
  \itemsep0em
  \item \nameDescription
  \item \xmlAttr{subType}, \xmlDesc{required string attribute}, specifies the
  code that needs to be associated to this Model.
  %
  \nb See Section~\ref{sec:existingInterface} for a list of currently supported
  codes.
  %
\end{itemize}
\vspace{-5mm}

\subnodesIntro
%
\begin{itemize}
  \item \xmlNode{executable} \xmlDesc{string, required field} specifies the path
  of the executable to be used.
  %
  \nb Either an absolute or relative path can be used.
  \item \aliasSystemDescription{Code}
  %
\end{itemize}

An example  is shown  below:

\begin{lstlisting}[style=XML]
<Models>
    <Code name="MyCobraTF" subType="CTF">
      <executable>path/to/cobratf</executable>
    </Code>
</Models>
\end{lstlisting}

%%%%%%%%%%%%%%%%%%%%%%%%%%%%%%%%%%%%%%%%%%%%%%%%%%%%%%%%%%%%%%%%%%%%%%%%%%%%%%%%%%
\subsubsection{Files}
%%%%%%%%%%%%%%%%%%%%%%%%%%%%%%%%%%%%%%%%%%%%%%%%%%%%%%%%%%%%%%%%%%%%%%%%%%%%%%%%%%
The \xmlNode{Files} XML node has to contain all the files required to run the external code  (CTF).
For RAVEN coupled with CTF, there are three input files. CTF input file (.inp) is required by the code. This input file includes all the geometry, boundary and calculation definitions.

\noindent There are two additional files (\textit{optional}) that can be used for model parameter perturbation(s) (vuq$\_$param.txt, vuq$\_$mult.txt). These two files can be used to change variables of models embedded in CTF. The ''vuq$\_$param.txt'' file includes parameter values, and ''vuq$\_$mult.txt'' file includes multipliers or additions to parameters. These files are not required by CTF unless a parameter exposure is desired. One, both or neither of them can be included in the simulation folder. The code first controls if these files exist and does modifications accordingly if needed.

The  \xmlNode{Files} XML node contains the information needed to execute CTF.

\attrsIntro
%
\vspace{-5mm}
\begin{itemize}
  \itemsep0em
  \item \nameDescription
  \item \xmlAttr{type}, \xmlDesc{required string attribute}, specifies the
  input type used by CTF (ctf, vuq$\_$mult, vuq$\_$param). Accepted types are as follows.
  \begin{itemize}
    \item CTF, \xmlDesc{required string attribute}, identifies the CTF input file (geometry, boundary, calculation options, etc.) and the code currently accept any name for input. One CTF input file is required.
    \item vuq\_mult, \xmlDesc{optional string attribute if closure modifiers are used}, identifies the closure term multiplier input file. If user needs to alter closure terms this file should be used and named \textcolor{red}{vuq$\_$mult.txt}. No other file name is accepted.
    \item vuq\_param, \xmlDesc{optional string attribute if model parameter modifiers are used}, identifies the model parameter input file. If user needs to change model parameters this file this model should be used and named \textcolor{red}{vuq$\_$params.txt}. No other file name is accepted.
    \item "", Empty type is also accepted by RAVEN input to perturb. Currently, CTF does not use any other input file that is not mentioned above, but to sample auxiliary files, this option can be used.
  \end{itemize}
\end{itemize}

\noindent Example:
\begin{lstlisting}[style=XML]
<Files>
  <Input name="CTF_input" type="ctf">case1.inp</Input>
  <Input name="vuq_param_input" type="vuq_param">vuq_param.txt</Input>
  <Input name="vuq_mult_input" type="vuq_mult">vuq_mult.txt</Input>
  <Input name="auxiliary_input" type="">auxiliaryInput</Input>
</Files>
\end{lstlisting}
The files mentioned in this section
 need, then, to be placed into the working directory specified
by the \xmlNode{workingDir} node in the \xmlNode{RunInfo} XML node.

\paragraph{Output Files Conversion}
Since RAVEN expects to receive a CSV file containing the outputs of the simulation, the results in the CTF output
files (.ctf.out or deck.out) need to be converted by the code interface.

\noindent \textcolor{red}{
\textbf{It is important to note that the interface output collection (i.e., the parser of the CTF output) is currently able to extract
major edit data (.ctf.out or deck.out) only. Only those variables printed in the "major edit" output files are exported and made available to RAVEN.} } \\
\\The following information is collected from CTF output file (.ctf.out or deck.out):
\begin{itemize}
  \item \textit{\textbf{average properties for channels}}
  \begin{lstlisting}[basicstyle=\tiny]


 ************************************************************************************************************************
          simulation time =      1.03030  seconds           aver. properties for channels
 node  dist.  quality    void fraction            mass flow               enthalpy incr.    enthalpy    heat added
  no.   (ft.)                                     (lbm/s)                    (btu/s)         (btu/s)      (btu/s)
                      liq.  vapor  entr.  liquid vapor entr.  integr.  liquid vapor integr.  mixture  liquid vapor integr.

  50   12.00  -.119   1.000 0.000 0.000   16.39  0.00  0.00  16.39     32.41  0.00  32.41   10683.98  32.37  0.00  32.37

\end{lstlisting}

   that will be converted in the following way (CSV):
   \begin{table}[h]
    \centering
    \caption{CSV transport info (average properties for channels)}
    \label{CSVaverageProperties}
    \tabcolsep=0.11cm
    \tiny
    \begin{tabular}{|c|c|c|c|c|c|c|c|c|c|c|}
     time & AVG\_ch\_ax50\_quality  & AVG\_ch\_ax50\_voidFractionLiquid & AVG\_ch\_ax50\_voidFractionVapor & AVG\_ch\_ax50\_volumeEntrainFraction & ...\\
     1.03030 & -.119 & 1.000  & 0.000   & 0.000  & ...
    \end{tabular}
   \end{table}

  \item \textit{\textbf{fluid properties for each sub-channel}}
  \begin{lstlisting}[basicstyle=\tiny]
              simulation time =      0.00000  seconds           fluid properties for channel   19
 node  dist. pressure  velocity             void fraction           flow rate          flow    heat added         gama
  no.  (ft.) (psi)     (ft/sec)                                      (lbm/s)           reg.     (btu/s)          (lbm/s)
                     liquid vapor entr. liquid  vapor  entr.   liquid  vapor   entr.         liquid    vapor


 155 0.00  1251.687  2.66   2.66  0.01  1.0000 0.0000 0.0000  0.12456  0.0000  0.00000  0   0.595E-01  0.000E+00   0.00

  \end{lstlisting}
   that will be converted in the following way (CSV):
   \begin{table}[h]
    \centering
    \caption{CSV transport info (fluid properties for channels)}
    \label{CSVfluidProperties}
    \tabcolsep=0.11cm
    \tiny
    \begin{tabular}{|c|c|c|c|c|c|c|c|c|c|c|}
     time & ch19\_ax155\_pressure  & ch19\_ax155\_velocityLiquid & ch19\_ax155\_velocityVapor & ch19\_ax155\_velocityEntrain & ch19\_ax155\_voidFractionLiquid & ...\\
     0.00 & 1251.687 & 2.66           & 2.66   & 0.01        & 1.00 & ...
    \end{tabular}
   \end{table}

  \item \textit{\textbf{nuclear fuel rod}}
  \begin{lstlisting}[basicstyle=\tiny]
          nuclear fuel rod no.  1                         simulation time =    0.00 seconds
             surface no.  1 of  1
          -----------------------        conducts heat to channels  1  0  0  0  0  0        geometry type =  1
                                         and azimuthally to surfaces   1 and   1            no. of radial nodes = 13

 **********************************************************************************************

   rod    axial    fluid temperatures  surface      heat       -clad temperatures-     gap        -fuel temperatures-
   node  location      (deg-f)         heat flux   transfer          (deg-f)        conductance         (deg-f)
   no.    (in.)    liquid  vapor       (b/h-ft2)    mode       outside  inside      (b/h-ft2-f)    surface   center
   ----  --------  ------  -----      ---------    --------    -------  ------      -----------    -------   ------

    10   22.80     464.1   467.1     0.5929E+04     spl        466.08   592.98       1594.2        859.58     2946.22
\end{lstlisting}
   that will be converted in the following way (CSV):
   \begin{table}[h]
     \centering
     \caption{CSV transport info (nuclear fuel rod)}
     \label{CSVfuelRod}
     \tabcolsep=0.11cm
     \tiny
     \begin{tabular}{|c|c|c|c|}
     time    & fuelRod10\_surface1\_ax10\_fluidTemperatureLiquid  & fuelRod10\_surface1\_ax10\_fluidTemperatureVapor & ... \\
     0.00 & 464.1 & 467.1  &  ...
     \end{tabular}
   \end{table}

   \item \textit{\textbf{cyclindrical tube}}

      \noindent \textcolor{red}{
   \textbf{Warning: CTF reports results for cylindrical tubes based on the flow type. Not every result will be available depending on the \underline{internal} or \underline{external} flow type. Check output file and see if the flow around heat slab is internal or external. If the user requests values that are not in the output file reported values will be from different columns and wrong. For example there is no outside surface liquid temperature when flow is internal.} } \\

   \begin{lstlisting}[basicstyle=\tiny]
    cylindrical tube rod no.  5                         simulation time =    2.00 seconds
           surface no.  1 of  4
    ------------------------        conducts heat to channels 10  0  0  0  0  0               geometry type =  2
                                    and azimuthally to surfaces   4 and   2                   no. of radial nodes =  2


 **********************************************************************************************************************

 rod    axial    *-------------- outside surface ----------------*   *---------------- inside surface ----------------*
 node  location   heat flux   h.t.  **** temperatures (deg-f) ****   **** temperatures (deg-f) ****    h.t.   heat flux
 no.    (in.)     (b/h-ft2)   mode     wall     vapor     liquid       liquid     vapor     wall       mode   (b/h-ftl)
 ----  --------   ---------   ----   -------   -------   -------       -------   -------   -------

  51    144.00  -0.4424E+03   spl    629.71     653.31    629.77                           629.68           0.0000E+00
  \end{lstlisting}
    that will be converted in the following way (CSV):
    \begin{table}[h]
      \centering
      \caption{CSV transport info (cylindrical tube)}
      \label{CSVcylTube}
      \tabcolsep=0.11cm
      \tiny
      \begin{tabular}{|c|c|c|c|}
      time    & cylRod10\_surface1\_ax51\_outsideSurfaceHeatFlux  & cylRod10\_surface1\_ax51\_outsideSurfaceWallTemperature &  ...  \\
      0.00 & -0.4424E+03 & 629.71  & ...
      \end{tabular}
    \end{table}

  \item \textit{\textbf{heat slab (tube)}}

   \noindent \textcolor{red}{
   \textbf{Warning: CTF reports results for heat slabs based on the flow type. Not every result will be available depending on the \underline{internal} or \underline{external} flow type. Check output file and see if the flow around heat slab is internal or external. If the user requests values that are not in the output file reported values will be from different columns and wrong. For example there is no outside surface liquid temperature when flow is internal.} } \\

   \begin{lstlisting}[basicstyle=\tiny]
          heat slab no.  1  (tube)                  simulation time =   20.00 seconds
                                                    fluid channel on inside surface =  1
                                                    fluid channel on outside surface =  0
                                                    geometry type =  1
                                                    no. of nodes =   2


 ***************************************************************************************************************

   rod    axial    *------------- outside surface ------------* *-------------- inside surface ----------------*
   node  location   heat flux  h.t.  ** temperatures (deg-f) ** ** temperatures (deg-f) ****    h.t.   heat flux
   no.    (in.)     (b/h-ft2)  mode   wall     vapor     liquid liquid     vapor     wall       mode   (b/h-ftl)
   ----  --------   ---------  ----  ------   -------   ------- -------   -------   -------

    21     19.69    999666E+00       200.00                     212.00    212.00    247.85     tran    0.2641E+02
  \end{lstlisting}
    that will be converted in the following way (CSV):
    \begin{table}[h]
      \centering
      \caption{CSV transport info (heat slab (tube) tube)}
      \label{CSVheatSlab}
      \tabcolsep=0.11cm
      \tiny
      \begin{tabular}{|c|c|c|c|}
      time    & heatSlab1\_ax21\_outsideSurfaceHeatFlux &  heatSlab1\_ax21\_outsideSurfaceWallTemperature &  ...  \\
      0.00 & 999666E+00 & 200.00  & ...
      \end{tabular}
    \end{table}

 \item \textit{\textbf{CTF's Output Variables and Corresponding Names in CSV file}}

   In CSV file, the output results obtained from the CTF output file (.ctf.out) will be saved with the names as described in Table \ref{CSVvariableNames}.

\end{itemize}

\captionsetup{justification=centering}
\captionof{table}{Variables Name List in CSV File \hspace{\textwidth} \textcolor{red}{NN: Axial Node Number; CN: Channel Number; \\ RN: Rod Number; SN: Surface Number: HN: Heat Slab Number}}
\label{CSVvariableNames}
\begingroup
% header and footer information
\scriptsize
\begin{longtable}{|l|l|}
\hline
\textbf{Output Variable} & \textbf{Name in CSV file} \\
\hline
\endhead
\hline
\endfoot
% body of table
      \scriptsize {simulation time  }&\scriptsize{ time  } \\
      channels' average height  & AVG\_ch\_ax\textcolor{red}{NN}\_height \\
      channels' average quality  & AVG\_ch\_ax\textcolor{red}{NN}\_quality \\
      channels' average void fraction (liquid)  & AVG\_ch\_ax\textcolor{red}{NN}\_voidFractionLiquid \\
      channels' average void fraction (vapor)  & AVG\_ch\_ax\textcolor{red}{NN}\_voidFractionVapor \\
      channels' average entrainment (volumetric) fraction  & AVG\_ch\_ax\textcolor{red}{NN}\_volumeEntrainFraction \\
      channels' average mass flow rate (liquid)  & AVG\_ch\_ax\textcolor{red}{NN}\_massFlowRateLiquid \\
      channels' average mass flow rate (vapor)  & AVG\_ch\_ax\textcolor{red}{NN}\_massFlowRateVapor \\
      channels' average entrainment rate (mass flow rate)  & AVG\_ch\_ax\textcolor{red}{NN}\_massFlowRateEntrain \\
      channels' average mass flow rate (integrated)  & AVG\_ch\_ax\textcolor{red}{NN}\_massFlowRateIntegrated \\
      channels' average enthalpy increase (liquid)  & AVG\_ch\_ax\textcolor{red}{NN}\_enthalpyIncreaseLiquid \\
      channels' average enthalpy increase (vapor)  & AVG\_ch\_ax\textcolor{red}{NN}\_enthalpyIncreaseVapor \\
      channels' average enthalpy increase (integrated)  & AVG\_ch\_ax\textcolor{red}{NN}\_enthalpyIncreaseIntegrated \\
      channels' average mixture enthalpy  & AVG\_ch\_ax\textcolor{red}{NN}\_enthalpyMixture \\
      channels' average heat added to liquid & AVG\_ch\_ax\textcolor{red}{NN}\_heatAddedToLiquid \\
      channels' average heat added to vapor & AVG\_ch\_ax\textcolor{red}{NN}\_heatAddedToVapor \\
      channels' average heat added (integrated)  & AVG\_ch\_ax\textcolor{red}{NN}\_heatAddedIntegrated' \\
      channel height & ch\textcolor{red}{CN}\_ax\textcolor{red}{NN}\_height \\
      channel pressure & ch\textcolor{red}{CN}\_ax\textcolor{red}{NN}\_pressure \\
      channel liquid velocity  & ch\textcolor{red}{CN}\_ax\textcolor{red}{NN}\_velocityLiquid \\
      channel vapor velocity  & ch\textcolor{red}{CN}\_ax\textcolor{red}{NN}\_velocityVapor \\
      channel entrainment rate (velocity) & ch\textcolor{red}{CN}\_ax\textcolor{red}{NN}\_velocityEntrain \\
      channel void fraction (liquid) & ch\textcolor{red}{CN}\_ax\textcolor{red}{NN}\_voidFractionLiquid \\
      channel void fraction (vapor)  & ch\textcolor{red}{CN}\_ax\textcolor{red}{NN}\_voidFractionVapor \\
      channel volume fraction of entrainment liquid  & ch\textcolor{red}{CN}\_ax\textcolor{red}{NN}\_volumeEntrainFraction \\
      channel mass flow rate (liquid) & ch\textcolor{red}{CN}\_ax\textcolor{red}{NN}\_massFlowRateLiquid \\
      channel mass flow rate (vapor)  & ch\textcolor{red}{CN}\_ax\textcolor{red}{NN}\_massFlowRateVapor \\
      channel entrainment rate (mass flow rate) & ch\textcolor{red}{CN}\_ax\textcolor{red}{NN}\_massFlowRateEntrain \\
      channel flow regime ID & ch\textcolor{red}{CN}\_ax\textcolor{red}{NN}\_flowRegimeID \\
      channel heat added to liquid & ch\textcolor{red}{CN}\_ax\textcolor{red}{NN}\_heatAddedToLiquid \\
      channel heat added to vapor  & ch\textcolor{red}{CN}\_ax\textcolor{red}{NN}\_heatAddedToVapor \\
      channel evaporation rate  & ch\textcolor{red}{CN}\_ax\textcolor{red}{NN}\_evaporationRate \\
      channel enthalpy of vapor  & ch\textcolor{red}{CN}\_ax\textcolor{red}{NN}\_enthalpyVapor \\
      channel enthalpy of saturated vapor & ch\textcolor{red}{CN}\_ax\textcolor{red}{NN}\_enthalpySaturatedVapor \\
      channel enthalpy difference between vapor and saturated vapor & ch\textcolor{red}{CN}\_ax\textcolor{red}{NN}\_enthalpyVapor-SaturatedVapor \\
      channel enthalpy of liquid & ch\textcolor{red}{CN}\_ax\textcolor{red}{NN}\_enthalpyLiquid \\
      channel enthalpy of saturated liquid & ch\textcolor{red}{CN}\_ax\textcolor{red}{NN}\_enthalpySaturatedLiquid \\
      channel enthalpy difference between liquid and saturated liquid & ch\textcolor{red}{CN}\_ax\textcolor{red}{NN}\_enthalpyLiquid-SaturatedLiquid \\
      channel enthalpy of mixture & ch\textcolor{red}{CN}\_ax\textcolor{red}{NN}\_enthalpyMixture \\
      channel density of liquid  & ch\textcolor{red}{CN}\_ax\textcolor{red}{NN}\_densityLiquid \\
      channel density of vapor  & ch\textcolor{red}{CN}\_ax\textcolor{red}{NN}\_densityVapor \\
      channel density of mixture & ch\textcolor{red}{CN}\_ax\textcolor{red}{NN}\_densityMixture \\
      channel net entrainment rate & \\ (difference between entrainment rate and de-entrainment rate)  & ch\textcolor{red}{CN}\_ax\textcolor{red}{NN}\_netEntrainRate \\
      % gas volumetric analysis
      channel enthalpy of the mixture of non-condensable gases & ch\textcolor{red}{CN}\_ax\textcolor{red}{NN}\_enthalpyNonCondensableMixture \\
      channel density of the mixture of non-condensable gases & ch\textcolor{red}{CN}\_ax\textcolor{red}{NN}\_densityNonCondensableMixture \\
      channel steam volume fraction [0-100] & ch\textcolor{red}{CN}\_ax\textcolor{red}{NN}\_volumeFractionSteam \\
      channel air volume fraction [0-100] & ch\textcolor{red}{CN}\_ax\textcolor{red}{NN}\_volumeFractionAir \\
      channel total equivalent diameter of the liquid droplets & \\ (all droplets as a single big one) (diam-ld) & ch\textcolor{red}{CN}\_ax\textcolor{red}{NN}\_equiDiameterLiquidDroplet \\
      channel averaged diameter of liquid droplets field (diam-sd) & ch\textcolor{red}{CN}\_ax\textcolor{red}{NN}\_avgDiameterLiquidDroplet \\
      channel averaged flow rate of liquid droplets field (flow-sd) & ch\textcolor{red}{CN}\_ax\textcolor{red}{NN}\_avgFlowRateLiquidDroplet \\
      channel averaged velocity of liquid droplets field (veloc-sd) & ch\textcolor{red}{CN}\_ax\textcolor{red}{NN}\_avgVelocityLiquidDroplet \\
      channel evaporation rate of liquid droplets field (gamsd) & ch\textcolor{red}{CN}\_ax\textcolor{red}{NN}\_evaporationRateLiquidDroplet \\
      % fuel rod
      fuel rod height & fuelRod\textcolor{red}{RN}\_surface\textcolor{red}{SN}\_ax\textcolor{red}{NN}\_height \\
      fuel rod fluid temperatures (liquid) & fuelRod\textcolor{red}{RN}\_surface\textcolor{red}{SN}\_ax\textcolor{red}{NN}\_fluidTemperatureLiquid \\
      fuel rod fluid temperatures (vapor)  & fuelRod\textcolor{red}{RN}\_surface\textcolor{red}{SN}\_ax\textcolor{red}{NN}\_fluidTemperatureVapor \\
      fuel rod surface heat flux & fuelRod\textcolor{red}{RN}\_surface\textcolor{red}{SN}\_ax\textcolor{red}{NN}\_surfaceHeatflux \\
      %fuel rod surface heat transfer mode [-] & fuelRod\textcolor{red}{(Rod Node Number)}_surfaceN\textcolor{red}{SN}_ax\textcolor{red}{NN}\_heatTransferMode \\
      clad outer surface temperature  & fuelRod\textcolor{red}{RN}\_surface\textcolor{red}{SN}\_ax\textcolor{red}{NN}\_cladOutTemperature \\
      clad iRNer surface temperature  & fuelRod\textcolor{red}{RN}\_surface\textcolor{red}{SN}\_ax\textcolor{red}{NN}\_cladInTemperature \\
      gap conductance  & fuelRod\textcolor{red}{RN}\_surface\textcolor{red}{SN}\_ax\textcolor{red}{NN}\_gapConductance \\
      fuel outer suface temperature  & fuelRod\textcolor{red}{RN}\_surface\textcolor{red}{SN}\_ax\textcolor{red}{NN}\_fuelTemperatureSurface \\
      fuel center temperature  & fuelRod\textcolor{red}{RN}\_surface\textcolor{red}{SN}\_ax\textcolor{red}{NN}\_fuelTemperatureCenter \\
      % cylindrical rod
      cylindrical tube height & cylRod\textcolor{red}{RN}\_surface\textcolor{red}{SN}\_ax\textcolor{red}{NN}\_height \\
      cylindrical tube outside surface heat flux & cylRod\textcolor{red}{RN}\_surface\textcolor{red}{SN}\_ax\textcolor{red}{NN}\_outsideSurfaceHeatFlux \\
      cylindrical tube outside surface wall temperature & cylRod\textcolor{red}{RN}\_surface\textcolor{red}{SN}\_ax\textcolor{red}{NN}\_outsideSurfaceWallTemperature \\
      cylindrical tube outside surface vapor temperature & cylRod\textcolor{red}{RN}\_surface\textcolor{red}{SN}\_ax\textcolor{red}{NN}\_outsideSurfaceVaporTemperature \\
      cylindrical tube outside surface liquid temperature & cylRod\textcolor{red}{RN}\_surface\textcolor{red}{SN}\_ax\textcolor{red}{NN}\_outsideSurfaceLiquidTemperature \\
      cylindrical tube inside surface wall temperature & cylRod\textcolor{red}{RN}\_surface\textcolor{red}{SN}\_ax\textcolor{red}{NN}\_insideSurfaceWallTemperature \\
      cylindrical tube inside surface vapor temperature & cylRod\textcolor{red}{RN}\_surface\textcolor{red}{SN}\_ax\textcolor{red}{NN}\_insideSurfaceVaporTemperature \\
      cylindrical tube inside surface liquid temperature & cylRod\textcolor{red}{RN}\_surface\textcolor{red}{SN}\_ax\textcolor{red}{NN}\_insideSurfaceLiquidTemperature \\
      cylindrical tube inside surface heat flux & cylRod\textcolor{red}{RN}\_surface\textcolor{red}{SN}\_ax\textcolor{red}{NN}\_insideSurfaceHeatFlux \\
      % heat slab
      heat slab (tube) height & heatSlab\textcolor{red}{HN}\_ax\textcolor{red}{NN}\_height \\
      heat slab (tube) outside surface heat flux & heatSlab\textcolor{red}{HN}\_ax\textcolor{red}{NN}\_outsideSurfaceHeatFlux\\
      heat slab (tube) outside surface wall temperature & heatSlab\textcolor{red}{HN}\_ax\textcolor{red}{NN}\_outsideSurfaceWallTemperature\\
      heat slab (tube) outside surface vapor temperature & heatSlab\textcolor{red}{HN}\_ax\textcolor{red}{NN}\_outsideSurfaceVaporTemperature\\
      heat slab (tube) outside surface liquid temperature & heatSlab\textcolor{red}{HN}\_ax\textcolor{red}{NN}\_outsideSurfaceLiquidTemperature\\
      heat slab (tube) inside surface wall temperature & heatSlab\textcolor{red}{HN}\_ax\textcolor{red}{NN}\_insideSurfaceWallTemperature\\
      heat slab (tube) inside surface vapor temperature & heatSlab\textcolor{red}{HN}\_ax\textcolor{red}{NN}\_insideSurfaceVaporTemperature\\
      heat slab (tube) inside surface liquid temperature & heatSlab\textcolor{red}{HN}\_ax\textcolor{red}{NN}\_insideSurfaceLiquidTemperature\\
      heat slab (tube) inside surface heat flux & heatSlab\textcolor{red}{HN}\_ax\textcolor{red}{NN}\_insideSurfaceHeatFlux
\end{longtable}
\endgroup

\textbf{\textit{\nb RAVEN, regonizes failed or crashed CTF runs and no data will be saved from those.}}

%%%%%%%%%%%%%%%%%%%%%%%%%%%%%%%%%%%%%%%%%%%%%%%%%%%%%%%%%
\subsubsection{Distributions}
%%%%%%%%%%%%%%%%%%%%%%%%%%%%%%%%%%%%%%%%%%%%%%%%%%%%%%%%%
The \xmlNode{Distribution} block defines the distributions that are going
to be used for the sampling of the variables defined in the \xmlNode{Samplers} block.
%
For all the possibile distributions and all their possible inputs please see the
chapter about Distributions (see~\ref{sec:distributions}).
%
Here we report an example of a Normal distribution:

\begin{lstlisting}[style=XML,morekeywords={name,debug}]
<Distributions verbosity='debug'>
    <Normal name="GridLossCoeff">
      <mean>0.7</mean>
      <sigma>0.1</sigma>
      <upperBound>0.9</upperBound>
      <lowerBound>0.6</lowerBound>
    </Normal>
    <Uniform name="DB1dist">
      <upperBound>0.026</upperBound>
      <lowerBound>0.020</lowerBound>
    </Uniform>
    <Uniform name="DB2dist">
      <upperBound>0.9</upperBound>
      <lowerBound>0.7</lowerBound>
    </Uniform>
    <Uniform name="DB3dist">
      <upperBound>0.5</upperBound>
      <lowerBound>0.3</lowerBound>
    </Uniform>
 </Distributions>
\end{lstlisting}

\noindent
It is good practice to name the distribution something similar to what kind of
variable is going to be sampled, since there might be many variables with the
same kind of distributions but different input parameters.

%%%%%%%%%%%%%%%%%%%%%%%%%%%%%%%%%%%%%%%%%%%%%%%%%%%%%%%%%
\subsubsection{Samplers}
%%%%%%%%%%%%%%%%%%%%%%%%%%%%%%%%%%%%%%%%%%%%%%%%%%%%%%%%%
In the \xmlNode{Samplers} block we want to define the variables that are going to be sampled.

\noindent The perturbation or optimization of the input of any CTF sequence is performed using the approach detailed in the \textit{Generic Interface} section (see \ref{subsec:genericInterface}). Briefly, this approach uses
 ``wild-cards'' (placed in the original input files) for injecting the perturbed values.
 For example, if the original input file (that needs to be perturbed) is the following:

\textbf{Example}:
We want to do the sampling of 1 single variable:
\begin{itemize}
  \item The Grid Loss Coefficient Data is used from sampled values.
\end{itemize}

\noindent We are going to sample this variable using two different sampling methods: Grid and MonteCarlo.
The RAVEN input is then written as follows:

\begin{lstlisting}[style=XML,morekeywords={name,type,construction,lowerBound,steps,limit,initialSeed}]
<Samplers verbosity='debug'>
  <Grid name='Grid_Sampler' >
    <variable name='GrdLss'>
      <distribution>GridLossCoeff</distribution>
      <grid type='CDF' construction='equal'  steps='10'>0.001 0.999</grid>
    </variable>
  </Grid>
  <MonteCarlo name='MC_Sampler'>
     <samplerInit>
       <limit>1000</limit>
     </samplerInit>
    <variable name='GrdLss'>
      <distribution>GridLossCoeff</distribution>
    </variable>
    <variable name='DB1'>
      <distribution>DB1dist</distribution>
    </variable>
    <variable name='DB2'>
      <distribution>DB2dist</distribution>
    </variable>
    <variable name='DB3'>
      <distribution>DB3dist</distribution>
    </variable>
  </MonteCarlo>
</Samplers>
\end{lstlisting}

CTF input file should be modified with wild-cards in the following way.
\begin{lstlisting}[basicstyle=\tiny]
***********************************************************************************************
*GROUP 7 - Grid Loss Coefficient Data                                                         *
***********************************************************************************************
**NGR
    7
*Card 7.1
**  NCD NGT  IFGQF IFSDRP  IFESPV  IFTPE  IGTEMP  NFBS  NDM9 NDM10 NDM11 NDM12 NDM13 NDM14
     21   0      0      0       0      0       0     0     0     0     0     0     0     0
*Card 7.2
**       CDL      J   CD1   CD2   CD3   CD4   CD5   CD6   CD7   CD8   CD9  CD10  CD11  CD12
 $RAVEN-GrdLss$   1     1     2     3     4     5     6     7     8     9    10    11    12
     0.90700      1    13    14    15    16    17    18    19    20    21    22    23    24
     0.90700      1    25    26    27    28    29    30    31    32    33    34    35    36

\end{lstlisting}

It is also possible to modify input values in parameter exposure input files.\\

\textbf{Example}:
We want to do the sampling of 3 correlation parameters (Dittus-Boelter parameter modification, DB1 $\times$ Re$^{DB2}$ $\times$ Pr$^{DB3}$):

\begin{itemize}
  \item DB1, DB2, DB3 values will be sampled and vuq\_param.txt will be modified with sampled values.
\end{itemize}

vuq$\_$param.txt and vuq$\_$mult.txt files are modified similarly with defined variable names.

\begin{lstlisting}[basicstyle=\tiny]
k_chen_1 = 0.24
k_chen_2 = 0.75
k_db_1 = $RAVEN-DB1$
k_db_2 = $RAVEN-DB2$
k_db_3 = $RAVEN-DB3$
k_db_4 = 7.86
k_wf_1 = 1.691
k_wf_2 = 0.43
\end{lstlisting}

\noindent It can be seen that each variable is connected with a proper distribution
defined in the \\\xmlNode{Distributions} block (from the previous example).

%%%%%%%%%%%%%%%%%%%%%%%%%%%%%%%%%%%%%%%%%%%%%%%%%%%%%%%%%%%
\subsubsection{Steps}
For a CTF interface, the \xmlNode{MultiRun} step type will most likely be
used. But \xmlNode{SingleRun} step can also be used for plotting and data extraction purposes.
%
First, the step needs to be named: this name will be one of the names used in
the \xmlNode{sequence} block.
%
In our example, \texttt{Grid\_Sampler} and \texttt{MC\_Sampler}.
%
\begin{lstlisting}[style=XML,morekeywords={name,debug,re-seeding}]
     <MultiRun name='Grid_Sampler' verbosity='debug'>
\end{lstlisting}

With this step, we need to import all the files needed for the simulation:
\begin{itemize}
  \item CTF input files
\end{itemize}
\begin{lstlisting}[style=XML,morekeywords={name,class,type}]
    <Input class="Files" type="ctf">CTFinput</Input>
    <Input class="Files" type="vuq_param">vuq_param</Input>
    <Input class="Files" type="vuq_mult" >vuq_mult</Input>
\end{lstlisting}
We then need to define which model will be used:
\begin{lstlisting}[style=XML]
    <Model  class='Models' type='Code'>MyCobraTF</Model>
\end{lstlisting}
We then need to specify which Sampler is used, and this can be done as follows:
\begin{lstlisting}[style=XML]
    <Sampler class='Samplers' type='Grid'>Grid_Sampler</Sampler>
\end{lstlisting}
And lastly, we need to specify what kind of output the user wants.
%
For example the user might want to make a database (in RAVEN the database
created is an HDF5 file).
%
Here is a classical example:
\begin{lstlisting}[style=XML,morekeywords={class,type}]
    <Output  class='Databases' type='HDF5'>Grid_out</Output>
\end{lstlisting}
Following is the example of two MultiRun steps which use different sampling
methods (Grid and Monte Carlo), and creating two different databases for each
one:
\begin{lstlisting}[style=XML]
<Steps verbosity='debug'>
  <MultiRun name='Grid_Sampler' verbosity='debug'>
    <Input   class='Files' type="ctf">CTFinput</Input>
    <Input   class="Files" type="vuq_param">vuq_param</Input>
    <Input   class="Files" type="vuq_mult" >vuq_mult</Input>
    <Model   class='Models'    type='Code'>MyCobraTF</Model>
    <Sampler class='Samplers'  type='Grid'>Grid_Sampler</Sampler>
    <Output  class='Databases' type='HDF5'>Grid_out</Output>
    <Output  class='DataObjects' type='PointSet'   >GridCTFPointSet</Output>
    <Output  class='DataObjects' type='HistorySet'>GridCTFHistorySet</Output>
  </MultiRun>
  <MultiRun name='MC_Sampler' verbosity='debug' re-seeding='210491'>
    <Input   class='Files' type=''>CTFinput</Input>
    <Input   class="Files" type="">vuq_param</Input>
    <Input   class="Files" type="" >vuq_mult</Input>
    <Model   class='Models'    type='Code'>MyCobraTF</Model>
    <Sampler class='Samplers'  type='MonteCarlo'>MC_Sampler</Sampler>
    <Output  class='Databases' type='HDF5'      >MC_out</Output>
    <Output  class='DataObjects' type='PointSet'   >MonteCarloCobraPointSet</Output>
    <Output  class='DataObjects' type='HistorySet'>MonteCarloCobraHistorySet</Output>
  </MultiRun>
</Steps>
\end{lstlisting}

%%%%%%%%%%%%%%%%%%%%%%%%%%%%%%%%%%%%%%%%%%%%%%%%%%%%%%
%%%%%%%%%%%%%%%%% Saphire Interface %%%%%%%%%%%%%%%%%%
%%%%%%%%%%%%%%%%%%%%%%%%%%%%%%%%%%%%%%%%%%%%%%%%%%%%%%
\subsection{SAPHIRE Interface}
\label{subsec:saphireInterface}
This section covers the input specification for running SAPHIRE through RAVEN. It is important to notice
that this short explanation assumes that the reader already knows how to use SAPHIRE.

\subsubsection{Files}
In the \xmlNode{Files} section, as specified before, all the files needed for the code to
run should be specified. In the case of SAPHIRE, the files typically needed are the following:
\begin{itemize}
  \item SAPHIRE compressed project inputs with file extension `.zip';
  \item SAPHIRE macro input file with file extension `.mac'.
\end{itemize}

Example:
\begin{lstlisting}[style=XML]
  <Files>
    <Input name="macro" type="">changeSet.mac</Input>
    <Input name="saphireInput" type="">saphireInput.zip</Input>
  </Files>
\end{lstlisting}

%%%%%%%%%%%%%%%%%%%%%%%%%%%%%%%%%%%%%%%%%%%%%%%%%%%%%X
\subsubsection{Models}
In the \xmlNode{Models} block SAPHIRE executable needs to be specified. Here is a standard example of what
can be used:
\begin{lstlisting}[style=XML]
  <Models>
    <Code name="saphire" subType="Saphire">
      <executable>"C:\Saphire 8\tools\SAPHIRE.exe"</executable>
      <clargs arg="macro" extension=".mac" type="input" delimiter="="/>
      <clargs arg="project" extension=".zip" type="input" delimiter="="/>
      <outputFile>fixed_output.csv</outputFile>
      <codeOutput type="uncertainty">et_uq.csv</codeOutput>
      <codeOutput type="uncertainty">ft_uq.csv</codeOutput>
    </Code>
  </Models>
\end{lstlisting}

The \xmlNode{Code} XML node contains the information needed to execute the specific External Code. This
XML node accepts the following attributes:
\begin{itemize}
  \item \xmlAttr{name}, \xmlDesc{required string attribute}, user-defined identifier of this model.
    \nb As with other objects, this identifier can be used to reference this specific entity from other input
    blocks in the XML.
  \item \xmlAttr{subType}, \xmlDesc{required string attribute}, specifies the code that needs to be
    associated to this Model.
\end{itemize}
This model can be initialized with the following children:
\begin{itemize}
  \item \xmlNode{executable}, \xmlDesc{string, required field}, specifies the path of the executable to
    be used; \nb Either an absolute or relative path can be used.
  \item \xmlNode{clargs}, \xmlDesc{string, required field}, allows addition of command-line arguments to
    the execution command. This node is used to specify the input files that are required by SAPHIRE.
    This node accepts the following attributes:
    \begin{itemize}
      \item \xmlAttr{type}, \xmlDesc{require string attribute}, specifies the type of command-line argument
        to add. The current option is \xmlString{input}
      \item \xmlAttr{arg}, \xmlDesc{string, required field} specifies the flag to be used before the entry.
      \item \xmlAttr{extension}, \xmlDesc{string, required field}, specifies the type of file extension
        to use. This links the \xmlNode{Input} file in the \xmlNode{Steps} to this location in the execution
        command. Currently only accepts `.zip' and `.mac'.
      \item \xmlAttr{delimiter}, \xmlDesc{string, required field}, uses to link the \xmlAttr{arg} and the
        \xmlNode{Input} with the extension given by \xmlAttr{extension}
    \end{itemize}
    \nb As shown in previous example, the following command will be generated:
    \begin{lstlisting}
      "C:\Saphire 8\tools\SAPHIRE.exe" project=path/to/saphireInput.zip macro=path/to/changeSet.mac
    \end{lstlisting}
  \item \xmlNode{outputFile}, \xmlDesc{string, optional field}, uses to specify the output file name (CSV only). In this case, the code
    interface always produce a CSV file named ``fixed\_output.csv''.
  \item \xmlNode{codeOutput}, \xmlDesc{string, required field}, uses to specify output file generated by SAPHIRE that will be processed
    via the code interface. The following attributes can be specified:
    \begin{itemize}
      \item \xmlAttr{type}, \xmlDesc{required string attribute}, the actual type of the provided file. The
        only type accepted here is \xmlString{uncertainty}
    \end{itemize}
\end{itemize}

In this example, two output files ``eq\_uq.csv'' amd ``ft\_uq.csv'' will be processed by the SAPHIRE code
interface, and the results will be saved in output file with name ``fixed\_output.csv''.

%%%%%%%%%%%%%%%%%%%%%%%%%%%%%%%%%%%%%%%%%%%%%%%%%%%%%%
\subsubsection{Distributions}
The \xmlNode{Distributions} block defines the distributions that are going to be used for the sampling
of the variables defined in the \xmlNode{Samplers} block. For all the possible distributions and all
their possible inputs, please see the chapter about Distributions (see~\ref{sec:distributions}).
%
Here we give a general example:
\begin{lstlisting}[style=XML]
  <Distributions>
    <Normal name="allEvents">
        <mean>0.1</mean>
        <sigma>0.025</sigma>
        <lowerBound>0.05</lowerBound>
        <upperBound>0.15</upperBound>
    </Normal>
    <Normal name="mov1Event">
        <mean>0.5</mean>
        <sigma>0.1</sigma>
        <lowerBound>0.3</lowerBound>
        <upperBound>0.8</upperBound>
    </Normal>
    <Normal name="single1">
        <mean>0.2</mean>
        <sigma>0.05</sigma>
        <lowerBound>0.1</lowerBound>
        <upperBound>0.3</upperBound>
    </Normal>
  </Distributions>
\end{lstlisting}

It is good practice to name the distribution similar to what kind of variable is going to be sampled.
%
%%%%%%%%%%%%%%%%%%%%%%%%%%%%%%%%%%%%%%%%%%%%%%%%%%%%%%
\subsubsection{Samplers}
In the \xmlNode{Samplers} block we want to define the variables that are going to be sampled.
The perturbation of the input of SAPHIRE MACRO is performed using the approach detailed in the
\textit{Generic Interface} section (see \ref{subsec:genericInterface}). This approach uses the
``wild-cards'' (placed in the original input files) for injecting the perturbed values. For example,
if one wants to perturb the event tree probabilities of the original input file, i.e.
\begin{lstlisting}[style=XML]
<change set>
  <unmark></unmark>
  <delete>
    <name>ALL-EVENTS</name>
  </delete>
  <add>
    <name>ALL-EVENTS</name>
    <description>Class change all events Change Set</description>
    <class>
      <event name>*</event name>
      <suscept>1</suscept>
      <probability>1.E-2</probability>
    </class>
  </add>
  <mark name>ALL-EVENTS</mark name>
  <generate></generate>
</change set>
...
<change set>
  <unmark></unmark>
  <delete>
    <name>MOV-1-EVENTS</name>
  </delete>
  <add>
    <name>MOV-1-EVENTS</name>
    <description>Class change subset events Change Set</description>
    <class>
      <event name>?-MOV-CC-1</event name>
      <calc type>1</calc type>
      <probability>5E-3</probability>
    </class>
  </add>
    <mark name>MOV-1-EVENTS</mark name>
    <generate></generate>
</change set>
...

<change set>
  <unmark></unmark>
  <delete>
    <name>SINGLE-1</name>
  </delete>
  <add>
    <name>SINGLE-1</name>
    <description>Single Event Change Set</description>
    <single>
      <event name>E-MOV-CC-A</event name>
      <calc type>1</calc type>
      <probability>4E-1</probability>
    </single>
  </add>
  <mark name>SINGLE-1</mark name>
  <generate></generate>
</change set>
\end{lstlisting}

One need to use the RAVEN ``wild-cards`` to inject the perturbed values, i.e.
\begin{lstlisting}[style=XML]
<change set>
  <unmark></unmark>
  <delete>
    <name>ALL-EVENTS</name>
  </delete>
  <add>
    <name>ALL-EVENTS</name>
    <description>Class change all events Change Set</description>
    <class>
      <event name>*</event name>
      <suscept>1</suscept>
      <probability>$RAVEN-allEventsPb$</probability>
    </class>
  </add>
  <mark name>ALL-EVENTS</mark name>
  <generate></generate>
</change set>
...
<change set>
  <unmark></unmark>
  <delete>
    <name>MOV-1-EVENTS</name>
  </delete>
  <add>
    <name>MOV-1-EVENTS</name>
    <description>Class change subset events Change Set</description>
    <class>
      <event name>?-MOV-CC-1</event name>
      <calc type>1</calc type>
      <probability>$RAVEN-mov1EventPb$</probability>
    </class>
  </add>
    <mark name>MOV-1-EVENTS</mark name>
    <generate></generate>
</change set>
...

<change set>
  <unmark></unmark>
  <delete>
    <name>SINGLE-1</name>
  </delete>
  <add>
    <name>SINGLE-1</name>
    <description>Single Event Change Set</description>
    <single>
      <event name>E-MOV-CC-A</event name>
      <calc type>1</calc type>
      <probability>$RAVEN-single1Pb$</probability>
    </single>
  </add>
  <mark name>SINGLE-1</mark name>
  <generate></generate>
</change set>
\end{lstlisting}

The RAVEN \xmlNode{Samplers} input will be

\textbf{Example}:
\begin{lstlisting}[style=XML]
  <Samplers>
    <MonteCarlo name="mcSaphire">
        <samplerInit>
            <limit>2</limit>
        </samplerInit>
        <variable name="allEventsPb">
            <distribution>allEvents</distribution>
        </variable>
        <variable name="mov1EventPb">
            <distribution>mov1Event</distribution>
        </variable>
        <variable name="single1Pb">
            <distribution>single1</distribution>
        </variable>
    </MonteCarlo>
\end{lstlisting}

%%%%%%%%%%%%%%%%%%%%%%%%%%%%%%%%%%%%%%%%%%%%%%%%%%%%%%
\subsubsection{Steps}
In this section, the \xmlNode{MultiRun} will be used. As shown in the following, two SAPHIRE
input files listed in \xmlNode{Files} are linked here using \xmlNode{Input}, the \xmlNode{Model}
and \xmlNode{Sampler} defined in previous sections will be used in this \xmlNode{MultiRun}. The
outputs will be saved in the \textbf{DataObject} ``saphireDump'', and will be printed via
\textbf{OutStreams}.

\begin{lstlisting}[style=XML]
  <Steps>
    <MultiRun name="sample">
      <Input class="Files" type="">macro</Input>
      <Input class="Files" type="">saphireInput</Input>
      <Model class="Models" type="Code">saphire</Model>
      <Sampler class="Samplers" type="MonteCarlo">mcSaphire</Sampler>
      <Output class="DataObjects" type="PointSet">saphireDump</Output>
      <Output class="OutStreams" type="Print">saphirePrint</Output>
    </MultiRun>
  </Steps>
\end{lstlisting}
%%%%%%%%%%%%%%%%%%%%%%%%%%%%%%%%%%%%%%%%%%%%%%%%%
%%%%%%%%%%%%% PHISICS INTERFACE %%%%%%%%%%%%%%%%%
%%%%%%%%%%%%%%%%%%%%%%%%%%%%%%%%%%%%%%%%%%%%%%%%%
\subsection{PHISICS Interface}
\label{subsec:PhisicsInterface}
%
\subsubsection{General Information}
%
This section covers the input specification for running PHISICS through RAVEN.
The interface can be used to perturb the PHISICS input files including the INSTANT and MRTAU input decks
and the following libraries: cross sections, fission yield, decay, fission Q-values, decay Q-values and the XML material input.
The interface also the cabablity to work in MRTAU standlone calculations, in INSTANT/MRTAU mode (PHISICS) and
in PHISICS/RELAP5 coupled mode (see~\ref{subsec:PhisicsRelap5Interface}).
%
\subsubsection{Files}
\label{subsubsec:PhisicsInterfaceFiles}
\xmlNode{Files} includes two attributes \xmlAttr{name} and \xmlAttr{type} entries, identically as other interfaces.
It also includes two optional attributes \xmlAttr{perturbable} and \xmlAttr{subDirectory}.

The \xmlAttr{name} attribute is a user-defined internal name for the file contained in the node.
Default: None (required entry).

The \xmlAttr{type} attribute identifies which base-level parser an input file is used within.  The \xmlAttr{type} has to be specified as long as the
file is parsed by the interface or an interface's parser. \xmlAttr{type} is hardcoded for this speciific inputs, in order to assign each input to its corresponding RAVEN parser.
Default: None (required entry if parsed).

The corresponding hardcoded flags accepted by RAVEN are given in Table~\ref{TypeTable}. The type attributes are case-incensitive.
\begin{table}[]
  \centering
  \caption{Correspondance between the type attributes required and the PHISICS input files}\label{TypeTable}
    \begin{tabular}{l|l|l}
\textit{Type Attribute} & \textit{Corresponding PHISICS input}                           & \textit{Perturbable} \\
\hline
decay                   & MrTau decay library                          & Yes                  \\
inp                     & XML Instant input                            & No                   \\
path                    & XML MrTau path-to-libraries file             & No                   \\
material                & XML MrTau material input                     & Yes                  \\
depletion\_input        & XML MrTau depletion input                    & No                   \\
Xs-Library              & XML MrTau library input                      & No                   \\
FissionYield            & MrTau fission yield library                  & Yes                  \\
FissQValue              & MrTau fission Q-values                       & Yes                  \\
AlphaDecay              & MrTau $\alpha$ decay library                 & Yes                  \\
Beta+Decay              & MrTau $\beta^+$ decay library                & Yes                  \\
Beta+xDecay             & MrTau $\beta^{+*}$ decay library             & Yes                  \\
BetaDecay               & MrTau $\beta$ decay library                  & Yes                  \\
BetaxDecay              & MrTau $\beta^*$ decay library                & Yes                  \\
IntTraDecay             & MrTau internal transition decay library      & Yes                  \\
XS                      & XML XS scaling factors file                  & Yes                  \\
N,2N                    & MrTau n,2n library                           & No                   \\
N,ALPHA                 & MrTau n,$\alpha$ library                     & No                   \\
N,G                     & MrTau n,$\gamma$ library                     & No                   \\
N,Gx                    & MrTau n,$\gamma^*$ library                   & No                   \\
N,P                     & MrTau n,proton library                       & No                   \\
budep                   & MrTau burn-up history                        & No                   \\
CRAM\_coeff\_PF         & MrTau CRAM coefficients                      & No                   \\
IsotopeList             & MrTau isotope list input                     & No                   \\
mass                    & MrTau mass input                             & No                   \\
tabMap             & XML tabulation mapping                       & No
    \end{tabular}
\end{table}

The cross section libraries files can be defined with any \xmlAttr{type} attribute.
The tabulation mapping file is optional. If a \xmlAttr{type}=\xmlString{tabMap} is found in an \xmlNode{Input} node, the cross section
parser will be based on the tabulation points provided in the tabulation mapping file. If no tabulation mapping file is provided,
the cross section parser will print the perturbed cross section in one single tabulation point.
The \xmlAttr{perturbable} attribute indicates whether the input file can be perturbed. It is an optional boolean attribute.
Default: False

The \xmlAttr{subDirectory} indicates the subdirectory to which RAVEN search an input file. It is an optional attribute.
Default: . / (relative path of the working directory)

The \xmlString{Input File} string is the user-defined input file name.
The XML file specifying the library input paths corresponding to the decay, fission yields, the Q-values and the Mrtau standalone inputs will be automatically populated according to the user-input file names.
The Instant-MrTau input files material, library, instant control, library path and depletion are also user-defined in the input section. The user does not have to use the default file names.
Example:
\begin{lstlisting}[style=XML]
<File>
  <Input name="path"  type="path"  perturbable="False"                    >pathMrTau.xml</Input>
  <Input name="dec"   type="decay" perturbable="True" subDirectory="libF" >decayLibrary.dat</Input>
  <Input name="input" type="inp"   perturbable="False"                    >inpInstant.xml</Input>
</File>
\end{lstlisting}
in the example, the path-to-MrTau-libraries is pointed by the \xmlAttr{type}="path". The file name associated to the path-to-MrTau-libraries file is then user-defined as \xmlString{pathMrTau.xml}.
It is located in the working directory and it cannot be perturbed.

The decay library is pointed by the \xmlAttr{type}="decay". The file name associated to the decay library is then user-defined as \xmlString{decayLibrary.dat}.
The decay library is located at the relative path "libF". This path and name will be populated in the \xmlString{pathMrTau.xml} file automatically. The decay library can be perturbed.

The Instant XML input is pointed by the \xmlAttr{type}="inp". The file name associated to the Instant input is then user-defined as \xmlString{inpInstant.xml}.
it is located in the working directory and it cannot be perturbed.
%
%%%%%%%%%%%%%%%%%%%%%%%%%%%%%%%%%%%%%%%%%%%%%%%%%%
\subsubsection{Models}
\label{subsubsection:PhisicsModel}
The user provides paths to executables for sampled variables within the \xmlNode{Models} block.

The \xmlNode{Code} block will contain attributes \xmlAttr{name} and
\xmlAttr{subType}. The \xmlAttr{name} identifies that particular \xmlNode{Code} model within RAVEN, and
\xmlAttr{subType} specifies which code interface the model will use.

The \xmlNode{executable} block contains the absolute or relative path (with respect to the current working directory)
to PHISICS that RAVEN will use to run generated input files.

The \xmlNode{mrtauStandAlone} node informs whether or not MrTau is ran in standalone mode. The
\xmlNode{mrtauStandAlone} accepts only a boolean entry (\xmlString{true}, \xmlString{t}, \xmlString{false}
,\xmlString{f}). It is case insensitive.
Default: false.
If \xmlNode{mrtauStandAlone} is false, a coupled INSTANT+MrTau calculation is ran, using the Phisics executable.

The \xmlNode{printSpatialRR} node indicates if the spatial reaction rates computed by PHISICS are to be included
in the RAVEN csv output. The entry is case insensitive.
. If False, the total reaction rates are printed instead. Default: False (spatial reaction rate not printed).

The \xmlNode{printSpatialFlux} node indicates if the spatial neutron fluxes computed by PHISICS are to be included in the RAVEN csv output. The entry is case insensitive. Default: False (spatial fluxes not printed).

An example of the \xmlNode{Models} block is given below:

\begin{lstlisting}[style=XML]
  <Models>
    <Code name="PHISICS" subType="Phisics">
       <executable>./path/to/instant/executable</executable>
       <mrtauStandAlone>F</mrtauStandAlone>
       <printSpatialRR>F</printSpatialRR>
       <printSpatialFlux>T</printSpatialFlux>
    </Code>
  </Models>
\end{lstlisting}
In the example, note that because \xmlNode{mrtauStandAlone} is false.
If  \xmlNode{mrtauStandAlone} is changed to \xmlString{true}, the path \xmlString{/path/to/mrtau/executable} will be read to get the MrTau executable.
In the example, RAVEN is used in PHISICS standalone mode. Spatial neutron fluxes are printed a and no spatial reaction rates are printed in the RAVEN output.

%%%%%%%%%%%%%%%%%%%%%%%%%%%%%%%%%%%%%%%%%%%%%%%%%%
\subsubsection{Distributions}
The \xmlNode{Distributions} block defines all distributions used to
sample variables in the current RAVEN run.

For all the possible distributions and their possible inputs please
refer to the Distributions chapter (see~\ref{sec:distributions}).
%
It is good practice to name a distribution and its corresponding sampled variable with identical root names, and appending sufix to the distribution name, since there might be many variables with the
same kind of distributions but different input parameters.

%%%%%%%%%%%%%%%%%%%%%%%%%%%%%%%%%%%%%%%%%%%%%%%%%%
\subsubsection{Samplers}
\label{subsubsection:SamplersPhisics}
The \xmlNode{Samplers} block defines the variables to be sampled.
After defining a sampling scheme, the variables to be sampled and
their distributions are identified in the \xmlNode{variable} blocks.
The \xmlAttr{name} must be formatted according to library which the variable belongs to.
The description of the \xmlString{variable} template is detailed in the next sub-sections for the decay
 constants (\ref{decayPara}), the fission yields (\ref{FYPara}), the number densities (\ref{NDPara}),
the fission Q-values (\ref{FQPara}), the $\alpha$ decay Q-values (\ref{AlphaPara}), the $\beta^{+}$ Q-values
 (\ref{BetaPlusPara}), $\beta^{+*}$ Q-values (\ref{BetaPlusStarPara}), $\beta$ Q-values (\ref{BetaPara}),
$\beta^{*}$ Q-values (\ref{BetaStarPara}), the internal transition decay Q-values (\ref{IntTraPara}) and the cross
section scaling factors (\ref{XSPara}).

\paragraph{Decay constant variable} \label{decayPara}

The \xmlString{variable} template is: DECAY$\vert$TYPE\_OF\_DECAY$\vert$ISOTOPE.
The type of decay (TYPE\_OF\_DECAY) is the decay mode relative to the isotope's decay constant perturbed.
The type of decay depends on the isotope perturbed.
If the isotope is an actinide, the available decay modes are:
\begin{enumerate}
  \item [$-$]BETA;
  \item [$-$]BETA+;
  \item [$-$]ALPHA.
\end{enumerate}
If the isotope is a fission product, the available decay modes are:
\begin{enumerate}
  \item [$-$]BETA;
  \item [$-$]BETA*;
  \item [$-$]BETA+;
  \item [$-$]BETA+*;
  \item [$-$]ALPHA;
  \item [$-$]INTER\_TRAN.
\end{enumerate}

The decay types are immediately parsed from the MRTAU decay library. Hence, If the user modifies the decay labels
 in the decay library,
the user will have to modify the her/his decay type in the RAVEN input. The isotope defined in the variable has
 to originally exist in the decay library.

\paragraph{Fission yield variable} \label{FYPara}
The \xmlString{variable} template is FY$\vert$SPECTRUM$\vert$FISSION\_ISOTOPE$\vert$FISSION\_PRODUCT.

The types of spectrum (SPECTRUM) available are: FAST; THERMAL.
The fission isotopes (FISSION\_ISOTOPE) and fission products (FISSION\_PRODUCT) in the variable have to
originally exist in the fission yield library.

\paragraph{Number density variable} \label{NDPara}
The \xmlString{variable} template is DENSITY$\vert$MATERIAL\_ID$\vert$ISOTOPE.
The Material ID (MATERIAL\_ID) has to originally exist in the material XML input. The isotope in the variable has to
be originally defined within the material ID aforementioned.

\paragraph{Fission Q-values variables} \label{FQPara}
The \xmlString{variable} template is QVALUES$\vert$ ISOTOPE.

The isotope (ISOTOPE) in the variable has to originally exist in the fission Q-values library.

\paragraph{$\alpha$ decay variable} \label{AlphaPara}
The \xmlString{variable} template is ALPHADECAY$\vert$ISOTOPE.

The isotope (ISOTOPE) in the variable has to originally exist in the $\alpha$ decay Q-values library.

\paragraph{$\beta^{+}$ decay variable} \label{BetaPlusPara}
The \xmlString{variable} template is BETA+DECAY$\vert$ ISOTOPE.

The isotope (ISOTOPE) in the variable has to originally exist in the $\beta^+$ decay Q-values library.

\paragraph{$\beta^{+*}$ decay variable} \label{BetaPlusStarPara}
The \xmlString{variable} template is BETA+XDECAY$\vert$ISOTOPE.

The isotope (ISOTOPE) in the variable has to originally exist in the $\beta^{+*}$ decay Q-values library.

\paragraph{$\beta$ decay variable} \label{BetaPara}
The \xmlString{variable} template is BETADECAY$\vert$ISOTOPE.

The isotope (ISOTOPE) in the variable has to originally exist in the $\beta$ decay Q-values library.

\paragraph{$\beta^{*}$ decay variable} \label{BetaStarPara}
The \xmlString{variable} template is BETAXDECAY$\vert$ISOTOPE.

The isotope (ISOTOPE) in the variable has to originally exist in the $\beta^*$ decay Q-values library.

\paragraph{Internal transition decay variable} \label{IntTraPara}
The \xmlString{variable} template is INTTRADECAY$\vert$ISOTOPE.

The isotope (ISOTOPE) in the variable has originally to exist in the internal transition decay Q-value library.

\paragraph{Cross section scaling factors} \label{XSPara}
The \xmlString{variable} template is:

XS$\vert$TABULATION\_POINT$\vert$MATERIAL\_ID$\vert$ISOTOPE$\vert$OPERATOR$\vert$XS\_TYPE$\vert$GROUP\_NUMBER.
\begin{enumerate}
  \item [$\cdot$]The tabulation point (TABULATION\_POINT) is the integer referrencing to the tabulation point.
  The tabulation numbering is given by the XML input file 'tabMapping.xml' (Section \ref{additionalInput}).
   If there are no tabulations, the tabulation number has to be 1.
  \item [$\cdot$]The material ID (MATERIAL\_ID) is the string referring to the material in which the isotope is defined.
  The Material ID  has to originally exist in the material XML file.
   \item [$\cdot$]The isotope (ISOTOPE) is the ISOTOPE that the user desires to perturbed. The isotope perturbed
   has to originally exist in the material ID.
   \item [$\cdot$]The operator (OPERATOR) determines how the cross section is perturbed from its nominal value.
   The operators available are:
   \begin{enumerate}
      \item[$-$]ADDITIVE; The additive operator adds the user-defined value to the nominal cross section value.
      \item[$-$]MULTIPLIER; the multiplier operator multiplies the user-defined factor to the nominal value.
      \item[$-$]ABSOLUTE; the absolute operator replaces the nominal value by the user-defined value.
  \end{enumerate}
  \item [$\cdot$]The types of cross sections (CROSS\_SECTION\_TYPE) available are:
  \begin{enumerate}
     \item[$-$]FISSIONXS; Fission cross section.
     \item[$-$]NPXS; Neutron to proton capture cross section.
     \item[$-$]NGXS; Neutron to gamma capture cross section.
     \item[$-$]NUFISSIONXS; $\nu$*fission cross section. The $\nu$*fission is coordinated with the
        fission cross section so that only the coefficient $\nu$ is perturbed.
      \item[$-$]SCATTERINGXS. Total scattering cross section.
      \item[$-$]N2NXS. n,2n cross section.
      \item[$-$]NALPHAXS. Neutron to alpha capture cross section.
      \item[$-$]KAPPAXS. Kappa coefficient.
   \end{enumerate}
   \item The group number (GROUP\_NUMBER) is the group number of the cross section perturbed.
    The group number is an integer and has to be inferior or equal to the number of groups used in the cross section library.
\end{enumerate}
An example is given for each type of variable in Table \ref{VariableExampleTable}.

\begin{table}[]
\centering
\caption{Examples of possible variable names}\label{VariableExampleTable}
\begin{tabular}{l|l|l}
\textit{Input file perturbed} & \textit{Variable perturbed}  & \textit{Example}  \\
\hline
Decay lib.               & Decay constant              & DECAY$\vert$BETAX$\vert$SE78     \\
Fission yield lib.       & Fission yield               & FY$\vert$FAST$\vert$U235$\vert$NB93       \\
XML Material             & Number densitiy             & DENSITY$\vert$FUEL1$\vert$PU239     \\
Fission Q-values lib.    & Fission Q-value              & QVALUES$\vert$U235            \\
$\alpha$ decay lib.      & $\alpha$ decay Q-value       & ALPHADECAY$\vert$U234         \\
$\beta^{+}$ decay lib.     & $\beta^{+}$ decay Q-value      & BETA+DECAY$\vert$U235         \\
$\beta^{+*}$ decay lib.  & $\beta^{+*}$ decay Q-value   & BETA+XDECAY$\vert$U236        \\
$\beta$ decay lib.       & $\beta$ decay Q-value       & BETADECAY$\vert$CM242         \\
$\beta^{*}$ decay lib.     & $\beta^{*}$ decay Q-value      & BETAXDECAY$\vert$PU240        \\
Internal transition lib. & Internal transition Q-value  & INTTRADECAY$\vert$U238        \\
XS scaling factors       & XS scaling factor           & XS$\vert$1$\vert$FUEL1$\vert$U238$\vert$ABSOLUTE$\vert$N2NXS$\vert$4 \\
\end{tabular}
\end{table}

The following example is a Monte Carlo-based sampler with two variables. The $\beta$ decay constant of
 uranium $^{235}$U and the $^{135}$Xe n,p cross section in group 12 at tabulation point 1 within the material
  ID "F1" are to be perturbed. The absolute operator is chosen for the n,p cross section, which means the
  nominal value will be replaced by the averaged value defined in the distribution block, with its corresponding distribution.

\begin{lstlisting}[style=XML]
  <Samplers>
    <MonteCarlo name="MC_samp">
      <samplerInit>
        <limit>100</limit>
      </samplerInit>
      <variable name="DECAY|BETA|U235">
        <distribution>DECAY|BETA|U235_dis</distribution>
      </variable>
      <variable name="XS|1|FUEL1|XE135|ABSOLUTE|NPXS|12">
        <distribution>XS|1|FUEL1|XE135|ABSOLUTE|NPXS|12_dis                   </distribution>
      </variable>
    </MonteCarlo>
  </Samplers>
\end{lstlisting}

%%%%%%%%%%%%%%%%%%%%%%%%%%%%%%%%%%%%%%%%%%%%%%%%%
\subsubsection{Steps}
For a PHISICS interface, the \xmlNode{MultiRun} step type will most likely be used. First, the step needs
to be named: this name will be one of the names used in the \xmlNode{Sequence} block.
In the example, the step is called \xmlString{testDummyStep}.
%
\begin{lstlisting}[style=XML]
<Steps>
  <MultiRun name='testDummyStep' verbosity='debug'>
    <Input   class='Files' type='decay'>decay.dat</Input>
    <Input   class='Files' type='inp'>inp.xml</Input>
    <Input   class='Files' type='XS'>xs.xml</Input>
    <Model   class='Models' type='Code'>PHISICS</Model>
    <Sampler class='Samplers' type='MonteCarlo'>MC_samp</Sampler>
    <Output  class='Databases' type='HDF5'>DataB_REL5_1</Ouput>
  </MultiRun>
</Steps>
\end{lstlisting}
%
%%%%%%%%%%%%%%%%%%%%%%%%%%%%%%%%%%%%%%%%%%%%%%%%%
\subsubsection{Additional Input}
\label{subsubsection:PhisicsAdditionalInput}
In addition to the usual PHISICS inputs required (INSTANT input, depletion input, material
 input, library input, path input) and the regular MrTau libraries, additional inputs may be required, depending on the user's needs.
\begin{enumerate}
\item [$\cdot$]A file 'tabMapping.xml' is (optional) maps the cross section tabulation points for interpolation purposes.
The tabulation mapping assigns an integer to a given tabulation in order to identify it in the RAVEN variable definition.
The format of the tabulation mapping is the following:
\begin{lstlisting}[style=XML]
<mapping>
  <tabulation set="1">
    <tab name="mod_temperature">559.0</tab>
    <tab name="BURN-UP">0.0</tab>
  </tabulation>
  <tabulation set="2">
    <tab name="mod_temperature">1000</tab>
    <tab name="BURN-UP">0.0</tab>
  </tabulation>
  <tabulation set="3">
    <tab name="mod_temperature">1000</tab>
    <tab name="BURN-UP">100</tab>
  </tabulation>
</mapping>
\end{lstlisting}
In \xmlNode{tabulation}, the \xmlAttr{set} refers to the user-defined number assigned to a given tabulation point.
This \xmlAttr{set} number corresponds to the second argument of the cross section scaling factor variable.
The user enters the tabulation parameters and corresponding tabulation values with the \xmlNode{tabulation},
 using the sub-node \xmlNode{tab}.

\item [$\cdot$]If the user perturbs cross sections, an XML file 'scaled\_xs.xml' will be generated at each perturbation
 in the output folder.
The file 'scaled\_xs.xml' is automatically created from the cross section variables defined in the RAVEN
\xmlNode{Sampler} block (see \ref{XSPara}). The cross section file 'scaled\_xs.xml' has the following format:

\begin{lstlisting}[style=XML]
<scaling_library>
  <tabulation>
    <tab name="mod_temperature">559.0</tab>
    <tab name="BURN-UP">0.0</tab>
    <library lib_name="fuel1" >
      <isotope id="xe135" type="absolute">
        <npxs g="8,3,12">8.889E+02,3.333E+02,1.212E+05</npxs>
      </isotope>
      <isotope id="u235" type="multiplier">
        <fissionxs g="1">2.019E-02</fissionxs>
      </isotope>
    </library>
  </tabulation>
<scaling_library>
\end{lstlisting}
The tabulation points \xmlNode{tab} are optional have to agree with the tabulation points defined in the
 ``tabMapping.xml'' if they are provided.
The \xmlNode{library} has one required attribute \xmlAttr{lib\_name} corresponding to one of the libraries
listed in the PHISICS library input.
The \xmlNode{isotope} provides the information related to an isotope included in the library aforementioned.
The \xmlAttr{id} gives the isotope ID (no dash allowed).
The \xmlAttr{type} specifies the type of operator used (\xmlString{additive}, \xmlString{multiplier} or
 \xmlString{absolute}).
The sub-node \xmlNode{XS} (where 'XS' is the perturb-able type of cross sections listed in section~\ref{XSPara})
 provides the cross section information.
The \xmlAttr{g} attribute refers to the group numbers to be perturbed, separated by commas.
The \xmlString{XS} provides the scaling factors or the new cross section values.
\end{enumerate}
%
%%%%%%%%%%%%%%%%%%%%%%%%%%%%%%%%%%%%%%%%%%%%%%%%%
\subsubsection{Output Files Conversion}
\label{subsubsection:PhisicsOutFileConv}

The PHISICS output available for RAVEN post-processing are described in this section. The PHISICS outputs
 are by convention separated by '$\vert$' if they are contained in a matrix form such as group-wise or region-wise values.
In the PHISICS mode (i.e. \xmlNode{mrtauStandalone} is \xmlString{False}) The variables available for RAVEN
 post-processing are:
\begin{enumerate}
  \item [$-$]the MrTau time;
  \item [$-$]the multiplication factor;
  \item [$-$]the multiplication factor error;
  \item [$-$]the spatial reaction rates (only if \xmlNode{printSpatialRR} is \xmlString{True});
  \item [$-$]the spatial power (only if \xmlNode{printSpatialRR} is \xmlString{True});
  \item [$-$]the spatial fluxes by region (only if \xmlNode{printSpatialRR} is \xmlString{True});
  \item [$-$]the neutron fluxes by cell (only if \xmlNode{printSpatialFlux} is \xmlString{True});
  \item [$-$]the neutron fluxes by material (only if \xmlNode{printSpatialFlux} is \xmlString{True});
  \item [$-$]the total reaction rates (only if \xmlNode{printSpatialRR} is \xmlString{False});
  \item [$-$]the decay heat (only if decay heat flag is on in the PHISICS input);
  \item [$-$]the burnup;
  \item [$-$]the cross section values (only for perturbed cross sections);
  \item [$-$]the PHISICS cpu time.
\end{enumerate}

In the MrTau standalone mode (i.e \xmlNode{mrtauStandalone} is \xmlString{True}) The variables available for RAVEN post-processing are:
\begin{enumerate}
  \item [$-$]the MRTAU time;
  \item [$-$]the isotope number densities;
  \item [$-$]the decay heat.
\end{enumerate}

The variable template is provided in Table \ref{VariableTemplateTable}. In the table, the region number is taken equal to 4,
 the group number is taken equal to 7, the cell number equal to 2 and the material number equal to 3.
Those values are only examples and can be adapted to the user's convenience.

\begin{table}[]
\centering
\caption{template of the RAVEN output variables}\label{VariableTemplateTable}
\begin{tabular}{l|l|l}
\textit{Variable} & \textit{Variable template}  & \textit{Comment}  \\
\hline
MrTau Time                     & timeMrTau                            & \\
Multiplication Factor          & keff                                 & \\
Multiplication Factor Error    & errorKeff                            & \\
n2n Reaction Rate              & n2n$\vert$gr7$\vert$reg4             & only if \xmlNode{printSpatialRR} is \xmlString{True})    \\
Power                          & power$\vert$gr7$\vert$reg4           & only if \xmlNode{printSpatialRR} is \xmlString{True})    \\
Absorption Reaction Rate       & absorption$\vert$gr7$\vert$reg4      & only if \xmlNode{printSpatialRR} is \xmlString{True})    \\
Fission Reaction Rate          & fission$\vert$gr7$\vert$reg4         & only if \xmlNode{printSpatialRR} is \xmlString{True})    \\
Neutron Flux                   & flux$\vert$gr7$\vert$reg4            & only if \xmlNode{printSpatialRR} is \xmlString{True})    \\
$\nu$ Fission Reaction Rate    & neutron$\vert$gr7$\vert$reg4         & only if \xmlNode{printSpatialRR} is \xmlString{True})    \\
Neutron Flux by Cell           & flux$\vert$cell2$\vert$gr7           & only if \xmlNode{printSpatialFlux} is \xmlString{True})  \\
Neutron Flux by Material       & flux$\vert$mat3$\vert$gr4            & only if \xmlNode{printSpatialFlux} is \xmlString{True})  \\
Total n2n Reaction Rate        & n2n$\vert$Total                      & only if \xmlNode{printSpatialRR} is \xmlString{False})  \\
Total Power reaction Rate      & power$\vert$Total                    & only if \xmlNode{printSpatialRR} is \xmlString{False})  \\
Total Absorption Reaction Rate & absorption$\vert$Total               & only if \xmlNode{printSpatialRR} is \xmlString{False})  \\
Total Fission Reaction Rate    & fission$\vert$Total                  & only if \xmlNode{printSpatialRR} is \xmlString{False})  \\
Total Neutron Flux             & flux$\vert$Total                     & only if \xmlNode{printSpatialRR} is \xmlString{False})  \\
Total $\nu$ Fis. Reaction Rate & neutron$\vert$Total                  & only if \xmlNode{printSpatialRR} is \xmlString{False})  \\
Decay Heat                     & decay$\vert$Fuel1$\vert$gr4          & only if the decay heat flag is turned on in PHISICS  \\
Cross sections                 & fuel1$\vert$xe135$\vert$npxs$\vert$8 & only if cross sections are perturbed  \\
PHISIC CPU time                & cpuTime                              & \\
\end{tabular}
\end{table}
$\nu$ is the average number of neutrons generated after fission. Note that the material number in the neutron
 flux by material corresponds to the material ID number in the PHISICS csv output,
while the material string ID in the decay heat corresponds to the material name given by the user in the xml
material file. Hence, the neutron flux by material will always have the format matX, where X is an integer.
The decay heat material is user-defined in the xml material file via the attribute \xmlAttr{id} of the \xmlNode{mat} node.
%%%%%%%%%%%%%%%%%%%%%%%%%%%%%%%%%%%%%%%%%%%%%%%%%
%%%%%%%%%%%%%%%%%%%%%%%%%%%%%%%%%%%%%%%%%%%%%%%%%
%%%%%%%%%%%%% PHISICS/RELAP5 INTERFACE %%%%%%%%%%
%%%%%%%%%%%%%%%%%%%%%%%%%%%%%%%%%%%%%%%%%%%%%%%%%
\subsection{PHISICS/RELAP5 Interface}
\label{subsec:PhisicsRelap5Interface}
%
\subsubsection{General Information}
%
This section covers the input specification for running PHISICS/RELAP5 through RAVEN.
This interface can be used to perturb the PHISICS and/or RELAP5 input files. This interface is strongly built around the PHISICS and RELAP5
standalone interfaces, hence this sections covers the additional cautions to take care of to run the coupled PHISICS/RELAP5 code. The
user will find additional information regarding PHISICS in section \ref{subsec:PhisicsInterface} or RELAP5 in section \ref{subsec:RELAP5Interface}.
%
\subsubsection{Files}
\label{subsec:PhisicsRelap5Files}
\xmlNode{Files} includes two attributes \xmlAttr{name} and \xmlAttr{type} entries, identically as other interfaces.
It also includes two optional attributes \xmlAttr{perturbable} and \xmlAttr{subDirectory}.

The \xmlAttr{name} attribute is a user-defined internal name for the file contained in the node.
Default: None (required entry).

For the files parsed by the PHISICS interface or the PHISICS interface's parsers, some of the \xmlAttr{type} attributes are hardcoded.
The accepted PHISICS \xmlAttr{type} attributes are given in Table \ref{TypeTable} and are not repeated here. Additional information can be found in section \ref{subsubsec:PhisicsInterfaceFiles}.
All the necessary RELAP5 input files need to have a \xmlAttr{type} attribute starting with the string 'relap'.
The necessary RELAP5 files for use in the coupled PHISICS/RELAP mode within RAVEN are given in Table~\ref{PhisicsRelap5TypeTable}. The files associated
with a  \xmlAttr{type} that does not start with the string 'relap' will be treated by the PHISICS interface.
The  \xmlAttr{type} attributes are case-incensitive.
\begin{table}[]
  \centering
  \caption{Example of RELAP5 type attributes in coupled PHISICS/RELAP5 mode}\label{PhisicsRelap5TypeTable}
    \begin{tabular}{l|l|l}
\textit{Type Attribute} & \textit{Corresponding RELAP5 input} & \textit{Perturbable} \\
\hline
relapFluid              & fluid properties                  & No  \\
relapInp                & Relap input File                  & yes \\
relapLicense            & license for the RELAP5 executable & No  \\
    \end{tabular}
\end{table}

Example of acceptable RELAP5 entries within PHISICS/RELAP5:
\begin{lstlisting}[style=XML]
<File>
  <Input name="H2O"       type="relaph2o"     perturbable="False">tph2o</Input>
  <Input name="H2"        type="relaph2"      perturbable="False">tph2</Input>
  <Input name="inputDeck" type="relapInput"   perturbable="True" >inp.i</Input>
  <Input name="lic"       type="relapLicence" perturbable="False">license.bin</Input>
</File>
\end{lstlisting}
%
\subsubsection{Models}
The user has to provide the paths to executables for the sampled variables within the \xmlNode{Models} block.

The \xmlNode{Code} block will contain attributes \xmlAttr{name} and \xmlAttr{subType}.
The \xmlAttr{name} identifies the particular \xmlNode{Code} model within RAVEN, and \xmlAttr{subType}
specifies which code interface the model will use. \xmlAttr{subType}=\xmlString{PhisicsRelap5} is the class name
currently used for PHISICS/RELAP5 coupled calculations.

The \xmlNode{executable} block contains the absolute or relative path (with respect to the current working directory) to PHISICS/RELAP5
that RAVEN will use to run the code. The additional nodes in the \xmlNode{Models} applicable to PHISICS standalone and RELAP5 standalone
are valid in coupled mode and can be consulted in section~\ref{subsubsection:PhisicsModel} and section~\ref{subsubsection:Relap5Models} respectively.
Exception: the use of MrTau in standalone mode (i.e. \xmlNode{mrtauStandAlone} set to \xmlString{True}) is not allowed in PHISICS/RELAP5 coupled
calculations.

An example of the \xmlNode{Models} block is given below:

\begin{lstlisting}[style=XML]
  <Models>
    <Code name="PHISICS_RELAP5" subType="PhisicsRelap5">
       <executable>./path/to/instant/executable</executable>
    </Code>
  </Models>
\end{lstlisting}
%%%%%%%%%%%%%%%%%%%%%%%%%%%%%%%%%%%%%%%%%%%%%%%%%%
\subsubsection{Distributions}
The \xmlNode{Distributions} block defines all distributions used to
sample variables in the current RAVEN run.

For all the possible distributions and their possible inputs please
refer to the Distributions chapter (see~\ref{sec:distributions}).
%
%%%%%%%%%%%%%%%%%%%%%%%%%%%%%%%%%%%%%%%%%%%%%%%%%%
\subsubsection{Samplers}\label{SamplerPhisicsRelap5}
The \xmlNode{Samplers} block defines the variables to be sampled.
After defining a sampling scheme, the variables to be sampled and
their distributions are identified in the \xmlNode{variable} blocks.
The \xmlAttr{name} must be formatted according to the PHISICS library which the variable belongs to.
Information relative to PHISICS distributions are in section~\ref{subsubsection:SamplersPhisics},
as well as specifications on PHISICS variable names.
An example of a \xmlNode{Samplers} block is given below:
\begin{lstlisting}[style=XML]
 <Samplers>
    <MonteCarlo name="MC_samp">
      <samplerInit>
        <limit>10</limit>
      </samplerInit>
      <variable name="DENSITY|FUEL1|U238">
        <distribution>DENSITY|FUEL1|U238_distrib</distribution>
      </variable>
      <variable name="20100154:2">
        <distribution>heat_capacity_154</distribution>
      </variable>
    </MonteCarlo>
  </Samplers>
\end{lstlisting}
In this example, the variable \xmlString{DENSITY$\vert$FUEL1$\vert$U238} is relative to PHISICS and the variable \xmlString{20100154:2} is relative to RELAP5.
%%%%%%%%%%%%%%%%%%%%%%%%%%%%%%%%%%%%%%%%%%%%%%%%%
\subsubsection{Steps}
The tasks performed by RAVEN need to be defined in the \xmlNode{Steps} block. Each task needs to be defined with a \xmlAttr{name}. This \xmlAttr{name} is later on used in the
the \xmlNode{Sequence} block. In the example, the step is called \xmlString{testDummyStep}.
%
\begin{lstlisting}[style=XML]
<Steps>
  <MultiRun name='testDummyStep' verbosity='debug'>
    <Input class='Files' type='decay'>decay.dat</Input>
    <Input class='Files' type='inp'>inp.xml</Input>
    <Input class='Files' type='XS'>xs.xml</Input>
    <Input class="Files" type="relapFluid">tpfhe</Input>
    <Input class="Files" type="relapExec">relap5Exec.x</Input>
    <Model class="Models" type="Code">PHISICS_RELAP5</Model>
    <Sampler class="Samplers" type="MonteCarlo">MC_samp</Sampler>
    <Output class="Databases" type="HDF5">DataB_REL5_1</Output>
    <Output class="DataObjects" type="PointSet">collset</Output>
    </MultiRun>
  </MultiRun>
</Steps>
\end{lstlisting}
%
%%%%%%%%%%%%%%%%%%%%%%%%%%%%%%%%%%%%%%%%%%%%%%%%%
\subsubsection{Additional Input}
\label{subsubsection:PhisicsRelap5AdditionalInput}
The PHISICS additional inputs are described in section~\ref{subsubsection:PhisicsAdditionalInput}. The RELAP5 additional inputs are described in
section ~\ref{subsubsection:Relap5Models}.
%
%%%%%%%%%%%%%%%%%%%%%%%%%%%%%%%%%%%%%%%%%%%%%%%%%
\subsubsection{Output Files Conversion}
\label{subsubsection:PhisicsRelap5OutputFileConversion}

The PHISICS output available for RAVEN post-processing are described in section~\ref{subsubsection:PhisicsOutFileConv}.
The output printed from PHISICS and RELAP5 are synchronized in the RAVEN csv output. The synchronization scheme is explained in this section.

At t = 0 seconds, the RELAP5 initialized output are printed in the csv output, while the output variables from PHISICS are taken equal to 0.
Then, the RAVEN/PHISICS/RELAP5 post-processor finds the time step number at the end of each PHISICS
burn step based on the \xmlNode{tab\_time\_step} values, and prints the RELAP minor edits according to the \xmlNode{TH\_between\_BURN} values.

Let's consider the following example:
\begin{enumerate}
\item [$-$]\xmlNode{tab\_time\_step} \xmlString{5 3 2} \xmlNode{/tab\_time\_step} (in the PHISICS depletion file);
\item [$-$]\xmlNode{TH\_between\_BURN} \xmlString{1.0 2.0} \xmlNode{TH\_between\_BURN}() in the PHISICS input file);
\item [$-$]\xmlNode{tabulation\_boundaries} \xmlString{5.0 35.0 45.0}\xmlNode{tabulation\_boundaries} (upper burn step boundaries in the PHISICS depletion file).
\item [$-$]in the RELAP5 input, a 3.0 seconds steady state is considered, with minor edits every 0.5 seconds.
\end{enumerate}
The first PHISICS/RELAP5 output line printed will be at t = 0 seconds. The PHISICS outputs are set to 0.0, the RELAP5 values are obtained at the end of the initialization.
The second line printed will be at the PHISICS time step \xmlString{5} (end of the first burn step, corresponding to t = 5.0 seconds),
and prints the RELAP5 minor edits as long as the time from the minor edits is lower than the first \xmlNode{TH\_between\_BURN} value \xmlString{1.0}.
The RELAP5 minor edits are printed along with the PHISICS burn step \xmlString{5} as long as time in the minor edits is smaller or equal to \xmlString{1.0}.
When the RELAP5 time in the minor edits time is greater than \xmlString{1.0}, the end of the second PHISICS burn step is targetted, from the
\xmlNode{tab\_time\_step}: \xmlString{3}. This corresponds to a PHISICS time equal to 35.0 seconds. The RELAP5 minor edits are printed along with the PHISICS values at t = 35.0 s,
as long as the minor data time is smaller than the \xmlNode{TH\_between\_BURN} equal to \xmlString{2.0}.
Finally, the last PHISICS time step at 45.0 seconds is printed along with the RELAP5 minor edits.
Overall, the output variables printed will be:

mrtauTime, n2n$\vert$gr1$\vert$reg4, httemp\_3001010\_1, time

0.000, 0.000, 1.526, 0.000

5.000, 1.111, 1.859, 0.500

5.000, 1.111, 2.369, 1.000

35.00, 7.800, 3.666, 1.500

35.00, 7.800, 4.789, 2.000

45.00, 9.330, 4.225, 3.000
%%%%%%%%%%%%%%%%%%%%%%%%%%%%%%%%%%%%%%%%%%%%%%%%%%%%%%
%%%%%%%%%%%%%%%%% Neutrino Interface %%%%%%%%%%%%%%%%%%
%%%%%%%%%%%%%%%%%%%%%%%%%%%%%%%%%%%%%%%%%%%%%%%%%%%%%%
\subsection{Neutrino Interface}
\label{subsec:neutrinoInterface}
This section covers the input specification for running Neutrino through RAVEN. It is important to notice
that this explanation assumes that the reader already knows how to use Neutrino. The existing inteface can be used to modify the particle size.
However, the interface can be modified to alter other parameters by using a similar method to the
existing particle size modification included in the interface.

\subsubsection{Files}
In the \xmlNode{Files} section, as specified before, all the files needed for the code to
run should be specified. In the case of Neutrino, the file needed is the following:
\begin{itemize}
  \item Neutrino input file with file extension `.nescene';
\end{itemize}
The Neutrino input file name must be NeutrinoInput.nescene. Otherwise, the Neutrino interface must be modified.
%
Example:
\begin{lstlisting}[style=XML]
  <Files>
    <Input name="neutrinoInput" type="">NeutrinoInput.nescene</Input>
  </Files>
\end{lstlisting}

%%%%%%%%%%%%%%%%%%%%%%%%%%%%%%%%%%%%%%%%%%%%%%%%%%%%%X
\subsubsection{Models}
In the \xmlNode{Models} block, the Neutrino executable needs to be specified. The entire path to the Neutrino executable must be included.
 Here is a standard example of what
can be used:
\begin{lstlisting}[style=XML]
  <Models>
    <Code name="neutrinoCode" subType="Neutrino">
      <executable>"C:\Program Files\Neutrino_02_22_19\Neutrino.exe"</executable>
    </Code>
  </Models>
\end{lstlisting}

The \xmlNode{Code} XML node contains the information needed to execute the specific External Code. This
XML node accepts the following attributes:
\begin{itemize}
  \item \xmlAttr{name}, \xmlDesc{required string attribute}, user-defined identifier of this model.
    \nb As with other objects, this identifier can be used to reference this specific entity from other input
    blocks in the XML.
  \item \xmlAttr{subType}, \xmlDesc{required string attribute}, specifies the code that needs to be
    associated to this Model.
\end{itemize}
This model can be initialized with the following children:
\begin{itemize}
  \item \xmlNode{executable}, \xmlDesc{string, required field}, specifies the path of the executable to
    be used; \nb Either an absolute or relative path can be used.
\end{itemize}


%%%%%%%%%%%%%%%%%%%%%%%%%%%%%%%%%%%%%%%%%%%%%%%%%%%%%%
\subsubsection{Distributions}
The \xmlNode{Distributions} block defines the distributions that are going to be used for the sampling
of the variables defined in the \xmlNode{Samplers} block. For all the possible distributions and all
their possible inputs, please see the chapter about Distributions (see~\ref{sec:distributions}). Here is an example of a
Uniform distribution:
\begin{lstlisting}[style=XML]
  <Distributions>
    <Uniform name="uni">
        <lowerBound>0.1</lowerBound>
        <upperBound>0.2</upperBound>
    </Uniform>
  </Distributions>
\end{lstlisting}
%
%%%%%%%%%%%%%%%%%%%%%%%%%%%%%%%%%%%%%%%%%%%%%%%%%%%%%%
\subsubsection{Samplers}
The \xmlNode{Samplers} block defines the variables to be sampled. After defining a sampling scheme, the variables to be sampled and
their distributions are identified in the \xmlNode{variable} blocks.
The \xmlAttr{name} must be formatted according to the Neutrino parameter name, which for the particle size is
'ParticleSize'.
An example of a \xmlNode{Samplers} block is given below:
\begin{lstlisting}[style=XML]
 <Samplers>
    <MonteCarlo name="myMC">
      <samplerInit>
        <limit>5</limit>
      </samplerInit>
      <variable name='ParticleSize'>
        <distribution>uni</distribution>
      </variable>
    </MonteCarlo>
  </Samplers>
\end{lstlisting}

%%%%%%%%%%%%%%%%%%%%%%%%%%%%%%%%%%%%%%%%%%%%%%%%%%%%%%
\subsubsection{Steps}
In this section, the \xmlNode{MultiRun} will be used. As shown in the following, a Neutrino
input file is listed in \xmlNode{Files} and is linked here using \xmlNode{Input}, the \xmlNode{Model}
and \xmlNode{Sampler} defined in previous sections will be used in this \xmlNode{MultiRun}. The
outputs will be saved in the \textbf{DataObject} 'resultPointSet'.

\begin{lstlisting}[style=XML]
  <Steps>
    <MultiRun name="run">
      <Input class="Files" type="">neutrinoInput</Input>
      <Model class="Models" type="Code">neutrinoCode</Model>
      <Sampler class="Samplers" type="MonteCarlo">myMC</Sampler>
      <Output class="DataObjects" type="PointSet">resultPointSet</Output>
    </MultiRun>
  </Steps>
\end{lstlisting}

%%%%%%%%%%%%%%%%%%%%%%%%%%%%%%%%%%%%%%%%%%%%%%%%%%%%%%
\subsubsection{Output File Conversion}
The Neutrino measurement field output is a CSV output. However, labels must be added to the Neutrino output
and it must be moved for RAVEN. These are both done in the Neutrino interface. The labels that are added to the
output file are 'time' and 'result'. These labels would be used in the \xmlNode{DataObjects} specification.
If different labels are wanted, they would need to be changed directly in the Neutrino interface.

%%%%%%%%%%%%%%%%%%%%%%%%%%%%%%%%%%%%%%%%%%%%%%%%%%%%%%
\subsubsection{Additional Information}
The Neutrino interface is used to alter the particle size by modifying the Neutrino input file. The Neutrino interface
searches for the default SPH solver parameter name: '\texttt{NIISphSolver\_1}'. If the SPH solver name is changed in the
Neutrino input file, the Neutrino interface must also be changed. Similarly, the Neutrino interface searches for the
output in the default Measurement field name: '\texttt{MeasurementField\_1}'. Again, this would need to modified in the interface
if the measurement field name was changed.

\subsection{Prescient Interface}
\label{subsec:prescientInterface}

\subsubsection{General Information}
The Prescient Interface is used to run the open source Prescient
production cost modeling platform available from
\url{https://github.com/grid-parity-exchange/Prescient}

This allows inputs to be perturbed and data to be read out.

\subsubsection{Sampler}

For perturbing inputs, the sampled variable needs to be placed inside
of \$( )\$ like \verb'$(var)$'. The sampled variable can have a
constant added or a multiplication factor like \verb'$(var+3.2)$' or
\verb'$(var*2.1)$' or \verb'$(var*5.0+7.0)$' or \verb'$(a*-2.0)$'
These can be placed in any of the .dat or .csv files that are listed
in the \xmlNode{Files} section as \xmlAttr{type="PrescientInput"} An
example line could be: \verb'Abel 1 $(var)$'

\begin{lstlisting}[style=XML]
  <Samplers>
    <Grid name="grid">
      <variable name="var">
        <distribution>dist</distribution>
        <grid construction="equal" steps="1" type="CDF">0.0 1.0</grid>
      </variable>
    </Grid>
  </Samplers>
\end{lstlisting}


\subsubsection{Files}

There are two types of inputs in the \xmlNode{Files} section.  The
\xmlAttr{type="PrescientRunnerInput"} ones are passed as an argument
to the \texttt{runner.py} If multiple PrescientRunnerInput files are
specified then \texttt{runner.py} will be called multiple times (which
can be used to run a populate and then simulate command).  The
\xmlAttr{type="PrescientInput"} are just used as additional inputs
that have the data in them perturbed.

\begin{lstlisting}[style=XML]
  <Files>
  <Input name="simulate" type="PrescientRunnerInput"
  >simulate_day.txt</Input>
    <Input name="structure" type="PrescientInput"
    subDirectory="scenarios/pyspdir_twostage/2020-07-10/"
    >ScenarioStructure.dat</Input>
    <Input name="scenario_1" type="PrescientInput"
    subDirectory="scenarios/pyspdir_twostage/2020-07-10/"
    >Scenario_1.dat</Input>
    <Input name="actuals" type="PrescientInput"
    subDirectory="scenarios/pyspdir_twostage/2020-07-10/"
    >Scenario_actuals.dat</Input>
    <Input name="forcasts" type="PrescientInput"
    subDirectory="scenarios/pyspdir_twostage/2020-07-10/"
    >Scenario_forecasts.dat</Input>
    <Input name="scenarios" type="PrescientInput"
    subDirectory="scenarios/pyspdir_twostage/2020-07-10/"
    >scenarios.csv</Input>
  </Files>
\end{lstlisting}

\subsubsection{Models}

The \xmlNode{Code} model can be used with the
\xmlAttr{subType="Prescient"} to run the Prescient Code Interface.
The block currently does not have any option xml nodes.

\begin{lstlisting}[style=XML]
  <Models>
    <Code name="TestPrescient" subType="Prescient">
      <executable>
      </executable>
    </Code>
  </Models>
\end{lstlisting}

\subsubsection{Output Files Conversion}

The code interface reads in the \texttt{hourly\_summary.csv} and the
\texttt{bus\_detail.csv} files. It will generate a \texttt{Date\_Hour}
variable that can be used as the \xmlNode{pivotParameter} and is a
string with the date and hour. It also generates an \texttt{Hour}
variable that is the hour as an integer.  From the hourly summary it
will generate variables like TotalCosts and the other data that
appears there.  For each of the busses in the bus detail file it
generates variables like \texttt{Clay\_LMP} that can be used.

Exactly which variables will appear will vary depending on the
Prescient input files, but typical ones include \texttt{TotalCosts},
\texttt{FixedCosts}, \texttt{VariableCosts}, \texttt{LoadShedding},
\texttt{OverGeneration}, \texttt{ReserveShortfall},
\texttt{RenewablesUsed}, \texttt{RenewablesCurtailment},
\texttt{Demand}, \texttt{Price}, and \texttt{NetDemand}. Variables
that can be included for a typical bus could include ones like
\texttt{Abel\_LMP}, \texttt{Abel\_LMP\_DA}, \texttt{Abel\_Shortfall},
and \texttt{Abel\_Overgeneration}.

\begin{lstlisting}[style=XML]
    <HistorySet name="samples">
      <Input>var</Input>
      <Output>Date_Hour, TotalCosts, FixedCosts, VariableCosts, LoadShedding, OverGeneration, ReserveShortfall, RenewablesUsed, RenewablesCurtailment, Demand, Price, NetDemand, Abel_LMP, Clay_LMP </Output>
      <options>
        <pivotParameter>Date_Hour</pivotParameter>
      </options>
    </HistorySet>
\end{lstlisting}


\subsubsection{Installation of Libraries}

Installing Prescient so that RAVEN can run it requires that RAVEN and
Prescient have a superset of the libraries that they use so that both
can run.  One way to set this up is to install RAVEN, and then source
the conda load script and inside of the conda raven libraries
environment do the Prescient and Egret install.  This is shown in the
following listing:

\begin{lstlisting}
  #first clone raven, Egret and Prescient into a directory
  git clone git@github.com:idaholab/raven.git
  git clone git@github.com:grid-parity-exchange/Prescient.git
  git clone git@github.com:grid-parity-exchange/Egret.git
  #Switch to raven directory
  cd raven
  #install raven libraries
  ./scripts/establish_conda_env.sh --install
  #switch to using raven libraries
  source ./scripts/establish_conda_env.sh --load
  #Switch to Prescient and install
  cd ../Prescient
  python setup.py develop --user
  conda install -c conda-forge coincbc
  #Switch to Egret and install
  cd ../Egret/
  pip install --user -e .
\end{lstlisting}

Note that the path to \texttt{runner.py} may need to be added to the PATH variable via a command like: \verb'PATH="$PATH:$HOME/.local/bin"'


    % ---------------------------------------------------------------------- %
    % References
    %
    \clearpage
    % If hyperref is included, then \phantomsection is already defined.
    % If not, we need to define it.
    \providecommand*{\phantomsection}{}
    \phantomsection
    \addcontentsline{toc}{section}{References}
    \bibliographystyle{ieeetr}
    \bibliography{raven_user_manual.bib}


    % ---------------------------------------------------------------------- %
    %

    % \printindex

    %\include{distribution}

\end{document}
