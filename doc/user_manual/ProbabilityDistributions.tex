\section{Distributions\\ \vspace{2em}}

\newcommand{\distname}[1]{\textbf{#1}}

\newcommand{\distattrib}[1]{\textit{#1}}

%Bernoulli - BasicBernoulliDistribution - discrete
%p

All of the probability distributions fuctions could be truncated by using: 
\begin{lstlisting}[style=XML]
<low>***</low>
<hi>***</hi>
\end{lstlisting}
In case the probability distribution function demanded specific boundary values, those are set by default, and the user can change them.

\begin{itemize}


\item The \distname{Bernoulli} distribution is a discrete distribution of
the outcome of a single experiment with success probability
\distattrib{p}.
\begin{lstlisting}[style=XML]
<Bernoulli name='...'>
<p>***</p>
\end{lstlisting}
%binomial - BasicBinomialDistribution - discrete
%p 
%n

\item The \distname{Beta} distribution is a continuous distribution.  It has
four parameters, \distattrib{low} for the lowest value,
\distattrib{high} for the highest value, \distattrib{alpha} and
\distattrib{beta} for the shape parameters.

\begin{lstlisting}[style=XML]
<Beta name='...'>
<low>***</low>
<hi>***</hi>
<alpha>***</alpha>
<beta>***</beta>
\end{lstlisting}

\item The \distname{Binomial} distribution is a discrete distribution of the
outcome of \distattrib{n} experiments with each experiment having the
success probability \distattrib{p}.

\begin{lstlisting}[style=XML]
<Binomial name='...'>
<n>***</n>
<p>***</p>
\end{lstlisting}

%exponential - BasicExponentialDistribution - continuous
%lambda

\item The \distname{Exponential} distribution is a continuous distribution
of which can be used to model the time between independent events that
happen at a constant average time.  It uses the rate parameter
\distattrib{lambda}.

\begin{lstlisting}[style=XML]
<Exponential name='...'>
<lambda>***</lambda>
\end{lstlisting}

%logistic - BasicLogisticDistribution - continuous
%location
%scale

\item The \distname{Gamma} distribution is a continuous distribution.  It
has three parameters, \distattrib{low} for the lowest value,
\distattrib{alpha} is the shape parameter, and \distattrib{beta} which
is 1/scale or the inverse scale parameter.

\begin{lstlisting}[style=XML]
<Gamma name='...'>
<low>***</low>
<alpha>***</alpha>
<beta>***</beta>
\end{lstlisting}


\item The \distname{Logistic} distribution is a continuous distribution
similar to the normal distribution with a CDF that is an instance of a
logistic function.  It has two parameters, \distattrib{location} with
is the most common value and the center, and \distattrib{scale} which
determines the shape.

\begin{lstlisting}[style=XML]
<Logistic name='...'>
<location>***</location>
<scale>***</scale>
\end{lstlisting}

%log-normal - BasicLogNormalDistribution - continuous
%mean
%sigma

\item The \distname{LogNormal} distribution is a continuous distribution
with the logarithm of the random variable being normally distributed.
It has two parameters, \distattrib{mean} which is the expected value,
and \distattrib{sigma} which is the standard deviation.

\begin{lstlisting}[style=XML]
<LogNormal name='...'>
<mean>***</mean>
<sigma>***</sigma>
\end{lstlisting}

%normal - BasicNormalDistribution - continuous
%mean
%sigma

\item The \distname{Normal} distribution (or Gaussian) distribution is a
continuous distribution which because of the central limit theorem,
the mean of many distributions approximates a normal distribution.  It
has two parameters, \distattrib{mean} which is the middle value, and
\distattrib{sigma} which is the standard deviation.

\begin{lstlisting}[style=XML]
<Normal name='...'>
<mean>***</mean>
<sigma>***</sigma>
\end{lstlisting}


%Poisson - BasicPoissonDistribution - discrete
%mu

\item The \distname{Poisson} distribution is a discrete distribution that
expresses the probability of the number of events occurring in a fixed
period of time.  It has one parameter, \distattrib{mu} the mean rate
of events/time.

\begin{lstlisting}[style=XML]
<Poisson name='...'>
<mu>***</mu>
\end{lstlisting}

%triangular - BasicTriangularDistribution - continuous
%apex
%min
%max

\item The \distname{Triangular} distribution is a continuous distribution
that has a triangular shape for the PDF.  The peak falls at the
\distattrib{apex} and the values run from \distattrib{min} to
\distattrib{max}.

\begin{lstlisting}[style=XML]
<Triangular name='...'>
<apex>***</apex>
<min>***</min>
<max>***</max>
\end{lstlisting}


%uniform - BasicUniformDistribution - continuous 
%low
%high

\item The \distname{Uniform} distribution is a continuous distribution with
a rectangular shaped PDF.  It goes from \distattrib{low} to
\distattrib{high}.

\begin{lstlisting}[style=XML]
<Uniform name='...'>
<low>***</low>
<hi>***</hi>
\end{lstlisting}

%weibull - BasicWeibullDistribution - continuous
%k
%lambda

\item The \distname{Weibull} distribution is a continuous distribution that
can be used is failure analysis.  It takes two parameters,
\distattrib{k} or the shape parameter, and \distattrib{lambda} or the
scale parameter.

\begin{lstlisting}[style=XML]
<Weibull name='...'>
<lambda>***</lambda>
<k>***</k>
\end{lstlisting}

\end{itemize}
%gamma - BasicGammaDistribution - continuous 
%low
%alpha
%beta



%beta - BasicBetaDistribution - continuous
%low
%high
%alpha
%beta


