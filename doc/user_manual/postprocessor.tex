\subsection{PostProcessor}
\label{sec:models_postProcessor}
A Post-Processor (PP) can be considered as an action performed on a set of data
or other type of objects.
%
Most of the post-processors contained in RAVEN, employ a mathematical operation
on the data given as ``input''.
%
RAVEN supports several different types of PPs.

Currently, the following types are available in RAVEN:
\begin{itemize}
  \itemsep0em
  \item \textbf{BasicStatistics}
  \item \textbf{ComparisonStatistics}
  \item \textbf{ImportanceRank}
  \item \textbf{SafestPoint}
  \item \textbf{LimitSurface}
  \item \textbf{LimitSurfaceIntegral}
  \item \textbf{External}
  \item \textbf{TopologicalDecomposition}
  \item \textbf{RavenOutput}
  \item \textbf{DataMining}
  \item \textbf{Metric}
  \item \textbf{CrossValidation}
  %\item \textbf{PrintCSV}
  %\item \textbf{LoadCsvIntoInternalObject}
\end{itemize}

The specifications of these types must be defined within the XML block
\xmlNode{PostProcessor}.
%
This XML node needs to contain the attributes:
\vspace{-5mm}
\begin{itemize}
  \itemsep0em
  \item \xmlAttr{name}, \xmlDesc{required string attribute}, user-defined
  identifier of this post-processor.
  %
  \nb As with other objects, this is the name that can be used to refer to this
  specific entity from other input XML blocks.
  \item \xmlAttr{subType}, \xmlDesc{required string attribute}, defines which of
  the post-processors needs to be used, choosing among the previously reported
  types.
  %
  This choice conditions the subsequent required and/or optional
  \xmlNode{PostProcessor} sub nodes.
  %
\end{itemize}
\vspace{-5mm}

As already mentioned, all the types and meaning of the remaining sub-nodes
depend on the post-processor type specified in the attribute \xmlAttr{subType}.
%
In the following sections the specifications of each type are reported.

%%%%% PP BasicStatistics %%%%%%%
\subsubsection{BasicStatistics}
\label{BasicStatistics}
The \textbf{BasicStatistics} post-processor is the container of the algorithms
to compute many of the most important statistical quantities. It is important to notice that this
post-processor can accept as input both \textit{\textbf{PointSet}} and \textit{\textbf{HistorySet}}
data objects, depending on the type of statistics the user wants to compute:
\begin{itemize}
  \item \textit{\textbf{PointSet}}: Static Statistics;
  \item \textit{\textbf{HistorySet}}: Dynamic Statistics. Depending on a ``pivot parameter'' (e.g. time)
  the post-processor is going to compute the statistics for each value of it (e.g. for each time step).
  In case an \textbf{HistorySet} is provided as Input, the Histories needs to be synchronized (use
    \textit{\textbf{Interfaced}} post-processor of type  \textbf{HistorySetSync}).
\end{itemize}
%
\ppType{BasicStatistics post-processor}{BasicStatistics}
\begin{itemize}
  \item \xmlNode{(metric)}, \xmlDesc{comma separated string or node list, required field},
    specifications for the metric to be calculated.  The name of each node is the requested metric.  There are
    two forms for specifying the requested parameters of the metric.  For scalar values such as
    \xmlNode{expectedValue} and \xmlNode{variance}, the text of the node is a comma-separated list of the
    parameters for which the metric should be calculated.  For matrix values such as \xmlNode{sensitivty} and
    \xmlNode{covariance}, the matrix node requires two sub-nodes, \xmlNode{targets} and \xmlNode{features},
    each of which is a comma-separated list of the targets for which the metric should be calculated, and the
    features for which the metric should be calculated for that target.  See the example below.

    \nb When defining the metrics to use, it is possible to have multiple nodes with the same name.  For
    example, if a problem has inputs $W$, $X$, $Y$, and $Z$, and the responses are $A$, $B$, and $C$, it is possible that
    the desired metrics are the \xmlNode{sensitivity} of $A$ and $B$ to $X$ and $Y$, as well as the
    \xmlNode{sensitivity} of $C$ to $W$ and $Z$, but not the sensitivity of $A$ to $W$.   In this event, two
    copies of the \xmlNode{sensitivity} node are added to the input.  The first has targets $A,B$ and features
    $X,Y$, while the second node has target $C$ and features $W,Z$.  This could reduce some computation effort
    in problems with many responses or inputs.  An example of this is shown below.
  %
  \\ Currently the scalar quantities available for request are:
  \begin{itemize}
    \item \textbf{expectedValue}: expected value or mean
    \item \textbf{minimum}: The minimum value of the samples.
    \item \textbf{maximum}: The maximum value of the samples.
    \item \textbf{median}: median
    \item \textbf{variance}: variance
    \item \textbf{sigma}: standard deviation
    \item \textbf{percentile}: the percentile. If this quantity is inputted as \textit{percentile} the $5\%$ and $95\%$ percentile(s) are going to be computed.
                               Otherwise the user can specify this quantity as \textit{percentile\_$X\%$}, where \textit{X} represents the requested
                               percentile (an integer value between 1 and 100)
    \item \textbf{variationCoefficient}: coefficient of variation, i.e. \textbf{sigma}/\textbf{expectedValue}. \nb If the \textbf{expectedValue} is zero,
    the \textbf{variationCoefficient} will be \textbf{INF}.
    \item \textbf{skewness}: skewness
    \item \textbf{kurtosis}: excess kurtosis (also known as Fisher's kurtosis)
    \item \textbf{samples}: the number of samples in the data set used to determine the statistics.
  \end{itemize}
  The matrix quantities available for request are:
  \begin{itemize}
    \item \textbf{sensitivity}: matrix of sensitivity coefficients, computed via linear regression method.
    \item \textbf{covariance}: covariance matrix
    \item \textbf{pearson}: matrix of correlation coefficients
    \item \textbf{NormalizedSensitivity}: matrix of normalized sensitivity
    coefficients. \nb{It is the matrix of normalized VarianceDependentSensitivity}
    \item \textbf{VarianceDependentSensitivity}: matrix of sensitivity coefficients dependent on the variance of the variables
  \end{itemize}
  If all the quantities need to be computed, this can be done through the \xmlNode{all} node, which
  requires the \xmlNode{targets} and \xmlNode{features} sub-nodes.

   %
  \nb If the weights are present in the system then weighted quantities are calculated automatically. In addition, if a matrix quantity is requested (e.g. Covariance matrix, etc.), only the weights in the output space are going to be used for both input and output space (the computation of the joint probability between input and output spaces is not implemented yet).
  \\
  \nb Certain ROMs provide their own statistical information (e.g., those using
  the sparse grid collocation sampler such as: \xmlString{GaussPolynomialRom}
  and \xmlString{HDMRRom}) which can be obtained by printing the ROM to file
  (xml). For these ROMs, computing the basic statistics on data generated from
  one of these sampler/ROM combinations may not provide the information that the
  user expects.
  \\
  %
   \item \xmlNode{pivotParameter}, \xmlDesc{string, optional field}, name of the parameter that needs
   to be used for the computation of the Dynamic BasicStatistics (e.g. time). This node needs to
   be inputted just in case an \textbf{HistorySet} is used as Input. It represents the reference
   monotonic variable based on which the statistics is going to be computed (e.g. time-dependent
   statistical moments).
    \default{None}
  %
  \item \xmlNode{biased}, \xmlDesc{string (boolean), optional field}, if \textit{True} biased
  quantities are going to be calculated, if \textit{False} unbiased.
  \default{False}
  %
  \item \xmlNode{methodsToRun}, \xmlDesc{comma separated string, optional
  field}, specifies the method names of an external Function that need to be run
  before computing any of the predefined quantities.
  %
  If this XML node is specified, the \xmlNode{Function} node must be present.
  %
  \default{None}
% Assembler Objects
  \item \textbf{Assembler Objects} This object is required in case the  \xmlNode{methodsToRun}
  node is specified.
  %
  The object must be listed with a rigorous syntax that, except for the xml
  node tag, is common among all the objects.
  %
  Each of these nodes must contain 2 attributes that are used to map those
  within the simulation framework:
  \begin{itemize}
    \item \xmlAttr{class}, \xmlDesc{required string attribute}, it is the main
    ``class'' the listed object is from;
    \item \xmlAttr{type}, \xmlDesc{required string attribute}, it is the object
    identifier or sub-type.
    %
  \end{itemize}
  The \textbf{BasicStatistics} post-processor approach optionally accepts the
  following object type:
  \begin{itemize}
    \item \xmlNode{Function}, \xmlDesc{string, required field}, The body of
    this xml block needs to contain the name of an External Function defined
    within the \xmlNode{Functions} main block (see section \ref{sec:functions}).
    %
    This object needs to contain the methods listed in the node
    \xmlNode{methodsToRun}.
  \end{itemize}
\end{itemize}
\textbf{Example (Static Statistics):}  This example demonstrates how to request the expected value of
\xmlString{x01} and \xmlString{x02}, along with the sensitivity of both \xmlString{x01} and \xmlString{x02} to
\xmlString{a} and \xmlString{b}.
\begin{lstlisting}[style=XML,morekeywords={name,subType,debug}]
<Simulation>
  ...
  <Models>
    ...
    <PostProcessor name='aUserDefinedName' subType='BasicStatistics' verbosity='debug'>
      <expectedValue>x01,x02</expectedValue>
      <sensitivity>
        <targets>x01,x02</targets>
        <features>a,b</features>
      </sensitivity>
      <methodsToRun>failureProbability</methodsToRun>
    </PostProcessor>
    ...
  </Models>
  ...
</Simulation>
\end{lstlisting}

\textbf{Example (Static using \xmlNode{all}):} This example is similar to the one above, but shows using the
\xmlNode{all} node.
\begin{lstlisting}[style=XML,morekeywords={name,subType,debug}]
<Simulation>
  ...
  <Models>
    ...
    <PostProcessor name='aUserDefinedName' subType='BasicStatistics' verbosity='debug'>
      <all>
        <targets>x01,x02</targets>
        <features>a,b</features>
      </all>
    </PostProcessor>
    ...
  </Models>
  ...
</Simulation>
\end{lstlisting}

\textbf{Example (Static, multiple matrix nodes):} This example shows how multiple nodes can specify
particular metrics multiple times to include different target/feature combinations.  This postprocessor
calculates the expected value of $A$, $B$, and $C$, as well as the sensitivity of both $A$ and $B$ to $X$ and
$Y$ as well as the sensitivity of $C$ to $W$ and $Z$.
\begin{lstlisting}[style=XML,morekeywords={name,subType,debug}]
<Simulation>
  ...
  <Models>
    ...
    <PostProcessor name='aUserDefinedName' subType='BasicStatistics' verbosity='debug'>
      <expectedValue>A,B,C</expectedValue>
      <sensitivity>
        <targets>A,B</targets>
        <features>x,y</features>
      </sensitivity>
      <sensitivity>
        <targets>C</targets>
        <features>w,z</features>
      </sensitivity>
    </PostProcessor>
    ...
  </Models>
  ...
</Simulation>
\end{lstlisting}
\textbf{Example (Dynamic Statistics):}
\begin{lstlisting}[style=XML,morekeywords={name,subType,debug}]
<Simulation>
  ...
  <Models>
    ...
    <PostProcessor name='aUserDefinedNameForDynamicPP' subType='BasicStatistics' verbosity='debug'>
      <expectedValue>x01,x02</expectedValue>
      <sensitivity>
        <targets>x01,x02</targets>
        <features>a,b</features>
      </sensitivity>
      <methodsToRun>failureProbability</methodsToRun>
      <pivotParameter>time</pivotParameter>
    </PostProcessor>
    ...
  </Models>
  ...
</Simulation>
\end{lstlisting}

%%%%% PP ComparisonStatistics %%%%%%%
\subsubsection{ComparisonStatistics}
\label{ComparisonStatistics}
The \textbf{ComparisonStatistics} post-processor computes statistics
for comparing two different dataObjects.  This is an experimental
post-processor, and it will definitely change as it is further
developed.

There are four nodes that are used in the post-processor.

\begin{itemize}
\item \xmlNode{kind}: specifies information to use for comparing the
  data that is provided.  This takes either uniformBins which makes
  the bin width uniform or equalProbability which makes the number
  of counts in each bin equal.  It can take the following attributes:
  \begin{itemize}
  \item \xmlAttr{numBins} which takes a number that directly
    specifies the number of bins
  \item \xmlAttr{binMethod} which takes a string that specifies the
    method used to calculate the number of bins.  This can be either
    square-root or sturges.
  \end{itemize}
\item \xmlNode{compare}: specifies the data to use for comparison.
  This can either be a normal distribution or a dataObjects:
  \begin{itemize}
  \item \xmlNode{data}: This will specify the data that is used.  The
    different parts are separated by $|$'s.
  \item \xmlNode{reference}: This specifies a reference distribution
    to be used.  It takes distribution to use that is defined in the
    distributions block.  A name parameter is used to tell which
    distribution is used.
  \end{itemize}
\item \xmlNode{fz}: If the text is true, then extra comparison
  statistics for using the $f_z$ function are generated.  These take
  extra time, so are not on by default.
\item \xmlNode{interpolation}: This switches the interpolation used
  for the cdf and the pdf functions between the default of quadratic
  or linear.
\end{itemize}

The \textbf{ComparisonStatistics} post-processor generates a variety
of data.  First for each data provided, it calculates bin boundaries,
and counts the numbers of data points in each bin.  From the numbers
in each bin, it creates a cdf function numerically, and from the cdf
takes the derivative to generate a pdf.  It also calculates statistics
of the data such as mean and standard deviation. The post-processor
can generate a CSV file only.

The post-processor uses the generated pdf and cdf function to
calculate various statistics.  The first is the cdf area difference which is:
\begin{equation}
  cdf\_area\_difference = \int_{-\infty}^{\infty}{\|CDF_a(x)-CDF_b(x)\|dx}
\end{equation}
This given an idea about how far apart the two pieces of data are, and
it will have units of $x$.

The common area between the two pdfs is calculated.  If there is
perfect overlap, this will be 1.0, if there is no overlap, this will
be 0.0.  The formula used is:
\begin{equation}
  pdf\_common\_area = \int_{-\infty}^{\infty}{\min(PDF_a(x),PDF_b(x))}dx
\end{equation}

The difference pdf between the two pdfs is calculated.  This is calculated as:
\begin{equation}
  f_Z(z) = \int_{-\infty}^{\infty}f_X(x)f_Y(x-z)dx
\end{equation}
This produces a pdf that contains information about the difference
between the two pdfs.  The mean can be calculated as (and will be
calculated only if fz is true):
\begin{equation}
  \bar{z} = \int_{-\infty}^{\infty}{z f_Z(z)dz}
\end{equation}
The mean can be used to get an signed difference between the pdfs,
which shows how their means compare.

The variance of the difference pdf can be calculated as (and will be
calculated only if fz is true):
\begin{equation}
  var = \int_{-\infty}^{\infty}{(z-\bar{z})^2 f_Z(z)dz}
\end{equation}

The sum of the difference function is calculated if fz is true, and is:
\begin{equation}
  sum = \int_{-\infty}^{\infty}{f_z(z)dz}
\end{equation}
This should be 1.0, and if it is different that
points to approximations in the calculation.


\textbf{Example:}
\begin{lstlisting}[style=XML]
<Simulation>
   ...
   <Models>
      ...
      <PostProcessor name="stat_stuff" subType="ComparisonStatistics">
      <kind binMethod='sturges'>uniformBins</kind>
      <compare>
        <data>OriData|Output|tsin_TEMPERATURE</data>
        <reference name='normal_410_2' />
      </compare>
      <compare>
        <data>OriData|Output|tsin_TEMPERATURE</data>
        <data>OriData|Output|tsout_TEMPERATURE</data>
      </compare>
      </PostProcessor>
      <PostProcessor name="stat_stuff2" subType="ComparisonStatistics">
        <kind numBins="6">equalProbability</kind>
        <compare>
          <data>OriData|Output|tsin_TEMPERATURE</data>
        </compare>
        <Distribution class='Distributions' type='Normal'>normal_410_2</Distribution>
      </PostProcessor>
      ...
   </Models>
   ...
   <Distributions>
      <Normal name='normal_410_2'>
         <mean>410.0</mean>
         <sigma>2.0</sigma>
      </Normal>
   </Distributions>
</Simulation>
\end{lstlisting}

%%%%% PP ImportanceRank %%%%%%%
\subsubsection{ImportanceRank}
\label{ImportanceRank}
The \textbf{ImportanceRank} post-processor is specifically used
to compute sensitivity indices and importance indices with respect to input parameters
associated with multivariate normal distributions. In addition, the user can also request the transformation
matrix and the inverse transformation matrix when the PCA reduction is used.
%
\ppType{ImportanceRank}{ImportanceRank}
%
\begin{itemize}
  \item \xmlNode{what}, \xmlDesc{comma separated string, required field},
  %
  List of quantities to be computed.
  %
  Currently the quantities available are:
  \begin{itemize}
    \item \xmlString{SensitivityIndex}: used to measure the impact of sensitivities on the model.
    \item \xmlString{ImportanceIndex}: used to measure the impact of sensitivities and input uncertainties on the model.
    \item \xmlString{PCAIndex}: the indices of principal component directions, used to measure the impact
    of principal component directions on input covariance matrix.
    \nb \xmlString{PCAIndex} can be only requested when subnode \xmlNode{latent} is defined in \xmlNode{features}.
    \item \xmlString{transformation}: the transformation matrix used to map the latent variables to the manifest variables in the original input space.
    \item \xmlString{InverseTransformation}: the inverse transformation matrix used to map the manifest variables to the latent variables in the transformed space.
    \item \xmlString{ManifestSensitivity}: the sensitivity coefficients of \xmlNode{target} with respect to \xmlNode{manifest} variables defined in \xmlNode{features}.

    \nb In order to request \xmlString{transformation} matrix or \xmlString{InverseTransformation} matrix or \xmlString{ManifestSensitivity},
    the subnodes \xmlNode{latent} and \xmlNode{manifest} under \xmlNode{features} are required (more details can be found in the following).
    %
  \end{itemize}
   %
  If all the quantities need to be computed, the user can input in the body of \xmlNode{what} the string \xmlString{all}.
  \nb \xmlString{all} equivalent to \xmlString {SensitivityIndex, ImportanceIndex, PCAIndex}.

  Since the transformation and InverseTransformation matrix can be very large, they are not printed with option \xmlString{all}.
  In order to request the transformation matrix (or inverse transformation matrix) from this post processor,
  the user need to specify \xmlString{transformation} or \xmlString{InverseTransformation} in \xmlNode{what}. In addition,
  both  \xmlNode{manifest} and \xmlNode{latent} subnodes are required and should be defined in node \xmlNode{features}. For example, let $\mathbf{L, P}$ represent
  the transformation and inverse transformation matrices, respectively. We will define vectors $\mathbf x$ as manifest variables and vectors $\mathbf y$
  as latent variables. If a absolute covariance matrix is used in given distribution, the following equation will be used:

  $
  \mathbf{\delta x} = \mathbf L * \mathbf y
  $

  $
  \mathbf y = \mathbf P * \mathbf \delta \mathbf x
  $

  If a relative covariance matrix is used in given distribution, the following equation will be used:

  $
  \frac{\mathbf \delta \mathbf x}{\mathbf \mu} = \mathbf L * \mathbf y
  $

  $
  \mathbf y = \mathbf P * {\frac{\mathbf \delta \mathbf x}{\mathbf \mu}}
  $

  where $\mathbf{\delta x}$ denotes the changes in the input vector $\mathbf x$, and $\mathbf \mu$ denotes the mean values of the input vector $\mathbf x$.

  %
  %
  \item \xmlNode{features}, \xmlDesc{XML node, required parameter}, used to specify the information for the input variables.
  In this xml-node, the following xml sub-nodes need to be specified:
    \begin{itemize}
      \item \xmlNode{manifest},\xmlDesc{XML node, optional parameter}, used to indicate the input variables belongs to the original input space.
      It can accept the following child node:
        \begin{itemize}
          \item \xmlNode{variables},\xmlDesc{comma separated string, required field}, lists manifest variables.
          \item \xmlNode{dimensions}, \xmlDesc{comma separated integer, optional field}, lists the dimensions corresponding to the manifest variables.
          If not provided, the dimensions are determined by the order indices of given manifest variables.
        \end{itemize}
      \item \xmlNode{latent},\xmlDesc{XML node, optional parameter}, used to indicate the input variables belongs to the transformed space.
      It can accept the following child node:
        \begin{itemize}
          \item \xmlNode{variables},\xmlDesc{comma separated string, required field}, lists latent variables.
          \item \xmlNode{dimensions}, \xmlDesc{comma separated integer, optional field}, lists the dimensions corresponding to the latent variables.
          If not provided, the dimensions are determined by the order indices of given latent variables.
        \end{itemize}
      \nb At least one of the subnodes, i.e. \xmlNode{manifest} and \xmlNode{latent} needs to be specified.
    \end{itemize}
  %
  \item \xmlNode{targets}, \xmlDesc{comma separated string, required field}, lists output responses.
  %
  \item \xmlNode{mvnDistribution}, \xmlDesc{string, required field}, specifies the
  multivariate normal distribution name. The \xmlNode{MultivariateNormal} node must be present.
\end{itemize}
  %
  %
  Here is an example to show the user how to request the transformation matrix, the inverse transformation matrix, the
  manifest sensitivities and other quantities.
  %

\textbf{Example:}
\begin{lstlisting}[style=XML,morekeywords={name,subType,debug}]
<Simulation>
  ...
  <Models>
    ...
    <PostProcessor name='aUserDefinedName' subType='ImportanceRank'>
      <what>SensitivityIndex,ImportanceIndex,Transformation, InverseTransformation,ManifestSensitivity</what>
      <features>
        <manifest>
          <variables>x1,x2</variables>
          <dimensions>1,2</dimensions>
        </manifest>
        <latent>
          <variables>latent_1, latent_2</variables>
          <dimensions>1,2</dimensions>
        </latent>
      </features>
      <targets>y1,y2</targets>
      <mvnDistribution>MVN</mvnDistribution>
    </PostProcessor>
    ...
  </Models>
  ...
</Simulation>
\end{lstlisting}

%%%%% PP SafestPoint %%%%%%%
\subsubsection{SafestPoint}
\label{SafestPoint}
The \textbf{SafestPoint} post-processor provides the coordinates of the farthest
point from the limit surface that is given as an input.
%
The safest point coordinates are expected values of the coordinates of the
farthest points from the limit surface in the space of the ``controllable''
variables based on the probability distributions of the ``non-controllable''
variables.

The term ``controllable'' identifies those variables that are under control
during the system operation, while the ``non-controllable'' variables are
stochastic parameters affecting the system behaviour randomly.

The ``SafestPoint'' post-processor requires the set of points belonging to the
limit surface, which must be given as an input.
%
The probability distributions as ``Assembler Objects'' are required in the
``Distribution'' section for both ``controllable'' and ``non-controllable''
variables.

The sampling method used by the ``SafestPoint'' is a ``value'' or ``CDF'' grid.
%
At present only the ``equal'' grid type is available.

\ppType{Safest Point}{SafestPoint}

\begin{itemize}
  \item \xmlNode{Distribution}, \xmlDesc{Required}, represents the probability
  distributions of the ``controllable'' and ``non-controllable'' variables.
  %
  These are \textbf{Assembler Objects}, each of these nodes must contain 2
  attributes that are used to identify those within the simulation framework:
        \begin{itemize}
    \item \xmlAttr{class}, \xmlDesc{required string attribute}, is the main
    ``class'' the listed object is from.
                \item \xmlAttr{type}, \xmlDesc{required string attribute}, is the object
    identifier or sub-type.
        \end{itemize}
        \item \xmlNode{controllable} lists the controllable variables.
  %
  Each variable is associated with its name and the two items below:
        \begin{itemize}
                \item \xmlNode{distribution} names the probability distribution associated
    with the controllable variable.
    %
                \item \xmlNode{grid} specifies the \xmlAttr{type}, \xmlAttr{steps}, and
    tolerance of the sampling grid.
    %
        \end{itemize}
        \item \xmlNode{non-controllable} lists the non-controllable variables.
  %
  Each variable is associated with its name and the two items below:
        \begin{itemize}
                \item \xmlNode{distribution} names the probability distribution associated
    with the non-controllable variable.
    %
                \item \xmlNode{grid} specifies the \xmlAttr{type}, \xmlAttr{steps}, and
    tolerance of the sampling grid.
    %
                \end{itemize}
\end{itemize}

\textbf{Example:}
\begin{lstlisting}[style=XML,morekeywords={name,subType,class,type,steps}]
<Simulation>
  ...
    <Models>
    ...
    <PostProcessor name='SP' subType='SafestPoint'>
      <Distribution  class='Distributions'  type='Normal'>x1_dst</Distribution>
      <Distribution  class='Distributions'  type='Normal'>x2_dst</Distribution>
      <Distribution  class='Distributions'  type='Normal'>gammay_dst</Distribution>
      <controllable>
        <variable name='x1'>
          <distribution>x1_dst</distribution>
          <grid type='value' steps='20'>1</grid>
        </variable>
        <variable name='x2'>
          <distribution>x2_dst</distribution>
          <grid type='value' steps='20'>1</grid>
        </variable>
      </controllable>
      <non-controllable>
        <variable name='gammay'>
          <distribution>gammay_dst</distribution>
          <grid type='value' steps='20'>2</grid>
        </variable>
      </non-controllable>
    </PostProcessor>
    ...
  </Models>
  ...
</Simulation>
\end{lstlisting}
%%%%% PP LimitSurface %%%%%%%
\subsubsection{LimitSurface}
\label{LimitSurface}
The \textbf{LimitSurface} post-processor is aimed to identify the transition
zones that determine a change in the status of the system (Limit Surface).

\ppType{LimitSurface}{LimitSurface}

\begin{itemize}
  \item \xmlNode{parameters}, \xmlDesc{comma separated string, required field},
  lists the parameters that define the uncertain domain and from which the LS
  needs to be computed.
  \item \xmlNode{tolerance}, \xmlDesc{float, optional field}, sets the absolute
  value (in CDF) of the convergence tolerance.
 %
  This value defines the coarseness of the evaluation grid.
 %
 \default{1.0e-4}
  \item \xmlNode{side}, \xmlDesc{string, optional field}, in this node the user can specify
  which side of the limit surface needs to be computed. Three options are available:
  \\ \textit{negative},  Limit Surface corresponding to the goal function value of ``-1'';
  \\ \textit{positive}, Limit Surface corresponding to the goal function value of ``1'';
  \\ \textit{both}, either positive and negative Limit Surface is going to be computed.
  %
  %
\default{negative}
  % Assembler Objects
  \item \textbf{Assembler Objects} These objects are either required or optional
  depending on the functionality of the Adaptive Sampler.
  %
  The objects must be listed with a rigorous syntax that, except for the xml
  node tag, is common among all the objects.
  %
  Each of these nodes must contain 2 attributes that are used to map those
  within the simulation framework:
   \begin{itemize}
    \item \xmlAttr{class}, \xmlDesc{required string attribute}, is the main
    ``class'' of the listed object.
    %
    For example, it can be ``Models,'' ``Functions,'' etc.
    \item \xmlAttr{type}, \xmlDesc{required string attribute}, is the object
    identifier or sub-type.
    %
    For example, it can be ``ROM,'' ``External,'' etc.
    %
  \end{itemize}
  The \textbf{LimitSurface} post-processor requires or optionally accepts the
  following objects' types:
   \begin{itemize}
    \item \xmlNode{ROM}, \xmlDesc{string, optional field}, body of this xml
    node must contain the name of a ROM defined in the \xmlNode{Models} block
    (see section \ref{subsec:models_ROM}).
    \item \xmlNode{Function}, \xmlDesc{string, required field}, the body of
    this xml block needs to contain the name of an External Function defined
    within the \xmlNode{Functions} main block (see section \ref{sec:functions}).
    %
    This object represents the boolean function that defines the transition
    boundaries.
    %
    This function must implement a method called
    \textit{\_\_residuumSign(self)}, that returns either -1 or 1, depending on
    the system conditions (see section \ref{sec:functions}).
    %
    \end{itemize}
\end{itemize}

\textbf{Example:}
\begin{lstlisting}[style=XML,morekeywords={name,subType,debug,class,type}]
<Simulation>
 ...
 <Models>
  ...
    <PostProcessor name="computeLimitSurface" subType='LimitSurface' verbosity='debug'>
      <parameters>x0,y0</parameters>
      <ROM class='Models' type='ROM'>Acc</ROM>
      <!-- Here, you can add a ROM defined in Models block.
           If it is not Present, a nearest neighbor algorithm
           will be used.
       -->
      <Function class='Functions' type='External'>
        goalFunctionForLimitSurface
      </Function>
    </PostProcessor>
    ...
  </Models>
  ...
</Simulation>
\end{lstlisting}

%%%%% PP LimitSurfaceIntegral %%%%%%%

\subsubsection{LimitSurfaceIntegral}
\label{LimitSurfaceIntegral}
The \textbf{LimitSurfaceIntegral} post-processor is aimed to compute the likelihood (probability) of the event, whose boundaries are
represented by the Limit Surface (either from the LimitSurface post-processor or Adaptive sampling strategies).
The inputted Limit Surface needs to be, in the  \textbf{PostProcess} step, of type  \textbf{PointSet} and needs to contain
both boundary sides (-1.0, +1.0).
\\ The \textbf{LimitSurfaceIntegral} post-processor accepts as outputs both files (CSV) and/or  \textbf{PointSet}s.

\ppType{LimitSurfaceIntegral}{LimitSurfaceIntegral}
\begin{itemize}
\item \variableDescription
 \variableChildIntro
 \begin{itemize}
    \item   \xmlNode{distribution}, \xmlDesc{string,
               optional field}, name of the distribution that is associated to this variable.
              Its name needs to be contained in the \xmlNode{Distributions} block explained
              in Section \ref{sec:distributions}. If this node is not present, the  \xmlNode{lowerBound}
              and  \xmlNode{upperBound} XML nodes must be inputted.
   \item   \xmlNode{lowerBound}, \xmlDesc{float,
               optional field}, lower limit of integration domain for this dimension (variable).
               If this node is not present, the  \xmlNode{distribution} XML node must be inputted.
   \item   \xmlNode{upperBound}, \xmlDesc{float,
               optional field}, upper limit of integration domain for this dimension (variable).
               If this node is not present, the  \xmlNode{distribution} XML node must be inputted.
  \end{itemize}

    \item  \xmlNode{tolerance}, \xmlDesc{float, optional field}, specifies the tolerance for
               numerical integration confidence.
                \default{1.0e-4}
     \item  \xmlNode{integralType}, \xmlDesc{string, optional field}, specifies the type of integrations that
                need to be used. Currently only MonteCarlo integration is available
                \default{MonteCarlo}
     \item  \xmlNode{seed}, \xmlDesc{integer, optional field}, specifies the random number generator seed.
                \default{20021986}
     \item  \xmlNode{target}, \xmlDesc{string, optional field}, specifies the target name that represents
                the $f\left ( \bar{x} \right )$ that needs to be integrated.
                \default{last output found in the inputted PointSet}
\end{itemize}

\textbf{Example:}
\begin{lstlisting}[style=XML,morekeywords={name,subType,debug,class,type}]
<Simulation>
 ...
 <Models>
  ...
    <PostProcessor name="LimitSurfaceIntegralDistributions" subType='LimitSurfaceIntegral'>
        <tolerance>0.0001</tolerance>
        <integralType>MonteCarlo</integralType>
        <seed>20021986</seed>
        <target>goalFunctionOutput</target>
        <variable name='x0'>
          <distribution>x0_distrib</distribution>
        </variable>
        <variable name='y0'>
          <distribution>y0_distrib</distribution>
        </variable>
    </PostProcessor>
    <PostProcessor name="LimitSurfaceIntegralLowerUpperBounds" subType='LimitSurfaceIntegral'>
        <tolerance>0.0001</tolerance>
        <integralType>MonteCarlo</integralType>
        <seed>20021986</seed>
        <target>goalFunctionOutput</target>
        <variable name='x0'>
          <lowerBound>-2.0</lowerBound>
          <upperBound>12.0</upperBound>
        </variable>
        <variable name='y0'>
            <lowerBound>-1.0</lowerBound>
            <upperBound>11.0</upperBound>
        </variable>
    </PostProcessor>
    ...
  </Models>
  ...
</Simulation>
\end{lstlisting}



%%%%% PP External %%%%%%%
\subsubsection{External}
\label{External}
The \textbf{External} post-processor will execute an arbitrary python function
defined externally using the \textit{Functions} interface (see
Section~\ref{sec:functions} for more details).
%

\ppType{External}{External}

\begin{itemize}
  \item \xmlNode{method}, \xmlDesc{comma separated string, required field},
  lists the method names of an external Function that will be computed (each
  returning a post-processing value). The name of the method represents a new
  variable that can be stored in a new \textbf{\textit{DataObjects}} entity.
  \item \xmlNode{Function}, \xmlDesc{xml node, required string field}, specifies
  the name of a Function where the \textit{methods} listed above are defined.
  %
  \nb This name should match one of the Functions defined in the
  \xmlNode{Functions} block of the input file.
  %
  The objects must be listed with a rigorous syntax that, except for the XML
  node tag, is common among all the objects.
  %
  Each of these sub-nodes must contain 2 attributes that are used to map them
  within the simulation framework:

   \begin{itemize}
     \item \xmlAttr{class}, \xmlDesc{required string attribute}, is the main
     ``class'' the listed object is from, the only acceptable class for
     this post-processor is \xmlString{Functions};
     \item \xmlAttr{type}, \xmlDesc{required string attribute}, is the object
     identifier or sub-type, the only acceptable type for this post-processor is
     \xmlString{External}.
  \end{itemize}
\end{itemize}

  This Post-Processor accepts as Input/Output both \xmlString{PointSet} and \xmlString{HistorySet}:
   \begin{itemize}
    \item If a \xmlString{PointSet}  is used as Input, the parameters are passed in the external  \xmlString{Function}
  as numpy arrays. The methods' return type must be either a new array or a scalar. In the following it is reported an example
  with two methods, one that returns a scalar and the other one that returns an array:
      \begin{lstlisting}[language=python]
import numpy as np
def sum(self):
  return np.sum(self.aParameterInPointSet)

def sumTwoArraysAndReturnAnotherone(self):
  return self.aParamInPointSet1+self.aParamInPointSet2
      \end{lstlisting}
    \item If a \xmlString{HistorySet}  is used as Input, the parameters are passed in the external  \xmlString{Function}
     as a list of numpy arrays. The methods' return type must be either a new list of arrays (if the Output is another
     \xmlString{HistorySet}), a scalar or a single array (if the  Output is  \xmlString{PointSet} . In the following it
     is reported an example
     with two methods, one that returns a new list of arrays (Output = HistorySet) and the other one that returns an array (Output =
     PointSet):
      \begin{lstlisting}[language=python]
import numpy as np
def newHistorySetParameter(self):
  x = []*len(self.time)
  for history in range(len(self.time)):
    for ts in range(len(self.time[history])):
      if self.time[history][ts] >= 0.001: break
    x[history] = self.x[history][ts:]
  return x

def aNewPointSetParameter(self):
  x = []*len(self.time)
  for history in range(len(self.time)):
    x[history] = self.x[history][-1]
  return x
      \end{lstlisting}
   \end{itemize}

\textbf{Example:}
\begin{lstlisting}[style=XML,morekeywords={subType,debug,name,class,type}]
<Simulation>
  ...
  <Models>
    ...
    <PostProcessor name="externalPP" subType='External' verbosity='debug'>
      <method>Delta,Sum</method>
      <Function class='Functions' type='External'>operators</Function>
        <!-- Here, you can add a Function defined in the
             Functions block. This should be present or
             else RAVEN will not know where to find the
             defined methods. -->
    </PostProcessor>
    ...
  </Models>
  ...
</Simulation>
\end{lstlisting}

%%%%% PP TopologicalDecomposition %%%%%%%
\subsubsection{TopologicalDecomposition}
\label{TopologicalDecomposition}
The \textbf{TopologicalDecomposition} post-processor will compute an
approximated hierarchical Morse-Smale complex which will add two columns to a
dataset, namely \texttt{minLabel} and \texttt{maxLabel} that can be used to
decompose a dataset.
%

The topological post-processor can also be run in `interactive' mode, that is
by passing the keyword \texttt{interactive} to the command line of RAVEN's
driver.
%
In this way, RAVEN will initiate an interactive UI that allows one to explore
the topological hierarchy in real-time and adjust the simplification setting
before adjusting a dataset. Use in interactive mode will replace the parameter
\xmlNode{simplification} described below with whatever setting is set in the UI
upon exiting it.

In order to use the \textbf{TopologicalDecomposition} post-processor, the user
needs to set the attribute \xmlAttr{subType}:
\xmlNode{PostProcessor \xmlAttr{subType}=\xmlString{TopologicalDecomposition}}.
The following is a list of acceptable sub-nodes:
\begin{itemize}
  \item \xmlNode{graph} \xmlDesc{, string, optional field}, specifies the type
  of neighborhood graph used in the algorithm, available options are:
  \begin{itemize}
    \item \texttt{beta skeleton}
    \item \texttt{relaxed beta skeleton}
    \item \texttt{approximate knn}
    %\item Delaunay \textit{(disabled)}
  \end{itemize}
  \default{\texttt{beta skeleton}}
  \item \xmlNode{gradient}, \xmlDesc{string, optional field}, specifies the
  method used for estimating the gradient, available options are:
  \begin{itemize}
    \item \texttt{steepest}
    %\item \xmlString{maxflow} \textit{(disabled)}
  \end{itemize}
  \default{\texttt{steepest}}
  \item \xmlNode{beta}, \xmlDesc{float in the range: (0,2], optional field}, is
  only used when the \xmlNode{graph} is set to \texttt{beta skeleton} or
  \texttt{relaxed beta skeleton}.
  \default{1.0}
  \item \xmlNode{knn}, \xmlDesc{integer, optional field}, is the number of
  neighbors when using the \xmlString{approximate knn} for the \xmlNode{graph}
  sub-node and used to speed up the computation of other graphs by using the
  approximate knn graph as a starting point for pruning. -1 means use a fully
  connected graph.
  \default{-1}
  \item \xmlNode{weighted}, \xmlDesc{boolean, optional}, a flag that specifies
  whether the regression models should be probability weighted.
  \default{False}
  \item \xmlNode{interactive}, if this node is present \emph{and} the user has
  specified the keyword \texttt{interactive} at the command line, then this will
  initiate a graphical interface for exploring the different simplification
  levels of the topological hierarchy. Upon exit of the graphical interface, the
  specified simplification level will be updated to use the last value of the
  graphical interface before writing any ``output'' results.
  \item \xmlNode{persistence}, \xmlDesc{string, optional field}, specifies how
  to define the hierarchical simplification by assigning a value to each local
  minimum and maximum according to the one of the strategy options below:
  \begin{itemize}
    \item \texttt{difference} - The function value difference between the
    extremum and its closest-valued neighboring saddle.
    \item \texttt{probability} - The probability integral computed as the
    sum of the probability of each point in a cluster divided by the count of
    the cluster.
    \item \texttt{count} - The count of points that flow to or from the
    extremum.
    % \item \xmlString{area} - The area enclosed by the manifold that flows to
    % or from the extremum.
  \end{itemize}
  \default{\texttt{difference}}
  \item \xmlNode{simplification}, \xmlDesc{float, optional field}, specifies the
  amount of noise reduction to apply before returning labels.
  \default{0}
  \item \xmlNode{parameters}, \xmlDesc{comma separated string, required field},
  lists the parameters defining the input space.
  \item \xmlNode{response}, \xmlDesc{string, required field}, is a single
  variable name defining the scalar output space.
\end{itemize}
\textbf{Example:}
\begin{lstlisting}[style=XML,morekeywords={subType}]
<Simulation>
  ...
  <Models>
    ...
    <PostProcessor name="***" subType='TopologicalDecomposition'>
      <graph>beta skeleton</graph>
      <gradient>steepest</gradient>
      <beta>1</beta>
      <knn>8</knn>
      <normalization>None</normalization>
      <parameters>X,Y</parameters>
      <response>Z</response>
      <weighted>true</weighted>
      <simplification>0.3</simplification>
      <persistence>difference</persistence>
    </PostProcessor>
    ...
  <Models>
  ...
<Simulation>
\end{lstlisting}

%%%%% PP DataMining %%%%%%%
\subsubsection{DataMining}
\label{subsubsec:DataMining}

Knowledge discovery in databases (KDD) is the process of discovering
 useful knowledge from a collection of data. This widely used data
mining technique is a process that includes data preparation and
selection, data cleansing, incorporating prior knowledge on data
sets and interpreting accurate solutions from the observed results.
Major KDD application areas include marketing, fraud detection,
telecommunication and manufacturing.

DataMining is the analysis step of the KDD process. The overall of
the data mining process is to extract information from a data set
and transform it into an understandable structure for further use.
The actual data mining task is the
automatic or semi-automatic analysis of large quantities of data
to extract previously unknown, interesting patterns such as groups
 of data records (cluster analysis), unusual records (anomaly
detection), and dependencies (association rule mining).
\\
%
In order to use the \textbf{DataMining} post-processor, the user
needs to set the attribute \xmlAttr{subType}: \\
\\
\xmlNode{PostProcessor \xmlAttr{subType}=
\xmlString{DataMining}}. \\
\\
The following is a list of acceptable sub-nodes:
\begin{itemize}
  \item \xmlNode{KDD} \xmlDesc{string,required field}, the subnodes specifies
  the necessary information for the algorithm to be used in the postprocessor.
  The \xmlNode{KDD} has the required attribute: \xmlAttr{lib}, the name of the
  library the algorithm belongs to. Current algorithms applied in the KDD model
  is based on SciKit-Learn library. Thus currently there is only one library:
  \begin{itemize}
    \item \xmlString{SciKitLearn}
  \end{itemize}
  The \xmlNode{KDD} has the optional attribute: \xmlAttr{labelFeature}, the name
  associated to labels or dimensions generated by the \textbf{DataMining}
  post-processor.
  The default name depends on the type of algorithm employed.
  For clustering and mixture models it is the name of the PostProcessor
  followed by ``Labels'' (e.g., if the name of a clustering PostProcessor is
  ``kMeans'' then the default name associated to the labels is ``kMeansLabels''
  if not specified in the attribute \xmlAttr{labelFeature}).
  For decomposition and manifold models, the default names are the name of the
  PostProcessor followed by ``Dimension'' and an integer identifier beginning
  with 1. (e.g., if the name of a dimensionality reduction PostProcessor is
  ``dr'' and the user specifies 3 components, then the output dataObject will
  have three new outputs named ``drDimension1,'' ``drDimension2,'' and
  ``drDimension3.'').
  \nb The ``Labels'' are automatically added to the output \textbf{DataObjects}. It
  is also accessible by the users using the variable name defined above.
\end{itemize}


\paragraph{SciKitLearn}
\xmlString{SciKitLearn} is based on algorithms in SciKit-Learn library, and it performs data mining over PointSet and HistorySet. Note that for HistorySet's \xmlString{SciKitLearn} performs the task given in \xmlNode{SKLType} (see below) for each time step, and so only synchronized HistorySet can be used as input to this model. For unsynchronized HistorySet, use \xmlString{HistorySetSync} method in \xmlString{Interfaced} post-processor to synchronize the input data before using \xmlString{SciKitLearn}. The rest of this subsection and following subsection is dedicated to the \xmlString{SciKitLearn} library.

The temporal variable for a HistorySet \xmlString{SciKitLearn} is specified in the \xmlNode{pivotParameter} node:
\begin{itemize}
  \item \xmlNode{pivotParameter}, \xmlDesc{string, optional parameter} specifies the pivot variable (e.g., time, etc) in the input HistorySet.
      \default{None}.
\end{itemize}

The algorithm for the dataMining is chosen by the subnode \xmlNode{SKLType} under the parent node
\xmlNode{KDD}. The format is same as in \ref{subsubsec:SciKitLearn}. However, for the completeness
sake, it is repeated here.

The data that are used in the training of the \textbf{DataMining}
postprocessor are suplied with subnode \xmlNode{Features} in the parent node
 \xmlNode{KDD}.


\begin{itemize}
  \item \xmlNode{SKLtype}, \xmlDesc{vertical bar (\texttt{|}) separated string,
  required field}, contains a string that represents the data mining algorithm
  to be used.
  %
  As mentioned, its format is:\\
  \xmlNode{SKLtype}\texttt{mainSKLclass|algorithm}\xmlNode{/SKLtype} where the
  first word (before the ``\texttt{|}'' symbol) represents the main class of
  algorithms, and the second word (after the ``\texttt{|}'' symbol) represents
  the specific algorithm.
  %
  \item \xmlNode{Features}, \xmlDesc{string, required field}, defines the data
  to be used for training the data mining algorithm. It can be:
  \begin{itemize}
	\item the name of the variable in the defined dataObject entity
	\item the location (i.e. input or output). In this case the data mining
        is applied to all the variables in the defined space.
  \end{itemize}
\end{itemize}

The \xmlNode{KDD} node can have either optional or required subnodes depending
 on the dataMining algorithm used. The possible subnodes will be described separately
 for each algorithm below. The time dependent clustering data mining algorithms have a \xmlNode{reOrderStep} option that will try and keep the same labels on the clusters.  The higher the number, the longer the history that the clustering algorithm will look through to maintain the same labeling between time steps.

All the available algorithms are described in the following sections.

\paragraph{Gaussian mixture models}
\label{paragraph:GMM}

A Gaussian mixture model is a probabilistic model that assumes all
 the data points are generated from a mixture of a finite number of
 Gaussian distributions with unknown parameters.
\\
Scikit-learn implements different classes to estimate Gaussian
mixture models, that correspond to different estimation strategies,
 detailed below.

\subparagraph{ GMM classifier} \hfill
\label{subparagraph:GMMClass}

The GMM object implements the expectation-maximization (EM)
algorithm for fitting mixture-of-Gaussian models. The GMM comes with different options
 to constrain the covariance of  the difference classes estimated: spherical, diagonal, tied or
 full covariance.

\skltype{Gaussian Mixture Model}{mixture|GMM}
\begin{itemize}
	\item \xmlNode{n\_components}, \xmlDesc{integer, optional
field} Number of mixture components. \default{1}
	\item \xmlNode{covariance\_type}, \xmlDesc{string, optional
field}, describes the type of covariance parameters to use.
Must be one of ‘spherical’, ‘tied’, ‘diag’, ‘full’. \default{diag}
	\item \xmlNode{random\_state}, \xmlDesc{integer seed or random
 number generator instance, optional field},  A random number
generator instance \default{0 or None}
	\item \xmlNode{min\_covar}, \xmlDesc{float, optional field},
 Floor on the diagonal of the covariance matrix to prevent overfitting.
 \default{1e-3}.
	\item \xmlNode{thresh}, \xmlDesc{float, optional field},
convergence threshold. \default{0.01}
	\item \xmlNode{n\_iter}, \xmlDesc{integer, optional field},
Number of EM iterations to perform. \default{100}
	\item \xmlNode{n\_init}, \xmlDesc{integer, optional},
Number of initializations to perform. the best results is kept.
\default{1}
	\item \xmlNode{init\_params}, \xmlDesc{string, optional
field},  The method used to initialize the weights, the means and the precision. Must be one of 
 ``kmeans'' (responsibilities are initialized using kmeans) or ``random'' (responsibilities are 
 initialized randomly)
 \default{kmeans}
\end{itemize}

\textbf{Example:}
\begin{lstlisting}[style=XML,morekeywords={subType}]
<Simulation>
  ...
  <Models>
    ...
      <PostProcessor name='PostProcessorName' subType='DataMining'>
          <KDD lib='SciKitLearn'>
              <Features>variableName</Features>
              <SKLtype>mixture|GMM</SKLtype>
              <n_components>2</n_components>
              <covariance_type>spherical</covariance_type>
          </KDD>
      </PostProcessor>
    ...
  <Models>
  ...
<Simulation>
\end{lstlisting}

\subparagraph{ Variational GMM Classifier (VBGMM)} \hfill
\label{subparagraph:VBGMM}

The VBGMM object implements a variant of the Gaussian mixture model
 with variational inference algorithms. The API is identical to GMM.

\skltype{Variational Gaussian Mixture Model}{mixture|VBGMM}
\begin{itemize}
	\item \xmlNode{n\_components}, \xmlDesc{integer, optional
field} Number of mixture components. \default{1}
	\item \xmlNode{covariance\_type}, \xmlDesc{string, optional
field}, describes the type of covariance parameters to use.
Must be one of ‘spherical’, ‘tied’, ‘diag’, ‘full’. \default{diag}
	\item \xmlNode{alpha}, \xmlDesc{float, optional field},
represents the concentration parameter of the Dirichlet process.
Intuitively, the Dirichlet Process is as likely to start a new cluster
 for a point as it is to add that point to a cluster with alpha
 elements. A higher alpha means more clusters, as the expected
number of clusters is ${\alpha*log(N)}$. \default{1}.
\end{itemize}

\paragraph{ Clustering }
\label{paragraph:Clustering}

Clustering of unlabeled data can be performed with this subType of
 the DataMining PostProcessor.

An overview of the different clustering algorithms is given in
Table\ref{tab:clustering}.

\begin{table}[!htbp]
  \centering
  \caption{Overview of Clustering Methods}
  \label{tab:clustering}
  \begin{tabular}{| L{2.5cm} | L{2.5cm} | L{2.5cm} | L{3.5cm} | L{2.75cm} |} \hline
    {\bf Method name} & {\bf Parameters} & {\bf Scalability} & {\bf
Usecase} & {\bf Geometry (metric used)} \\ \hline
    K-Means  & number of clusters & Very large n\_samples, medium
n\_clusters with MiniBatch code & General-purpose, even cluster size,
flat geometry, not too many clusters & Distances between points
 \\ \hline
    Affinity propagation & damping, sample preference & Not scalable with
n\_samples & Many clusters, uneven cluster size, non-flat geometry &
Graph distance (e.g. nearest-neighbor graph)       \\ \hline
    Mean-shift & bandwidth & Not scalable with n\_samples & Many clusters,
 uneven cluster size, non-flat geometry & Distances between points \\ \hline
    Spectral clustering & number of clusters & Medium n\_samples, small
n\_clusters & Few clusters, even cluster size, non-flat geometry &
Graph distance (e.g. nearest-neighbor graph)       \\ \hline
    Ward hierarchical clustering & number of clusters & Large n\_samples
and n\_clusters & Many clusters, possibly connectivity constraints &
Distances between points       \\ \hline
    Agglomerative clustering & number of clusters, linkage type, distance
 & Large n\_samples and n\_clusters & Many clusters, possibly
connectivity constraints, non Euclidean distances & Any pairwise
distance       \\ \hline
    DBSCAN & neighborhood size & Very large n\_samples, medium n\_clusters
 & Non-flat geometry, uneven cluster sizes & Distances between nearest
 points       \\ \hline
    Gaussian mixtures & many & Not scalable & Flat geometry, good for
 density estimation & Mahalanobis distances to centers       \\ \hline
  \end{tabular}
\end{table}

\FloatBarrier

\subparagraph{K-Means Clustering} \hfill
\label{subparagraph:KMeans}

The KMeans algorithm clusters data by trying to separate samples in n groups
of equal variance, minimizing a criterion known as the inertia or within-cluster
sum-of-squares. This algorithm requires the number of clusters to be specified.
 It scales well to large number of samples and has been used across a large
range of application areas in many different fields

\skltype{ K-Means Clustering}{cluster|KMeans}
\begin{itemize}
	\item \xmlNode{n\_clusters}, \xmlDesc{integer, optional field}
The number of clusters to form as well as the number of centroids to
generate. \default{8}
	\item \xmlNode{max\_iter}, \xmlDesc{integer, optional field},
Maximum number of iterations of the k-means algorithm for a single run.
\default{300}
	\item \xmlNode{n\_init}, \xmlDesc{integer, optional field},
Number of time the k-means algorithm will be run with different centroid
 seeds. The final results will be the best output of n\_init consecutive
 runs in terms of inertia. \default{3}
	\item \xmlNode{init}, \xmlDesc{string, optional},
Method for initialization, k-means++’, ‘random’ or an ndarray:
		\begin{itemize}
			\item ‘k-means++’ : selects initial cluster
centers for k-mean clustering in a smart way to speed up convergence.
			\item ‘random’: choose k observations (rows) at
 random from data for the initial centroids.
			\item If an ndarray is passed, it should be of
 shape (n\_clusters, n\_features) and gives the initial centers.
		\end{itemize}
	\item \xmlNode{precompute\_distances}, \xmlDesc{boolean, optional
field}, Precompute distances (if true faster but takes more memory).
\default{true}
	\item \xmlNode{tol}, \xmlDesc{float, optional field}, Relative
tolerance with regards to inertia to declare convergence. \default{1e-4}
	\item \xmlNode{n\_jobs}, \xmlDesc{integer, optional field}, The number
of jobs to use for the computation. This works by breaking down the pairwise
 matrix into n jobs even slices and computing them in parallel. If -1 all CPUs
 are used. If 1 is given, no parallel computing code is used at all, which is
useful for debugging. For n\_jobs below -1, (n\_cpus + 1 + n\_jobs) are used. Thus
 for n\_jobs = -2, all CPUs but one are used. \default{1}
	\item \xmlNode{random\_state}, \xmlDesc{integer or numpy. RandomState,
 optional field} The generator used to initialize the centers. If an integer
 is given, it fixes the seed. \default{the global numpy random number generator}.
\end{itemize}

\textbf{Example:}
\begin{lstlisting}[style=XML,morekeywords={subType}]
<Simulation>
  ...
  <Models>
    ...
      <PostProcessor name='PostProcessorName' subType='DataMining'>
          <KDD lib='SciKitLearn'>
              <Features>variableName</Features>
              <SKLtype>cluster|KMeans</SKLtype>
              <n_clusters>2</n_clusters>
              <tol>0.0001</tol>
              <init>random</init>
          </KDD>
      </PostProcessor>
    ...
  <Models>
  ...
<Simulation>
\end{lstlisting}


\subparagraph{  Mini Batch K-Means } \hfill
\label{subparagraph:MiniBatch}

The MiniBatchKMeans is a variant of the KMeans algorithm which uses
mini-batches to reduce the computation time, while still attempting
 to optimize the same objective function. Mini-batches are subsets of
 the input data, randomly sampled in each training iteration.

MiniBatchKMeans converges faster than KMeans, but the quality of the
results is reduced. In practice this difference in quality can be
 quite small.

\skltype{ Mini Batch K-Means Clustering}{cluster|MiniBatchKMeans}
\begin{itemize}
	\item \xmlNode{n\_clusters}, \xmlDesc{integer, optional field}
The number of clusters to form as well as the number of centroids to
generate. \default{8}
	\item \xmlNode{max\_iter}, \xmlDesc{integer, optional field},
Maximum number of iterations of the k-means algorithm for a single run.
\default{100}
	\item \xmlNode{max\_no\_improvement}, \xmlDesc{integer, optional
firld}, Control early stopping based on the consecutive number of mini
 batches that does not yield an improvement on the smoothed inertia.
To disable convergence detection based on inertia, set
max\_no\_improvement to None. \default{10}
	\item \xmlNode{tol}, \xmlDesc{float, optional field}, Control
 early stopping based on the relative center changes as measured by a
smoothed, variance-normalized of the mean center squared position changes.
 This early stopping heuristics is closer to the one used for the batch
 variant of the algorithms but induces a slight computational and memory
overhead over the inertia heuristic. To disable convergence detection
based on normalized center change, set tol to 0.0 (default). \default{0.0}
	\item \xmlNode{batch\_size}, \xmlDesc{integer, optional field},
Size of the mini batches. \default{100}
	\item{init\_size}, \xmlDesc{integer, optional field}, Number of
samples to randomly sample for speeding up the initialization
 (sometimes at the expense of accuracy): the only algorithm is initialized
 by running a batch KMeans on a random subset of the data.
\textit{This needs to be larger than k.}, \default{3 * \xmlNode{batch\_size}}
	\item \xmlNode{init}, \xmlDesc{string, optional},
Method for initialization, k-means++’, ‘random’ or an ndarray:
		\begin{itemize}
			\item ‘k-means++’ : selects initial cluster
centers for k-mean clustering in a smart way to speed up convergence.
			\item ‘random’: choose k observations (rows) at
 random from data for the initial centroids.
			\item If an ndarray is passed, it should be of
 shape (n\_clusters, n\_features) and gives the initial centers.
		\end{itemize}
	\item \xmlNode{precompute\_distances}, \xmlDesc{boolean, optional
field}, Precompute distances (if true faster but takes more memory).
\default{true}
	\item \xmlNode{n\_init}, \xmlDesc{integer, optional field},
Number of time the k-means algorithm will be run with different centroid
 seeds. The final results will be the best output of n\_init consecutive
 runs in terms of inertia. \default{3}
	\item \xmlNode{compute\_labels}, \xmlDesc{boolean, optional field},
Compute label assignment and inertia for the complete dataset once the
 minibatch optimization has converged in fit. \default{True}
	\item \xmlNode{random\_state}, \xmlDesc{integer or numpy.RandomState,
 optional field} The generator used to initialize the centers. If an integer
 is given, it fixes the seed. \default{the global numpy random number generator}.
	\item{reassignment\_ratio}, \xmlNode{float, optional field}, Control
the fraction of the maximum number of counts for a center to be reassigned.
A higher value means that low count centers are more easily reassigned, which
 means that the model will take longer to converge, but should converge in a
better clustering. \default{0.01}
\end{itemize}

\subparagraph{Affinity Propagation} \hfill
\label{subparagraph:Affinity}

AffinityPropagation creates clusters by sending messages between pairs of
samples until convergence. A dataset is then described using a small number
of exemplars, which are identified as those most representative of other
samples. The messages sent between pairs represent the suitability for one
sample to be the exemplar of the other, which is updated in response to the
values from other pairs. This updating happens iteratively until convergence,
 at which point the final exemplars are chosen, and hence the final clustering
 is given.

\skltype{ AffinityPropogation Clustering}{cluster|AffinityPropogation}
\begin{itemize}
	\item \xmlNode{damping}, \xmlDesc{float, optional field}, Damping factor
 between 0.5 and 1. \default{0.5}
	\item \xmlNode{convergence\_iter}, \xmlDesc{integer, optional field},
Number of iterations with no change in the number of estimated clusters that
stops the convergence. \default{15}
	\item \xmlNode{max\_iter}, \xmlDesc{integer, optional field}, Maximum
 number of iterations. \default{200}
	\item \xmlNode{copy}, \xmlDesc{boolean, optional field}, Make a copy of
input data or not. \default{True}
	\item \xmlNode{preference}, \xmlDesc{array-like, shape (n\_samples,)
or float, optional field}, Preferences for each point - points with larger
values of preferences are more likely to be chosen as exemplars. The number
of exemplars, ie of clusters, is influenced by the input preferences value.
\default{If the preferences are not passed as arguments, they will be set to the median of
 the input similarities.}
	\item \xmlNode{affinity}, \xmlDesc{string, optional field},Which affinity to use.
 At the moment precomputed and euclidean are supported. euclidean uses the negative squared
euclidean distance between points. \default{``euclidean``}
	\item \xmlNode{verbose}, \xmlDesc{boolean, optional field}, Whether to be verbose.
\default{False}
\end{itemize}

\subparagraph{ Mean Shift } \hfill
\label{subparagraph:MeanShift}

MeanShift clustering aims to discover blobs in a smooth density of samples. It is
 a centroid based algorithm, which works by updating candidates for centroids to be
the mean of the points within a given region. These candidates are then filtered in
a post-processing stage to eliminate near-duplicates to form the final set of centroids.

\skltype{ Mean Shift Clustering}{cluster|MeanShift}
\begin{itemize}
	\item \xmlNode{bandwidth}, \xmlDesc{float, optional field}, Bandwidth used
in the RBF kernel. If not given, the bandwidth is estimated using
\textit{sklearn.cluster.estimate\_bandwidth}; see the documentation for that function for
 hints on scalability.
	\item \xmlNode{seeds}, \xmlDesc{array, shape=[n\_samples, n\_features],
optional field}, Seeds used to initialize kernels. If not set, the seeds are
calculated by \textit{clustering.get\_bin\_seeds} with bandwidth as the grid size and
 default values for other parameters.
	\item \xmlNode{bin\_seeding}, \xmlDesc{boolean, optional field}, If true,
 initial kernel locations are not locations of all points, but rather the
 location of the discretized version of points, where points are binned onto
 a grid whose coarseness corresponds to the bandwidth. Setting this option
to True will speed up the algorithm because fewer seeds will be initialized.
 \default{False} Ignored if seeds argument is not None.
	\item \xmlNode{min\_bin\_freq}, \xmlDesc{integer, optional field},
To speed up the algorithm, accept only those bins with at least min\_bin\_freq
 points as seeds. \default{1}.
	\item \xmlNode{cluster\_all}, \xmlDesc{boolean, optional field}, If true,
 then all points are clustered, even those orphans that are not within any
kernel. Orphans are assigned to the nearest kernel. If false, then orphans
are given cluster label -1. \default{True}
\end{itemize}


\subparagraph{Spectral clustering} \hfill
\label{subparagraph:Spectral}

SpectralClustering does a low-dimension embedding of the affinity matrix between
 samples, followed by a \textit{KMeans} in the low dimensional space. It is
especially efficient if the affinity matrix is sparse and the pyamg module is
installed.

\skltype{Spectral Clustering}{cluster|Spectral}
\begin{itemize}
	\item \xmlNode{n\_clusters}, \xmlDesc{integer, optional field},
	The dimension of the projection subspace.\default{8}
	%
	\item \xmlNode{affinity}, \xmlDesc{string, array-like or callable, optional
	  field}, If a string, this may be one of:
	\begin{itemize}
		\item ‘nearest\_neighbors’,
		\item ‘precomputed’,
		\item ‘rbf’ or
		\item one of the kernels supported by \textit{sklearn.metrics.pairwise\_kernels}.
	\end{itemize}
	Only kernels that produce similarity scores (non-negative values that increase
	with similarity) should be used. This property is not checked by the clustering
	 algorithm. \default{‘rbf’}
	%
	\item \xmlNode{gamma}, \xmlDesc{float, optional field}, Scaling factor of RBF,
	polynomial, exponential $chi^2$ and sigmoid affinity kernel.
	Ignored for $affinity='nearest\_neighbors'$. \default{1.0}
	%
	\item \xmlNode{degree}, \xmlDesc{float, optional field}, Degree of the polynomial
	 kernel. Ignored by other kernels. \default{3}
	%
	\item \xmlNode{coef0}, \xmlDesc{float, optional field}, Zero coefficient for
	polynomial and sigmoid kernels. Ignored by other kernels. \default{1}
	%
	\item \xmlNode{n\_neighbors}, \xmlDesc{integer, optional field}, Number of neighbors
	to use when constructing the affinity matrix using the nearest neighbors method.
	Ignored for affinity='rbf'. \default{10}
	%
	\item \xmlNode{eigen\_solver} \xmlDesc{string, optional field},  The eigenvalue
	decomposition strategy to use:
	\begin{itemize}
		\item None,
		\item ‘arpack’,
		\item ‘lobpcg’, or
		\item ‘amg’
	\end{itemize}
	\nb{AMG requires pyamg to be installed. It can be faster on very large, sparse
	problems, but may also lead to instabilities}
	%
	\item \xmlNode{random\_state}, \xmlDesc{integer seed, RandomState instance,
	 or None, optional field}, A pseudo random number generator used for the
	initialization of the lobpcg eigen vectors decomposition when $eigen_solver == ‘amg’$
	 and by the K-Means initialization. \default{None}
	%
	\item \xmlNode{n\_init}, \xmlDesc{integer, optional field}, Number of time the
	 k-means algorithm will be run with different centroid seeds. The final results
	 will be the best output of n\_init consecutive runs in terms of inertia.
	\default{10}
	%
	\item \xmlNode{eigen\_tol}, \xmlDesc{float, optional field}, Stopping criterion
	 for eigendecomposition of the Laplacian matrix when using arpack eigen\_solver.
	\default{0.0}
	%
	\item \xmlNode{assign\_labels}, \xmlDesc{string, optional field}, The strategy to
	use to assign labels in the embedding space. There are two ways to assign labels
	after the laplacian embedding:
	\begin{itemize}
		\item ‘kmeans’,
		\item ‘discretize’
	\end{itemize}
	 k-means can be applied and is a popular choice. But it can also be sensitive
	to initialization. Discretization is another approach which is less sensitive
	 to random initialization. \default{‘kmeans’}
	%
	\item \xmlNode{kernel\_params}, \xmlDesc{dictionary of string to any, optional
	 field}, Parameters (keyword arguments) and values for kernel passed as
	callable object. Ignored by other kernels. \default{None}
\end{itemize}

\textbf{Notes} \\
If you have an affinity matrix, such as a distance matrix, for which 0 means identical
elements, and high values means very dissimilar elements, it can be transformed in a
similarity matrix that is well suited for the algorithm by applying the Gaussian
 (RBF, heat) kernel:
\begin{equation}
np.exp(- X ** 2 / (2. * delta ** 2))
\end{equation}
Another alternative is to take a symmetric version of the k nearest neighbors
connectivity matrix of the points.
If the \textit{pyamg} package is installed, it is used: this greatly speeds
up computation.

\subparagraph{DBSCAN Clustering} \hfill
\label{subparagraph:DBSCAN}

The Density-Based Spatial Clustering of Applications with Noise (DBSCAN)
 algorithm views clusters as areas of high density separated by
areas of low density. Due to this rather generic view, clusters found by
DBSCAN can be any shape, as opposed to k-means which assumes that clusters
 are convex shaped.

\skltype{DBSCAN Clustering}{cluster|DBSCAN}
\begin{itemize}
	\item \xmlNode{eps}, \xmlDesc{float, optional field}, The maximum
	distance between two samples for them to be considered as in the
	 same neighborhood. \default{0.5}
	%
	\item \xmlNode{min\_samples}, \xmlDesc{integer, optional field},
	The number of samples in a neighborhood for a point to be
	considered as a core point. \default{5}
	%
	\item \xmlNode{metric}, \xmlDesc{string, or callable, optional field}
	The metric to use when calculating distance between instances in
	 a feature array. If metric is a string or callable, it must be one
	of the options allowed by \textit{metrics.pairwise.calculate\_distance}
	 for its metric parameter. If metric is “precomputed”, X is assumed
	 to be a distance matrix and must be square. \default{'euclidean'}
	%
	\item \xmlNode{random\_state}, \xmlDesc{numpy.RandomState,
	 optional field}, The generator used to initialize the centers.
	\default{numpy.random}.
\end{itemize}

\subparagraph{Agglomerative Clustering } \hfill
\label{subparagraph:agglomerative}

Hierarchical clustering is a general family of clustering algorithms that build nested clusters by merging or splitting them successively.
This hierarchy of clusters is represented as a tree (or dendrogram).
The root of the tree is the unique cluster that gathers all of the samples, the leaves being the clusters with only one sample.
The AgglomerativeClustering object performs a hierarchical clustering using a bottom up approach: each observation starts in its own cluster,
and clusters are successively merged together. The linkage criteria determines the metric used for the merge strategy:
\begin{itemize}
  \item Ward: it minimizes the sum of squared differences within all clusters. It is a variance-minimizing approach and in this sense
is similar to the k-means objective function but tackled with an agglomerative hierarchical approach.
  \item Maximum or complete linkage: it minimizes the maximum distance between observations of pairs of clusters.
  \item Average linkage: it minimizes the average of the distances between all observations of pairs of clusters.
\end{itemize}

AgglomerativeClustering can also scale to large number of samples when it is used jointly with a connectivity matrix,
but is computationally expensive when no connectivity constraints are added between samples: it considers at each step all of the possible merges.

\skltype{Agglomerative Clustering}{cluster|Agglomerative}
\begin{itemize}
  \item \xmlNode{n\_clusters}, \xmlDesc{int, optional field}, The number of clusters to find. \default{2}
  \item \xmlNode{connectivity}, \xmlDesc{array like or callable, optional field}, Connectivity matrix. Defines for each sample the neighboring samples
   following a given structure of the data. This can be a connectivity matrix itself or a callable that transforms the data into a connectivity matrix,
   such as derived from kneighbors graph. Default is None, i.e, the hierarchical clustering algorithm is unstructured. \default{None}
  \item \xmlNode{affinity}, \xmlDesc{string or callable, optional field}, Metric used to compute the linkage. Can be ``euclidean'', ``$l1$'', ``$l2$'',
   ``manhattan'',``cosine'', or ``precomputed''. If linkage is ``ward'', only ``euclidean'' is accepted. \default{euclidean}
%  \item \xmlNode{memory}, \xmlDesc{Instance of joblib.Memory or string, optional field}, Used to cache the output of the computation of the tree.
%   By default, no caching is done. If a string is given, it is the path to the caching directory.
  \item \xmlNode{n\_components}, \xmlDesc{int, optional field}, Number of connected components. If None the number of connected components is estimated
  from the connectivity matrix. NOTE: This parameter is now directly determined from the connectivity matrix and will be removed in $0.18$.
%  \item \xmlNode{compute\_full\_tree}, \xmlDesc{bool or 'auto', optional field}, Stop early the construction of the tree at \xmlNode{n\_clusters}.
%  This is useful to decrease computation time if the number of clusters is not small compared to the number of samples. This option is useful only
%  when specifying a connectivity matrix. Note also that when varying the number of clusters and using caching, it may be advantageous to compute the full tree.
  \item \xmlNode{linkage}, \xmlDesc{{ward,complete,average}, optional field}, Which linkage criterion to use. The linkage criterion determines which distance
  to use between sets of observation. The algorithm will merge the pairs of cluster that minimize this criterion. Ward minimizes the variance of the clusters being merged.
  Average uses the average of the distances of each observation of the two sets. Complete or maximum linkage uses the maximum distances between all observations
  of the two sets.. \default{ward}
%  \item \xmlNode{pooling\_func}, \xmlDesc{callable, optional field}, This combines the values of agglomerated features into a single value, and
%  should accept an array of shape $[M, N]$ and the keyword argument axis=1, and reduce it to an array of size $[M]$. \default{np.mean}
\end{itemize}


\subparagraph{Clustering performance evaluation} \hfill
\label{subparagraph:ClusterPerformance}

Evaluating the performance of a clustering algorithm is not as trivial as
counting the number of errors or the precision and recall of a supervised
 classification algorithm. In particular any evaluation metric should not
 take the absolute values of the cluster labels into account but rather if
 this clustering define separations of the data similar to some ground truth
 set of classes or satisfying some assumption such that members belong to
the same class are more similar that members of different classes according
to some similarity metric.

If the ground truth labels are not known, evaluation must be performed using
 the model itself. The \textbf{Silhouette Coefficient} is an example of
such an evaluation, where a higher Silhouette Coefficient score relates to
 a model with better defined clusters. The Silhouette Coefficient is defined
 for each sample and is composed of two scores:
\begin{enumerate}
	\item The mean distance between a sample and all other points in the
	 same class.
	%
	\item The mean distance between a sample and all other points in the
	 next nearest cluster.
\end{enumerate}

The Silhoeutte Coefficient s for a single sample is then given as:
\begin{equation}
s = \frac{b - a}{max(a, b)}
\end{equation}
The Silhouette Coefficient for a set of samples is given as the mean of the
 Silhouette Coefficient for each sample. In normal usage, the Silhouette
 Coefficient is applied to the results of a cluster analysis.

\begin{description}
	\item[Advantages] \hfill \\
	\begin{itemize}
		\item The score is bounded between -1 for incorrect
		clustering and +1 for highly dense clustering. Scores around
		 zero indicate overlapping clusters.
		\item The score is higher when clusters are dense and well
		 separated, which relates to a standard concept of a cluster.
	\end{itemize}
	\item[Drawbacks] \hfill \\
	The Silhouette Coefficient is generally higher for convex clusters
	than other concepts of clusters, such as density based clusters like
	 those obtained through DBSCAN.
\end{description}

\paragraph{Decomposing signals in components (matrix factorization problems)}
\label{paragraph:Decomposing}
\subparagraph{Principal component analysis (PCA)}
\label{subparagraph:PCA}

\begin{itemize}
	\item \textbf{Exact PCA and probabilistic interpretation} \\
	Linear Dimensionality reduction using Singular Value Decomposition of
	the data and keeping only the most significant singular vectors to
	 project the data to a lower dimensional space.
	\skltype{Exact PCA}{decomposition|PCA}
	\begin{itemize}
		\item \xmlNode{n\_components}, \xmlDesc{integer, None or String,
		optional field}, Number of components to keep. if
		\item \xmlNode{n\_components} is not set all components are kept,
		\default{all components}
		\item \xmlNode{copy}, \xmlDesc{boolean, optional field}, If False,
		 data passed to fit are overwritten and running fit(X).transform(X)
 		will not yield the expected results, use fit\_transform(X) instead.
		\default{True}
		\item \xmlNode{whiten}, \xmlDesc{boolean, optional field}, When True
		the components\_ vectors are divided by n\_samples times singular
		 values to ensure uncorrelated outputs with unit component-wise
		variances. Whitening will remove some information from the transformed
		 signal (the relative variance scales of the components) but can
		sometime improve the predictive accuracy of the downstream estimators
		 by making there data respect some hard-wired assumptions. \default{False}
	\end{itemize}
\textbf{Example:}
\begin{lstlisting}[style=XML,morekeywords={subType}]
<Simulation>
  ...
  <Models>
    ...
      <PostProcessor name='PostProcessorName' subType='DataMining'>
          <KDD lib='SciKitLearn'>
              <Features>variable1,variable2,variable3, variable4,variable5</Features>
              <SKLtype>decomposition|PCA</SKLtype>
              <n_components>2</n_components>
          </KDD>
      </PostProcessor>
    ...
  <Models>
  ...
<Simulation>
\end{lstlisting}


	\item \textbf{Randomized {(Approximate)} PCA} \\
	Linear Dimensionality reduction using Singular Value Decomposition of the data
	and keeping only the most significant singular vectors to project the data to a
	 lower dimensional space.
	\skltype{Randomized PCA}{decomposition|RandomizedPCA}
	\begin{itemize}
		\item \xmlNode{n\_components}, \xmlDesc{interger, None or String,
		optional field}, Number of components to keep. if n\_components is
		not set all components are kept.\default{all components}
		\item \xmlNode{copy}, \xmlDesc{boolean, optional field}, If False,
		 data passed to fit are overwritten and running fit(X).transform(X)
 		will not yield the expected results, use fit\_transform(X) instead.
		\default{True}
		\item \xmlNode{iterated\_power}, \xmlDesc{integer, optional field},
		Number of iterations for the power method. \default{3}
		\item \xmlNode{whiten}, \xmlDesc{boolean, optional field}, When True
		the components\_ vectors are divided by n\_samples times singular
		 values to ensure uncorrelated outputs with unit component-wise
		variances. Whitening will remove some information from the transformed
		 signal (the relative variance scales of the components) but can
		sometime improve the predictive accuracy of the downstream estimators
		 by making there data respect some hard-wired assumptions. \default{False}
		\item \xmlNode{random\_state}, \xmlDesc{int, or Random State instance
		or None, optional field}, Pseudo Random Number generator seed control.
		 If None, use the numpy.random singleton. \default{None}
	\end{itemize}
	\item \textbf{Kernel PCA} \\
	Non-linear dimensionality reduction through the use of kernels.
	\skltype{Kernel PCA}{decomposition|KernelPCA}
	\begin{itemize}
		\item \xmlNode{n\_components}, \xmlDesc{interger, None or String,
		optional field}, Number of components to keep. if n\_components is
		not set all components are kept.\default{all components}
		\item \xmlNode{kernel}, \xmlDesc{string, optional field}, name of
		the kernel to be used, options are:
		\begin{itemize}
			\item linear
			\item poly
			\item rbf
			\item sigmoid
			\item cosine
			\item precomputed
		\end{itemize}
		\default{linear}
		\xmlNode{degree}, \xmlDesc{integer, optional field}, Degree for poly
		 kernels, ignored by other kernels. \default{3}
		\xmlNode{gamma}, \xmlDesc{float, optional field}, Kernel coefficient
		 for rbf and poly kernels, ignored by other kernels. \default{1/n\_features}
		\item \xmlNode{coef0}, \xmlDesc{float, optional field}, independent term in
		 poly and sigmoig kernels, ignored by other kernels.
		\item \xmlNode{kernel\_params}, \xmlDesc{mapping of string to any, optional
		 field}, Parameters (keyword arguments) and values for kernel passed as
		callable object. Ignored by other kernels. \default{3}
		\item{alpha}, \xmlDesc{int, optional field}, Hyperparameter of the ridge
		regression that learns the inverse transform (when fit\_inverse\_transform=True).
		\default{1.0}
		\item \xmlNode{fit\_inverse\_transform}, \xmlDesc{bool, optional field},
		Learn the inverse transform for non-precomputed kernels. (i.e. learn to find
		 the pre-image of a point) \default{False}
		\item \xmlNode{eigen\_solver}, \xmlDesc{string, optional field}, Select eigensolver
		 to use. If n\_components is much less than the number of training samples,
		arpack may be more efficient than the dense eigensolver. Options are:
		\begin{itemize}
			\item auto
			\item dense
			\item arpack
		\end{itemize} \default{False}
		\item{tol}, \xmlDesc{float, optional field}, convergence tolerance for arpack.
		\default{0 (optimal value will be chosen by arpack)}
		\item{max\_iter}, \xmlDesc{int, optional field}, maximum number of iterations
		for arpack. \default{None (optimal value will be chosen by arpack)}
		\item \xmlNode{remove\_zero\_eig}, \xmlDesc{boolean, optional field}, If True,
		 then all components with zero eigenvalues are removed, so that the number of
		 components in the output may be < n\_components (and sometimes even zero due
		 to numerical instability). When n\_components is None, this parameter is
		 ignored and components with zero eigenvalues are removed regardless. \default{True}
	\end{itemize}
	\item \textbf{Sparse PCA} \\
	Finds the set of sparse components that can optimally reconstruct the data. The amount
	of sparseness is controllable by the coefficient of the L1 penalty, given by the
	parameter alpha.
	\skltype{Sparse PCA}{decomposition|SparsePCA}
	\begin{itemize}
		\item \xmlNode{n\_components}, \xmlDesc{integer, optional field}, Number of
		sparse atoms to extract. \default{None}
		\item \xmlNode{alpha}, \xmlDesc{float, optional field}, Sparsity controlling
		 parameter. Higher values lead to sparser components. \default{1.0}
		\item \xmlNode{ridge\_alpha}, \xmlDesc{float, optional field}, Amount of ridge
		 shrinkage to apply in order to improve conditioning when calling the transform
		 method. \default{0.01}
		\item \xmlNode{max\_iter}, \xmlDesc{float, optional field}, maximum number of
		iterations to perform. \default{1000}
		\item \xmlNode{tol}, \xmlDesc{float, optional field}, convergence tolerance.
		\default{1E-08}
		\item \xmlNode{method}, \xmlDesc{string, optional field}, method to use,
		options are:
		\begin{itemize}
			\item lars: uses the least angle regression method to solve the lasso
			 problem (linear\_model.lars\_path)
			\item cd: uses the coordinate descent method to compute the Lasso
			solution (linear\_model.Lasso)
		\end{itemize}
		Lars will be faster if the estimated components are sparse. \default{lars}
		\item \xmlNode{n\_jobs}, \xmlDesc{int, optional field}, number of parallel
		 runs to run. \default{1}
		\item \xmlNode{U\_init}, \xmlDesc{array of shape (n\_samples, n\_components)
		, optional field}, Initial values for the loadings for warm restart scenarios
		\default{None}
		\item \xmlNode{V\_init}, \xmlDesc{array of shape (n\_components, n\_features),
		 optional field}, Initial values for the components for warm restart scenarios
		\default{None}
		\item{verbose}, \xmlDesc{boolean, optional field}, Degree of verbosity of the
		 printed output. \default{False}
		\item{random\_state}, \xmlDesc{int or Random State, optional field}, Pseudo
		number generator state used for random sampling. \default{None}
	\end{itemize}
	\item \textbf{Mini Batch Sparse PCA} \\
	Finds the set of sparse components that can optimally reconstruct the data. The amount
	 of sparseness is controllable by the coefficient of the L1 penalty, given by the
	parameter alpha.
	\skltype{Mini Batch Sparse PCA}{decomposition|MiniBatchSparsePCA}
	\begin{itemize}
		\item \xmlNode{n\_components}, \xmlDesc{integer, optional field}, Number of
		 sparse atoms to extract. \default{None}
		\item \xmlNode{alpha}, \xmlDesc{float, optional field}, Sparsity controlling
		parameter. Higher values lead to sparser components. \default{1.0}
		\item \xmlNode{ridge\_alpha}, \xmlDesc{float, optional field}, Amount of ridge
		 shrinkage to apply in order to improve conditioning when calling the transform
		 method. \default{0.01}
		\item \xmlNode{n\_iter}, \xmlDesc{float, optional field}, number of iterations
		to perform per mini batch. \default{100}
		\item \xmlNode{callback}, \xmlDesc{callable, optional field}, callable that
		gets invoked every five iterations. \default{None}
		\item \xmlNode{batch\_size}, \xmlDesc{int, optional field}, the number of
		features to take in each mini batch. \default{3}
		\item \xmlNode{verbose}, \xmlDesc{boolean, optional field}, Degree of verbosity
		 of the printed output. \default{False}
		\item \xmlNode{shuffle}, \xmlDesc{boolean, optional field}, whether to shuffle
		the data before splitting it in batches. \default{True}
		\item \xmlNode{n\_jobs}, \xmlDesc{integer, optional field}, Parameters (keyword
		arguments) and values for kernel passed as callable object. Ignored by other
		kernels. \default{3}
		\item \xmlNode{metho}, \xmlDesc{string, optional field}, method to use,
		options are:
		\begin{itemize}
			\item lars: uses the least angle regression method to solve the lasso
			 problem (linear\_model.lars\_path),
			\item cd: uses the coordinate descent method to compute the Lasso solution
			 (linear\_model.Lasso)
		\end{itemize}
		Lars will be faster if the estimated components are sparse. \default{lars}
		\item \xmlNode{random\_state}, \xmlDesc{integer or Random State, optional field},
		 Pseudo number generator state used for random sampling. \default{None}
	\end{itemize}
\end{itemize}

\subparagraph{Truncated singular value decomposition} \hfil \\
\label{subparagraph:TruncatedSVD}
Dimensionality reduction using truncated SVD (aka LSA).
\skltype{Truncated SVD}{decomposition|TruncatedSVD}
\begin{itemize}
	\item \xmlNode{n\_components}, \xmlDesc{integer, optional field}, Desired dimensionality
	of output data. Must be strictly less than the number of features. The default value is
	useful for visualization. For LSA, a value of 100 is recommended. \default{2}
	\item \xmlNode{algorithm}, \xmlDesc{string, optional field}, SVD solver to use:
	\begin{itemize}
		\item Randomized: randomized algorithm
		\item Arpack: ARPACK wrapper in.
	\end{itemize}
	\default{Randomized}
	\item \xmlNode{n\_iter}, \xmlDesc{float, optional field}, number of iterations randomized
	SVD solver. Not used by ARPACK. \default{5}
	\item \xmlNode{random\_state}, \xmlDesc{int or Random State, optional field}, Pseudo number
	generator state used for random sampling. If not given, the numpy.random singleton is used.
	\default{None}
	\item \xmlNode{tol}, \xmlDesc{float, optional field}, Tolerance for ARPACK. 0 means machine
	precision. Ignored by randomized SVD solver. \default{0.0}
\end{itemize}

\subparagraph{Fast ICA} \hfil \\
\label{subparagraph:FastICA}
A fast algorithm for Independent Component Analysis.
\skltype{Fast ICA}{decomposition|FastICA}
\begin{itemize}
	\item \xmlNode{n\_components}, \xmlDesc{integer, optional field}, Number of components to
	use. If none is passed, all are used. \default{None}
	\item \xmlNode{algorithm}, \xmlDesc{string, optional field}, algorithm used in FastICA:
	\begin{itemize}
		\item parallel,
		\item deflation.
	\end{itemize}
	\default{parallel}
	\item \xmlNode{fun}, \xmlDesc{string or function, optional field}, The functional form of
	the G function used in the approximation to neg-entropy. Could be either:
	\begin{itemize}
		\item logcosh,
		\item exp, or
		\item cube.
	\end{itemize}
	One can also provide own function. It should return a tuple containing the value of the
	 function, and of its derivative, in the point. \default{logcosh}
	\item \xmlNode{fun\_args}, \xmlDesc{dictionary, optional field}, Arguments to send to the
	functional form. If empty and if fun=’logcosh’, fun\_args will take value {‘alpha’ : 1.0}.
	\default{None}
	\item \xmlNode{max\_iter}, \xmlDesc{float, optional field}, maximum number of iterations
	 during fit. \default{200}
	\item \xmlNode{tol}, \xmlDesc{float, optional field}, Tolerance on update at each iteration.
	\default{0.0001}
	\item \xmlNode{w\_init}, \xmlDesc{None or an (n\_components, n\_components) ndarray,
	optional field}, The mixing matrix to be used to initialize the algorithm. \default{None}
	\item \xmlNode{randome\_state}, \xmlDesc{int or Random State, optional field}, Pseudo number
	 generator state used for random sampling. \default{None}
\end{itemize}

\paragraph{Manifold learning}
\label{paragraph:Manifold}
A manifold is a topological space that resembles a Euclidean space locally at each point. Manifold
learning is an approach to non-linear dimensionality reduction. It assumes that the data of interest
 lie on an embedded non-linear manifold within the higher-dimensional space. If this manifold is of
 low dimension, data can be visualized in the low-dimensional space. Algorithms for this task are
based on the idea that the dimensionality of many data sets is only artificially high.
\subparagraph{Isomap} \hfil \\
\label{subparagraph:Isomap}
Non-linear dimensionality reduction through Isometric Mapping (Isomap).
\skltype{Isometric Mapping}{manifold|Isomap}
\begin{itemize}
	\item \xmlNode{n\_neighbors}, \xmlDesc{integer, optional field}, Number of neighbors to
	consider for each point. \default{5}
	\item \xmlNode{n\_components}, \xmlDesc{integer, optional field}, Number of coordinates to
	manifold. \default{2}
	\item \xmlNode{eigen\_solver}, \xmlDesc{string, optional field}, eigen solver to use:
	\begin{itemize}
		\item auto: Attempt to choose the most efficient solver for the given problem,
		\item arpack: Use Arnoldi decomposition to find the eigenvalues and eigenvectors
		\item dense: Use a direct solver (i.e. LAPACK) for the eigenvalue decomposition
	\end{itemize}
	\default{auto}
	\item \xmlNode{tol}, \xmlDesc{float, optional field}, Convergence tolerance passed to
	arpack or lobpcg. not used if eigen\_solver is ‘dense’. \default{0.0}
	\item \xmlNode{max\_iter}, \xmlDesc{float, optional field}, Maximum number of iterations
	for the arpack solver. not used if eigen\_solver == ‘dense’. \default{None}
	\item \xmlNode{path\_method}, \xmlDesc{string, optional field}, Method to use in finding
	 shortest path. Could be either:
	\begin{itemize}
		\item Auto: attempt to choose the best algorithm
		\item FW: Floyd-Warshall algorithm
		\item D: Dijkstra algorithm with Fibonacci Heaps
	\end{itemize}
	\default{auto}
	\item \xmlNode{neighbors\_algorithm}, \xmlDesc{string, optional field}, Algorithm to use
	 for nearest neighbors search, passed to neighbors.NearestNeighbors instance.
	\begin{itemize}
		\item auto,
		\item brute
		\item kd\_tree
		\item ball\_tree
	\end{itemize}
	\default{auto}
\end{itemize}

\textbf{Example:}
\begin{lstlisting}[style=XML,morekeywords={subType}]
<Simulation>
  ...
  <Models>
    ...
      <PostProcessor name='PostProcessorName' subType='DataMining'>
          <KDD lib='SciKitLearn'>
              <Features>input</Features>
              <SKLtype>manifold|Isomap</SKLtype>
              <n_neighbors>5</n_neighbors>
	      <n_components>3</n_components>
	      <eigen_solver>arpack</eigen_solver>
	      <neighbors_algorithm>kd_tree</neighbors_algorithm>
            </KDD>
      </PostProcessor>
    ...
  <Models>
  ...
<Simulation>
\end{lstlisting}


\subparagraph{Locally Linear Embedding} \hfil \\
\label{subparagraph:LLE}
\skltype{Locally Linear Embedding}{manifold|LocallyLinearEmbedding}
\begin{itemize}
	\item \xmlNode{n\_neighbors}, \xmlDesc{integer, optional field}, Number of neighbors to
	consider for each point. \default{5}
	\item \xmlNode{n\_components}, \xmlDesc{integer, optional field}, Number of coordinates to
	 manifold. \default{2}
	\item \xmlNode{reg}, \xmlDesc{float, optional field}, regularization constant, multiplies
	the trace of the local covariance matrix of the distances. \default{0.01}
	\item \xmlNode{eigen\_solver}, \xmlDesc{string, optional field}, eigen solver to use:
	\begin{itemize}
		\item auto: Attempt to choose the most efficient solver for the given problem,
		\item arpack: use arnoldi iteration in shift-invert mode.
		\item dense: use standard dense matrix operations for the eigenvalue
	\end{itemize}
	\default{auto}
	\item \xmlNode{tol}, \xmlDesc{float, optional field}, Convergence tolerance passed to arpack.
	 not used if eigen\_solver is ‘dense’. \default{1E-06}
	\item \xmlNode{max\_iter}, \xmlDesc{int, optional field}, Maximum number of iterations for the
	 arpack solver. not used if eigen\_solver == ‘dense’. \default{100}
	\item \xmlNode{method}, \xmlDesc{string, optional field}, Method to use. Could be either:
	\begin{itemize}
		\item Standard: use the standard locally linear embedding algorithm
		\item hessian: use the Hessian eigenmap method
		\item itsa: use local tangent space alignment algorithm
	\end{itemize}
	\default{standard}
	\item \xmlNode{hessian\_tol}, \xmlDesc{float, optional field}, Tolerance for Hessian eigenmapping
	 method. Only used if method == 'hessian' \default{0.0001}
	\item \xmlNode{modified\_tol}, \xmlDesc{float, optional field}, Tolerance for modified LLE method.
	 Only used if method == 'modified' \default{0.0001}
	\item \xmlNode{neighbors\_algorithm}, \xmlDesc{string, optional field}, Algorithm to use for nearest
	 neighbors search, passed to neighbors.NearestNeighbors instance.
	\begin{itemize}
		\item auto,
		\item brute
		\item kd\_tree
		\item ball\_tree
	\end{itemize}
	\default{auto}
	\item \xmlNode{random\_state}, \xmlDesc{int or numpy random state, optional field}, the generator
	or seed used to determine the starting vector for arpack iterations. \default{None}
\end{itemize}
\subparagraph{Spectral Embedding} \hfil \\
\label{subparagraph:Spectral}
Spectral embedding for non-linear dimensionality reduction, it forms an affinity matrix given by the
 specified function and applies spectral decomposition to the corresponding graph laplacian. The resulting
 transformation is given by the value of the eigenvectors for each data point
\skltype{Spectral Embedding}{manifold|SpectralEmbedding}
\begin{itemize}
	\item \xmlNode{n\_components}, \xmlDesc{integer, optional field}, the dimension of projected
	sub-space. \default{2}
	\item \xmlNode{eigen\_solver}, \xmlDesc{string, optional field}, the eigen value decomposition
	 strategy to use:
	\begin{itemize}
		\item none,
		\item arpack.
		\item lobpcg,
		\item amg
	\end{itemize}
	\default{none}
	\item \xmlNode{random\_state}, \xmlDesc{integer or numpy random state, optional field}, A
	pseudo random number generator used for the initialization of the lobpcg eigen vectors
	decomposition when eigen\_solver == ‘amg. \default{None}
	\item \xmlNode{affinity}, \xmlDesc{string or callable, optional field}, How to construct
	 the affinity matrix:
	\begin{itemize}
		\item \textit{nearest\_neighbors} : construct affinity matrix by knn graph
		\item \textit{rbf} : construct affinity matrix by rbf kernel
		\item \textit{precomputed} : interpret X as precomputed affinity matrix
		\item \textit{callable} : use passed in function as affinity the function takes
		 in data matrix (n\_samples, n\_features) and return affinity matrix (n\_samples, n\_samples).
	\end{itemize}
	\default{nearest\_neighbor}
	\item \xmlNode{gamma}, \xmlDesc{float, optional field}, Kernel coefficient for rbf kernel.
	\default{None}
	\item \xmlNode{n\_neighbors}, \xmlDesc{int, optional field}, Number of nearest neighbors for
	 nearest\_neighbors graph building. \default{None}
\end{itemize}

\subparagraph{Multi-dimensional Scaling (MDS)} \hfil \\
\label{subparagraph:MDS}

\skltype{Multi Dimensional Scaling}{manifold|MDS}
\begin{itemize}
	\item \xmlNode{metric}, \xmlDesc{boolean, optional field}, compute metric or nonmetric SMACOF
	 (Scaling by Majorizing a Complicated Function) algorithm \default{True}
	\item \xmlNode{n\_components}, \xmlDesc{integer, optional field}, number of dimension in
	which to immerse the similarities overridden if initial array is provided. \default{2}
	\item \xmlNode{n\_init}, \xmlDesc{integer, optional field}, Number of time the smacof
	algorithm will be run with different initialisation. The final results will be the best
	 output of the n\_init consecutive runs in terms of stress. \default{4}
	\item \xmlNode{max\_iter}, \xmlDesc{integer, optional field}, Maximum number of iterations
	of the SMACOF algorithm for a single run \default{300}
	\item \xmlNode{verbose}, \xmlDesc{integer, optional field}, level of verbosity \default{0}
	\item \xmlNode{eps}, \xmlDesc{float, optional field}, relative tolerance with respect
	to stress to declare converge \default{1E-06}
	\item \xmlNode{n\_jobs}, \xmlDesc{integer, optional field}, The number of jobs to use for
	the computation. This works by breaking down the pairwise matrix into n\_jobs even slices
	 and computing them in parallel. If -1 all CPUs are used. If 1 is given, no parallel
	 computing code is used at all, which is useful for debugging. For n\_jobs below -1,
	(n\_cpus + 1 + n\_jobs) are used. Thus for n\_jobs = -2, all CPUs but one are used.
	\default{1}
	\item \xmlNode{random\_state}, \xmlNode{integer or numpy random state, optional field},
	The generator used to initialize the centers. If an integer is given, it fixes the seed.
	Defaults to the global numpy random number generator. \default{None}
	\item \xmlNode{dissimilarity}, \xmlDesc{string, optional field}, Which dissimilarity
	measure to use. Supported are ‘euclidean’ and ‘precomputed’. \default{euclidean}
\end{itemize}


\paragraph{Scipy}
\xmlString{Scipy} provides a Hierarchical clustering that performs clustering over PointSet and HistorySet.
This algorithm also automatically generates a dendrogram in .pdf format (i.e., dendrogram.pdf).

\begin{itemize}
  \item \xmlNode{SCIPYtype}, \xmlDesc{string, required field}, SCIPY algorithm to be employed.
  \item \xmlNode{Features}, \xmlDesc{string, required field}, defines the data to be used for training the data mining algorithm. It can be:
    \begin{itemize}
      \item the name of the variable in the defined dataObject entity
      \item the location (i.e. input or output). In this case the data mining is applied to all the variables in the defined space.
    \end{itemize}
  \item \xmlNode{method},         \xmlDesc{string, required field}, The linkage algorithm to be used  \default{single, complete, weighted, centroids, median, ward}.
  \item \xmlNode{metric},         \xmlDesc{string, required field}, The distance metric to be used \default{ ‘braycurtis’, ‘canberra’, ‘chebyshev’, ‘cityblock’,
                                                                    ‘correlation’, ‘cosine’, ‘dice’, ‘euclidean’, ‘hamming’, ‘jaccard’, ‘kulsinski’,
                                                                    ‘mahalanobis’, ‘matching’, ‘minkowski’, ‘rogerstanimoto’, ‘russellrao’, ‘seuclidean’,
                                                                    ‘sokalmichener’, ‘sokalsneath’, ‘sqeuclidean’, ‘yule’}.
  \item \xmlNode{level},          \xmlDesc{float, required field},  Clustering distance level where actual clusters are formed.
  \item \xmlNode{criterion},      \xmlDesc{string, required field}, The criterion to use in forming flat clusters. This can be any of the following values:
    \begin{itemize}
      \item   ``inconsistent''     : If a cluster node and all its descendants have an inconsistent value less than or equal to `t` then all its leaf descendants
                                     belong to the same flat cluster. When no non-singleton cluster meets this criterion, every node is assigned to its own
                                     cluster. (Default)
      \item   ``distance''         : Forms flat clusters so that the original observations in each flat cluster have no greater a cophenetic distance than $t$.
      \item   ``maxclust''         : Finds a minimum threshold ``r'' so that the cophenetic distance between any two original observations in the same flat cluster
                                     is no more than ``r'' and no more than $t$ flat clusters are formed.
      \item   ``monocrit''         : Forms a flat cluster from a cluster node c with index i when $monocrit[j] <= t$.
       \item  ``maxclust\_monocrit'' : Forms a flat cluster from a non-singleton cluster node ``c'' when $monocrit[i] <= r$ for all cluster indices ``i''
                                       below and including ``c''. ``r'' is minimized such that no more than ``t'' flat clusters are formed. monocrit must be
                                       monotonic.
    \end{itemize}
  \item \xmlNode{dendrogram},     \xmlDesc{boolean, required field}, If True the dendrogram is actually created.
  \item \xmlNode{truncationMode}, \xmlDesc{string, required field}, The dendrogram can be hard to read when the original observation matrix from which the
                                                                    linkage is derived is large. Truncation is used to condense the dendrogram. There are several
                                                                    modes:
                                                                    \begin{itemize}
                                                                      \item ``None'': No truncation is performed (Default).
                                                                      \item ``lastp'': The last p non-singleton formed in the linkage are the only non-leaf nodes
                                                                       in the linkage; they correspond to rows $Z[n-p-2:end]$ in Z. All other non-singleton
                                                                       clusters are contracted into leaf nodes.
                                                                      \item ``level''/``mtica'': No more than p levels of the dendrogram tree are displayed.
                                                                      This corresponds to Mathematica behavior.
                                                                    \end{itemize}
  \item \xmlNode{p},              \xmlDesc{int, required field},     The $p$ parameter for truncationMode.
  \item \xmlNode{leafCounts},     \xmlDesc{boolean, required field}, When True the cardinality non singleton nodes contracted into a leaf node is indiacted in
                                                                     parenthesis.
  \item \xmlNode{showContracted}, \xmlDesc{boolean, required field}, When True the heights of non singleton nodes contracted into a leaf node are plotted as
                                                                     crosses along the link connecting that leaf node.
  \item \xmlNode{annotatedAbove}, \xmlDesc{float, required field},  Clustering level above which the branching level is annotated.
\end{itemize}

\textbf{Example:}
\begin{lstlisting}[style=XML,morekeywords={subType}]
<Simulation>
  ...
  <Models>
    ...
    <PostProcessor name="hierarchical" subType="DataMining" verbosity="quiet">
      <KDD lib="Scipy" labelFeature='labels'>
        <SCIPYtype>cluster|Hierarchical</SCIPYtype>
        <Features>output</Features>
        <method>single</method>
        <metric>euclidean</metric>
        <level>75</level>
        <criterion>distance</criterion>
        <dendrogram>true</dendrogram>
        <truncationMode>lastp</truncationMode>
        <p>20</p>
        <leafCounts>True</leafCounts>
        <showContracted>True</showContracted>
        <annotatedAbove>10</annotatedAbove>
      </KDD>
    </PostProcessor>
    ...
  <Models>
  ...
<Simulation>
\end{lstlisting}




%%%%% PP PrintCSV %%%%%%%
%\paragraph{PrintCSV}
%\label{PrintCSV}
%TO BE MOVED TO STEP ``IOSTEP''
%%%%% PP LoadCsvIntoInternalObject %%%%%%%
%\paragraph{LoadCsvIntoInternalObject}
%\label{LoadCsvIntoInternalObject}
%TO BE MOVED TO STEP ``IOSTEP''
%

%%%%% PP External %%%%%%%
\subsubsection{Interfaced}
\label{Interfaced}
The \textbf{Interfaced} post-processor is a Post-Processor that allows the user
to create its own Post-Processor. While the External Post-Processor (see
Section~\ref{External} allows the user to create case-dependent
Post-Processors, with this new class the user can create new general
purpose Post-Processors.
%

\ppType{Interfaced}{Interfaced}

\begin{itemize}
  \item \xmlNode{method}, \xmlDesc{comma separated string, required field},
  lists the method names of a method that will be computed (each
  returning a post-processing value). All available methods need to be included
  in the ``/raven/framework/PostProcessorFunctions/'' folder
\end{itemize}

\textbf{Example:}
\begin{lstlisting}[style=XML,morekeywords={subType,debug,name,class,type}]
<Simulation>
  ...
  <Models>
    ...
    <PostProcessor name="example" subType='InterfacedPostProcessor'verbosity='debug'>
       <method>testInterfacedPP</method>
       <!--Here, the xml nodes required by the chosen method have to be
       included.
        -->
    </PostProcessor>
    ...
  </Models>
  ...
</Simulation>
\end{lstlisting}

All the \textbf{Interfaced} post-processors need to be contained in the
``/raven/framework/PostProcessorFunctions/'' folder. In fact, once the
\textbf{Interfaced} post-processor is defined in the RAVEN input file, RAVEN
search that the method of the post-processor is located in such folder.

The class specified in the \textbf{Interfaced} post-processor has to inherit the
PostProcessorInterfaceBase class and the user must specify this set of
methods:
\begin{itemize}
  \item initialize: in this method, the internal parameters of the
  post-processor are initialized. Mandatory variables that needs to be
  specified are the following:
\begin{itemize}
  \item self.inputFormat: type of dataObject expected in input
  \item self.outputFormat: type of dataObject generated in output
\end{itemize}
  \item readMoreXML: this method is in charge of reading the PostProcessor xml
  node, parse it and fill the PostProcessor internal variables.
  \item run: this method performs the desired computation of the dataObject.
\end{itemize}

\begin{lstlisting}[language=python]
from PostProcessorInterfaceBaseClass import PostProcessorInterfaceBase
class testInterfacedPP(PostProcessorInterfaceBase):
  def initialize(self)
  def readMoreXML(self,xmlNode)
  def run(self,inputDic)
\end{lstlisting}

\paragraph{Data Format}
The user is not allowed to modify directly the DataObjects, however the
content of the DataObjects is available in the form of a python dictionary.
Both the dictionary give in input and the one generated in the output of the
PostProcessor are structured as follows:

\begin{lstlisting}[language=python]
inputDict = {'data':{}, 'metadata':{}}
\end{lstlisting}

where:

\begin{lstlisting}[language=python]
inputDict['data'] = {'input':{}, 'output':{}}
\end{lstlisting}

In the input dictonary, each input variable is listed as a dictionary that
contains a numpy array with its own values as shown below for a simplified
example

\begin{lstlisting}[language=python]
inputDict['data']['input'] = {'inputVar1': array([ 1.,2.,3.]),
                              'inputVar2': array([4.,5.,6.])}
\end{lstlisting}

Similarly, if the dataObject is a PointSet then the output dictionary is
structured as follows:

\begin{lstlisting}[language=python]
inputDict['data']['output'] = {'outputVar1': array([ .1,.2,.3]),
                               'outputVar2':array([.4,.5,.6])}
\end{lstlisting}

Howevers, if the dataObject is a HistorySet then the output dictionary is
structured as follows:

\begin{lstlisting}[language=python]
inputDict['data']['output'] = {'hist1': {}, 'hist2':{}}
\end{lstlisting}

where

\begin{lstlisting}[language=python]
inputDict['output']['data'][hist1] = {'time': array([ .1,.2,.3]),
                              'outputVar1':array([ .4,.5,.6])}
inputDict['output']['data'][hist2] = {'time': array([ .1,.2,.3]),
                              'outputVar1':array([ .14,.15,.16])}
\end{lstlisting}

\paragraph{Method: HistorySetSampling}
This Post-Processor performs the conversion from HistorySet to HistorySet
The conversion is made so that each history H is re-sampled accordingly  to a
specific sampling strategy.
It can be used to reduce the amount of space required by the HistorySet.

In the \xmlNode{PostProcessor} input block, the following XML sub-nodes are required,
independent of the \xmlAttr{subType} specified:

\begin{itemize}
   \item \xmlNode{samplingType}, \xmlDesc{string, required field}, specifies the type of sampling method to be used (uniform, firstDerivative secondDerivative, filteredFirstDerivative or
   filteredSecondDerivative).
   \item \xmlNode{numberOfSamples}, \xmlDesc{integer, optional field}, number of samples (required only for the following sampling types: uniform, firstDerivative secondDerivative)
   \item \xmlNode{pivotParameter}, \xmlDesc{string, required field}, ID of the temporal variable
   \item \xmlNode{interpolation}, \xmlDesc{string, optional field}, type of interpolation to be employed for the history recostruction (required only for the following sampling types: uniform,
   firstDerivative secondDerivative). Valid types of interpolation to specified: linear, nearest, zero, slinear, quadratic, cubic, intervalAverage;
   \item \xmlNode{tolerance}, \xmlDesc{string, optional field}, tolerance level (required only for the following sampling types: filteredFirstDerivative or filteredSecondDerivative)
\end{itemize}

\paragraph{Method: HistorySetSync}
This Post-Processor performs the conversion from HistorySet to HistorySet
The conversion is made so that all histories are synchronized in time.
It can be used to allow the histories to be sampled at the same time instant.

There are two possible synchronization methods, specified through the \xmlNode{syncMethod} node.  If the
\xmlNode{syncMethod} is \xmlString{grid}, a \xmlNode{numberOfSamples} node is specified,
which yields an equally-spaced grid of time points. The output values for these points will be linearly derived
using nearest sampled time points, and the new HistorySet will contain only the new grid points.

The other methods are used by specifying \xmlNode{syncMethod} as \xmlString{all}, \xmlString{min}, or
\xmlString{max}.  For \xmlString{all}, the postprocessor will iterate through the
existing histories, collect all the time points used in any of them, and use these as the new grid on which to
establish histories, retaining all the exact original values and interpolating linearly where necessary.
In the event of \xmlString{min} or \xmlString{max}, the postprocessor will find the smallest or largest time
history, respectively, and use those time values as nodes to interpolate between.

In the \xmlNode{PostProcessor} input block, the following XML sub-nodes are required,
independent of the \xmlAttr{subType} specified:

\begin{itemize}
   \item \xmlNode{pivotParameter}, \xmlDesc{string, required field}, ID of the temporal variable
   \item \xmlNode{extension}, \xmlDesc{string, required field}, type of extension when the sync process goes outside the boundaries of the history (zeroed or extended)
   \item \xmlNode{syncMethod}, \xmlDesc{string, required field}, synchronization strategy to employ (see
     description above).  Options are \xmlString{grid}, \xmlString{all}, \xmlString{max}, \xmlString{min}.
   \item \xmlNode{numberOfSamples}, \xmlDesc{integer, optional field}, required if \xmlNode{syncMethod} is
     \xmlString{grid}, number of new time samples
\end{itemize}

\paragraph{Method: HistorySetSnapShot}
This Post-Processor performs the conversion from HistorySet to PointSet
The conversion is made so that each history H is converted to a single point P.
There are several methods that can be employed to choose the single point from the history:
\begin{itemize}
  \item min: Take a time slice when the \xmlNode{pivotVar} is at its smallest value,
  \item max: Take a time slice when the \xmlNode{pivotVar} is at its largest value,
  \item average: Take a time slice when the \xmlNode{pivotVar} is at its time-weighted average value,
  \item value: Take a time slice when the \xmlNode{pivotVar} \emph{first passes} its specified value,
  \item timeSlice: Take a time slice index from the sampled time instance space.
\end{itemize}
To demonstrate the timeSlice, assume that each history H is a dict of n output variables $x_1=[...],
x_n=[...]$, then the resulting point P is at time instant index t: $P=[x_1[t],...,x_n[t]]$.

Choosing one the these methods for the \xmlNode{type} node will take a time slice for all the variables in the
output space based on the provided parameters.  Alternatively, a \xmlString{mixed} type can be used, in which
each output variable can use a different time slice parameter.  In other words, you can take the max of one
variable while taking the minimum of another, etc.

In the \xmlNode{PostProcessor} input block, the following XML sub-nodes are required,
independent of the \xmlAttr{subType} specified:

\begin{itemize}
  \item \xmlNode{type}, \xmlDesc{string, required field}, type of operation: \xmlString{min}, \xmlString{max},
                        \xmlString{average}, \xmlString{value}, \xmlString{timeSlice}, or \xmlString{mixed}
   \item \xmlNode{extension}, \xmlDesc{string, required field}, type of extension when the sync process goes outside the boundaries of the history (zeroed or extended)
   \item \xmlNode{pivotParameter}, \xmlDesc{string, optional field}, name of the temporal variable.  Required for the
     \xmlString{average} and \xmlString{timeSlice} methods.
\end{itemize}

If a \xmlString{timeSlice} type is in use, the following nodes also are required:
\begin{itemize}
   \item \xmlNode{timeInstant}, \xmlDesc{integer, required field}, required and only used in the
     \xmlString{timeSlice} type.  Location of the time slice (integer index)
   \item \xmlNode{numberOfSamples}, \xmlDesc{integer, required field}, number of samples
\end{itemize}

If instead a \xmlString{min}, \xmlString{max}, \xmlString{average}, or \xmlString{value} is used, the following nodes
are also required:
\begin{itemize}
   \item \xmlNode{pivotVar}, \xmlDesc{string, required field},  Name of the chosen indexing variable (the
         variable whose min, max, average, or value is used to determine the time slice)
       \item \xmlNode{pivotVal}, \xmlDesc{float, optional field},  required for \xmlString{value} type, the value for the chosen variable
\end{itemize}

Lastly, if a \xmlString{mixed} approach is used, the following nodes apply:
\begin{itemize}
  \item \xmlNode{max}, \xmlDesc{string, optional field}, the names of variables whose output should be their
    own maximum value within the history.
  \item \xmlNode{min}, \xmlDesc{string, optional field}, the names of variables whose output should be their
    own minimum value within the history.
  \item \xmlNode{average}, \xmlDesc{string, optional field}, the names of variables whose output should be their
    own average value within the history. Note that a \xmlNode{pivotParameter} node is required to perform averages.
  \item \xmlNode{value}, \xmlDesc{string, optional field}, the names of variables whose output should be taken
    at a time slice determined by another variable.  As with the non-mixed \xmlString{value} type, the first
    time the \xmlAttr{pivotVar} crosses the specified \xmlAttr{pivotVal} will be the time slice taken.
    This node requires two attributes, if used:
    \begin{itemize}
      \item \xmlAttr{pivotVar}, \xmlDesc{string, required field}, the name of the variable on which the time
        slice will be performed.  That is, if we want the value of $y$ when $t=0.245$,
        this attribute would be \xmlString{t}.
      \item \xmlAttr{pivotVal}, \xmlDesc{float, required field}, the value of the \xmlAttr{pivotVar} on which the time
        slice will be performed.  That is, if we want the value of $y$ when $t=0.245$,
        this attribute would be \xmlString{0.245}.
    \end{itemize}
  Note that all the outputs of the \xmlNode{DataObject} output of this postprocessor must be listed under one
  of the \xmlString{mixed} node types in order for values to be returned.
\end{itemize}

\textbf{Example (mixed):}
This example will output the average value of $x$ for $x$, the value of $y$ at
time$=0.245$ for $y$, and the value of $z$ at $x=4.0$ for $z$.
\begin{lstlisting}[style=XML,morekeywords={subType,debug,name,class,type}]
<Simulation>
  ...
  <Models>
    ...
    <PostProcessor name="mampp2" subType="InterfacedPostProcessor">
      <method>HistorySetSnapShot</method>
      <type>mixed</type>
      <average>x</average>
      <value pivotVar="time" pivotVal="0.245">y</value>
      <value pivotVar="x" pivotVal="4.0">z</value>
      <pivotParameter>time</pivotParameter>
      <extension>zeroed</extension>
    </PostProcessor>
    ...
  </Models>
  ...
</Simulation>
\end{lstlisting}


\paragraph{Method: HSPS}

This Post-Processor performs the conversion from HistorySet to PointSet
The conversion is made so that each history H is converted to a single point P.
Assume that each history H is a dict of n output variables $x_1=[...],x_n=[...]$, then the resulting point P is as follows; $P=[x_1,...,x_n]$
Note: it is here assumed that all histories have been sync so that they have the same length, start point and end point. If you are not sure, do a pre-processing the the original history set.

In the \xmlNode{PostProcessor} input block, the following XML sub-nodes are required,
independent of the \xmlAttr{subType} specified (min, max, avg and value case):

\begin{itemize}
   \item \xmlNode{pivotParameter}, \xmlDesc{string, optional field}, ID of the temporal variable (only for avg)
\end{itemize}

\paragraph{Method: TypicalHistoryFromHistorySet}
This Post-Processor performs a simplified procedure of \cite{wilcox2008users} to form a ``typical'' time series from multiple time series. The input should be a HistorySet, with each history in the HistorySet synchronized. For HistorySet that is not synchronized, use Post-Processor method \textbf{HistorySetSync}  to synchronize the data before running this method.

Each history in input HistorySet is first converted to multiple histories each has maximum time specified in \xmlNode{outputLen} (see below). Each converted history $H_i$ is divided into a set of subsequences $\{H_i^j\}$, and the division is guided by the \xmlNode{subseqLen} node specified in the input XML. The value of \xmlNode{subseqLen} should be a list of positive numbers that specify the length of each subsequence. If the number of subsequence for each history is more than the number of values given in \xmlNode{subseqLen}, the values in \xmlNode{subseqLen} would be reused.

For each variable $x$, the method first computes the empirical CDF (cumulative density function) by using all the data values of $x$ in the HistorySet. This CDF is termed as long-term CDF for $x$. Then for each subsequence $H_i^j$, the method computes the empirical CDF by using all the data values of $x$ in $H_i^j$. This CDF is termed as subsequential CDF. For the first interval window (i.e., $j=1$), the method computes the Finkelstein-Schafer (FS) statistics \cite{finkelstein1971improved} between the long term CDF and the subsequential CDF of $H_i^1$ for each $i$. The FS statistics is defined as following.
\begin{align*}
FS & = \sum_x FS_x\\
FS_x &= \frac{1}{N}\sum_{n=1}^N\delta_n
\end{align*}
where $N$ is the number of value reading in the empirical CDF and $\delta_n$ is the absolute difference between the long term CDF and the subsequential CDF at value $x_n$. The subsequence $H_i^1$ with minimal FS statistics will be selected as the typical subsequence for the interval window $j=1$. Such process repeats for $j=2,3,\dots$ until all subsequences have been processed. Then all the typical subsequences will be concatenated to form a complete history.

In the \xmlNode{PostProcessor} input block, the following XML sub-nodes are required,
independent of the \xmlAttr{subType} specified:

\begin{itemize}
   \item \xmlNode{pivotParameter}, \xmlDesc{string, optional field}, ID of the temporal variable
   \default{Time}
   \item \xmlNode{subseqLen}, \xmlDesc{integers, required field}, length of the divided subsequence (see above)
   \item \xmlNode{outputLen}, \xmlDesc{integer, optional field}, maximum value of the temporal variable for the generated typical history
   \default{Maximum value of the variable with name of \xmlNode{pivotParameter}}
\end{itemize}

For example, consider history of data collected over three years in one-second increments,
where the user wants a single \emph{typical year} extracted from the data.
The user wants this data constructed by combining twelve equal \emph{typical month}
segments.  In this case, the parameter \xmlNode{outputLen} should be \texttt{31536000} (the number of seconds
in a year), while the parameter \xmlNode{subseqLen} should be \texttt{2592000} (the number of seconds in a
month).  Using a value for \xmlNode{subseqLen} that is either much, much smaller than \xmlNode{outputLen} or
of equal size to \xmlNode{outputLen} might have unexpected results.  In general, we recommend using a
\xmlNode{subseqLen} that is roughly an order of magnitude smaller than \xmlNode{outputLen}.

\paragraph{Method: dataObjectLabelFilter}
This Post-Processor allows to filter the portion of a dataObject, either PointSet or HistorySet, with a given clustering label.
A clustering algorithm associates a unique cluster label to each element of the dataObject (PointSet or HistorySet).
This cluster label is a natural number ranging from $0$ (or $1$ depending on the algorithm) to $N$ where $N$ is the number of obtained clusters.
Recall that some clustering algorithms (e.g., K-Means) receive $N$ as input while others (e.g., Mean-Shift) determine $N$ after clustering has been performed.
Thus, this Post-Processor is naturally employed after a data-mining clustering techniques has been performed on a dataObject so that each clusters
can be analyzed separately.

In the \xmlNode{PostProcessor} input block, the following XML sub-nodes are required,
independently of the \xmlAttr{subType} specified:

\begin{itemize}
   \item \xmlNode{label}, \xmlDesc{string, required field}, name of the clustering label
   \item \xmlNode{clusterIDs}, \xmlDesc{integers, required field}, ID of the selected clusters. Note that more than one ID can be provided as input
\end{itemize}

%%FIXME: which one is the correct one? The previous one or this one
\paragraph{Method: dataObjectLabelFilter}
The \xmlNode{HSPS} Post-Processor performs a filtering of the dataObject. This particular filtering is based on the labels generated by any clustering algorithm.
Given the selected label, this Post-Processor filters out all histories or points having a different label.
In the \xmlNode{PostProcessor} input block, the following XML sub-nodes are required:

\begin{itemize}
   \item \xmlNode{dataType}, \xmlDesc{string, required field}, type of dataObject (HistorySet or PointSet)
   \item \xmlNode{label}, \xmlDesc{string, required field}, varaiable which contains the cluster labels
   \item \xmlNode{clusterIDs}, \xmlDesc{int, required field}, cluster labels considered
\end{itemize}

\paragraph{Method: Discrete Risk Measures}
This Post-Processor calculates a series of risk importance measures from a PointSet. This calculation if performed for a set of input paramteres given an output target.

The user is required to provide the following information:
\begin{itemize}
   \item the set of input variables. For each variable the following need to be specified:
     \begin{itemize}
       \item the set of values that imply a reliability value equal to $1$ for the input variable
       \item the set of values that imply a reliability value equal to $0$ for the input variable
     \end{itemize}
   \item the output target variable. For this variable it is needed to specify the values of the output target variable that defines the desired outcome.
\end{itemize}

The following variables are first determined for each input variable $i$:
\begin{itemize}
   \item $R_0$ Probability of the outcome of the output target variable (nominal value)
   \item $R^{+}_i$ Probability of the outcome of the output target variable if reliability of the input variable is equal to $0$
   \item $R^{-}_i$ Probability of the outcome of the output target variable if reliability of the input variable is equal to $1$
\end{itemize}

Available measures are:
\begin{itemize}
   \item Risk Achievement Worth (RAW): $RAW = R^{+}_i / R_0 $
   \item Risk Achievement Worth (RRW): $RRW = R_0 / R^{-}_i$
   \item Fussell-Vesely (FV): $FV = (R_0 - R^{-}_i) / R_0$
   \item Birnbaum (B): $B = R^{+}_i - R^{-}_i$
\end{itemize}

In the \xmlNode{PostProcessor} input block, the following XML sub-nodes are required,
independent of the \xmlAttr{subType} specified:

\begin{itemize}
   \item \xmlNode{measures}, \xmlDesc{string, required field}, desired risk importance measures that have to be computed (RRW, RAW, FV, B)
   \item \xmlNode{variable}, \xmlDesc{string, required field}, ID of the input variable. This node is provided for each input variable. This nodes needs to contain also these attributes:
     \begin{itemize}
       \item \xmlAttr{R0values}, \xmlDesc{float, required field}, interval of values (comma separated values) that implies a reliability value equal to $0$ for the input variable
       \item \xmlAttr{R1values}, \xmlDesc{float, required field}, interval of values (comma separated values) that implies a reliability value equal to $1$ for the input variable
     \end{itemize}
   \item \xmlNode{target}, \xmlDesc{string, required field}, ID of the output variable. This nodes needs to contain also the attribute \xmlAttr{values}, \xmlDesc{string, required field}, interval of
                                                             values of the output target variable that defines the desired outcome
\end{itemize}

\textbf{Example:}
This example shows an example where it is desired to calculate all available risk importance measures for two input variables (i.e., pumpTime and valveTime)
given an output target variable (i.e., Tmax).
A value of the input variable pumpTime in the interval $[0,240]$ implies a reliability value of the input variable pumpTime equal to $0$.
A value of the input variable valveTime in the interval $[0,60]$ implies a reliability value of the input variable valveTime equal to $0$.
A value of the input variables valveTime and pumpTime in the interval $[1441,2880]$ implies a reliability value of the input variables equal to $1$.
The desired outcome of the output variable Tmax occurs in the interval $[2200,2500]$.
\begin{lstlisting}[style=XML,morekeywords={subType,debug,name,class,type}]
<Simulation>
  ...
  <Models>
    ...
    <PostProcessor name="riskMeasuresDiscrete" subType="InterfacedPostProcessor">
      <method>RiskMeasuresDiscrete</method>
      <measures>B,FV,RAW,RRW</measures>
      <variable R0values='0,240' R1values='1441,2880'>pumpTime</variable>
      <variable R0values='0,60'  R1values='1441,2880'>valveTime</variable>
      <target   values='2200,2500'                  >Tmax</target>
    </PostProcessor>
    ...
  </Models>
  ...
</Simulation>
\end{lstlisting}

This Post-Processor allows the user to consider also multiple datasets (a data set for each initiating event) and calculate the global risk importance measures.
This can be performed by:
\begin{itemize}
  \item Including all datasets in the step
\begin{lstlisting}[style=XML,morekeywords={subType,debug,name,class,type}]
<Simulation>
  ...
  </Steps>
    ...
    <PostProcess name="PP">
      <Input   class="DataObjects"  type="PointSet"        >outRun1</Input>
      <Input   class="DataObjects"  type="PointSet"        >outRun2</Input>
      <Model   class="Models"       type="PostProcessor"   >riskMeasuresDiscrete</Model>
      <Output  class="DataObjects"  type="PointSet"        >outPPS</Output>
      <Output  class="OutStreams"   type="Print"           >PrintPPS_dump</Output>
    </PostProcess>
  </Steps>
  ...
</Simulation>
\end{lstlisting}
  \item Adding in the Post-processor the frequency of the initiating event associated to each dataset
\begin{lstlisting}[style=XML,morekeywords={subType,debug,name,class,type}]
<Simulation>
  ...
  <Models>
    ...
    <PostProcessor name="riskMeasuresDiscrete" subType="InterfacedPostProcessor">
      <method>riskMeasuresDiscrete</method>
      <measures>FV,RAW</measures>
      <variable R1values='-0.1,0.1' R0values='0.9,1.1'>Astatus</variable>
      <variable R1values='-0.1,0.1' R0values='0.9,1.1'>Bstatus</variable>
      <variable R1values='-0.1,0.1' R0values='0.9,1.1'>Cstatus</variable>
      <variable R1values='-0.1,0.1' R0values='0.9,1.1'>Dstatus</variable>
      <target   values='0.9,1.1'>outcome</target>
      <data     freq='0.01'>outRun1</data>
      <data     freq='0.02'>outRun2</data>
    </PostProcessor>
    ...
  </Models>
  ...
</Simulation>
\end{lstlisting}

\end{itemize}

This post-processor can be made time dependendent if a single HistorySet is provided among the other data objects.
The HistorySet contains the temporal profiles of a subset of the input variables. This temporal profile can be only
boolean, i.e., 0 (component offline) or 1 (component online).
Note that the provided history set must contains a single History; multiple Histories are not allowed.
When this post-processor is in a dynamic configuration (i.e., time-dependent), the user is required to specify an xml
node \xmlNode{temporalID} that indicates the ID of the temporal variable.
For each time instant, this post-processor determines the temporal profiles of the desired risk importance measures.
Thus, in this case, an HistorySet must be chosen as an output data object.
An example is shown below:
\begin{lstlisting}[style=XML,morekeywords={subType,debug,name,class,type}]
<Simulation>
  ...
  <Models>
    ...
    <PostProcessor name="riskMeasuresDiscrete" subType="InterfacedPostProcessor">
      <method>riskMeasuresDiscrete</method>
      <measures>B,FV,RAW,RRW,R0</measures>
      <variable R1values='-0.1,0.1' R0values='0.9,1.1'>Astatus</variable>
      <variable R1values='-0.1,0.1' R0values='0.9,1.1'>Bstatus</variable>
      <variable R1values='-0.1,0.1' R0values='0.9,1.1'>Cstatus</variable>
      <target   values='0.9,1.1'>outcome</target>
      <data     freq='1.0'>outRun1</data>
      <temporalID>time</temporalID>
    </PostProcessor>
    ...
  </Models>
  ...
  <Steps>
    ...
    <PostProcess name="PP">
      <Input     class="DataObjects"  type="PointSet"        >outRun1</Input>
      <Input     class="DataObjects"  type="HistorySet"      >timeDepProfiles</Input>
      <Model     class="Models"       type="PostProcessor"   >riskMeasuresDiscrete</Model>
      <Output    class="DataObjects"  type="HistorySet"      >outHS</Output>
      <Output    class="OutStreams"   type="Print"           >PrintHS</Output>
    </PostProcess>
    ...
  </Steps>
  ...
</Simulation>
\end{lstlisting}

%
%%%%%%%%%%%%%%%%%%%%%%%%%%%%
%%%%%% RavenOutput PP   %%%%%%
%%%%%%%%%%%%%%%%%%%%%%%%%%%%
%

\subsubsection{RavenOutput}
\label{RavenOutput}
The \textbf{RavenOutput} post-processor is specifically used
to gather data from RAVEN output files and generate a PointSet suitable for plotting or other analysis.
It can do this in two modes: static and dynamic.  In static mode, the
PostProcessor reads from from several static XML output files produced by RAVEN.  In dynamic mode, the PostProcessor
reads from a single dynamic XML output file and builds a PointSet where the pivot parameter (e.g. time) is the
input and the requested values are returned for each of the pivot parameter values (e.g. points in time).  The
name for the pivot parameter will be taken directly from the XML structure.
%
Note: by default the PostProcessor operates in static mode; to read a dynamic file, the \xmlNode{dynamic} node must
be specified.
%
\ppType{RavenOutput}{RavenOutput}
%
\begin{itemize}
  \item \xmlNode{dynamic}, \xmlDesc{string, optional field}, if included will trigger reading a single dynamic
  file instead of multiple static files, unless the text of this field is \xmlString{false}, in which case it
  will return to the default (multiple static files).  \default(False)
  \item \xmlNode{File}, \xmlDesc{XML Node, required field}
  %
  For each file to be read by this postprocessor, an entry in the \xmlNode{Files} node must be added, and a
  \xmlNode{File} node must be added to the postprocessor input block.  The \xmlNode{File} requires two
  identifying attributes:
  \begin{itemize}
    \item \xmlAttr{name}, \xmlDesc{string, required field}, the RAVEN-assigned name of the file,
    \item \xmlAttr{ID}, \xmlDesc{float, optional field}, the floating point ID that will be unique to this
      file.  This will appear as an entry in the output \xmlNode{DataObject} and the corresponding column are
      the values extracted from this file.  If not specified, RAVEN will attempt to find a suitable integer ID
      to use, and a warning will be raised.

      When defining the \xmlNode{DataObject} that this postprocessor will write to, and when using the static
      (non-\xmlNode{dynamic}) form of the postprocessor, the \xmlNode{input} space should be given as
      \xmlString{ID}, and the output variables should be the outputs specified in the postprocessor. See the
      examples below.  In the data object, the variable values will be keyed on the \xmlString{ID} parameter.
  \end{itemize}
  Each value that needs to be extracted from the file needs to be specified by one of the following
  \xmlNode{output} nodes within the \xmlNode{File} node:
  \begin{itemize}
    \item \xmlNode{output}, \xmlDesc{|-separated string, required field},
           the specification of the output to extract from the file.
           RAVEN uses \texttt{xpath} as implemented in Python's \texttt{xml.etree} module to specify locations
           in XML.  For example, to search tags, use a path
           separated by forward slash characters (``/''), starting under the root; this means the root node should not
           be included in the path. See the example.  For more details on xpath options available, see
           \url{https://docs.python.org/2/library/xml.etree.elementtree.html#xpath-support}.
           %
           The \xmlNode{output} node requires the following attribute:
      \begin{itemize}
        \item \xmlAttr{name}, \xmlDesc{string, required field}, specifies the entry in the Data Object that
          this value should be stored under.
      \end{itemize}

  \end{itemize}
  %
\end{itemize}
\textbf{Example (Static):}
Using an example, let us have two input files, named \emph{in1.xml} and \emph{in2.xml}.  They appear as
follows.  Note that the name of the variables we want changes slightly between the XML; this is fine.

\textbf{\emph{in1.xml}}
\begin{lstlisting}[style=XML]
<BasicStatistics>
  <ans>
    <val1>6</val1>
    <val2>7</val2>
  </ans>
</BasicStatistics>
\end{lstlisting}
\textbf{\emph{in2.xml}}
\begin{lstlisting}[style=XML]
<ROM>
  <ans>
    <first>6.1</first>
    <second>7.1</second>
  </ans>
</BasicStatistics>
\end{lstlisting}

The RAVEN input to extract this information would appear as follows.
We include an example of defining the \xmlNode{DataObject} that this postprocessor will write out to, for
further clarity.

\begin{lstlisting}[style=XML]
<Simulation>
 ...
 <Files>
   <Input name='in1'>inp1.xml</Input>
   <Input name='in2'>inp2.xml</Input>
 </Files>
 ...
 <Models>
   ...
   <PostProcessor name='pp' subType='RavenOutput'>
     <File name='in1' ID='1'>
       <output name='first'>ans/val1</output>
       <output name='second'>ans/val2</output>
     </File>
     <File name='in2' ID='2'>
       <output name='first'>ans/first</output>
       <output name='second'>ans/second</output>
     </File>
   </PostProcessor>
   ...
 </Models>
 ...
 <DataObjects>
   ...
   <PointSet name='pointSetName'>
     <input>ID</input>
     <output>first,second</output>
   </PointSet>
   ...
 </DataObjects>
 ...
</Simulation>
\end{lstlisting}

\textbf{Example (Dynamic):}
For a dynamic example, consider this time-evolution of values example.  \emph{inFile.xml} is a RAVEN dynamic
XML output.

\textbf{\emph{in1.xml}}
\begin{lstlisting}[style=XML]
<BasicStatistics type='Dynamic'>
  <time value='0.0'>
    <ans>
      <val1>6</val1>
      <val2>7</val2>
    </ans>
  <\time>
  <time value='1.0'>
    <ans>
      <val1>9</val1>
      <val2>10</val2>
    </ans>
  <\time>
</BasicStatistics>
\end{lstlisting}
The RAVEN input to extract this information would appear as follows:
\begin{lstlisting}[style=XML]
<Simulation>
 ...
 <Files>
   <Input name='inFile'>inFile.xml</Input>
 </Files>
 ...
 <Models>
   ...
   <PostProcessor name='pp' subType='RavenOut'>
     <dynamic>true</dynamic>
     <File name='inFile'>
       <output name='first'>ans|val1</output>
     </File>
   </PostProcessor>
   ...
 </Models>
 ...
</Simulation>
\end{lstlisting}
The resulting PointSet has \emph{time} as an input and \emph{first} as an output.

%%%%%%%%%%%%%% Metric PP %%%%%%%%%%%%%%%%%%%

\subsubsection{Metric}
\label{MetricPP}
The \textbf{Metric} post-processor is specifically used to calculate the distance values among points and histories,
while the \textbf{Metrics} block (See Chapter \ref{sec:Metrics}) allows the user to specify the similarity/dissimilarity metrics to be used in this
post-processor. It is important to notice that this post-processor currently can only accept \textbf{PointSet} data
object and does not accept \textbf{HistorySet} data.  If the name of the variable is unique, it can be used, otherwise the variable can be specified with DataObjectName|InputOrOutput|variablename like other places in RAVEN.  Some of the Metrics also accept distributions to calculate the distance against.  These are specified by using the name of the distribution.
%
\ppType{Metric}{Metric}
%
\begin{itemize}
  \item \xmlNode{Features}, \xmlDesc{comma separated string, required field}, specifies the names of the features.
    This xml-node accepts the following attribute:
    \begin{itemize}
      \item \xmlAttr{type}, \xmlDesc{required string attribute}, the type of provided features. Currently only
        accept `variable'.
    \end{itemize}
  \item \xmlNode{Targets}, \xmlDesc{comma separated string, required field}, contains a comma separated list of
    the targets. \nb Each target is paired with a feature listed in xml node \xmlNode{Features}. In this case, the
    number of targets should be equal to the number of features.
    This xml-node accepts the following attribute:
    \begin{itemize}
      \item \xmlAttr{type}, \xmlDesc{required string attribute}, the type of provided features. Currently only
        accept `variable'.
    \end{itemize}
  \item \xmlNode{Metric}, \xmlDesc{string, required field}, specifies the \textbf{Metric} name that is defined via
    \textbf{Metrics} entity. In this xml-node, the following xml attributes need to be specified:
    \begin{itemize}
      \item \xmlAttr{class}, \xmlDesc{required string attribute}, the class of this metric (e.g. Metrics)
      \item \xmlAttr{type}, \xmlDesc{required string attribute}, the sub-type of this Metric (e.g. SKL, Minkowski)
    \end{itemize}
\end{itemize}

\textbf{Example:}

\begin{lstlisting}[style=XML]
<Simulation>
 ...
  <Models>
    ...
    <PostProcessor name="pp1" subType="Metric">
      <Features type="variable">ans</Features>
      <Targets type="variable">ans2</Targets>
      <Metric class="Metrics" type="SKL">euclidean</Metric>
      <Metric class="Metrics" type="SKL">rbf</Metric>
      <Metric class="Metrics" type="SKL">poly</Metric>
      <Metric class="Metrics" type="SKL">sigmoid</Metric>
      <Metric class="Metrics" type="SKL">polynomial</Metric>
      <Metric class="Metrics" type="SKL">linear</Metric>
      <Metric class="Metrics" type="SKL">cosine</Metric>
      <Metric class="Metrics" type="SKL">cityblock</Metric>
      <Metric class="Metrics" type="SKL">l1</Metric>
      <Metric class="Metrics" type="SKL">l2</Metric>
      <Metric class="Metrics" type="SKL">manhattan</Metric>
      <Metric class="Metrics" type="SKL">laplacian</Metric>
    </PostProcessor>
    ...
  </Models>
 ...
</Simulation>
\end{lstlisting}

%%%%%%%%%%%%%% Cross Validation PP %%%%%%%%%%%%%%%%%%%

\subsubsection{CrossValidation}
\label{CVPP}
The \textbf{CrossValidation} post-processor is specifically used to evaluate estimator (i.e. ROMs) performance.
Cross-validation is a statistical method of evaluating and comparing learning algorithms by dividing data into
two portions: one used to `train' a surrogate model and the other used to validate the model, based on specific
scoring metrics. In typical cross-validation, the training and validation sets must crossover in successive
rounds such that each data point has a chance of being validated against the various sets. The basic form of
cross-validation is k-fold cross-validation. Other forms of cross-validation are special cases of k-fold or involve
repeated rounds of k-fold cross-validation. \nb It is important to notice that this post-processor currently can
only accept \textbf{PointSet} data object.
%
\ppType{CrossValidation}{CrossValidation}
%
\begin{itemize}
  \item \xmlNode{SciKitLearn}, \xmlDesc{string, required field}, the subnodes specifies the necessary information
    for the algorithm to be used in the post-processor. `SciKitLearn' is based on algorithms in SciKit-Learn
    library, and currently it performs cross-validation over \textbf{PointSet} only.
  \item \xmlNode{Metric}, \xmlDesc{string, required field}, specifies the \textbf{Metric} name that is defined via
    \textbf{Metrics} entity. In this xml-node, the following xml attributes need to be specified:
    \begin{itemize}
      \item \xmlAttr{class}, \xmlDesc{required string attribute}, the class of this metric (e.g. Metrics)
      \item \xmlAttr{type}, \xmlDesc{required string attribute}, the sub-type of this Metric (e.g. SKL, Minkowski)
    \end{itemize}
    \nb Currently, cross-validation post-processor only accepts \xmlNode{SKL} metrics with \xmlNode{metricType}
    \xmlString{mean\_absolute\_error}, \xmlString{explained\_variance\_score}, \xmlString{r2\_score},
    \xmlString{mean\_squared\_error}, and \xmlString{median\_absolute\_error}.
\end{itemize}

\paragraph{SciKitLearn}

The algorithm for cross-validation is chosen by the subnode \xmlNode{SKLtype} under the parent node \xmlNode{SciKitLearn}.
In addition, a special subnode \xmlNode{average} can be used to obtain the average cross validation results.

\begin{itemize}
  \item \xmlNode{SKLtype}, \xmlDesc{string, required field}, contains a string that
    represents the cross-validation algorithm to be used. As mentioned, its format is:

    \xmlNode{SKLtype}algorithm\xmlNode{/SKLtype}.
  \item \xmlNode{average}, \xmlDesc{boolean, optional field}, if `True`, dump the average cross validation results into the
    output files.
\end{itemize}


Based on the \xmlNode{SKLtype} several different algorithms are available. In the following paragraphs a brief
explanation and the input requirements are reported for each of them.

\paragraph{K-fold}
\textbf{KFold} divides all the samples in $k$ groups of samples, called folds (if $k=n$, this is equivalent to the
\textbf{Leave One Out} strategy), of equal sizes (if possible). The prediction function is learned using $k-1$ folds,
and fold left out is used for test.
In order to use this algorithm, the user needs to set the subnode:
\xmlNode{SKLtype}KFold\xmlNode{/SKLtype}.
In addition to this XML node, several others are available:
\begin{itemize}
  \item \xmlNode{n\_splits}, \xmlDesc{integer, optional field}, number of folds, must be at least 2. \default{3}
  \item \xmlNode{shuffle}, \xmlDesc{boolean, optional field}, whether to shuffle the data before splitting into
    batches.
  \item \xmlNode{random\_state}, \xmlDesc{None, integer or RandomState, optional field}, when shuffle=True,
    pseudo-random number generator state used for shuffling. If None, use default numpy RNG for shuffling.
\end{itemize}

\paragraph{Stratified k-fold}
\textbf{StratifiedKFold} is a variation of \textit{k-fold} which returns stratified folds: each set contains approximately
the same percentage of samples of each target class as the complete set.
In order to use this algorithm, the user needs to set the subnode:

\xmlNode{SKLtype}StratifiedKFold\xmlNode{/SKLtype}.

In addition to this XML node, several others are available:
\begin{itemize}
  \item \xmlNode{y}, \xmlDesc{array-like, [n\_samples], required field}, samples to split in K folds.
  \item \xmlNode{n\_splits}, \xmlDesc{integer, optional field}, number of folds, must be at least 2. \default{3}
  \item \xmlNode{shuffle}, \xmlDesc{boolean, optional field}, whether to shuffle the data before splitting into
    batches.
  \item \xmlNode{random\_state}, \xmlDesc{None, integer or RandomState, optional field}, when shuffle=True,
    pseudo-random number generator state used for shuffling. If None, use default numpy RNG for shuffling.
\end{itemize}

\paragraph{Label k-fold}
\textbf{LabelKFold} is a variation of \textit{k-fold} which ensures that the same label is not in both testing and
training sets. This is necessary for example if you obtained data from different subjects and you want to avoid
over-fitting (i.e., learning person specific features) by testing and training on different subjects.
In order to use this algorithm, the user needs to set the subnode:

\xmlNode{SKLtype}LabelKFold\xmlNode{/SKLtype}.

In addition to this XML node, several others are available:
\begin{itemize}
  \item \xmlNode{labels}, \xmlDesc{array-like with shape (n\_samples, ), required field}, contains a label for
    each sample. The folds are built so that the same label does not appear in two different folds.
  \item \xmlNode{n\_splits}, \xmlDesc{integer, optional field}, number of folds, must be at least 2. \default{3}
\end{itemize}

\paragraph{Leave-One-Out - LOO}
\textbf{LeaveOneOut} (or LOO) is a simple cross-validation. Each learning set is created by taking all the samples
except one, the test set being the sample left out. Thus, for $n$ samples, we have $n$ different training sets and
$n$ different tests set. This is cross-validation procedure does not waste much data as only one sample is removed from
the training set.
In order to use this algorithm, the user needs to set the subnode:

\xmlNode{SKLtype}LeaveOneOut\xmlNode{/SKLtype}.

\paragraph{Leave-P-Out - LPO}
\textbf{LeavePOut} is very similar to \textbf{LeaveOneOut} as it creates all the possible training/test sets by removing
$p$ samples from the complete set. For $n$ samples, this produces $(^n_p)$ train-test pairs. Unlike \textbf{LeaveOneOut}
and \textbf{KFold}, the test sets will overlap for $p > 1$.
In order to use this algorithm, the user needs to set the subnode:

\xmlNode{SKLtype}LeavePOut\xmlNode{/SKLtype}.

In addition to this XML node, several others are available:
\begin{itemize}
  \item \xmlNode{p}, \xmlDesc{integer, required field}, size of the test sets
\end{itemize}

\paragraph{Leave-One-Label-Out - LOLO}
\textbf{LeaveOneLabelOut} (LOLO) is a cross-validation scheme which holds out the samples according to a third-party
provided array of integer labels. This label information can be used to encode arbitrary domain specific pre-defined
cross-validation folds. Each training set is thus constituted by all samples except the ones related to a specific
label.
In order to use this algorithm, the user needs to set the subnode:

\xmlNode{SKLtype}LeaveOneLabelOut\xmlNode{/SKLtype}.

In addition to this XML node, several others are available:
\begin{itemize}
  \item \xmlNode{labels}, \xmlDesc{array-like of integer with shape (n\_samples,), required field}, arbitrary
    domain-specific stratificatioin of the data to be used to draw the splits.
\end{itemize}

\paragraph{Leave-P-Label-Out}
\textbf{LeavePLabelOut} is imilar as \textit{Leave-One-Label-Out}, but removes samples related to $P$ labels for
each training/test set.
In order to use this algorithm, the user needs to set the subnode:

\xmlNode{SKLtype}LeavePLabelOut\xmlNode{/SKLtype}.

In addition to this XML node, several others are available:
\begin{itemize}
  \item \xmlNode{labels}, \xmlDesc{array-like of integer with shape (n\_samples,), required field}, arbitrary
    domain-specific stratificatioin of the data to be used to draw the splits.
  \item \xmlNode{p}, \xmlDesc{integer, optional field}, number of samples to leave out in the test split.
\end{itemize}

\paragraph{ShuffleSplit}
\textbf{ShuffleSplit} iterator will generate a user defined number of independent train/test dataset splits. Samples
are first shuffled and then split into a pair of train and test sets. it is possible to control the randomness for
reproducibility of the results by explicitly seeding the \xmlNode{random\_state} pseudo random number generator.
In order to use this algorithm, the user needs to set the subnode:

\xmlNode{SKLtype}ShuffleSplit\xmlNode{/SKLtype}.

In addition to this XML node, several others are available:
\begin{itemize}
  \item \xmlNode{n\_iter}, \xmlDesc{integer, optional field}, number of re-shuffling and splitting iterations
    \default{10}.
  \item \xmlNode{test\_size}, \xmlDesc{float, integer or None}, if float, should be between 0.0 and 1.0 and
    represent the proportion of the dataset to include in the test split. \default{0.1}
    If integer, represents the absolute number of test samples. If None, the value is automatically set to
    the complement of the train size.
  \item \xmlNode{train\_size}, \xmlDesc{float, integer or None}, if float, should be between 0.0 and 1.0 and represent
    the proportion of the dataset to include in the train split. If integer, represents the absolute number of train
    samples. If None, the value is automatically set to the complement of the test size. \default{None}
  \item \xmlNode{random\_state}, \xmlDesc{None, integer or RandomState, optional field}, when shuffle=True,
    pseudo-random number generator state used for shuffling. If None, use default numpy RNG for shuffling.
\end{itemize}

\paragraph{Label-Shuffle-Split}
\textbf{LabelShuffleSplit} iterator behaves as a combination of \textbf{ShuffleSplit} and \textbf{LeavePLabelOut},
and generates a sequence of randomized partitions in which a subset of labels are held out for each split.
In order to use this algorithm, the user needs to set the subnode:

\xmlNode{SKLtype}LabelShuffleSplit\xmlNode{/SKLtype}.

In addition to this XML node, several others are available:
\begin{itemize}
  \item \xmlNode{labels}, \xmlDesc{array, [n\_samples]}, labels of samples.
  \item \xmlNode{n\_iter}, \xmlDesc{integer, optional field}, number of re-shuffling and splitting iterations
    \default{10}.
  \item \xmlNode{test\_size}, \xmlDesc{float, integer or None}, if float, should be between 0.0 and 1.0 and
    represent the proportion of the dataset to include in the test split. \default{0.1}
    If integer, represents the absolute number of test samples. If None, the value is automatically set to
    the complement of the train size.
  \item \xmlNode{train\_size}, \xmlDesc{float, integer or None}, if float, should be between 0.0 and 1.0 and represent
    the proportion of the dataset to include in the train split. If integer, represents the absolute number of train
    samples. If None, the value is automatically set to the complement of the test size. \default{None}
  \item \xmlNode{random\_state}, \xmlDesc{None, integer or RandomState, optional field}, when shuffle=True,
    pseudo-random number generator state used for shuffling. If None, use default numpy RNG for shuffling.
\end{itemize}

\textbf{Example:}

\begin{lstlisting}[style=XML]
<Simulation>
 ...
  <Files>
    <Input name="output_cv" type="">output_cv.xml</Input>
    <Input name="output_cv.csv" type="">output_cv.csv</Input>
  </Files>
  <Models>
    ...
    <ROM name="surrogate" subType="SciKitLearn">
      <SKLtype>linear_model|LinearRegression</SKLtype>
      <Features>x1,x2</Features>
      <Target>ans</Target>
      <fit_intercept>True</fit_intercept>
      <normalize>True</normalize>
    </ROM>
    <PostProcessor name="pp1" subType="CrossValidation">
        <SciKitLearn>
            <SKLtype>KFold</SKLtype>
            <n_splits>3</n_splits>
            <shuffle>False</shuffle>
            <random_state>None</random_state>
        </SciKitLearn>
        <Metric class="Metrics" type="SKL">m1</Metric>
    </PostProcessor>
    ...
  </Models>
  <Metrics>
    <SKL name="m1">
      <metricType>mean_absolute_error</metricType>
    </SKL>
  </Metrics>
  <Steps>
    <PostProcess name="PP1">
        <Input class="DataObjects" type="PointSet">outputDataMC</Input>
        <Input class="Models" type="ROM">surrogate</Input>
        <Model class="Models" type="PostProcessor">pp1</Model>
        <Output class="Files" type="">output_cv</Output>
        <Output class="Files" type="">output_cv.csv</Output>
    </PostProcess>
  </Steps>
 ...
</Simulation>
\end{lstlisting}
