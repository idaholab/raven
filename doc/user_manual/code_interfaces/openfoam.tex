%%%%%%%%%%%%%%%%%%%%%
%%% OpenFOAM INTERFACE %%%
%%%%%%%%%%%%%%%%%%%%%
\subsection{OpenFOAM Interface}
This section presents the main aspects of the interface between RAVEN and OpenFOAM (\url{https://www.openfoam.com}).
OpenFOAM (Open Source Field Operation and Manipulation) is a free, open-source CFD toolbox written in modern C++ 
that provides a modular set of libraries and more than 100 ready-to-run solvers for fluid flow, heat transfer, multiphase, 
turbulence, chemical reactions, electro-magnetics, solid mechanics, and coupled multi-physics problems. Built on a flexible 
finite-volume framework, it supports arbitrary polyhedral meshes, dynamic mesh motion, and high-order discretization schemes; 
users can extend or compose functionality simply by adding new classes without altering the core. Native MPI parallelization, 
in-situ post-processing, and integration with ParaView enable scalable simulations from workstations to HPC clusters, while 
command-line utilities (e.g., mesh generation, field manipulation, data conversion) and Python/PyVista bindings streamline 
pre- and post-processing workflows. Maintained under the GPL (version 3), OpenFOAM enjoys an active international community that 
contributes tutorials, documentation, and continuous releases, making it a widely adopted platform for both academic 
research and industrial engineering analysis.

\nb \textcolor{red}{OpenFOAM outputs are parsed using the external library \href{https://pyvista.org}{ \textcolor{red}{pyvista}}, 
                             which \textbf{IS NOT} a default RAVEN library and consequentially \textbf{REQUIRES} to install RAVEN libraries
                             (as detailed in Sections~\ref{sec:clone raven} and ~\ref{subsec:installPythonLibs} )
                             using the ``--code-interface-deps'' command line option.} 

In this section, the OpenFOAM-specific RAVEN input adjustments and the modifications of the OpenFOAM
input files are reported. \noindent
%%%%%%%%%%%%%%%%%%%%%%%%%%%%%%%%%%%%%%%%%%%%%%%%%%%
\subsubsection{Sequence}
%%%%%%%%%%%%%%%%%%%%%%%%%%%%%%%%%%%%%%%%%%%%%%%%%%%
In the \xmlNode{Sequence} section, the names of the steps declared in the
\xmlNode{Steps} block should be specified.
%
As an example, if we called the first MultiRun ``MC\_Sample'' and the second
MultiRun ``plotResults'' in the sequence section we should see this:

\begin{lstlisting}[style=XML]
<Sequence>MC_Sampler,plotResults</Sequence>
\end{lstlisting}

%%%%%%%%%%%%%%%%%%%%%%%%%%%%%%%%%%%%%%%%%%%%%%%%%%%
\subsubsection{batchSize and mode}
%%%%%%%%%%%%%%%%%%%%%%%%%%%%%%%%%%%%%%%%%%%%%%%%%%%
For the \xmlNode{batchSize} and \xmlNode{mode} sections please refer to the
\xmlNode{RunInfo} block in the previous chapters.

%%%%%%%%%%%%%%%%%%%%%%%%%%%%%%%%%%%%%%%%%%%%%%%%%%%%
\subsubsection{RunInfo}
%%%%%%%%%%%%%%%%%%%%%%%%%%%%%%%%%%%%%%%%%%%%%%%%%%%%
After all of these blocks are filled out, a standard example RunInfo block may
look like the example below: \\

\begin{lstlisting}[style=XML]
<RunInfo>
  <WorkingDir>~/workingDir</WorkingDir>
  <Sequence>MC_Sampler,plotResults</Sequence>
  <batchSize>8</batchSize>
</RunInfo>
\end{lstlisting}
In this example, the \xmlNode{batchSize} is set to $8$; this means that 8 simultaneous (parallel) instances
of OpenFOAM are going to be executed when a sampling strategy is employed.

%%%%%%%%%%%%%%%%%%%%%%%%%%%%%%%%%%%%%%%%%%%%%%%%%%%%%%%%%
\subsubsection{Models}
\label{subsub:OpenFOAMModels}
%%%%%%%%%%%%%%%%%%%%%%%%%%%%%%%%%%%%%%%%%%%%%%%%%%%%%%%%%
As any other Code, in order to activate the OpenFOAM interface, a \xmlNode{Code} XML node needs to be inputted, within the
main XML node \xmlNode{Models}.
\\The  \xmlNode{Code} XML node contains the
information needed to execute the specific External Code.

\attrsIntro
%

\begin{itemize}
  \itemsep0em
  \item \nameDescription
  \item \xmlAttr{subType}, \xmlDesc{required string attribute}, specifies the
  code that needs to be associated to this Model. To use OpenFOAM, the \xmlAttr{subType} needs to be set to ``OpenFOAM'' 
  %
\end{itemize}


\subnodesIntro
%
\begin{itemize}
  \item \xmlNode{executable}, \xmlDesc{string, required field} specifies the NAME
  of the executable (bash script) to be used. For the OpenFOAM interface, the ``executable'' node must contain the ``Allrun'' file (or similar), which is the bash script file
  that will be used to run OpenFOAM commands (e.g. meshing, kernel, postprocessing). The ``Allrun'' (or similar) is MANDATORY.
  %
  \nb In this node only the FILE NAME must be indicated. The ``Allrun'' (or similar) bash script MUST be located in the folder containing the ``\textbf{.foam}'' FILE (see section ~\ref{subsub:OpenFOAMfiles} for details about this file). The interface will use such location 
   \item \xmlNode{preexec}, \xmlDesc{string, optional field} specifies the name of a command (if in the SYSTEM PATH), executable or bash script that is executed before invoking the \xmlNode{executable} command. 
                                                    Generally, the pre-exec here can be used to open the openfoam shell, with all the environment variable set (e.g. invoking the command ``openfoam'' before executing the ``Allrun'' script).
                                                    \default{None}

   \item \xmlNode{directoriesPerturbableInputs}, \xmlDesc{comma separated list, required field} specifies the  list of directories that contains OpenFOAM input files and that, consequentially, will be ``candidate'' of perturbable input files. For example,
                                                                             the common folders are ``constant'', ``system'' and ``0'' or ``0.orig''.
                                                                             \nb If other folders are present in the case directory, they will still be copied into the run working directories  (e.g. ``stepName/1'', ``stepName/2'', etc.) in order for 
                                                                                   OpenFOAM to execute successfully but the files that are there contained will not be parsed to perturb variables. Consequentially, this list should contain all the folders whose
                                                                                   input files must be perturbed.
    \item \xmlNode{writeCentroids}, \xmlDesc{bool, optional field} boolean flag to request (to the interface) to write the mesh cell centroids in a dedicated CSV file (that will be written in the realization
                                                                                  subfolder (e.g. stepName/1)/                                                       
                                                                              \default{False}
   \item \xmlNode{outputVariables}, \xmlDesc{comma separated list, optional field} specifies the  list of variables that are collected from the OpenFOAM outputs and that will be made available to RAVEN for storage in DataObjects or Databases. 
                                                                      The  \xmlNode{outputVariables} serves as a ``speed up'' tool, since it constraints the amount of data collected from OpenFOAM. Indeed, if it is not present, all the OpenFOAM variables are collected.
                                                                       \default{None}
   \item \xmlNode{onlineStopCriteriaTimeInterval}, \xmlDesc{float, optional parameter}, time frequency (in seconds)) at which  the stopping function (if inputted)
                                                            is inquired (stoppingCriteriaFunction).
                                                            For time-dep calculation, a good time frequency (time interval) is of the order of a the CPU time
                                                            required for an iteration calculation. \default{5}.
   \item \xmlNode{StoppingFunction}, \xmlDesc{XML node, optional node},   the
                                                           body of this XML node must contain the name of an appropriate \textbf{Function} defined in the
                                                            \xmlNode{Functions} block (see Section~\ref{sec:functions}).  It is used as a
                                                           ``stopping'' function tool for online simulation monitoring. Indeed, for time-dep calculation, the user can provide a
                                                            function to halt the simulation if a certain condition is
                                                            met. The criterion or criteria is or are defined through the external RAVEN python function linked here through this XML node. 
                                                            The function must return a ``bool'':
                                                            \begin{itemize}
                                                              \item False, if the simulation can continue (i.e. the criteria are not met)
                                                              \item True, if the simulation must STOP (i.e. the criteria are met)
                                                            \end{itemize}
                                                            %    
                                                            Each variable defined in the \xmlNode{StoppingFunction} block is available in the
                                                            function as a python \texttt{raven.} member (\nb Since the OpenFOAM output variables are returned to RAVEN with not Python-variable-compatible characters in it (e.g. $($, $)$, $|$, etc.), 
                                                            the user should use the Alias System (as explained in the following bullet) to alias all the variables that are required to be used in \xmlNode{StoppingFunction}). 
                                                            In the following, an example of a
                                                            user-defined stopping condition function is reported (using the ``raven.cumulativeContErr'' output variable).
                                                            \nb if this function is used, the user can monitor the reason of a stop exporting the variable ``StoppingReason'' if requested in RAVEN DataObjects.
                                                            
\begin{lstlisting}[language=python]
def criteria(raven):

  if raven.cumulativeContErr[-1] > 1e-2:
    return True
  else:
    return False
\end{lstlisting}

  \item \aliasSystemDescription{Code}
  %
\end{itemize}

An example is shown  below:

\begin{lstlisting}[style=XML]
<Models>
    <Code name="MyOpenFOAMcode" subType="OpenFOAM">
      <!--
           Pre-exec here is used to open the openfoam shell
           (with all the environment variable set)
      -->
      <preexec>openfoam</preexec>
      <!--
        The executable is ALWAYS the Allrun script that the user must provide
        (this is a typical workflow in OpenFOAM simulations)
      -->
      <executable>Allrun</executable>
      
      <!--
        List of directories that contains OpenFOAM input files for this case
      -->
      <directoriesPerturbableInputs>0.orig, constant, system</directoriesPerturbableInputs>
      
      <!-- we ask the interface to write the centroids in a dedicated CSV file (that will be written in the realization
      subfolder (E.g. stepName/1) -->
      <writeCentroids>True</writeCentroids>
      <!-- in the following we collect 
           some function processed variables (e.g. average(p), average(U) )
           and mesh variables (e.g. T_12, p_12, etc.)
           the the outputVariables node is not present, all the variables are collected-->
      
      <outputVariables>cumulativeContErr, average(p), average(U)|x, average(U)|y, average(U)|z,
                       T|12, cellTypes|12, U|12|x, U|12|y, U|12|z, p|12</outputVariables>
      
      
        <!--
        For time-dependent calculation, the user can provide a
        function to halt the simulation if a certain condition is
        met. The criterion or criteria is or are defined through
        an external RAVEN python function (e.g. ``criteria'' in the
        following example). The time frequency of the check is defined
        by the node onlineStopCriteriaTimeInterval. By default, the output
        is checked every 5 seconds.
       -->
      <onlineStopCriteriaTimeInterval>
        15.0
      </onlineStopCriteriaTimeInterval>
      <StoppingFunction class="Functions" type="External">
             criteria
      </StoppingFunction>
    </Code>
</Models>

<Functions>
    <External file="stoppingCriteria.py" name="criteria">
         <!-- 
           variables below must be retrievable
           from the one of the OpenFOAM Outputs (see Output Files Conversion section)
         -->
        <variables>cumulativeContErr</variables>
    </External>
</Functions>
\end{lstlisting}

%%%%%%%%%%%%%%%%%%%%%%%%%%%%%%%%%%%%%%%%%%%%%%%%%%%%%%%%%%%%%%%%%%%%%%%%%%%%%%%%%%
\subsubsection{Files}
\label{subsub:OpenFOAMfiles}
%%%%%%%%%%%%%%%%%%%%%%%%%%%%%%%%%%%%%%%%%%%%%%%%%%%%%%%%%%%%%%%%%%%%%%%%%%%%%%%%%%
The \xmlNode{Files} XML node in the OpenFOAM interface has a different meaning with respect other Code Interfaces. Indeed,  since to run OpenFOAM, 
several inputs can be provided in multiple subfolders, which are listed using the \xmlNode{directoriesPerturbableInputs} XML node, (e.g. \textbf{constant}, \textbf{system}, \textbf{0} or \textbf{0.orig}.), the OpenFOAM interface accepts
only a single ``\textbf{.foam}'' file (as input), which  is simply used to ``mark'' the folder containing all the OpenFOAM data/inputs, including the running script  (e.g. \textbf{\textit{Allrun}}).
\\The interface will use such file to identify the path to the OpenFOAM data/input directory and it will scan for all the input files (perturbable) contained in the  \xmlNode{directoriesPerturbableInputs} sub-directories. 
The sub-folders listed in \xmlNode{directoriesPerturbableInputs} are identified starting from the location of the ``\textbf{.foam}'' input file provided.
\\In summary, for RAVEN coupled with OpenFOAM, only one file is required (the ``.foam'' input file) for RAVEN to be aware of, while the remaining ones are automatically detected by the interface. 
The  \xmlNode{Files} XML node contains the information needed to execute OpenFOAM.

\attrsIntro
%

\begin{itemize}
  \itemsep0em
  \item \nameDescription
  \item \xmlAttr{type}, \xmlDesc{required string attribute}, specifies the
  input type used by OpenFOAM (interface). The only accepted type (and required to be inputted) is: 
  \begin{itemize}
    \item openfoam, \xmlDesc{required string attribute}, identifies the OpenFOAM case file and the code currently accept any name for input as long as with the ``.foam'' extension. This \xmlAttr{type} (openfoam) is REQUIRED.
  \end{itemize}
  \item \xmlAttr{subDirectory}, \xmlDesc{required string attribute}, specifies the
   OpenFOAM case directory (directory containing all the OpenFOAM files, including this ``.foam'' file and the  \xmlNode{executable} (e.g. Allrun).
\end{itemize}

Example:

\begin{lstlisting}[style=XML]
<Files>
  <Input name="myFOAMcaseFile" type="openfoam" subDirectory="caseDirectory">case.foam</Input>
</Files>
\end{lstlisting}
The file mentioned in this section
need, then, to be placed into the ``subDirectory'' within the working directory specified
by the \xmlNode{workingDir} node in the \xmlNode{RunInfo} XML node. For example,
if the \xmlNode{workingDir} is ``myOpenFOAMsamplingAnalysis'' and the ``subDirectory'' is ``caseDirectory'', the file
must be placed in ``myOpenFOAMsamplingAnalysis/caseDirectory''.



\subsubsection{Output Files Conversion}
\label{subsubsec:openfoamInterfaceOutputConversion}

The \textbf{OpenFOAM/RAVEN output parser}\footnote{Implemented in \texttt{OpenFOAMoutputParser.py}.}
automates the extraction of the most common OpenFOAM post-processing files
and converts them into variables that can be consumed directly by RAVEN.
The OpenFOAM parser reads native OpenFOAM dictionaries and \texttt{vol*/surface*Field} files
via \texttt{PyVista}'s \texttt{POpenFOAMReader}.
The parser currently recognizes the file types listed in
Table~\ref{tab:ofAvailableFileTypes}.

\begin{table}[ht]
\centering
\caption{OpenFOAM output file categories handled by the parser.}
\label{tab:ofAvailableFileTypes}
\begin{tabular}{|l|p{9cm}|}
\hline
\textbf{File type} & \textbf{Typical location / contents} \\ \hline
\texttt{functionObjectProperties}          & Time-directory subfolder \texttt{uniform / functionObjectProperties}; contains scalar or vector reductions written by OpenFOAM function objects (e.g.\ \texttt{average(p)}, \texttt{max(U)}). \\ \hline
\texttt{cumulativeContErr} & Optional continuity–error monitor in \texttt{uniform/}; treated as scalar or vector depending on the solver. \\ \hline
\texttt{volScalarField}                    & Volume-centered scalar fields (e.g.\ \texttt{p}, \texttt{T}). \\ \hline
\texttt{volVectorField}                    & Volume-centered vector fields (e.g.\ \texttt{U}). \\ \hline
\end{tabular}
\end{table}

\paragraph{General mechanics.}
The parser receives the case directory (i.e. ``stepName/sampleNumber/\xmlAttr{subDirectory}''), 
the \texttt{.foam} file (The \xmlNode{Input} of the OpenFOAM interface (see section ~\ref{subsub:OpenFOAMfiles}), 
an optional list of variables to extract (the variables optionally listed in \xmlNode{outputVariables}; see section ~\ref{subsub:OpenFOAMModels}), and a flag that forces the export of cell
centroids (\texttt{centroids.csv}) when vector- or tensor-valued fields are
requested (flag that can be set via the \xmlNode{writeCentroids}).  Internally it
\begin{enumerate}
  \item scans all time directories (skipping ``\texttt{0}’’ by default),
  \item builds a single \texttt{POpenFOAMReader} instance restricted to the
        \texttt{internalMesh},
  \item reads every requested field at every time step and returns the data
        as stacked \texttt{NumPy} arrays.
\end{enumerate}
A one-dimensional array is always aligned with the automatically generated
\texttt{time} vector, so that the row index corresponds to the OpenFOAM
time directory.

\paragraph{Variable-naming convention.}
All OpenFOAM names are preserved \emph{verbatim}; the parser merely flattens
compound entities (vectors, tensors) into separate scalar variables using a
vertical bar (\texttt{|}).

\begin{itemize}
\item \emph{Scalars}: remain unchanged  
  (e.g.\ \texttt{p} $\rightarrow$ \texttt{p}).
\item \emph{Vectors of length~3}:  
  \texttt{U} $\rightarrow$ \texttt{U|x}, \texttt{U|y}, \texttt{U|z}.
\item \emph{Vectors of length $\neq3$}:  
  \texttt{gradT} ($N\!=\!4$) $\rightarrow$ \texttt{gradT|1}, \dots, \texttt{gradT|4}.
\item \emph{Rank-2 arrays} (e.g.\ stress $\sigma_{ij}$, size $3\times3$):  
  \texttt{sigma|i|j} where the first bar indexes the row, the second bar
  indexes the column  
  (e.g.\ \texttt{sigma|y|x} represents~$\sigma_{21}$).
\end{itemize}

\noindent
\textbf{Example.}  
Assume the case writes a vector field \texttt{U} and a function-object
average pressure:

\begin{lstlisting}[language=bash]
1.0/uniform/functionObjectProperties
------------------------------------
average(U)  (2.50 1.20 0.00);
average(p)  101325;

1.0/U # internal field excerpt
------------------------------
internalField   nonuniform List<vector>
3
(2.50 1.20 0.00)
(2.40 1.22 0.05)
(2.48 1.19 0.02)
...
\end{lstlisting}

The parser produces the variables summarized in
Table~\ref{tab:ofVariableExample}.

\begin{table}[ht]
\centering
\caption{Example of variables generated by the OpenFOAM parser.}
\label{tab:ofVariableExample}
\begin{tabular}{|c|c|c|c|c|}
\hline
\texttt{time} & \texttt{average(p)} & \texttt{average(U|x)} & \texttt{average(U|y)} & \texttt{average(U|z)} \\
\hline
1.0 & 101325 & 2.5 & 1.2 & 0.0 \\ \hline
\end{tabular}

\vspace{0.3cm}

\begin{tabular}{|c|c|c|c|c|}
\hline
\textbf{Cell} & \texttt{U|x} & \texttt{U|y} & \texttt{U|z} & \texttt{centroids.csv} \\
\hline
0 & 2.50 & 1.20 & 0.00 & $(x_0,y_0,z_0)$ \\
1 & 2.40 & 1.22 & 0.05 & $(x_1,y_1,z_1)$ \\
2 & 2.48 & 1.19 & 0.02 & $(x_2,y_2,z_2)$ \\
\vdots & \vdots & \vdots & \vdots & \vdots \\ \hline
\end{tabular}
\end{table}

\paragraph{Restrictions to user-defined lists.}
If the \xmlNode{outputVariables} argument is supplied, the parser keeps only the
requested prefixes (e.g.\ \texttt{``U''}, \texttt{``average(p)''}).  
An exception is raised if one or more requested variables cannot be found.

\paragraph{Optional centroid export.}
Setting \texttt{writeCentroids=True} instructs the parser to dump a
comma-separated file named \texttt{centroids.csv} in the parent folder of the
case, containing the mapping
\[
\texttt{cellID},x,y,z .
\]
This file can be post-processed in RAVEN to build spatially resolved meta-models.

\bigskip
In summary, the OpenFOAM parser delivers a flat, time-aligned table of
RAVEN-friendly variables while retaining the original field semantics and
component ordering, making the integration of CFD outputs into
uncertainty-quantification or optimization workflows seamless.

%%%%%%%%%%%%%%%%%%%%%%%%%%%%%%%%%%%%%%%%%%%%%%%%%%%%%%%%%
\subsubsection{Distributions}
%%%%%%%%%%%%%%%%%%%%%%%%%%%%%%%%%%%%%%%%%%%%%%%%%%%%%%%%%
The \xmlNode{Distribution} block defines the distributions that are going
to be used for the sampling of the variables defined in the \xmlNode{Samplers} block.
%
For all the possible distributions and all their possible inputs please see the
chapter about Distributions (Section~\ref{sec:distributions}).
%
Here we report an example of a Normal and Uniform distributions (for sampling the heat capacity (Cp) and pressure, respectively):

\begin{lstlisting}[style=XML,morekeywords={name,debug}]
<Distributions verbosity='debug'>
    <Normal name="Cp">
      <mean>1005</mean>
      <sigma>2.0</sigma>
      <lowerBound>1000</lowerBound>
      <upperBound>1010</upperBound>
    </Normal>
    <Uniform name="pressure">
      <lowerBound>1e5</lowerBound>
      <upperBound>2e5</upperBound>
    </Uniform>
 </Distributions>
\end{lstlisting}

\noindent
It is good practice to name the distribution something similar to what kind of
variable is going to be sampled, since there might be many variables with the
same kind of distributions but different input parameters.

%%%%%%%%%%%%%%%%%%%%%%%%%%%%%%%%%%%%%%%%%%%%%%%%%%%%%%%%%
\subsubsection{Samplers}
%%%%%%%%%%%%%%%%%%%%%%%%%%%%%%%%%%%%%%%%%%%%%%%%%%%%%%%%%
In the \xmlNode{Samplers} block we want to define the variables that are going to be sampled.

\noindent The perturbation or optimization of the input of any OpenFOAM sequence is performed using the approach detailed in the \textit{Generic Interface} section (see \ref{subsec:genericInterface}). 
Briefly, this approach uses
 ``wild-cards'' (placed in the original input files) for injecting the perturbed values.

\textbf{Example}:

We want to do the sampling of 2 variable:
\begin{itemize}
  \item Heat capacity ``Cp'';
  \item Pressure boundary condition ``pressure'';
\end{itemize}

\noindent We are going to sample this variable using a MonteCarlo method.
The RAVEN input is then written as follows:

\begin{lstlisting}[style=XML,morekeywords={name,type,construction,lowerBound,steps,limit,initialSeed}]
<Samplers verbosity='debug'>
  <MonteCarlo name='MC_Sampler'>
     <samplerInit> <limit>10</limit> </samplerInit>
    <variable name='MixCp'>
      <distribution>Cp</distribution>
    </variable>
    <variable name='pressure'>
      <distribution>pressure</distribution>
    </variable>
  </MonteCarlo>
</Samplers>
\end{lstlisting}

The ``Cp''  (MixCp) is reported in the OpenFoam file ``thermophysicalProperties'' located within the ``constant'' folder of the OpenFOAM case directory.
Such input file has be be modified with wild-cards in the following way.
\begin{lstlisting}[basicstyle=\tiny]
/*--------------------------------*- C++ -*----------------------------------*\
| =========                 |                                                 |
| \\      /  F ield         | OpenFOAM: The Open Source CFD Toolbox           |
|  \\    /   O peration     | Version:  v2412                                 |
|   \\  /    A nd           | Website:  www.openfoam.com                      |
|    \\/     M anipulation  |                                                 |
\*---------------------------------------------------------------------------*/
FoamFile
{
    version     2.0;
    format      ascii;
    class       dictionary;
    location    "constant";
    object      thermophysicalProperties;
}
// * * * * * * * * * * * * * * * * * * * * * * * * * * * * * * * * * * * * * //

thermoType
{
    type            heRhoThermo;
    mixture         pureMixture;
    transport       sutherland;
    thermo          hConst;
    equationOfState perfectGas;
    specie          specie;
    energy          sensibleEnthalpy;
}

mixture
{
    specie
    {
        molWeight       28.9;
    }
    thermodynamics
    {
        Cp              $RAVEN-MixCp$;
        Hf              0;
    }
    transport
    {
        As              1.4792e-06;
        Ts              116;
    }
}

dpdt    true;

// ************************************************************************* //
\end{lstlisting}

On the other hand, the ``pressure'' boundary condition is reported in the OpenFoam file ``p'' located within the ``0.orig'' folder of the OpenFOAM case directory.
Such input file has be be modified with wild-cards in the following way.
\begin{lstlisting}[basicstyle=\tiny]
/*--------------------------------*- C++ -*----------------------------------*\
| =========                 |                                                 |
| \\      /  F ield         | OpenFOAM: The Open Source CFD Toolbox           |
|  \\    /   O peration     | Version:  v2412                                 |
|   \\  /    A nd           | Website:  www.openfoam.com                      |
|    \\/     M anipulation  |                                                 |
\*---------------------------------------------------------------------------*/
FoamFile
{
    version     2.0;
    format      ascii;
    class       volScalarField;
    object      p;
}
// * * * * * * * * * * * * * * * * * * * * * * * * * * * * * * * * * * * * * //

dimensions      [1 -1 -2 0 0 0 0];

internalField   uniform $RAVEN-pressure$;

boundaryField
{
    #includeEtc "caseDicts/setConstraintTypes"

    "(walls|hole|outlet|inlet)"
    {
        type            zeroGradient;
    }

    overset
    {
        type            overset;
    }
}

// ************************************************************************* //
\end{lstlisting}

\noindent It can be seen that each variable is connected with a proper distribution
defined in the \xmlNode{Distributions} block (from the previous example).

%%%%%%%%%%%%%%%%%%%%%%%%%%%%%%%%%%%%%%%%%%%%%%%%%%%%%%%%%%%
\subsubsection{Steps}
For a OpenFOAM interface, the \xmlNode{MultiRun} step type will most likely be
used. But \xmlNode{SingleRun} step can also be used for plotting and data extraction purposes.
%
First, the step needs to be named: this name will be one of the names used in
the \xmlNode{sequence} block.
%
In our example,  \texttt{MC\_Sampler}.
%
\begin{lstlisting}[style=XML,morekeywords={name,debug,re-seeding}]
     <MultiRun name='MC_Sampler' verbosity='debug'>
\end{lstlisting}

With this step, we need to import the ``.foam'' case file that is needed by the interface to locate the case folder and then expose all the perturbable input files in the \xmlNode{directoriesPerturbableInputs} directories:
\begin{itemize}
  \item ``openfoam'' input file (extension ``.foam'')
\end{itemize}
\begin{lstlisting}[style=XML,morekeywords={name,class,type}]
    <Input class="Files" type="openfoam">myFOAMcaseFile</Input>
\end{lstlisting}

We then need to define which model will be used:
\begin{lstlisting}[style=XML]
    <Model  class='Models' type='Code'>MyOpenFOAMcode</Model>
\end{lstlisting}
We then need to specify which Sampler is used, and this can be done as follows:
\begin{lstlisting}[style=XML]
    <Sampler class='Samplers' type='MonteCarlo'>MC_Sampler</Sampler>
\end{lstlisting}
And lastly, we need to specify what kind of output the user wants.
%
For example the user might want to make a database (in RAVEN the database
created is an HDF5 file).
%
Here is a classical example:
\begin{lstlisting}[style=XML,morekeywords={class,type}]
    <Output  class='Databases' type='HDF5'>MC_out</Output>
\end{lstlisting}

Following is the example of two MultiRun step, which uses the MonteCarlo sampling
method, and create a database storing the results:
\begin{lstlisting}[style=XML]
<Steps verbosity='debug'>
  <MultiRun name='MC_Sampler' verbosity='debug' re-seeding='210491'>
    <Input   class='Files' type='openfoam'>
      myFOAMcaseFile
     </Input>
    <Model   class='Models' type='Code'>
      MyOpenFOAMcode
    </Model>
    <Sampler class='Samplers'  type='MonteCarlo'>
      MC_Sampler
    </Sampler>
    <Output  class='Databases' type='HDF5' >
      MC_out
    </Output>
    <Output  class='DataObjects' type='PointSet'>
      MonteCarloHTPIPEPointSet
    </Output>
    <Output  class='DataObjects' type='HistorySet'>
      MonteCarloHTPIPEHistorySet
    </Output>
  </MultiRun>
</Steps>
\end{lstlisting}

