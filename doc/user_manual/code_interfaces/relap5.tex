%%%%%%%%%%%%%%%%%%%%%%%%%%%%
%%%%%% RELAP5  INTERFACE  %%%%%%
%%%%%%%%%%%%%%%%%%%%%%%%%%%%
\subsection{RELAP5 Interface}
\label{subsec:RELAP5Interface}

\subsubsection{Sequence}
In the \xmlNode{Sequence} section, the names of the steps declared in the
\xmlNode{Steps} block should be specified.
%
As an example, if we called the first multirun ``Grid\_Sampler'' and the second
multirun ``MC\_Sampler'' in the sequence section we should see this:
\begin{lstlisting}[style=XML]
<Sequence>Grid_Sampler,MC_Sampler</Sequence>
\end{lstlisting}
%%%%%%%%%%%%%%%%%%%%%%%%%%%%%%%%%%%%%%%%%%%%%%%%%%%

\subsubsection{batchSize and mode}
For the \xmlNode{batchSize} and \xmlNode{mode} sections please refer to the
\xmlNode{RunInfo} block in the previous chapters.
%
%%%%%%%%%%%%%%%%%%%%%%%%%%%%%%%%%%%%%%%%%%%%%%%%%%%%
\subsubsection{RunInfo}
After all of these blocks are filled out, a standard example RunInfo block may
look like the example below:
\begin{lstlisting}[style=XML]
<RunInfo>
  <WorkingDir>~/workingDir</WorkingDir>
  <Sequence>Grid_Sampler,MC_Sampler</Sequence>
  <batchSize>1</batchSize>
  <mode>mpi</mode>
  <expectedTime>1:00:00</expectedTime>
  <ParallelProcNumb>1</ParallelProcNumb>
</RunInfo>
\end{lstlisting}
%%%%%%%%%%%%%%%%%%%%%%%%%%%%%%%%%%%%%%%%%%%%%%%%%%%%%%%%%%%
\subsubsection{Files}
In the \xmlNode{Files} section, as specified before, all of the files needed for
the code to run should be specified.
%
In the case of RELAP5, the files typically needed are:
\begin{itemize}
  \item RELAP5 Input file
  \item Table file or files that RELAP needs to run
\end{itemize}
Example:
\begin{lstlisting}[style=XML]
<Files>
  <Input name='tpfh2o' type=''>tpfh2o</Input>
  <Input name='inputrelap.i' type=''>X10.i</Input>
</Files>
\end{lstlisting}

It is a good practice to put inside the working directory all of these files and
also:
\begin{itemize}
  \item the RAVEN input file
  \item the license for the executable of RELAP5
\end{itemize}
\textcolor{red}{
\textbf{It is important to notice that the interface output collection relies on the MINOR EDITS. The user must specify the MINOR
EDITS block and those variables are the only one the INTERFACE will read and make available to RAVEN. In addition, it is important to notice that:}
\begin{itemize}
  \item \textbf{the simulation time is stored in a variable called \textit{``time''}};
  \item \textbf{all the variables specified in the MINOR EDIT block are going to be converted using underscores (e.g.  an edit such as
  $301 \:\:\: p \:\:\: 345010000$ will be named in the converted CSVs as $p\_345010000$).In addition, if a variable contains spaces, the trailing spaces
   are going to be removed and internal spaces are replaced with underscores (e.g. $HTTEMP 1131008 12$ will become $HTTEMP\_1131008\_12$}.
\end{itemize}
}

Remember also that a RELAP5 simulation run is considered successful (i.e., the simulation did not crash) if it terminates with the following
message:
\textcolor{red}{Transient terminated by end of time step cards}
or
\textcolor{red}{Transient terminated by trip}

If the a RELAP5 simulation run stops with messages other than this one (e.g., `` Transient terminated by failure.'') than the simulation is considered as
crashed, i.e., it will not be saved.
Hence, it is strongly recommended to set up the RELAP5 input file so that the simulation exiting conditions are set through control logic trip variables
(e.g., simulation mission time and clad temperature equal to clad failure temperature).

%%%%%%%%%%%%%%%%%%%%%%%%%%%%%%%%%%%%%%%%%%%%%%%%%%%%
\subsubsection{Models}
\label{subsubsection:Relap5Models}
For the \xmlNode{Models} block here is a standard example of how it would look
when using RELAP5 as the external model:
\begin{lstlisting}[style=XML]
<Models>
  <Code name='MyRELAP' subType='Relap5'>
    <executable>~/path_to_the_executable</executable>
  </Code>
</Models>
\end{lstlisting}
In case the \textbf{multi-deck} approach is used in RELAP5, the interface is going to load all the outputs in one CSV RAVEN is
going to read. This means that all the decks' outputs are going to be loaded in one of the Output of RAVEN. In case the user
wants to select the outputs coming from only one deck, the following XML node needs to be specified:
\begin{itemize}
   \item \xmlNode{outputDeckNumber}, \xmlDesc{integer, optional parameter}, the deck number from
   which the results needs to be retrieved. \default{all}.
\end{itemize}
In addition, if some command line parameters need to be passed to RELAP5 \\(e.g. ``-r
$\: restartFileWithCustomName.r$''), the user might use (optionally) the \xmlNode{clargs} XML nodes.
\begin{lstlisting}[style=XML]
<Models>
  <Code name='MyRELAP' subType='Relap5'>
    <executable>~/path_to_the_executable</executable>
    <outputDeckNumber>1</outputDeckNumber>
    <clargs type="text" arg="-r restartFileWithCustomName.r"/>
  </Code>
</Models>
\end{lstlisting}

An additional feature of the RELAP5 Code interface is the possibility to specify operation based on the value of user-inputted
cards. For example, let's assume the values in cards 1180801:2 and 1180802:2 must come from a calculation based on sampled variables (e.g. 20100154:2 and 20100155:2), the user can specify the following XML node:

\begin{itemize}
   \item \xmlNode{operator}, \xmlDesc{XML node, optional parameter}, The operator block.
    This XML node must contain the following attribute:
     \begin{itemize}
      \item \textit{variables}, \xmlDesc{comma separated list, required parameter}, The list of variables
       (coming from a Sampler) that will be used in the  \xmlNode{expression} XML node.
      \end{itemize}
     Within the  \xmlNode{operator} the following XML sub-nodes must be specified:
     \begin{itemize}
        \item \xmlNode{expression}, \xmlDesc{string, required parameter}, The string representing the expression to be
        performed. The ``card'' (if needed to be used) must be identified with the token \%card\% and it
         will be replaced with the values of the cards
        (specified in the XML node \xmlNode{cards} ) from the original input file. In this expression, all the functions available in
        the Python $math$ module can be used (e.g. $sqrt$, $exp$, $sin$, etc.).
         \item \xmlNode{cards}, \xmlDesc{comma separated list, required parameter},
          The list of cards in the original input file whose values need to be replaced by the value resulting from the expression
          contained in \xmlNode{expression}.
     \end{itemize}
    \nb the user can specify as many \xmlNode{operator} nodes as needed.
\end{itemize}
An example is reported below:
\begin{lstlisting}[style=XML]
<Models>
  <Code name='MyRELAP' subType='Relap5'>
    <executable>~/path_to_the_executable</executable>
    ...
    <operator variables="20100154:2,20100155:2">
      <expression> %card%*20100155:2*2./20100155:2</expression>
      <cards>1180801:2,1180802:2,1180901:3</cards>
    </operator>
    ...
  </Code>
</Models>
\end{lstlisting}


%%%%%%%%%%%%%%%%%%%%%%%%%%%%%%%%%%%%%%%%%%%%%%%%%%%%%%%%%
\subsubsection{Distributions}
The \xmlNode{Distribution} block defines the distributions that are going
to be used for the sampling of the variables defined in the \xmlNode{Samplers}
block.
%
For all the possible distributions and all their possible inputs please see the
chapter about Distributions (see~\ref{sec:distributions}).
%
Here we give a general example of three different distributions:
\begin{lstlisting}[style=XML,morekeywords={name,debug}]
<Distributions verbosity='debug'>
  <Triangular name='BPfailtime'>
    <apex>5.0</apex>
    <min>4.0</min>
    <max>6.0</max>
  </Triangular>
  <LogNormal name='BPrepairtime'>
    <mean>0.75</mean>
    <sigma>0.25</sigma>
  </LogNormal>
  <Uniform name='ScalFactPower'>
    <lowerBound>1.0</lowerBound>
    <upperBound>1.2</upperBound>
  </Uniform>
 </Distributions>
\end{lstlisting}

It is good practice to name the distribution something similar to what kind of
variable is going to be sampled, since there might be many variables with the
same kind of distributions but different input parameters.
%
%%%%%%%%%%%%%%%%%%%%%%%%%%%%%%%%%%%%%%%%%%%%%%%%%%%%%%%%%
\subsubsection{Samplers}
In the \xmlNode{Samplers} block we want to define the variables that are going
to be sampled.
%
\textbf{Example}:
We want to do the sampling of 3 variables:
\begin{itemize}
  \item Battery Fail Time
  \item Battery Repair Time
  \item Scaling Factor Power Rate
\end{itemize}

We are going to sample these 3 variables using two different sampling methods:
grid and MonteCarlo.

In RELAP5, the sampler reads the variable as, given the name, the first number
is the card number and the second number is the word number.
%
In this example we are sampling:
\begin{itemize}
  \item For card 0000588 (trip) the word 6 (battery failure time)
  \item For card 0000575 (trip) the word 6 (battery repair time)
  \item For card 20210000 (reactor power) the word 4 (reactor scaling factor)
\end{itemize}

We proceed to do so for both the Grid sampling and the MonteCarlo sampling.

\begin{lstlisting}[style=XML,morekeywords={name,type,construction,lowerBound,steps,limit,initialSeed}]
<Samplers verbosity='debug'>
  <Grid name='Grid_Sampler' >
    <variable name='0000588:6'>
      <distribution>BPfailtime</distribution>
      <grid type='value' construction='equal'  steps='10'>0.0 28800</grid>
    </variable>
    <variable name='0000575:6'>
      <distribution>BPrepairtime</distribution>
      <grid type='value' construction='equal' steps='10'>0.0 28800</grid>
    </variable>
    <variable name='20210000:4'>
      <distribution>ScalFactPower</distribution>
      <grid type='value' construction='equal' steps='10'>1.0 1.2</grid>
    </variable>
  </Grid>
  <MonteCarlo name='MC_Sampler'>
     <samplerInit>
       <limit>1000</limit>
     </samplerInit>
    <variable name='0000588:6'>
      <distribution>BPfailtime</distribution>
    </variable>
    <variable name='0000575:6'>
      <distribution>BPrepairtime</distribution>
    </variable>
    <variable name='20210000:4'>
      <distribution>ScalFactPower</distribution>
    </variable>
  </MonteCarlo>
</Samplers>
\end{lstlisting}

In case the RELAP5 input file is a multi-deck, the user can specify the deck to which each sampled variable
corresponds to. As an example, the following sampling strategy:

\begin{lstlisting}[style=XML,morekeywords={name,type,construction,lowerBound,steps,limit,initialSeed}]
<MonteCarlo name='MC_Sampler'>
   <samplerInit>
     <limit>1000</limit>
   </samplerInit>
  <variable name='1|0000588:6'>
    <distribution>BPfailtime</distribution>
  </variable>
  <variable name='2|0000575:6'>
    <distribution>BPrepairtime</distribution>
  </variable>
</MonteCarlo>
</Samplers>
\end{lstlisting}
performs:
\begin{itemize}
  \item the sampling of the distribution \\\xmlNode{BPfailtime} and it provides the sampled value
        to the 6th word of card 0000588 for the first deck
  \item the sampling of the distribution \\\xmlNode{BPrepairtime} and it provides the sampled value
        to the 6th word of card 0000575 for the second deck
\end{itemize}

It can be seen that each variable is connected with a proper distribution
defined in the \\\xmlNode{Distributions} block (from the previous example).
%
The following demonstrates how the input for the first variable is read.

We are sampling a a variable situated in word 6 of the card 0000588 using a Grid
sampling method.
%
The distribution that this variable is following is a Triangular distribution
(see section above).
%
We are sampling this variable beginning from 0.0 in 10 \textit{equal} steps of
2880.
%
In case of Dynamic Event Tree-based sampling, the input is very similar to the other
sampling strategies with the only ``limitation'' that the sampled variables
directly linked to the Dynamic Event Tree must be part of a RELAP5 trip.
In case, other variables must be sampled, they are considered ``epistemic'' variables and
should be sampled using the Hybrid Dynamic Event Tree approach.


For example, the Dynamic Event Tree sampling and the  Hybrid Dynamic Event Tree sampling would look
like the following:

\begin{lstlisting}[style=XML,morekeywords={name,type,construction,lowerBound,steps,limit,initialSeed}]
<Samplers verbosity='debug'>
  <DynamicEventTree name='DET'>
    <variable name='414:6'>
      <distribution>endtimedist</distribution>
      <grid type='CDF' construction='custom'>0.1 0.3 0.99</grid>
    </variable>
    <variable name='454:6'>
      <distribution>endtime2dist</distribution>
      <grid type='CDF' construction='custom'>0.11 0.5 0.99</grid>
    </variable>
  </DynamicEventTree>

  <DynamicEventTree name='HDET'>
    <HybridSampler type="MonteCarlo">
      <!-- in here we specify the epistemic like variables -->
      <samplerInit>
        <limit>10</limit>
      </samplerInit>
      <variable name="200:1">
        <distribution>missionTimeDist</distribution>
      </variable>
    </HybridSampler>
    <variable name='414:6'>
      <distribution>endtimedist</distribution>
      <grid type='CDF' construction='custom'>0.1 0.3 0.99</grid>
    </variable>
    <variable name='454:6'>
      <distribution>endtime2dist</distribution>
      <grid type='CDF' construction='custom'>0.11 0.5 0.99</grid>
    </variable>
  </DynamicEventTree>
</Samplers>
\end{lstlisting}

%%%%%%%%%%%%%%%%%%%%%%%%%%%%%%%%%%%%%%%%%%%%%%%%%%%%%%%%%%%
\subsubsection{Steps}
For a RELAP5 interface, the \xmlNode{MultiRun} step type will most likely be
used.
%
First, the step needs to be named: this name will be one of the names used in
the \xmlNode{Sequence} block.
%
In our example, \texttt{Grid\_Sampler} and \texttt{MC\_Sampler}.
%
\begin{lstlisting}[style=XML,morekeywords={name,debug,re-seeding}]
     <MultiRun name='Grid_Sampler' verbosity='debug'>
\end{lstlisting}

With this step, we need to import all the files needed for the simulation:
\begin{itemize}
  \item RELAP5 input file
  \item element tables -- tpfh2o
\end{itemize}
\begin{lstlisting}[style=XML,morekeywords={name,class,type}]
    <Input   class='Files' type=''>inputrelap.i</Input>
    <Input   class='Files' type=''>tpfh2o</Input>
\end{lstlisting}
We then need to define which model will be used:
\begin{lstlisting}[style=XML]
    <Model  class='Models' type='Code'>MyRELAP</Model>
\end{lstlisting}
We then need to specify which Sampler is used, and this can be done as follows:
\begin{lstlisting}[style=XML]
    <Sampler class='Samplers' type='Grid'>Grid_Sampler</Sampler>
\end{lstlisting}
And lastly, we need to specify what kind of output the user wants.
%
For example the user might want to make a database (in RAVEN the database
created is an HDF5 file).
%
Here is a classical example:
\begin{lstlisting}[style=XML,morekeywords={class,type}]
    <Output  class='Databases' type='HDF5'>Grid_out</Output>
\end{lstlisting}
Following is the example of two MultiRun steps which use different sampling
methods (grid and Monte Carlo), and creating two different databases for each
one:
\begin{lstlisting}[style=XML]
<Steps verbosity='debug'>
  <MultiRun name='Grid_Sampler' verbosity='debug'>
    <Input   class='Files' type=''>inputrelap.i</Input>
    <Input   class='Files'     type=''    >tpfh2o</Input>
    <Model   class='Models'    type='Code'>MyRELAP</Model>
    <Sampler class='Samplers'  type='Grid'>Grid_Sampler</Sampler>
    <Output  class='Databases' type='HDF5'>Grid_out</Output>
  </MultiRun>
  <MultiRun name='MC_Sampler' verbosity='debug' re-seeding='210491'>
    <Input   class='Files' type=''>inputrelap.i</Input>
    <Input   class='Files'     type=''          >tpfh2o</Input>
    <Model   class='Models'    type='Code'      >MyRELAP</Model>
    <Sampler class='Samplers'  type='MonteCarlo'>MC_Sampler</Sampler>
    <Output  class='Databases' type='HDF5'      >MC_out</Output>
  </MultiRun>
</Steps>
\end{lstlisting}
%%%%%%%%%%%%%%%%%%%%%%%%%%%%%%%%%%%%%%%%%%%%%%%%%%%%%%
\subsubsection{Databases}
As shown in the \xmlNode{Steps} block, the code is creating two database objects
called \texttt{Grid\_out} and \texttt{MC\_out}.
%
So the user needs to input the following:
\begin{lstlisting}[style=XML]
<Databases>
  <HDF5 name="Grid_out" readMode="overwrite"/>
  <HDF5 name="MC_out" readMode="overwrite"/>
</Databases>
\end{lstlisting}
As listed before, this will create two databases.
%
The files will have names corresponding to their \xmlAttr{name} appended with
the .h5 extension (i.e. \texttt{Grid\_out.h5} and \texttt{MC\_out.h5}).

\subsubsection{Modified Version of the Institute of Nuclear Safety System Incorporated (Japan)}
The Institute of Nuclear Safety System Incorporated (Japan) has modified the \textbf{RELAP5}  source code
in order to be able to control some additional parameters from an auxiliary input file (\textbf{modelPar.inp}).
\\In order to use this interface, the user needs to input the $subType$ attribute\textbf{Relap5inssJp}:
\begin{lstlisting}[style=XML]
<Models>
  <Code name='MyRELAP' subType='Relap5'>
    <executable>~/path_to_the_executable</executable>
    <!-- here is taking the output from the first deck only -->
    <outputDeckNumber>1</outputDeckNumber>
  </Code>
</Models>
\end{lstlisting}
For perturbing such input file, the approach presented in section \ref{subsec:genericInterface} (Generic Interface)
has been employed. For the standard \textbf{RELAP5} input, the same approach previously in this section is used.
\\For example, in the following Sampler block, the card $9100101$ is perturbed with the same approach used in standard \textbf{RELAP5}; in addition, the variable $modelParTest$  is going to be perturbed in the \textbf{modelPar.inp} input file.
\begin{lstlisting}[style=XML]
    <MonteCarlo name="mc_loca">
      <samplerInit>
        <limit>1</limit>
      </samplerInit>
      <variable name="9100101:3">
        <distribution>break_size</distribution>
      </variable>
      <variable name="modelParTest">
          <distribution>break_size</distribution>
      </variable>
    </MonteCarlo>
\end{lstlisting}
