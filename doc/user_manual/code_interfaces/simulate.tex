%%%%%%%%%%%%%%%%%%%%%%%%%%%
%%% SIMULATE3 INTERFACE %%%
%%%%%%%%%%%%%%%%%%%%%%%%%%%
\subsection{SIMULATE-3 Interface}
A recent interface between RAVEN and the neutronics code SIMULATE-3 was implemented.
The primary purpose of this section is to describe the interface associated 
with its application using RAVEN/Generic Algorithm in solving the Core optimization problem.
A brief explanation on the components of the SIMULATE-3 interface and how to use it with RAVEN 
are reported in the following subsections.

%%%%%%%%%%%%%%%%%%%%%%%%%%%%%%%%%%%%%%%%%%%%%%%%%%%%%%%%%
\subsubsection{Interface components}
%%%%%%%%%%%%%%%%%%%%%%%%%%%%%%%%%%%%%%%%%%%%%%%%%%%%%%%%%%%%%%%%%%%%%%%%%%%%%%%%%%
The interface was built based on Python 3. It included three main components: 
\begin{itemize}

    \item \texttt{SimulateInterface.py}: Connects and interacts with RAVEN main module
    \item \texttt{SimulateData.py}: Collects and extracts data from output file generated after each SIMULATE-3 simulation. The list of data can be extracted are:
    \begin{itemize}
  \item Time dependent multiplication factor $k_{eff}$;
    \item Time dependent $F_Q$;
    \item Time dependent $F_\Delta H$;
    \item Time dependent critical boron concentration;
  \end{itemize}
  \item \texttt{SpecificParser.py}: Generates SIMULATE-3 input from sample loading pattern provided by RAVEN-GA module. 
\end{itemize}

%%%%%%%%%%%%%%%%%%%%%%%%%%%%%%%%%%%%%%%%%%%%%%%%%%%%%%%%%%%%%%%%%%%%%%%%%%%%%%%%%%  
\subsubsection{Models}
%%%%%%%%%%%%%%%%%%%%%%%%%%%%%%%%%%%%%%%%%%%%%%%%%%%%%%%%%%%%%%%%%%%%%%%%%%%%%%%%%%
To successfully execute the code, users need to ensure that the code SIMULATE-3 is included in the 
\xmlNode{Models} section in a way shown below:

\begin{lstlisting}[style=XML]
  <Models>
   <Code name="MySimulate" subType="Simulate">
     <executable>...</executable>
     <sequence>simulate</sequence>
   </Code>
  </Models>
  \end{lstlisting}

%%%%%%%%%%%%%%%%%%%%%%%%%%%%%%%%%%%%%%%%%%%%%%%%%%%%%%%%%%%%%%%%%%%%%%%%%%%%%%%%%%
\subsubsection{Files}
%%%%%%%%%%%%%%%%%%%%%%%%%%%%%%%%%%%%%%%%%%%%%%%%%%%%%%%%%%%%%%%%%%%%%%%%%%%%%%%%%%
In order to execute the SIMULATE-3-RAVEN interface, user needs to provide three input files. 
Although the requested input file name can be arbitrary, they need to serve similar functions as shown below:
\begin{itemize}
    \item \xmlNode{sim3-perturb.xml}: Contains initial assigned FA type for 1/8 loading pattern following an index scheme shown below since this module was specifically designed to solve the core loading pattern in 17$\times$17 PWR core. 
    \item \xmlNode{input.inp}: Serves as a placeholder for perturbed SIMULATE-3 input deck.
    \item \xmlNode{sim3-param.xml}: Contains specification on SIMULATE-3 simulation and modeling for core including FA definition; cross section information.
\end{itemize}
The requested input will be identified in RAVEN input, then they will be loaded into RAVEN memory.
With the GA optimizer, a set of sampling for possible FA location mapping 
will be used to update \xmlNode{sim3-perturb.xml} file. From this point, the SIMULATE-3 interface will be called to
read the remaining requested inputs with the updated \xmlNode{sim3-perturb.xml} to generate input 
for SIMULATE-3 simulation. After the simulation is completed, the fitness function of GA 
optimizer will be evaluated and used to generate the population of the next generation. 
This iterative process will continue until the maximum number of generations as specified 
in the RAVEN input is reached.

Furthermore, constraints for the optimization problem can be also specified by using \xmlNode{Functions} 
, see Section~\ref{sec:functions} for more details

\begin{center}
  \begin{tabular}{ |p{0.5cm}|p{0.5cm}|p{0.5cm}|p{0.5cm}|p{0.5cm}|p{0.5cm}| } 
    \hline
    1& & & & &       \\ 
    \hline
    2&3& & & &        \\ 
    \hline
    4&5&6& & &        \\ 
    \hline
    7&8&9&10& &       \\ 
    \hline
    11&12&13&14&15&   \\ 
    \hline
    16&17&18&19&20&21  \\ 
    \hline
    22&23&24&25&26&27 \\ 
    \hline
    28&29&30&31&32&    \\ 
    \hline
    33&34&35&  &  &   \\ 
    \hline
  \end{tabular}
\end{center}
  
Example:
  \begin{lstlisting}[style=XML]
  <Files>
    <Input name="simulatedata_input" type="simulatedata">sim3-param.xml</Input>
    <Input name="simulateperturb_input" type="perturb">sim3-perturb.xml</Input>
    <Input name="input" type="input">input.inp</Input>
  </Files>
  \end{lstlisting}
Example for \xmlNode{SIMULATE-3-input-gen.xml}:
  \begin{lstlisting}[style=XML]
  <Sim3-input-gen>
   <pins> 17 </pins>
   <core_width> 15 </core_width>
   <load_point> 0.000 </load_point>
   <depletion> 20 </depletion>
   <axial_nodes> 25 </axial_nodes>
   <batch> 0 </batch>
   <pressure> 2250.0 </pressure>
   <boron> 900.0 </boron>
   <power> 100.0 </power>
   <flow> 100.0 </flow>
   <inlet_temperature> 550.0 </inlet_temperature>
   <map_size> quarter </map_size>
   <symmetry> octant </symmetry>
   <restart_file> cycle1.res </restart_file>
   <cs_lib> cms.pwr-all.lib </cs_lib>
   <number_assemblies> 157 </number_assemblies>
   <working-dir> SampleSpecificSim3 </working-dir>
   <reflector> TRUE </reflector>
   <FA-list>
    <FA name='FA1'  FAid ='0'  type ='2'/>
    <FA name='FA2'  FAid ='1'  type ='3'/>
    <FA name='FA3'  FAid ='2'  type ='5'/>
    <FA name='FA4'  FAid ='3'  type ='4'/>
    <FA name='FA5'  FAid ='4'  type ='6'/>
    <FA name='REF'  FAid ='5'  type ='1'/>
   </FA-list>
  </Sim3-input-gen>
  \end{lstlisting}
%%%%%%%%%%%%%%%%%%%%%%%%%%%%%%%%%%%%%%%%%%%%%%%%%%%%%%%%%%%%%%%%%%%%%%%%%%%%%%%%%%
\subsubsection{Sampler/Optimizer}
%%%%%%%%%%%%%%%%%%%%%%%%%%%%%%%%%%%%%%%%%%%%%%%%%%%%%%%%%%%%%%%%%%%%%%%%%%%%%%%%%%
As mentioned in previous sections, the variables that users desire to optimize are defined in 
the \xmlNode{Samplers} or  \xmlNode{Optimizers} block. In this interface, the GA optimizer is applied.
In the loading pattern optimization problem, the variables are the location of Fuel Assembly in the core layout
with the index scheme shown in this section.

For example, the original locations are specified in the \xmlNode{sim3-perturb.xml.xml} (that needs to be perturbed).
The name of the variables in the \xmlNode{Samplers} should be the same as the one in original \xmlNode{sim3-perturb.xml.xml}
file, namely \textbf{loc} and the associated index \textbf{index}.

Example:
\begin{lstlisting}[style=XML]
...
<Samplers>
  ...
  <variable name="loc1">
    <distribution>FA_dist</distribution>
  ...
  </variable>
</Samplers>
...
\end{lstlisting}

