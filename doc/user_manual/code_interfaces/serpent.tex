%%%%%%%%%%%%%%%%%%%%%%%%%%%%
%%%%%% SERPENT  INTERFACE  %%%%%%
%%%%%%%%%%%%%%%%%%%%%%%%%%%%

\subsection{SERPENT Interface}
\label{subsec:serpentInterface}
The Serpent interface is meant to run multiple user-defined SERPENT simulations
while storing the values contained in the following output file types: 
\begin{itemize}
\item Results' file (``\_res.m'')
\item Detectors' files  (``\_det[bu].m'')
\item Depletion file (``\_dep.m'')
\item Burn-up matrices' files (``\_depmtx\_[mat]\_[bu]\_[ss].m'')
\end{itemize}

where:
\begin{itemize}
\item  \texttt{bu} is the burn up step index
\item  \texttt{mat} is the material name
\item  \texttt{ss} is the sub step id (if depletion with sub-steps is activated).
\end{itemize}

The interface allows users to run SERPENT by varying values from a 
template input file using the wildcard approach available through the
generic code interface. In section \ref{subsec:genericInterface}  information
on how to use the wildcard approach and all the available options of the interface
can be found.

\subsubsection{Models}
\label{subsubsec:serpentInterfaceModels}
In order to run the code, make sure you have a valid SERPENT input file.

In the RAVEN input file, your \xmlNode{Models} node should look like:
\begin{lstlisting}[style=XML]
<Models>
    <Code name="SERPENT" subType="Serpent">
      <!-- path to your serpent executable -->
      <executable>/my/path/to/serpent/sss2</executable>
      <clargs arg="" extension=".serpent" type="input"/>
      <clargs arg="--noplot" type="postpend"/>
      <additionalFileTypes>
        DepletionReader, DetectorReader, DepmtxReader
      <additionalFileTypes/>
      <!-- EOL parameters -->
      <EOL target="absKeff_0">1.0</EOL>
      <EOL target="impKeff_0">1.0</EOL>
      <!-- Volume calculation -->
      <volumeCalculation nPoints="1e9">True</volumeCalculation>
      <!--
        For burnup calculation, the user can provide a
        function to halt the simulation if a certain condition is
        met. The criterion or criteria is or are defined through
        an external RAVEN python function (e.g. ``criteria'' in the
        following example). The time frequency of the check is defined
        by the node onlineStopCriteriaTimeInterval. By default, the output
        is checked every 10 seconds.
       -->
      <onlineStopCriteriaTimeInterval>
        15.0
      </onlineStopCriteriaTimeInterval>
      <StoppingFunction class="Functions" type="External">
             criteria
      </StoppingFunction>
    </Code>
</Models>

<Functions>
    <External file="stoppingCriteria.py" name="criteria">
         <!-- 
           variables below must be retrievable
           from the one of the Serpent Outputs (see Output Files Conversion section)
         -->
        <variables>impKeff_0 </variables>
    </External>
</Functions>
\end{lstlisting}
where the \xmlNode{executable} and \xmlNode{clargs} should me modified
to create the appropriate run command for SERPENT. In this example, the command
to run raven is:
\begin{lstlisting}[language=bash]
/my/path/to/serpent/sss2 [inputFile] --noplot
\end{lstlisting}
where the \texttt{inputFile} is defined in the \xmlNode{Files} node as:
\begin{lstlisting}[style=XML]
<Files>
    <Input name="originalInput" type="">serpent_input.serpent</Input>
</Files>
\end{lstlisting}

Additional XML sub-nodes can be optionally added in the \xmlNode{Code} XML node is:
\begin{itemize}
   \item \xmlNode{additionalFileTypes}, \xmlDesc{comma separated list, optional parameter},  additional output file types 
                                                              that the interface needs to process.  
                                                              Available options are:  
                                                              \begin{itemize}
                                                                \item  \textit{ResultsReader} (``*\_res.m''), Results’ file.
                                                                \item  \textit{DepletionReader} (``*\_dep.m''), Depletion file.
                                                                 \item \textit{DetectorReader}  (``*\_det[bu].m''), Detectors’ files.
                                                                \item  \textit{DepmtxReader} (``*\_depmtx\_[mat]\_[bu]\_[ss].m''), Burn-up matrices’ files.
                                                              \end{itemize}
                                                              \nb By default  the ``Results'' (``\_res.m'') file (using the ``ResultReader'') is the 
                                                               only output file processed (``*\_res.m'') by the interface. 
                                                               For variable processing conventions, see section  
                                                               \ref{subsubsec:serpentInterfaceOutputConversion}.
   \item \xmlNode{EOL}, \xmlDesc{float, optional parameter},  value of the ``target'' at which the time (in burnup calculations) of end of life should be recorded.
                                                             The  \xmlNode{EOL} node  must contain the attribute \xmlAttr{target}, which indicates the FOM in the Results file that should be
                                                             monitored (e.g. ``absKeff''). The user can specify as many  \xmlNode{EOL} as needed.
                                                              
                                                              \nb  The output parser will generate an output variable called ``EOL\_\textit{target}'' that represents the time at which the 
                                                              parameter ``target'' takes the value indicated in this node. For example,  if ``target'' == absKeff and the value of the XML node is 1.0, 
                                                              the variable ``EOL\_absKeff'' will contain the time (burnDays) at which the ``absKeff'' == 1.0.
   \item \xmlNode{volumeCalculation}, \xmlDesc{bool, optional parameter},  true to activate the stochastic material volume calculation via Serpent ``checkvolumes'' command. If True, the execution of 
                                                             SERPENT is always preceded with a volume calculation. If the user wants to use the volume calculation output file in its SERPENT template input, he needs to include 
                                                             the filename in the SERPENT template input (e.g. include ``myInputFile.mvol'')  value of the ``target'' at which the time (in burnup calculations) of end of life should be recorded.
                                                             The  \xmlNode{volumeCalculation} node  must contain the attribute \xmlAttr{nPoints}, which indicates the number of samples that needs to be used by SERPENT for the calculation of the volumes (e.g. 1e8).
   \item \xmlNode{onlineStopCriteriaTimeInterval}, \xmlDesc{float, optional parameter}, time frequency (in seconds)) at which  the stopping function (if inputted)
                                                            is inquired (stoppingCriteriaFunction).
                                                            For burnup calculation, a good time frequency (time interval) is of the order of a the CPU time
                                                            required for an eginevalue calc. \default{10}.
   \item \xmlNode{StoppingFunction}, \xmlDesc{XML node, optional node},  \xmlDesc{string, optional field}, the
                                                           body of this XML node must contain the name of an appropriate \textbf{Function} defined in the
                                                            \xmlNode{Functions} block (see Section~\ref{sec:functions}).  It is used as a
                                                           ``stopping'' function tool for online simulation monitoring. Indeed, for burnup calculation, the user can provide a
                                                            function to halt the simulation if a certain condition is
                                                            met. The criterion or criteria is or are defined through the external RAVEN python function linked here through this XML node. 
                                                            The function must return a ``bool'':
                                                            \begin{itemize}
                                                              \item False, if the simulation can continue (i.e. the criteria are not met)
                                                              \item True, if the simulation must STOP (i.e. the criteria are met)
                                                            \end{itemize}
                                                            %    
                                                            Each variable defined in the \xmlNode{StoppingFunction} block is available in the
                                                            function as a python \texttt{raven.} member. In the following, an example of a
                                                            user-defined stopping condition function is reported.
                                                            
\begin{lstlisting}[language=python]
def criteria(raven):
  if raven.impKeff_0[-1] < 1.0:
    return True
  else:
    return False
\end{lstlisting}

\end{itemize}

\subsubsection{Files}
The only input file needed is a complete SERPENT input file,
which means that it should either be self sufficient, or includes
the necessary files (e.g. geometry, material definition files).
The main \textit{SERPENT} input file must be tagged with the
type ``serpent'' (as shown below). The auxiliary files, included in the
main input file through the SERPENT include nomenclature (i.e., `` include "myAuxFile" ''),
do not need to be tagged with any specific type:

\begin{lstlisting}[style=XML]
  <Files>
    <Input name="sMainInput" type="serpent">
      mySerpentInputFile.serpent
     </Input>
    <Input name="detectors"  type="">
     detectors
     </Input>
    <Input name="geometry"  type="">
      geometry
    </Input>
  </Files>
\end{lstlisting}

\nb Since the Generic Code interface (see section  \ref{subsec:genericInterface})
      is used to perturb the SERPENT input file, the wild-carded variables to perturb
      can be present in any of the files needed by SERPENT (main input file and its ``include'' files).

\subsubsection{Samplers / Optimizers}
In the \xmlNode{Samplers} block the user can define the variables
to be sampled.
\textbf{Example:} If the user wants to vary depletion time from
10 to 100 days, the SERPENT input file \texttt{dep} definition is as such:
\begin{lstlisting}
dep daystep 
$RAVEN-deptime$ 
\end{lstlisting}

Then in the RAVEN input file, the \xmlNode{Samplers} block would be:
\begin{lstlisting}[style=XML]
  <Samplers>
    <Grid name="myGrid">
      <variable name="deptime">
        <distribution>timedist</distribution>
        <!-- equally spaced steps with lower and upper bound -->
        <grid construction="equal" steps="100" type="CDF">0.1 1</grid>
      </variable>
    </Grid>
  </Samplers>
\end{lstlisting}
where the distribution \texttt{timedist} is defined in the \xmlNode{Distributions}:
\begin{lstlisting}[style=XML]
  <Distributions>
    <Uniform name="timedist">
      <lowerBound>0.0</lowerBound>
      <upperBound>100</upperBound>
    </Uniform>
</Distributions>
\end{lstlisting}

Then this run would be defined in the \xmlNode{Steps} block as \xmlNode{MultiRun}:
\begin{lstlisting}[style=XML]
<MultiRun name="runGrid">
      <!-- runGrid runs serpent by 
           the number of steps with
           sampled variable 
       -->
      <Input class="Files" type="serpent">
        sMainInput
      </Input>
      <Input  class="Files" type="">
        detectors
      </Input>
      <Input class="Files" type="">
        geometry
      </Input>
      <Model class="Models" type="Code">
        SERPENT
      </Model>
      <Sampler class="Samplers" type="Grid" >
        myGrid
      </Sampler>
      <Output class="DataObjects" type="PointSet">
        outPointSet
      </Output>
</MultiRun>
\end{lstlisting}

\subsubsection{Output Files Conversion}
\label{subsubsec:serpentInterfaceOutputConversion}
SERPENT can generate several output files. The description of 
such output files can be found in SERPENT documentation page  
(available at \url{https://serpent.vtt.fi/mediawiki/index.php/Main_Page}).

As mentioned in previous sections, the interface processes, by default, the
main Results file \texttt{[input]\_res.m} (both in time-independent  and burnup calculations).
Additional output files can be requested, as described in Section 
\ref{subsubsec:serpentInterfaceModels}. Such output types are:
\begin{itemize}
   \item  \textit{ResultsReader} (``*\_res.m''), Results’ file.
   \item  \textit{DepletionReader} (``*\_dep.m''), Depletion file.
   \item \textit{DetectorReader}  (``*\_det[bu].m''), Detectors’ files.
   \item  \textit{DepmtxReader} (``*\_depmtx\_[mat]\_[bu]\_[ss].m''), Burn-up matrices’ files.
\end{itemize}

In this section, the naming conventions for the variables in each of the available output types 
are reported.\\

\textit{\textbf{\underline{ResultsReader}}}
\\The \textit{ResultsReader} is the default output parsing option. 
The reader is responsible to read the result files (``*\_res.m'').
The parser reads all the results (time-independent  or burnup dependent) that can
be found in the file.
The variable names are converted from using underscores (``\_'') to \textit{\textbf{camelBack}}. For example,
the variables \textit{\textbf{ANA\_KEFF}} and  \textit{\textbf{IMP\_KEFF}}  will be renamed
 as \textit{\textbf{anaKeff}} and \textit{\textbf{impKeff}}, respectively. 
 \\In addition, the variables characterized by multiple elements will be expanded in unique variables, 
 appending to the variable name the element index (e.g. 0, 1, 2, etc.). 
 \begin{lstlisting}[language={matlab}]
% _res.m file
ANA_KEFF  (idx, [1:   6]) = [1.0E+00 0.001 9.9E-01 0.001 1.0E-02 0.001];
IMP_KEFF  (idx, [1:   2]) = [1.0E+00 0.001];
\end{lstlisting}
For example, for the above snippet of the result file, the variables generated would be:

\begin{table}[ht]
\begin{tabular}{|c|c|c|c|c|c|c|c|}
\hline
\textit{anaKeff\_0} & \textit{anaKeff\_1} & \textit{anaKeff\_2} & \textit{anaKeff\_3} & \textit{anaKeff\_4} & \textit{anaKeff\_5} & \textit{impKeff\_0} & \textit{impKeff\_1} \\ \hline
1.E+00              & 0.001               & 9.9E-01             & 0.001               & 1.E-02              & 0.001               & 1.E+00              & 0.001               \\ \hline
\end{tabular}
\end{table}

For burnup calculations the same syntax is used with the different that multiple rows are generated each of them indexed by the burnup step.

\textit{\textbf{\underline{DepletionReader}}}
\\The \textit{DepletionReader} is aimed to parse and process depletion files.
The reader is responsible to read the depletion file (``*\_dep.m'').
The parser reads all the depletion results that can
be found in the file.
The variable names are converted from using underscores (``\_'') to \textit{\textbf{camelBack}}. For example,
the variable \textit{\textbf{ING\_TOX}} will be renamed
 as \textit{\textbf{ingTox}}.
 The parser parses the materials, isotopes and quantities for creating unique variables, based on the following nomenclature:
  \begin{lstlisting}[language={matlab}]
   [materialName]_[isotope]_[quantity]
\end{lstlisting}
For example, The variable referring to the  \textit{activity}  of the  \textit{U235}  isotope in material  \textit{UO2} will be called:
\textit{UO2\_U235\_activity}.
For global variables (not linked to specific isotopes) the same nomenclature reported above is used removing the ``isotope'' section.
For example, the burnup of the material \textit{UO2} will be named \textit{UO2\_burnup}.

\textit{\textbf{\underline{DetectorReader}}}
\\The \textit{DetectorReader} is aimed to parse and process the detector output files.
The reader is responsible to read the detector files (``*\_det[bu].m'') for both time-independent and burnup calculations.
The parser reads all the detector results that can
be found in the files.
For detector returning scalar quantities (E.g. total flux on a surface), the variable nomenclature is as follows:
  \begin{lstlisting}[language={matlab}]
   for the actual scalar result:
   [dName]
   for the uncertainty/error associated with the detector score:
   [dName]_err
\end{lstlisting}
For detector returning a field quantity (e.g. energy and spatial dependent flux), the variable nomenclature is as follows:
  \begin{lstlisting}[language={matlab}]
   for the actual detector score entry
   [dName]_[gridIndexName1]_[gridIndexValue1] (...)_[gridIndexNameN]_[gridIndexValueN]  ]
   for the uncertainty/error associated with the detector score entry:
   [dName]_[gridIndexName1]_[gridIndexValue1] (...)_[gridIndexNameN]_[gridIndexValueN]_err
\end{lstlisting}


\textit{\textbf{\underline{DepmtxReader}}}
\\The \textit{DepmtxReader} is aimed to parse and process Burn-up matrices’ files.
Since the matrices can be quite large, the parser reads the matrices and store them in a serialized file.
The filename is returned to RAVEN following the nomenclature:
  \begin{lstlisting}[language={matlab}]
    filename_depmtx_zai_[materialName]_[bu]_[ss]
\end{lstlisting}
 In addition, the 1 group flux used for the depletion is returned as follows:
  \begin{lstlisting}[language={matlab}]
     flx_[materialName]_[bu]_[ss]
\end{lstlisting} 

\subsubsection{SERPENT/RAVEN variable generation through script}
Since the number of output variables generated by SERPENT and processable by SERPENT/RAVEN interface is quite large and,
consequentially, manually listing them would be cumbersome, 
an utility script has been provided to create  \xmlNode{VariableGroups} (\xmlNode{Group}s XML nodes) containing all the
variables (with the correct nomenclature as seen in section \ref{subsubsec:serpentInterfaceOutputConversion} extractable by
specific output files.
\\In order to link the variable generation to the specific SERPENT case of interest, the script inspects pre-generated SERPENT output files
provided by the user.
\\The script is located at \path{RAVEN_HOME/ravenframework/CodeInterfaceClasses/SERPENT/utilityForCreatingVariables/generateRavenVariables.py} and
can be invoked like indicated (for example) below:
\begin{lstlisting}[language=bash]
python generateRavenVariables.py -fileTypes ResultsReader,DepletionReader,DetectorReader,DepmtxReader   -fileRoot mySerpentInputFile -o myOptionalScriptOutputFilename
\end{lstlisting}
where:
\begin{itemize}
\item  \texttt{-fileTypes}, Comma-separated list of output file types from which variable names need to be extracted. 
                                     Currently, the available types  are the ones indicated in \ref{subsubsec:serpentInterfaceOutputConversion} 
                                     (e.g. ResultsReader,DepletionReader,DetectorReader,DepmtxReader)
\item  \texttt{-fileRoot}, File name root from which all the output files are going to be inferred. 
                                    For example, if the ``fileRoot'' is ``myInputFile'', the ``ResultsReader'' file that the script will parse
                                    must be named ``myInputFile\_res.m''.
\item  \texttt{-o}, Optional output file name. If not provided, the script will store the  \xmlNode{VariableGroups} in file named \path{ravenOutputVariables.xml}.
\end{itemize}


