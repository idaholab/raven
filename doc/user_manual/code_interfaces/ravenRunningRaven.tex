%%%%%%%%%%%%%%%%%%%%%%%%%%%%%%%%%%%%%%%%%%%%%%%%
%%%%%% RAVEN  INTERFACE  (RAVEN running RAVEN) %%%%%%
%%%%%%%%%%%%%%%%%%%%%%%%%%%%%%%%%%%%%%%%%%%%%%%%
\subsection{RAVEN Interface}
\label{subsec:RAVENInterface}
The RAVEN interface is meant to provide the possibility to execute a RAVEN input file
driving a set of SLAVE RAVEN calculations. For example, if the user wants to optimize the parameters
of a surrogate model (e.g. minimizing the distance between the surrogate predictions and the real data), he
can achieve this task by setting up  a RAVEN input file (master) that performs an optimization on the feature
space characterized by the surrogate model parameters, whose training and validation assessment  is performed in the SLAVE
RAVEN runs.
\\ There are some limitations for this interface:
\begin{itemize}
\item only one  sub-level of RAVEN can be executed (i.e. if the SLAVE RAVEN input file contains the run of another RAVEN SLAVE, the MASTER RAVEN will error out)
\item only data from Outstreams of type Print can be collected by the MASTER RAVEN
\item only a maximum of two Outstreams can be collected (1 PointSet and 1 HistorySet)
\end{itemize}


Like for every other interface, most of the RAVEN workflow stays the same independently of which type of Model (i.e. Code) is used.
\\ Similarly to any other code interface, the user provides paths to executables and aliases for sampled variables within the
\xmlNode{Models} block.  The \xmlNode{Code} block will contain attributes \xmlAttr{name} and
\xmlAttr{subType}.  \xmlAttr{name} identifies that particular \xmlNode{Code} model within RAVEN, and
\xmlAttr{subType} specifies which code interface the model will use (In this case \xmlAttr{subType}=``RAVEN'').
The \xmlNode{executable}
block should contain the absolute or relative (with respect to the current working
directory) path to the RAVEN framework script (\textbf{raven\_framework}).
\\ In addition to the attributes and xml nodes reported above, the RAVEN accepts the following XML nodes (required and optional):
\begin{itemize} % nodes for code
  \item  \xmlNode{outputDatabase}, \xmlDesc{string, required parameter}
    will specify the \xmlNode{Database} that will be loaded as outputs of the INNER RAVEN. If this node
    is not specifed, \xmlNode{outputExportOutStreams} may be used instead.
  \item  \xmlNode{outputExportOutStreams}, \xmlDesc{comma separated list,
    required parameter} will specify the  \xmlNode{OutStreams} that will be loaded as outputs of the SLAVE RAVEN.
    Maximum two  \xmlNode{OutStreams} can be listed here (1 for PointSet and/or 1 for HistorySet).
  \item  \xmlNode{conversion}, \xmlDesc{Node,optional parameter} will specify details of conversion scripts to
    be used in creating the inner RAVEN input file.  This node contains the following nodes:
    \begin{itemize} % nodes for conversion
      \item \xmlNode{module}, \xmlDesc{Node, optional parameter} a module for directly manipulating the xml
        structure of perturbed input files. This can be used to modify the template input file in arbitrary
        ways; however, it should be used with caution, and is considered an advanced method.
        This node has the following attribute:
        \begin{itemize} % attributes for module
          \item \xmlAttr{source}, \xmlDesc{string, required} provides the path to the manipulation module including
              the module file itself.  The following method should be defined in order to perform the input
              manipulation:
              \begin{itemize} % functions for module
                 \item \textbf{\textit{modifyInput}}, manipulates the input in arbitrary ways. This method
                   takes two arguments. The first is the root \xmlNode{Simulation} node of the template input
                   file that has already been modified with the perturbed samples (the object is a Python
                   \texttt{xml.etree.ElementTree.Element} object). The second input is a dictionary with all
                   the modification information used to previously modify the template xml. The method should
                   return the modified root.
                 Example:
                  \begin{lstlisting}[language=python]
import xml.etree.ElementTree as ET
def modifyInput(root, modDict):
  """
    Manipulate the inner RAVEN xml input.
    @ In, root, ET.Element, perturbed RAVEN input
    @ In, modDict, dictionary, modifications made to the input
    @ Out, root, ET.Element, modified RAVEN input
  """
  # adds the file <Input name='aux_inp'>auxfile.txt</Input> to the <Files> node
  filesNode = root.find('Files')
  newNode = ET.Element('Input')
  newNode.text = 'auxfile.txt'
  newNode.attrib['name'] = 'aux_inp'
  filesNode.append(newNode)
  return root
                   \end{lstlisting}
              \end{itemize} % end functions for module
        \end{itemize} % end attributes for module

      \item \xmlNode{module}, \xmlDesc{Node, optional parameter} contains the information about a specific
        conversion module (python file).  This node can be repeated multiple times.
        This node has the following attribute:
        \begin{itemize} % attributes for module
          \item \xmlAttr{source}, \xmlDesc{string, required} provides the path to the conversion module including
              the module file itself.  There are two methods that can be placed in the conversion module:
              \begin{itemize} % functions for module
                 \item \textbf{\textit{manipulateScalarSampledVariables}}, a method that is aimed to manipulate sampled variables and to create more in case needed.
                 Example:
                  \begin{lstlisting}[language=python]
def manipulateScalarSampledVariables(sampledVariables):
  """
  This method is aimed to manipulate scalar variables.
  The user can create new variables based on the
  variables sampled by RAVEN
   @ In, sampledVariables, dict, dictionary of
       sampled variables ({"var1":value1,"var2":value2})
   @ Out, None, the new variables should be
                 added in the "sampledVariables" dictionary
  """
  newVariableValue =
    sampledVariables['Distributions|Uniform@name:a_dist|lowerBound']
    + 1.0
  sampledVariables['Distributions|Uniform@name:a_dist|upperBound'] =
    newVariableValue
  return
                   \end{lstlisting}

                 \item \textbf{\textit{convertNotScalarSampledVariables}}, a method that is aimed to convert not scalar variables (e.g. 1D arrays) into multiple scalar variables
                 (e.g.  \xmlNode{constant}(s) in a sampling strategy).
                  This method is going to be required in case not scalar variables are detected by the interface.
                  Example:
                  \begin{lstlisting}[language=python]
 def convertNotScalarSampledVariables(noScalarVariables):
   """
   This method is aimed to convert not scalar
   variables into multiple scalar variables. The user MUST
    create new variables based on the not Scalar Variables
     sampled (and passed in) by RAVEN
   @ In, noScalarVariables, dict, dictionary of sampled
        variables that are not scalar ({"var1":1Darray1,"var2":1Darray2})
   @ Out, newVars, dict,  the new variables that have
        been created based on the not scalar variables
        contained in "noScalarVariables" dictionary
   """
   oneDimensionalArray =
       noScalarVariables['temperatureHistory']
   newVars = {}
   for cnt, value in enumerate(oneDimensionalArray):
     newVars['Samplers|MonteCarlo@name:myMC|constant'+
                '@name=temperatureHistory'+str(cnt)] =
                oneDimensionalArray[cnt]
   return newVars
                  \end{lstlisting}
              \end{itemize} % end functions for module
        \end{itemize} % end attributes for module
        The \xmlNode{module} node also takes the following node:
        \begin{itemize} % nodes of module
          \item \xmlNode{variables}, \xmlDesc{comma-separated list, required} provides a comma-separated list of
            the variables from the MASTER RAVEN that need to be accessed by the conversion script module.  The
            variables listed here use the pipe naming system (un-aliased names).
        \end{itemize} % end nodes of module
    \end{itemize} % end nodes for conversion
\end{itemize} % end nodes for code

Code input example:
\begin{lstlisting}[style=XML]
<Code name="RAVENrunningRAVEN" subType="RAVEN">
  <executable>../../../raven_framework</executable>
  <outputExportOutStreams>
     HistorySetOutStream,PointSetOutStream
  </outputExportOutStreams>
  <conversion>
    <module source=/Users/username/whateverConversionModule.py>
      <variables>a,b,x,y</variables>
    </module>
  </conversion>
</Code>
\end{lstlisting}

Like for every other interface,  the syntax of the variable names is important to make the parser understand how to perturb an input file.
\\ For the RAVEN interface, a syntax inspired by the XPath nomenclature is used.
\begin{lstlisting}[style=XML]
<Samplers>
  <MonteCarlo name="MC_external">
     ...
    <variable name="Models|ROM@subType:SciKitLearn@name:ROM1|C">
      <distribution>C_distrib</distribution>
    </variable>
    <variable name="Models|ROM@subType:SciKitLearn@name:ROM1|tol">
      <distribution>toll_distrib</distribution>
    </variable>
    <variable name="Samplers|Grid@name:'+
          'GridName|variable@name:var1|grid@construction:equal@type:value@steps">
      <distribution>categorical_step_distrib</distribution>
    </variable>
    ...
  </MonteCarlo>
</Samplers>
\end{lstlisting}
In the above example, it can be inferred that each XML node (subnode) needs to be separated by a ``|'' separator. In addition,
every time an XML node has attributes, the user can specify them using the ``@'' separator to specify a value for them.
The first variable above will be pointing to the following XML sub-node ( \xmlNode{C}):
\begin{lstlisting}[style=XML]
<Models>
  <ROM name="ROM1" subType="SciKitLearn">
     ...
     <C>10.0</C>
    ...
 </ROM>
</Models>
\end{lstlisting}
The second variable above will be pointing to the following XML sub-node ( \xmlNode{tol}):
\begin{lstlisting}[style=XML]
<Models>
  <ROM name="ROM1" subType="SciKitLearn">
     ...
     <tol>0.0001</tol>
    ...
 </ROM>
</Models>
\end{lstlisting}
The third variable above will be pointing to the following XML attribute ( \xmlAttr{steps}):
\begin{lstlisting}[style=XML]
<Samplers>
  <Grid name="GridName">
     ...
    <variable name="var1">
       ...
      <grid construction="equal" type="value" steps="1">0 1</grid>
       ...
    </variable>

    ...
  </MonteCarlo>
</Samplers>
\end{lstlisting}

The above nomenclature must be used for all the variables to be sampled and for the variables generated by the two methods contained, in case, in
the module that gets specified by the \xmlNode{conversionModule} in the \xmlNode{Code} section.
\\ Finally the SLAVE RAVEN input file (s) must be ``tagged'' with the attribute  \xmlAttr{type="raven"} in the Files section. For example,

\begin{lstlisting}[style=XML]
<Files>
    <Input name="slaveRavenInputFile" type="raven" >
      test_rom_trainer.xml
    </Input>
</Files>
\end{lstlisting}

\subsubsection{ExternalXML and RAVEN interface}
Care must be taken if the SLAVE RAVEN uses \xmlNode{ExternalXML} nodes.  In this case, each file containing
external XML nodes must be added in the \xmlNode{Step} as an \xmlNode{Input} class \xmlAttr{Files} to make sure it gets copied to
the individual run directory.  The type for these files can be anything, with the exception of type
\xmlString{raven}.

