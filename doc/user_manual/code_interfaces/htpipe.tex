%%%%%%%%%%%%%%%%%%%%%
%%%       HTPIPE INTERFACE       %%%
%%%%%%%%%%%%%%%%%%%%%
\subsection{HTPIPE Interface}
This section presents the main aspects of the interface between RAVEN and HTPIPE~\cite{htpipeCodeInterface},
which is a hydrodynamic model designed to evaluate heat pipe performance. Originally developed for high-performance
 liquid metal heat pipes, it can be adapted for other working fluids as long as their thermal transport properties are available.
 The program accommodates multiple liquid return geometries, such as homogeneous wicks, screen-covered grooves, 
 circular arteries, and gutters. By using HTPIPE, one can optimize heat pipe designs, determine performance limits, and 
 analyze axial temperature and pressure profiles.

In this section, the HTPIPE-specific RAVEN input adjustments and the modifications of the HTPIPE
input file are reported. \noindent

\noindent In the following sections a short explanation on how to use RAVEN coupled with HTPIPE is reported.
%%%%%%%%%%%%%%%%%%%%%%%%%%%%%%%%%%%%%%%%%%%%%%%%%%%
\subsubsection{Sequence}
%%%%%%%%%%%%%%%%%%%%%%%%%%%%%%%%%%%%%%%%%%%%%%%%%%%
In the \xmlNode{Sequence} section, the names of the steps declared in the
\xmlNode{Steps} block should be specified.
%
As an example, if we called the first MultiRun ``MC\_Sample'' and the second
MultiRun ``plotResults'' in the sequence section we should see this:

\begin{lstlisting}[style=XML]
<Sequence>MC_Sampler,plotResults</Sequence>
\end{lstlisting}

%%%%%%%%%%%%%%%%%%%%%%%%%%%%%%%%%%%%%%%%%%%%%%%%%%%
\subsubsection{batchSize and mode}
%%%%%%%%%%%%%%%%%%%%%%%%%%%%%%%%%%%%%%%%%%%%%%%%%%%
For the \xmlNode{batchSize} and \xmlNode{mode} sections please refer to the
\xmlNode{RunInfo} block in the previous chapters.

%%%%%%%%%%%%%%%%%%%%%%%%%%%%%%%%%%%%%%%%%%%%%%%%%%%%
\subsubsection{RunInfo}
%%%%%%%%%%%%%%%%%%%%%%%%%%%%%%%%%%%%%%%%%%%%%%%%%%%%
After all of these blocks are filled out, a standard example RunInfo block may
look like the example below: \\

\begin{lstlisting}[style=XML]
<RunInfo>
  <WorkingDir>~/workingDir</WorkingDir>
  <Sequence>MC_Sampler,plotResults</Sequence>
  <batchSize>8</batchSize>
</RunInfo>
\end{lstlisting}
In this example, the \xmlNode{batchSize} is set to $8$; this means that 8 simultaneous (parallel) instances
of HTPIPE are going to be executed when a sampling strategy is employed.

%%%%%%%%%%%%%%%%%%%%%%%%%%%%%%%%%%%%%%%%%%%%%%%%%%%%%%%%%
\subsubsection{Models}
%%%%%%%%%%%%%%%%%%%%%%%%%%%%%%%%%%%%%%%%%%%%%%%%%%%%%%%%%
As any other Code, in order to activate the HTPIPE interface, a \xmlNode{Code} XML node needs to be inputted, within the
main XML node \xmlNode{Models}.
\\The  \xmlNode{Code} XML node contains the
information needed to execute the specific External Code.

\attrsIntro
%

\begin{itemize}
  \itemsep0em
  \item \nameDescription
  \item \xmlAttr{subType}, \xmlDesc{required string attribute}, specifies the
  code that needs to be associated to this Model. To use HTPIPE, the \xmlAttr{subType} needs to be set to ``HTPIPE'' 
  %
  \nb See Section~\ref{sec:existingInterface} for a list of currently supported
  codes.
  %
\end{itemize}


\subnodesIntro
%
\begin{itemize}
  \item \xmlNode{executable} \xmlDesc{string, required field} specifies the path
  of the executable to be used.
  %
  \nb Either an absolute or relative path can be used.
  \item \aliasSystemDescription{Code}
  %
\end{itemize}

An example  is shown  below:

\begin{lstlisting}[style=XML]
<Models>
    <Code name="MyHTPIPEcode" subType="HTPIPE">
      <executable>path/to/HTPIPE</executable>
    </Code>
</Models>
\end{lstlisting}

%%%%%%%%%%%%%%%%%%%%%%%%%%%%%%%%%%%%%%%%%%%%%%%%%%%%%%%%%%%%%%%%%%%%%%%%%%%%%%%%%%
\subsubsection{Files}
%%%%%%%%%%%%%%%%%%%%%%%%%%%%%%%%%%%%%%%%%%%%%%%%%%%%%%%%%%%%%%%%%%%%%%%%%%%%%%%%%%
The \xmlNode{Files} XML node has to contain all the files required to run the external code  (HTPIPE).
For RAVEN coupled with HTPIPE, only one file is required (the HTPIPE input file). 
\nb Since HTPIPE can create the input interactively, it is suggested to create a template input of the study that needs to be performed and then use the generated input in RAVEN.

The  \xmlNode{Files} XML node contains the information needed to execute HTPIPE.

\attrsIntro
%

\begin{itemize}
  \itemsep0em
  \item \nameDescription
  \item \xmlAttr{type}, \xmlDesc{required string attribute}, specifies the
  input type used by HTPIPE. The only accepted type (and required to be inputted) is: 
  \begin{itemize}
    \item htpipe, \xmlDesc{required string attribute}, identifies the HTPIPE input file and the code currently accept any name for input. This \xmlAttr{type} (htpipe) is REQUIRED.
  \end{itemize}
\end{itemize}

Example:

\begin{lstlisting}[style=XML]
<Files>
  <Input name="waterHeatPipeInput" type="htpipe">water</Input>
</Files>
\end{lstlisting}
The file mentioned in this section
need, then, to be placed into the working directory specified
by the \xmlNode{workingDir} node in the \xmlNode{RunInfo} XML node.

\paragraph{Output Files Conversion}
The HTPIPE interface is designed to read the output results from the ``plotfl'' generated by HTPIPE, which contains a formatted
table, whose data and variables depend on the calculation type, which is defined in the input file (first raw of the input file):
\begin{itemize}
  \item \textit{1}: Calculation of the operational limits (e.g. boiling limits, capillarity limits, etc.)
  \item \textit{2}: Calculation of pressure and temperature distribution along the heat pipe 
\end{itemize}

For calculation type  \textit{(1)} (Operational Limits), the following output values will be collected:
\begin{itemize}
  \item \textit{eetemp}: Operational Temperature (K), which is the independent variable for this calculation type (i.e. the \xmlNode{pivotParameter} in an \xmlNode{HistorySet})
  \item \textit{heat}: Capillarity Limit (W) 
  \item \textit{sonlim}: Sonic Limit (W)
  \item \textit{entlim}: Entrainment Limit (W)
  \item \textit{boillim}: Boiling Limit (W)
\end{itemize}

For this calculation type (1), the ``plotfl'' has the following structure:
  \begin{lstlisting}[basicstyle=\tiny]
      eetemp        heat      sonlim      entlim     boillim
     300.000     163.533    1179.676     163.533       0.000
     301.000     168.030    1250.606     168.030       0.000
     302.000     172.612    1325.214     172.612       0.000
     303.000     177.278    1403.654     177.278       0.000
\end{lstlisting}

For calculation type  \textit{(2)} (Pressure and Temperature distribution), the following output values will be collected:
\begin{itemize}
  \item \textit{dist}: Axial coordinate on the heat pipe (cm), which is the independent variable for this calculation type (i.e. the \xmlNode{pivotParameter} in an \xmlNode{HistorySet})
  \item \textit{pvap}: Vapor Pressure (Pa) 
  \item \textit{pliq}: Liquid Pressure (Pa)
  \item \textit{tempx}: Maximum Temperature (K)
\end{itemize}

For this calculation type (2), the ``plotfl'' has the following structure:
  \begin{lstlisting}[basicstyle=\tiny]
        dist        pvap        pliq       tempx
       0.000  187856.527  187856.527     332.002
       2.000  187855.229  187855.229     332.002
       4.000  187852.635  187852.635     332.002
       6.000  187848.743  187848.743     332.001
\end{lstlisting}

\textbf{\textit{\nb RAVEN, recognizes failed or crashed HTPIPE runs and no data will be saved from those.}}

%%%%%%%%%%%%%%%%%%%%%%%%%%%%%%%%%%%%%%%%%%%%%%%%%%%%%%%%%
\subsubsection{Distributions}
%%%%%%%%%%%%%%%%%%%%%%%%%%%%%%%%%%%%%%%%%%%%%%%%%%%%%%%%%
The \xmlNode{Distribution} block defines the distributions that are going
to be used for the sampling of the variables defined in the \xmlNode{Samplers} block.
%
For all the possible distributions and all their possible inputs please see the
chapter about Distributions (see~\ref{sec:distributions}).
%
Here we report an example of a Normal and Uniform distributions (for sampling the evaporator and condenser length, respectively):

\begin{lstlisting}[style=XML,morekeywords={name,debug}]
<Distributions verbosity='debug'>
    <Normal name="evapLength">
      <mean>10</mean>
      <sigma>2.0</sigma>
      <upperBound>20.0</upperBound>
      <lowerBound>5.0</lowerBound>
    </Normal>
    <Uniform name="condLength">
      <upperBound>30.0</upperBound>
      <lowerBound>20.0</lowerBound>
    </Uniform>
 </Distributions>
\end{lstlisting}

\noindent
It is good practice to name the distribution something similar to what kind of
variable is going to be sampled, since there might be many variables with the
same kind of distributions but different input parameters.

%%%%%%%%%%%%%%%%%%%%%%%%%%%%%%%%%%%%%%%%%%%%%%%%%%%%%%%%%
\subsubsection{Samplers}
%%%%%%%%%%%%%%%%%%%%%%%%%%%%%%%%%%%%%%%%%%%%%%%%%%%%%%%%%
In the \xmlNode{Samplers} block we want to define the variables that are going to be sampled.

\noindent The perturbation or optimization of the input of any HTPIPE sequence is performed using the approach detailed in the \textit{Generic Interface} section (see \ref{subsec:genericInterface}). 
Briefly, this approach uses
 ``wild-cards'' (placed in the original input files) for injecting the perturbed values.
 For example, if the original input file (that needs to be perturbed) is the following:

\textbf{Example}:

We want to do the sampling of 2 variable:
\begin{itemize}
  \item Condenser section length;
  \item Evaporator section length;
\end{itemize}

\noindent We are going to sample this variable using a MonteCarlo method.
The RAVEN input is then written as follows:

\begin{lstlisting}[style=XML,morekeywords={name,type,construction,lowerBound,steps,limit,initialSeed}]
<Samplers verbosity='debug'>
  <MonteCarlo name='MC_Sampler'>
     <samplerInit> <limit>10</limit> </samplerInit>
    <variable name='evapLength'>
      <distribution>evapLength</distribution>
    </variable>
    <variable name='condLength'>
      <distribution>condLength</distribution>
    </variable>
  </MonteCarlo>
</Samplers>
\end{lstlisting}

HTPIPE input file should be modified with wild-cards in the following way.
\begin{lstlisting}[basicstyle=\tiny]
         2               calculate q vs tee
         3               geometry: wickless pipe    
         5               working fluid: water     
         5               input choice: does not apply      
         5               input choice: does not apply      
         3               source coupling:does not apply      
         3               sink coupling:does not apply      
         0.00000         default value should be zero                                     
         0.00000         default value should be zero                                     
         0.00000         default value should be zero                                     
         0.00000         default value should be zero                                     
         0.00000         default value should be zero                                     
         $RAVEN-evapLength|8.5f$         evaporator length (cm)
         5.00000         adiabatic length (cm)
         $RAVEN-condLength|8.5f$         condenser length (cm)                                            
 (...) 
\end{lstlisting}

As shown above, for HTPIPE is \textbf{KEY} to specify the format of the variable (eg. ``8.5f'' for float, or ``d5'' for integer or any other formatting available).\\

\noindent It can be seen that each variable is connected with a proper distribution
defined in the \\\xmlNode{Distributions} block (from the previous example).

%%%%%%%%%%%%%%%%%%%%%%%%%%%%%%%%%%%%%%%%%%%%%%%%%%%%%%%%%%%
\subsubsection{Steps}
For a HTPIPE interface, the \xmlNode{MultiRun} step type will most likely be
used. But \xmlNode{SingleRun} step can also be used for plotting and data extraction purposes.
%
First, the step needs to be named: this name will be one of the names used in
the \xmlNode{sequence} block.
%
In our example,  \texttt{MC\_Sampler}.
%
\begin{lstlisting}[style=XML,morekeywords={name,debug,re-seeding}]
     <MultiRun name='MC_Sampler' verbosity='debug'>
\end{lstlisting}

With this step, we need to import all the files needed for the simulation (only one for HTPIPE):
\begin{itemize}
  \item ``htpipe'' input file
\end{itemize}
\begin{lstlisting}[style=XML,morekeywords={name,class,type}]
    <Input class="Files" type="htpipe">waterHeatPipeInput</Input>
\end{lstlisting}
We then need to define which model will be used:
\begin{lstlisting}[style=XML]
    <Model  class='Models' type='Code'>MyHTPIPEcode</Model>
\end{lstlisting}
We then need to specify which Sampler is used, and this can be done as follows:
\begin{lstlisting}[style=XML]
    <Sampler class='Samplers' type='MonteCarlo'>MC_Sampler</Sampler>
\end{lstlisting}
And lastly, we need to specify what kind of output the user wants.
%
For example the user might want to make a database (in RAVEN the database
created is an HDF5 file).
%
Here is a classical example:
\begin{lstlisting}[style=XML,morekeywords={class,type}]
    <Output  class='Databases' type='HDF5'>MC_out</Output>
\end{lstlisting}

Following is the example of two MultiRun step, which uses the MonteCarlo sampling
method, and create a database storing the results:
\begin{lstlisting}[style=XML]
<Steps verbosity='debug'>
  <MultiRun name='MC_Sampler' verbosity='debug' re-seeding='210491'>
    <Input   class='Files' type='htpipe'>
      waterHeatPipeInput
     </Input>
    <Model   class='Models' type='Code'>
      MyHTPIPEcode
    </Model>
    <Sampler class='Samplers'  type='MonteCarlo'>
      MC_Sampler
    </Sampler>
    <Output  class='Databases' type='HDF5' >
      MC_out
    </Output>
    <Output  class='DataObjects' type='PointSet'>
      MonteCarloHTPIPEPointSet
    </Output>
    <Output  class='DataObjects' type='HistorySet'>
      MonteCarloHTPIPEHistorySet
    </Output>
  </MultiRun>
</Steps>
\end{lstlisting}

