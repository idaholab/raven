\subsection{Generic Interface}
\label{subsec:genericInterface}
The GenericCode interface is meant to handle a wide variety of generic codes
that take straightforward input files and produce output CSV files.  There are
some limitations for this interface.
If a code: \vspace{-20pt}
\begin{itemize}
\item accepts a keyword-based input file with no cross-dependent inputs,
\item has no more than one filetype extension per command line flag,
\item and returns a CSV with the input parameters and output parameters,
\end{itemize}\vspace{-20pt}
the GenericCode interface should cover the code for RAVEN.

The GenericCode interface leverages a wildcard-based approach to editing input files. Using the
special wildcard format \texttt{\$RAVEN-\$}, RAVEN parses text-based inputs and replaces the
wildcards with sampled values. For example, consider RAVEN sampling variables named
\texttt{initial\_velocity} and \texttt{initial\_angle}. Assume we're using a projectile tracking model
with keyword based entry input files; for example,
\begin{lstlisting}[language=python]
  initial_height = 0     # starting height, m
  initial_angle = 35     # starting angle, degrees
  initial_velocity = 40  # starting velocity, m/s
  gravity = 9.8          # accel due to grav, m/s/s
  auxfile = gen.two      # additional properties file
  case = myOut           # output name (adds .csv)
 \end{lstlisting}
Since we want to sample \texttt{initial\_velocity} and \texttt{initial\_angle}, we create a new
template input and replace the values where samples should go with the wildcard and the variable
name:
\begin{lstlisting}[language=python]
  initial_height = 0     # starting height, m
  initial_angle = $RAVEN-initial_angle$ # starting angle, degrees
  initial_velocity = $RAVEN-initial_velocity$ # starting velocity, m/s
  gravity = 9.8          # accel due to grav, m/s/s
  auxfile = gen.two      # additional properties file
  case = myOut           # output name (adds .csv)
\end{lstlisting}
See more discussion of replacing the output case and auxiliary file names below. When RAVEN samples
values for the initial height and velocity, it will generate a new input file with those values in
place, for example,
\begin{lstlisting}[language=python]
  initial_height = 0     # starting height, m
  initial_angle = 22.7589 # starting angle, degrees
  initial_velocity = 47.2076 # starting velocity, m/s
  gravity = 9.8          # accel due to grav, m/s/s
  auxfile = gen.two      # additional properties file
  case = myOut           # output name (adds .csv)
\end{lstlisting}

If a code contains cross-dependent data, the generic interface is not able to
edit the correct values.  For example, if a geometry-building script specifies
inner\_radius, outer\_radius, and thickness, the generic interface cannot
calculate the thickness given the outer and inner radius, or vice versa.
In this case, the \textit{function} method explained in the Samplers (see \ref{sec:Samplers})
and Optimizers (see \ref{sec:Optimizers}) sections can be used.

 An example of the code interface is shown here.  The input parameters are read
 from the input files \texttt{gen.one} and \texttt{gen.two} respectively.
 The code is run using \texttt{python}, so that is part of the \xmlNode{clargs} node with the \xmlAttr{type} equal \xmlString{prepend}.
 The command line entry to normally run the code is
\begin{lstlisting}[language=bash]
python poly_inp.py -i gen.one -a gen.two -o myOut
\end{lstlisting}
and produces the output \texttt{myOut.csv}.

Example:
\begin{lstlisting}[style=XML]
    <Code name="poly" subType="GenericCode">
      <executable>GenericInterface/poly_inp.py</executable>
      <inputExtentions>.one,.two</inputExtentions>
      <clargs type='prepend' arg='python'/>
      <clargs type='input'   arg='-i' extension='.one'/>
      <clargs type='input'   arg='-a' extension='.two'/>
      <clargs type='output'  arg='-o'/>
    </Code>
\end{lstlisting}

If a code doesn't accept necessary Raven-editable auxiliary input files
or output filenames through the command line, the GenericCode interface
can also edit the input files and insert the filenames there.  For example,
in the previous example, say instead of \texttt{-a gen.two} and \texttt{-o myOut}
in the command line, \texttt{gen.one} has the following lines:
\begin{lstlisting}[language=python]
...
auxfile = gen.two
case = myOut
...
\end{lstlisting}
Then, our example XML for the code would be

Example:
\begin{lstlisting}[style=XML]
    <Code name="poly" subType="GenericCode">
      <executable>GenericInterface/poly_inp.py</executable>
      <inputExtentions>.one,.two</inputExtentions>
      <clargs   type='prepend' arg='python'/>
      <clargs   type='input'   arg='-i'  extension='.one'/>
      <fileargs type='input'   arg='two' extension='.two'/>
      <fileargs type='output'  arg='out'/>
    </Code>
\end{lstlisting}
and the corresponding template input file lines would be changed to read
\begin{lstlisting}[language=python]
...
auxfile = $RAVEN-two$
case = $RAVEN-out$
...
\end{lstlisting}


%%%%
If a code has hard-coded output file names that are not changeable,
the GenericCode interface can be invoked using the \xmlNode{outputFile}
node in which the output file name (CSV only) must be specified.
For example, in the previous example, say instead of \texttt{-a gen.two} and \texttt{-o myOut}
in the command line, the code always produce a CSV file named ``fixed\_output.csv'';

Then, our example XML for the code would be

Example:
\begin{lstlisting}[style=XML]
    <Code name="poly" subType="GenericCode">
      <executable>GenericInterface/poly_inp.py</executable>
      <inputExtentions>.one,.two</inputExtentions>
      <clargs   type='prepend' arg='python'/>
      <clargs   type='input'   arg='-i'  extension='.one'/>
      <fileargs type='input'   arg='two' extension='.two'/>
      <outputFile>fixed_output.csv</outputFile>
    </Code>
\end{lstlisting}

In addition, the ``wild-cards'' above can contain two special and optional symbols:
\begin{itemize}
  \item  \texttt{:}, that defines an eventual default value;
  \item  \texttt{|}, that defines the format of the value. The  Generic Interface currently supports the following formatting options (* in the examples means blank space):
    \begin{itemize}
       \item \textbf{plain integer}, in this case  the value that is going to be replaced by the Generic Interface, will be left-justified with a string length equal to the integer value specified here (e.g. ``\texttt{|}6'', the value is left-justified with a string length of 6);
      \item \textbf{d}, signed integer decimal, the value is going to be formatted as an integer (e.g.  if the value is 9 and the format ``\texttt{|}10d'', the replaced value will be formatted as follows: ``*********9'');
      \item \textbf{e}, floating point exponential format (lowercase), the value is going to be formatted as a float in scientific notation (e.g. if the value is 9.1234 and the format ``\texttt{|}10.3e'', the replaced value will be formatted as follows: ``*9.123e+00'' );
      \item \textbf{E}, floating point exponential format (uppercase), the value is going to be formatted as a float in scientific notation (e.g. if the value is 9.1234 and the format ``\texttt{|}10.3E'', the replaced value will be formatted as follows: ``*9.123E+00'' );
      \item \textbf{f or F}, floating point decimal format, the value is going to be formatted as a float in decimal notation (e.g. if the value is 9.1234 and the format ``\texttt{|}10.3f'', the replaced value will be formatted as follows: ``*****9.123'' );
      \item \textbf{g}, floating point format. Uses lowercase exponential format if exponent is less than -4 or not less than precision, decimal format otherwise (e.g. if the value is 9.1234 and the format ``\texttt{|}10.3g'', the replaced value will be formatted as follows: ``******9.12'' );
      \item \textbf{G}, floating point format. Uses uppercase exponential format if exponent is less than -4 or not less than precision, decimal format otherwise (e.g. if the value is 0.000009 and the format ``\texttt{|}10.3G'', the replaced value will be formatted as follows: ``*****9E-06'' ).
    \end{itemize}|
\end{itemize}
For example:
\begin{lstlisting}[language=python]
...
auxfile = $RAVEN-two:3$
case = $RAVEN-out:5|10$
...
\end{lstlisting}
Where,
\begin{itemize}
  \item  \texttt{:}, in case the variable ``two'' is not defined in the RAVEN XML input file, the Parser, will replace it with the value ``3''.;
  \item  \texttt{|}, the value that is going to be replaced by the Generic Interface, will be left- justified with a string length of ``10'';
\end{itemize}

The GenericCode interface also implements a mean for the user to stop a simulation while running if certain criteria are met, 
monitoring the output file (CSV in this case). 
Indeed, the user can provide a function to halt the simulation if a certain condition is met. The criterion or criteria is or are defined through
an external RAVEN python function (e.g. ``criteria'' in the  following example). The time frequency of the check is defined
by the node \xmlNode{onlineStopCriteriaTimeInterval} . By default, the output  is checked every 5 seconds.
\\To activate such capability, the following addition XML nodes can be inputted:
\begin{itemize}
   \item \xmlNode{onlineStopCriteriaTimeInterval}, \xmlDesc{float, optional parameter}, time frequency (in seconds) at which  the stopping function (if inputted)
                                                            is inquired (stoppingCriteriaFunction). \default{5}.
   \item \xmlNode{StoppingFunction}, \xmlDesc{XML node, optional node}, the
                                                           body of this XML node must contain the name of an appropriate \textbf{Function} defined in the
                                                            \xmlNode{Functions} block (see Section~\ref{sec:functions}). It is used as a
                                                           ``stopping'' function tool for online simulation monitoring. Indeed, the user can provide a
                                                            function to halt the simulation if a certain condition is
                                                            met. The criterion or criteria is or are defined through the external RAVEN python function linked here through this XML node. 
                                                            The function must return a ``bool'':
                                                            \begin{itemize}
                                                              \item False, if the simulation can continue (i.e. the criteria are not met)
                                                              \item True, if the simulation must STOP (i.e. the criteria are met)
                                                            \end{itemize}
                                                            %
                                                            Each variable defined in the \xmlNode{StoppingFunction} block is available in the
                                                            function as a python \texttt{raven.} member. In the following, an example of a
                                                            user-defined stopping condition function is reported.
                                                            \nb if this function is used, the user can monitor the reason of a stop exporting the variable ``StoppingReason'' if requested in RAVEN DataObjects.
\begin{lstlisting}[language=python]
def criteria(raven):
  if raven.var1[-1] < 1.0 or raven.var2[-1] < 1.0:
    return True
  else:
    return False
\end{lstlisting}
\end{itemize}

Then, our example XML for the code would be

Example:
\begin{lstlisting}[style=XML]
<Models>
    <Code name="poly" subType="GenericCode">
      <executable>GenericInterface/poly_inp.py</executable>
      <inputExtentions>.one,.two</inputExtentions>
      <clargs   type='prepend' arg='python'/>
      <clargs   type='input'   arg='-i'  extension='.one'/>
      <fileargs type='input'   arg='two' extension='.two'/>
      <outputFile>fixed_output.csv</outputFile>
      <!--
        Check the output every 15 seconds
       -->
      <onlineStopCriteriaTimeInterval>
        15.0
      </onlineStopCriteriaTimeInterval>
      <StoppingFunction class="Functions" type="External">
             criteria
      </StoppingFunction>
    </Code>
</Models>   
 
<Functions>
    <External file="stoppingCriteria.py" name="criteria">
         <!-- 
           variables below must be retrievable
           from the generic code interface csv output 
         -->
        <variables>var1, var2 </variables>
    </External>
</Functions>
    
\end{lstlisting}


