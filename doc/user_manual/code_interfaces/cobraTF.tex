%%%%%%%%%%%%%%%%%%%%%
%%% CTF INTERFACE %%%
%%%%%%%%%%%%%%%%%%%%%
\subsection{COBRA-TF (CTF) Interface}
This section presents the main aspects of the interface between RAVEN and CTF (COBRA-TF) system,
the consequent RAVEN input adjustments and the modifications of the CTF
files required to run the two coupled codes. \noindent

\noindent \textcolor{red}{
\textbf{\textit{\nb This interface is currently working only with the specific type of CTF output file (.ctf.out or deck.out (if input file name is deck.inp)) }}
}

\noindent In the following sections a short explanation on how to use RAVEN coupled with CTF is reported.
%%%%%%%%%%%%%%%%%%%%%%%%%%%%%%%%%%%%%%%%%%%%%%%%%%%
\subsubsection{Sequence}
%%%%%%%%%%%%%%%%%%%%%%%%%%%%%%%%%%%%%%%%%%%%%%%%%%%
In the \xmlNode{Sequence} section, the names of the steps declared in the
\xmlNode{Steps} block should be specified.
%
As an example, if we called the first MultiRun ``Grid\_Sampler'' and the second
MultiRun ``MC\_Sampler'' in the sequence section we should see this:

\begin{lstlisting}[style=XML]
<Sequence>Grid_Sampler, MC_Sampler</Sequence>
\end{lstlisting}

%%%%%%%%%%%%%%%%%%%%%%%%%%%%%%%%%%%%%%%%%%%%%%%%%%%
\subsubsection{batchSize and mode}
%%%%%%%%%%%%%%%%%%%%%%%%%%%%%%%%%%%%%%%%%%%%%%%%%%%
For the \xmlNode{batchSize} and \xmlNode{mode} sections please refer to the
\xmlNode{RunInfo} block in the previous chapters.

%%%%%%%%%%%%%%%%%%%%%%%%%%%%%%%%%%%%%%%%%%%%%%%%%%%%
\subsubsection{RunInfo}
%%%%%%%%%%%%%%%%%%%%%%%%%%%%%%%%%%%%%%%%%%%%%%%%%%%%
After all of these blocks are filled out, a standard example RunInfo block may
look like the example below: \\

\begin{lstlisting}[style=XML]
<RunInfo>
  <WorkingDir>~/workingDir</WorkingDir>
  <Sequence>Grid_Sampler,MC_Sampler</Sequence>
  <batchSize>8</batchSize>
</RunInfo>
\end{lstlisting}
In this example, the \xmlNode{batchSize} is set to $8$; this means that 8 simulatenous (parallel) instances
of CTF are going to be executed when a sampling strategy is employed.

%%%%%%%%%%%%%%%%%%%%%%%%%%%%%%%%%%%%%%%%%%%%%%%%%%%%%%%%%
\subsubsection{Models}
%%%%%%%%%%%%%%%%%%%%%%%%%%%%%%%%%%%%%%%%%%%%%%%%%%%%%%%%%
As any other Code, in order to activate the CTF interface, a \xmlNode{Code} XML node needs to be inputted, within the
main XML node \xmlNode{Models}.
\\The  \xmlNode{Code} XML node contains the
information needed to execute the specific External Code.

\attrsIntro
%
\vspace{-5mm}
\begin{itemize}
  \itemsep0em
  \item \nameDescription
  \item \xmlAttr{subType}, \xmlDesc{required string attribute}, specifies the
  code that needs to be associated to this Model.
  %
  \nb See Section~\ref{sec:existingInterface} for a list of currently supported
  codes.
  %
\end{itemize}
\vspace{-5mm}

\subnodesIntro
%
\begin{itemize}
  \item \xmlNode{executable} \xmlDesc{string, required field} specifies the path
  of the executable to be used.
  %
  \nb Either an absolute or relative path can be used.
  \item \aliasSystemDescription{Code}
  %
\end{itemize}

An example  is shown  below:

\begin{lstlisting}[style=XML]
<Models>
    <Code name="MyCobraTF" subType="CTF">
      <executable>path/to/cobratf</executable>
    </Code>
</Models>
\end{lstlisting}

%%%%%%%%%%%%%%%%%%%%%%%%%%%%%%%%%%%%%%%%%%%%%%%%%%%%%%%%%%%%%%%%%%%%%%%%%%%%%%%%%%
\subsubsection{Files}
%%%%%%%%%%%%%%%%%%%%%%%%%%%%%%%%%%%%%%%%%%%%%%%%%%%%%%%%%%%%%%%%%%%%%%%%%%%%%%%%%%
The \xmlNode{Files} XML node has to contain all the files required to run the external code  (CTF).
For RAVEN coupled with CTF, there are three input files. CTF input file (.inp) is required by the code. This input file includes all the geometry, boundary and calculation definitions.

\noindent There are two additional files (\textit{optional}) that can be used for model parameter perturbation(s) (vuq$\_$param.txt, vuq$\_$mult.txt). These two files can be used to change variables of models embedded in CTF. The ''vuq$\_$param.txt'' file includes parameter values, and ''vuq$\_$mult.txt'' file includes multipliers or additions to parameters. These files are not required by CTF unless a parameter exposure is desired. One, both or neither of them can be included in the simulation folder. The code first controls if these files exist and does modifications accordingly if needed.

The  \xmlNode{Files} XML node contains the information needed to execute CTF.

\attrsIntro
%
\vspace{-5mm}
\begin{itemize}
  \itemsep0em
  \item \nameDescription
  \item \xmlAttr{type}, \xmlDesc{required string attribute}, specifies the
  input type used by CTF (ctf, vuq$\_$mult, vuq$\_$param). Accepted types are as follows.
  \begin{itemize}
    \item CTF, \xmlDesc{required string attribute}, identifies the CTF input file (geometry, boundary, calculation options, etc.) and the code currently accept any name for input. One CTF input file is required.
    \item vuq\_mult, \xmlDesc{optional string attribute if closure modifiers are used}, identifies the closure term multiplier input file. If user needs to alter closure terms this file should be used and named \textcolor{red}{vuq$\_$mult.txt}. No other file name is accepted.
    \item vuq\_param, \xmlDesc{optional string attribute if model parameter modifiers are used}, identifies the model parameter input file. If user needs to change model parameters this file this model should be used and named \textcolor{red}{vuq$\_$params.txt}. No other file name is accepted.
    \item "", Empty type is also accepted by RAVEN input to perturb. Currently, CTF does not use any other input file that is not mentioned above, but to sample auxiliary files, this option can be used.
  \end{itemize}
\end{itemize}

\noindent Example:
\begin{lstlisting}[style=XML]
<Files>
  <Input name="CTF_input" type="ctf">case1.inp</Input>
  <Input name="vuq_param_input" type="vuq_param">vuq_param.txt</Input>
  <Input name="vuq_mult_input" type="vuq_mult">vuq_mult.txt</Input>
  <Input name="auxiliary_input" type="">auxiliaryInput</Input>
</Files>
\end{lstlisting}
The files mentioned in this section
 need, then, to be placed into the working directory specified
by the \xmlNode{workingDir} node in the \xmlNode{RunInfo} XML node.

\paragraph{Output Files Conversion}
Since RAVEN expects to receive a CSV file containing the outputs of the simulation, the results in the CTF output
files (.ctf.out or deck.out) need to be converted by the code interface.

\noindent \textcolor{red}{
\textbf{It is important to note that the interface output collection (i.e., the parser of the CTF output) is currently able to extract
major edit data (.ctf.out or deck.out) only. Only those variables printed in the "major edit" output files are exported and made available to RAVEN.} } \\
\\The following information is collected from CTF output file (.ctf.out or deck.out):
\begin{itemize}
  \item \textit{\textbf{average properties for channels}}
  \begin{lstlisting}[basicstyle=\tiny]


 ************************************************************************************************************************
          simulation time =      1.03030  seconds           aver. properties for channels
 node  dist.  quality    void fraction            mass flow               enthalpy incr.    enthalpy    heat added
  no.   (ft.)                                     (lbm/s)                    (btu/s)         (btu/s)      (btu/s)
                      liq.  vapor  entr.  liquid vapor entr.  integr.  liquid vapor integr.  mixture  liquid vapor integr.

  50   12.00  -.119   1.000 0.000 0.000   16.39  0.00  0.00  16.39     32.41  0.00  32.41   10683.98  32.37  0.00  32.37

\end{lstlisting}

   that will be converted in the following way (CSV):
   \begin{table}[h]
    \centering
    \caption{CSV transport info (average properties for channels)}
    \label{CSVaverageProperties}
    \tabcolsep=0.11cm
    \tiny
    \begin{tabular}{|c|c|c|c|c|c|c|c|c|c|c|}
     time & AVG\_ch\_ax50\_quality  & AVG\_ch\_ax50\_voidFractionLiquid & AVG\_ch\_ax50\_voidFractionVapor & AVG\_ch\_ax50\_volumeEntrainFraction & ...\\
     1.03030 & -.119 & 1.000  & 0.000   & 0.000  & ...
    \end{tabular}
   \end{table}

  \item \textit{\textbf{fluid properties for each sub-channel}}
  \begin{lstlisting}[basicstyle=\tiny]
              simulation time =      0.00000  seconds           fluid properties for channel   19
 node  dist. pressure  velocity             void fraction           flow rate          flow    heat added         gama
  no.  (ft.) (psi)     (ft/sec)                                      (lbm/s)           reg.     (btu/s)          (lbm/s)
                     liquid vapor entr. liquid  vapor  entr.   liquid  vapor   entr.         liquid    vapor


 155 0.00  1251.687  2.66   2.66  0.01  1.0000 0.0000 0.0000  0.12456  0.0000  0.00000  0   0.595E-01  0.000E+00   0.00

  \end{lstlisting}
   that will be converted in the following way (CSV):
   \begin{table}[h]
    \centering
    \caption{CSV transport info (fluid properties for channels)}
    \label{CSVfluidProperties}
    \tabcolsep=0.11cm
    \tiny
    \begin{tabular}{|c|c|c|c|c|c|c|c|c|c|c|}
     time & ch19\_ax155\_pressure  & ch19\_ax155\_velocityLiquid & ch19\_ax155\_velocityVapor & ch19\_ax155\_velocityEntrain & ch19\_ax155\_voidFractionLiquid & ...\\
     0.00 & 1251.687 & 2.66           & 2.66   & 0.01        & 1.00 & ...
    \end{tabular}
   \end{table}

  \item \textit{\textbf{nuclear fuel rod}}
  \begin{lstlisting}[basicstyle=\tiny]
          nuclear fuel rod no.  1                         simulation time =    0.00 seconds
             surface no.  1 of  1
          -----------------------        conducts heat to channels  1  0  0  0  0  0        geometry type =  1
                                         and azimuthally to surfaces   1 and   1            no. of radial nodes = 13

 **********************************************************************************************

   rod    axial    fluid temperatures  surface      heat       -clad temperatures-     gap        -fuel temperatures-
   node  location      (deg-f)         heat flux   transfer          (deg-f)        conductance         (deg-f)
   no.    (in.)    liquid  vapor       (b/h-ft2)    mode       outside  inside      (b/h-ft2-f)    surface   center
   ----  --------  ------  -----      ---------    --------    -------  ------      -----------    -------   ------

    10   22.80     464.1   467.1     0.5929E+04     spl        466.08   592.98       1594.2        859.58     2946.22
\end{lstlisting}
   that will be converted in the following way (CSV):
   \begin{table}[h]
     \centering
     \caption{CSV transport info (nuclear fuel rod)}
     \label{CSVfuelRod}
     \tabcolsep=0.11cm
     \tiny
     \begin{tabular}{|c|c|c|c|}
     time    & fuelRod10\_surface1\_ax10\_fluidTemperatureLiquid  & fuelRod10\_surface1\_ax10\_fluidTemperatureVapor & ... \\
     0.00 & 464.1 & 467.1  &  ...
     \end{tabular}
   \end{table}

   \item \textit{\textbf{cyclindrical tube}}

      \noindent \textcolor{red}{
   \textbf{Warning: CTF reports results for cylindrical tubes based on the flow type. Not every result will be available depending on the \underline{internal} or \underline{external} flow type. Check output file and see if the flow around heat slab is internal or external. If the user requests values that are not in the output file reported values will be from different columns and wrong. For example there is no outside surface liquid temperature when flow is internal.} } \\

   \begin{lstlisting}[basicstyle=\tiny]
    cylindrical tube rod no.  5                         simulation time =    2.00 seconds
           surface no.  1 of  4
    ------------------------        conducts heat to channels 10  0  0  0  0  0               geometry type =  2
                                    and azimuthally to surfaces   4 and   2                   no. of radial nodes =  2


 **********************************************************************************************************************

 rod    axial    *-------------- outside surface ----------------*   *---------------- inside surface ----------------*
 node  location   heat flux   h.t.  **** temperatures (deg-f) ****   **** temperatures (deg-f) ****    h.t.   heat flux
 no.    (in.)     (b/h-ft2)   mode     wall     vapor     liquid       liquid     vapor     wall       mode   (b/h-ftl)
 ----  --------   ---------   ----   -------   -------   -------       -------   -------   -------

  51    144.00  -0.4424E+03   spl    629.71     653.31    629.77                           629.68           0.0000E+00
  \end{lstlisting}
    that will be converted in the following way (CSV):
    \begin{table}[h]
      \centering
      \caption{CSV transport info (cylindrical tube)}
      \label{CSVcylTube}
      \tabcolsep=0.11cm
      \tiny
      \begin{tabular}{|c|c|c|c|}
      time    & cylRod10\_surface1\_ax51\_outsideSurfaceHeatFlux  & cylRod10\_surface1\_ax51\_outsideSurfaceWallTemperature &  ...  \\
      0.00 & -0.4424E+03 & 629.71  & ...
      \end{tabular}
    \end{table}

  \item \textit{\textbf{heat slab (tube)}}

   \noindent \textcolor{red}{
   \textbf{Warning: CTF reports results for heat slabs based on the flow type. Not every result will be available depending on the \underline{internal} or \underline{external} flow type. Check output file and see if the flow around heat slab is internal or external. If the user requests values that are not in the output file reported values will be from different columns and wrong. For example there is no outside surface liquid temperature when flow is internal.} } \\

   \begin{lstlisting}[basicstyle=\tiny]
          heat slab no.  1  (tube)                  simulation time =   20.00 seconds
                                                    fluid channel on inside surface =  1
                                                    fluid channel on outside surface =  0
                                                    geometry type =  1
                                                    no. of nodes =   2


 ***************************************************************************************************************

   rod    axial    *------------- outside surface ------------* *-------------- inside surface ----------------*
   node  location   heat flux  h.t.  ** temperatures (deg-f) ** ** temperatures (deg-f) ****    h.t.   heat flux
   no.    (in.)     (b/h-ft2)  mode   wall     vapor     liquid liquid     vapor     wall       mode   (b/h-ftl)
   ----  --------   ---------  ----  ------   -------   ------- -------   -------   -------

    21     19.69    999666E+00       200.00                     212.00    212.00    247.85     tran    0.2641E+02
  \end{lstlisting}
    that will be converted in the following way (CSV):
    \begin{table}[h]
      \centering
      \caption{CSV transport info (heat slab (tube) tube)}
      \label{CSVheatSlab}
      \tabcolsep=0.11cm
      \tiny
      \begin{tabular}{|c|c|c|c|}
      time    & heatSlab1\_ax21\_outsideSurfaceHeatFlux &  heatSlab1\_ax21\_outsideSurfaceWallTemperature &  ...  \\
      0.00 & 999666E+00 & 200.00  & ...
      \end{tabular}
    \end{table}

 \item \textit{\textbf{CTF's Output Variables and Corresponding Names in CSV file}}

   In CSV file, the output results obtained from the CTF output file (.ctf.out) will be saved with the names as described in Table \ref{CSVvariableNames}.

\end{itemize}

\captionsetup{justification=centering}
\captionof{table}{Variables Name List in CSV File \hspace{\textwidth} \textcolor{red}{NN: Axial Node Number; CN: Channel Number; \\ RN: Rod Number; SN: Surface Number: HN: Heat Slab Number}}
\label{CSVvariableNames}
\begingroup
% header and footer information
\scriptsize
\begin{longtable}{|l|l|}
\hline
\textbf{Output Variable} & \textbf{Name in CSV file} \\
\hline
\endhead
\hline
\endfoot
% body of table
      \scriptsize {simulation time  }&\scriptsize{ time  } \\
      channels' average height  & AVG\_ch\_ax\textcolor{red}{NN}\_height \\
      channels' average quality  & AVG\_ch\_ax\textcolor{red}{NN}\_quality \\
      channels' average void fraction (liquid)  & AVG\_ch\_ax\textcolor{red}{NN}\_voidFractionLiquid \\
      channels' average void fraction (vapor)  & AVG\_ch\_ax\textcolor{red}{NN}\_voidFractionVapor \\
      channels' average entrainment (volumetric) fraction  & AVG\_ch\_ax\textcolor{red}{NN}\_volumeEntrainFraction \\
      channels' average mass flow rate (liquid)  & AVG\_ch\_ax\textcolor{red}{NN}\_massFlowRateLiquid \\
      channels' average mass flow rate (vapor)  & AVG\_ch\_ax\textcolor{red}{NN}\_massFlowRateVapor \\
      channels' average entrainment rate (mass flow rate)  & AVG\_ch\_ax\textcolor{red}{NN}\_massFlowRateEntrain \\
      channels' average mass flow rate (integrated)  & AVG\_ch\_ax\textcolor{red}{NN}\_massFlowRateIntegrated \\
      channels' average enthalpy increase (liquid)  & AVG\_ch\_ax\textcolor{red}{NN}\_enthalpyIncreaseLiquid \\
      channels' average enthalpy increase (vapor)  & AVG\_ch\_ax\textcolor{red}{NN}\_enthalpyIncreaseVapor \\
      channels' average enthalpy increase (integrated)  & AVG\_ch\_ax\textcolor{red}{NN}\_enthalpyIncreaseIntegrated \\
      channels' average mixture enthalpy  & AVG\_ch\_ax\textcolor{red}{NN}\_enthalpyMixture \\
      channels' average heat added to liquid & AVG\_ch\_ax\textcolor{red}{NN}\_heatAddedToLiquid \\
      channels' average heat added to vapor & AVG\_ch\_ax\textcolor{red}{NN}\_heatAddedToVapor \\
      channels' average heat added (integrated)  & AVG\_ch\_ax\textcolor{red}{NN}\_heatAddedIntegrated' \\
      channel height & ch\textcolor{red}{CN}\_ax\textcolor{red}{NN}\_height \\
      channel pressure & ch\textcolor{red}{CN}\_ax\textcolor{red}{NN}\_pressure \\
      channel liquid velocity  & ch\textcolor{red}{CN}\_ax\textcolor{red}{NN}\_velocityLiquid \\
      channel vapor velocity  & ch\textcolor{red}{CN}\_ax\textcolor{red}{NN}\_velocityVapor \\
      channel entrainment rate (velocity) & ch\textcolor{red}{CN}\_ax\textcolor{red}{NN}\_velocityEntrain \\
      channel void fraction (liquid) & ch\textcolor{red}{CN}\_ax\textcolor{red}{NN}\_voidFractionLiquid \\
      channel void fraction (vapor)  & ch\textcolor{red}{CN}\_ax\textcolor{red}{NN}\_voidFractionVapor \\
      channel volume fraction of entrainment liquid  & ch\textcolor{red}{CN}\_ax\textcolor{red}{NN}\_volumeEntrainFraction \\
      channel mass flow rate (liquid) & ch\textcolor{red}{CN}\_ax\textcolor{red}{NN}\_massFlowRateLiquid \\
      channel mass flow rate (vapor)  & ch\textcolor{red}{CN}\_ax\textcolor{red}{NN}\_massFlowRateVapor \\
      channel entrainment rate (mass flow rate) & ch\textcolor{red}{CN}\_ax\textcolor{red}{NN}\_massFlowRateEntrain \\
      channel flow regime ID & ch\textcolor{red}{CN}\_ax\textcolor{red}{NN}\_flowRegimeID \\
      channel heat added to liquid & ch\textcolor{red}{CN}\_ax\textcolor{red}{NN}\_heatAddedToLiquid \\
      channel heat added to vapor  & ch\textcolor{red}{CN}\_ax\textcolor{red}{NN}\_heatAddedToVapor \\
      channel evaporation rate  & ch\textcolor{red}{CN}\_ax\textcolor{red}{NN}\_evaporationRate \\
      channel enthalpy of vapor  & ch\textcolor{red}{CN}\_ax\textcolor{red}{NN}\_enthalpyVapor \\
      channel enthalpy of saturated vapor & ch\textcolor{red}{CN}\_ax\textcolor{red}{NN}\_enthalpySaturatedVapor \\
      channel enthalpy difference between vapor and saturated vapor & ch\textcolor{red}{CN}\_ax\textcolor{red}{NN}\_enthalpyVapor-SaturatedVapor \\
      channel enthalpy of liquid & ch\textcolor{red}{CN}\_ax\textcolor{red}{NN}\_enthalpyLiquid \\
      channel enthalpy of saturated liquid & ch\textcolor{red}{CN}\_ax\textcolor{red}{NN}\_enthalpySaturatedLiquid \\
      channel enthalpy difference between liquid and saturated liquid & ch\textcolor{red}{CN}\_ax\textcolor{red}{NN}\_enthalpyLiquid-SaturatedLiquid \\
      channel enthalpy of mixture & ch\textcolor{red}{CN}\_ax\textcolor{red}{NN}\_enthalpyMixture \\
      channel density of liquid  & ch\textcolor{red}{CN}\_ax\textcolor{red}{NN}\_densityLiquid \\
      channel density of vapor  & ch\textcolor{red}{CN}\_ax\textcolor{red}{NN}\_densityVapor \\
      channel density of mixture & ch\textcolor{red}{CN}\_ax\textcolor{red}{NN}\_densityMixture \\
      channel net entrainment rate & \\ (difference between entrainment rate and de-entrainment rate)  & ch\textcolor{red}{CN}\_ax\textcolor{red}{NN}\_netEntrainRate \\
      % gas volumetric analysis
      channel enthalpy of the mixture of non-condensable gases & ch\textcolor{red}{CN}\_ax\textcolor{red}{NN}\_enthalpyNonCondensableMixture \\
      channel density of the mixture of non-condensable gases & ch\textcolor{red}{CN}\_ax\textcolor{red}{NN}\_densityNonCondensableMixture \\
      channel steam volume fraction [0-100] & ch\textcolor{red}{CN}\_ax\textcolor{red}{NN}\_volumeFractionSteam \\
      channel air volume fraction [0-100] & ch\textcolor{red}{CN}\_ax\textcolor{red}{NN}\_volumeFractionAir \\
      channel total equivalent diameter of the liquid droplets & \\ (all droplets as a single big one) (diam-ld) & ch\textcolor{red}{CN}\_ax\textcolor{red}{NN}\_equiDiameterLiquidDroplet \\
      channel averaged diameter of liquid droplets field (diam-sd) & ch\textcolor{red}{CN}\_ax\textcolor{red}{NN}\_avgDiameterLiquidDroplet \\
      channel averaged flow rate of liquid droplets field (flow-sd) & ch\textcolor{red}{CN}\_ax\textcolor{red}{NN}\_avgFlowRateLiquidDroplet \\
      channel averaged velocity of liquid droplets field (veloc-sd) & ch\textcolor{red}{CN}\_ax\textcolor{red}{NN}\_avgVelocityLiquidDroplet \\
      channel evaporation rate of liquid droplets field (gamsd) & ch\textcolor{red}{CN}\_ax\textcolor{red}{NN}\_evaporationRateLiquidDroplet \\
      % fuel rod
      fuel rod height & fuelRod\textcolor{red}{RN}\_surface\textcolor{red}{SN}\_ax\textcolor{red}{NN}\_height \\
      fuel rod fluid temperatures (liquid) & fuelRod\textcolor{red}{RN}\_surface\textcolor{red}{SN}\_ax\textcolor{red}{NN}\_fluidTemperatureLiquid \\
      fuel rod fluid temperatures (vapor)  & fuelRod\textcolor{red}{RN}\_surface\textcolor{red}{SN}\_ax\textcolor{red}{NN}\_fluidTemperatureVapor \\
      fuel rod surface heat flux & fuelRod\textcolor{red}{RN}\_surface\textcolor{red}{SN}\_ax\textcolor{red}{NN}\_surfaceHeatflux \\
      %fuel rod surface heat transfer mode [-] & fuelRod\textcolor{red}{(Rod Node Number)}_surfaceN\textcolor{red}{SN}_ax\textcolor{red}{NN}\_heatTransferMode \\
      clad outer surface temperature  & fuelRod\textcolor{red}{RN}\_surface\textcolor{red}{SN}\_ax\textcolor{red}{NN}\_cladOutTemperature \\
      clad iRNer surface temperature  & fuelRod\textcolor{red}{RN}\_surface\textcolor{red}{SN}\_ax\textcolor{red}{NN}\_cladInTemperature \\
      gap conductance  & fuelRod\textcolor{red}{RN}\_surface\textcolor{red}{SN}\_ax\textcolor{red}{NN}\_gapConductance \\
      fuel outer suface temperature  & fuelRod\textcolor{red}{RN}\_surface\textcolor{red}{SN}\_ax\textcolor{red}{NN}\_fuelTemperatureSurface \\
      fuel center temperature  & fuelRod\textcolor{red}{RN}\_surface\textcolor{red}{SN}\_ax\textcolor{red}{NN}\_fuelTemperatureCenter \\
      % cylindrical rod
      cylindrical tube height & cylRod\textcolor{red}{RN}\_surface\textcolor{red}{SN}\_ax\textcolor{red}{NN}\_height \\
      cylindrical tube outside surface heat flux & cylRod\textcolor{red}{RN}\_surface\textcolor{red}{SN}\_ax\textcolor{red}{NN}\_outsideSurfaceHeatFlux \\
      cylindrical tube outside surface wall temperature & cylRod\textcolor{red}{RN}\_surface\textcolor{red}{SN}\_ax\textcolor{red}{NN}\_outsideSurfaceWallTemperature \\
      cylindrical tube outside surface vapor temperature & cylRod\textcolor{red}{RN}\_surface\textcolor{red}{SN}\_ax\textcolor{red}{NN}\_outsideSurfaceVaporTemperature \\
      cylindrical tube outside surface liquid temperature & cylRod\textcolor{red}{RN}\_surface\textcolor{red}{SN}\_ax\textcolor{red}{NN}\_outsideSurfaceLiquidTemperature \\
      cylindrical tube inside surface wall temperature & cylRod\textcolor{red}{RN}\_surface\textcolor{red}{SN}\_ax\textcolor{red}{NN}\_insideSurfaceWallTemperature \\
      cylindrical tube inside surface vapor temperature & cylRod\textcolor{red}{RN}\_surface\textcolor{red}{SN}\_ax\textcolor{red}{NN}\_insideSurfaceVaporTemperature \\
      cylindrical tube inside surface liquid temperature & cylRod\textcolor{red}{RN}\_surface\textcolor{red}{SN}\_ax\textcolor{red}{NN}\_insideSurfaceLiquidTemperature \\
      cylindrical tube inside surface heat flux & cylRod\textcolor{red}{RN}\_surface\textcolor{red}{SN}\_ax\textcolor{red}{NN}\_insideSurfaceHeatFlux \\
      % heat slab
      heat slab (tube) height & heatSlab\textcolor{red}{HN}\_ax\textcolor{red}{NN}\_height \\
      heat slab (tube) outside surface heat flux & heatSlab\textcolor{red}{HN}\_ax\textcolor{red}{NN}\_outsideSurfaceHeatFlux\\
      heat slab (tube) outside surface wall temperature & heatSlab\textcolor{red}{HN}\_ax\textcolor{red}{NN}\_outsideSurfaceWallTemperature\\
      heat slab (tube) outside surface vapor temperature & heatSlab\textcolor{red}{HN}\_ax\textcolor{red}{NN}\_outsideSurfaceVaporTemperature\\
      heat slab (tube) outside surface liquid temperature & heatSlab\textcolor{red}{HN}\_ax\textcolor{red}{NN}\_outsideSurfaceLiquidTemperature\\
      heat slab (tube) inside surface wall temperature & heatSlab\textcolor{red}{HN}\_ax\textcolor{red}{NN}\_insideSurfaceWallTemperature\\
      heat slab (tube) inside surface vapor temperature & heatSlab\textcolor{red}{HN}\_ax\textcolor{red}{NN}\_insideSurfaceVaporTemperature\\
      heat slab (tube) inside surface liquid temperature & heatSlab\textcolor{red}{HN}\_ax\textcolor{red}{NN}\_insideSurfaceLiquidTemperature\\
      heat slab (tube) inside surface heat flux & heatSlab\textcolor{red}{HN}\_ax\textcolor{red}{NN}\_insideSurfaceHeatFlux
\end{longtable}
\endgroup

\textbf{\textit{\nb RAVEN, regonizes failed or crashed CTF runs and no data will be saved from those.}}

%%%%%%%%%%%%%%%%%%%%%%%%%%%%%%%%%%%%%%%%%%%%%%%%%%%%%%%%%
\subsubsection{Distributions}
%%%%%%%%%%%%%%%%%%%%%%%%%%%%%%%%%%%%%%%%%%%%%%%%%%%%%%%%%
The \xmlNode{Distribution} block defines the distributions that are going
to be used for the sampling of the variables defined in the \xmlNode{Samplers} block.
%
For all the possibile distributions and all their possible inputs please see the
chapter about Distributions (see~\ref{sec:distributions}).
%
Here we report an example of a Normal distribution:

\begin{lstlisting}[style=XML,morekeywords={name,debug}]
<Distributions verbosity='debug'>
    <Normal name="GridLossCoeff">
      <mean>0.7</mean>
      <sigma>0.1</sigma>
      <upperBound>0.9</upperBound>
      <lowerBound>0.6</lowerBound>
    </Normal>
    <Uniform name="DB1dist">
      <upperBound>0.026</upperBound>
      <lowerBound>0.020</lowerBound>
    </Uniform>
    <Uniform name="DB2dist">
      <upperBound>0.9</upperBound>
      <lowerBound>0.7</lowerBound>
    </Uniform>
    <Uniform name="DB3dist">
      <upperBound>0.5</upperBound>
      <lowerBound>0.3</lowerBound>
    </Uniform>
 </Distributions>
\end{lstlisting}

\noindent
It is good practice to name the distribution something similar to what kind of
variable is going to be sampled, since there might be many variables with the
same kind of distributions but different input parameters.

%%%%%%%%%%%%%%%%%%%%%%%%%%%%%%%%%%%%%%%%%%%%%%%%%%%%%%%%%
\subsubsection{Samplers}
%%%%%%%%%%%%%%%%%%%%%%%%%%%%%%%%%%%%%%%%%%%%%%%%%%%%%%%%%
In the \xmlNode{Samplers} block we want to define the variables that are going to be sampled.

\noindent The perturbation or optimization of the input of any CTF sequence is performed using the approach detailed in the \textit{Generic Interface} section (see \ref{subsec:genericInterface}). Briefly, this approach uses
 ``wild-cards'' (placed in the original input files) for injecting the perturbed values.
 For example, if the original input file (that needs to be perturbed) is the following:

\textbf{Example}:
We want to do the sampling of 1 single variable:
\begin{itemize}
  \item The Grid Loss Coefficient Data is used from sampled values.
\end{itemize}

\noindent We are going to sample this variable using two different sampling methods: Grid and MonteCarlo.
The RAVEN input is then written as follows:

\begin{lstlisting}[style=XML,morekeywords={name,type,construction,lowerBound,steps,limit,initialSeed}]
<Samplers verbosity='debug'>
  <Grid name='Grid_Sampler' >
    <variable name='GrdLss'>
      <distribution>GridLossCoeff</distribution>
      <grid type='CDF' construction='equal'  steps='10'>0.001 0.999</grid>
    </variable>
  </Grid>
  <MonteCarlo name='MC_Sampler'>
     <samplerInit>
       <limit>1000</limit>
     </samplerInit>
    <variable name='GrdLss'>
      <distribution>GridLossCoeff</distribution>
    </variable>
    <variable name='DB1'>
      <distribution>DB1dist</distribution>
    </variable>
    <variable name='DB2'>
      <distribution>DB2dist</distribution>
    </variable>
    <variable name='DB3'>
      <distribution>DB3dist</distribution>
    </variable>
  </MonteCarlo>
</Samplers>
\end{lstlisting}

CTF input file should be modified with wild-cards in the following way.
\begin{lstlisting}[basicstyle=\tiny]
***********************************************************************************************
*GROUP 7 - Grid Loss Coefficient Data                                                         *
***********************************************************************************************
**NGR
    7
*Card 7.1
**  NCD NGT  IFGQF IFSDRP  IFESPV  IFTPE  IGTEMP  NFBS  NDM9 NDM10 NDM11 NDM12 NDM13 NDM14
     21   0      0      0       0      0       0     0     0     0     0     0     0     0
*Card 7.2
**       CDL      J   CD1   CD2   CD3   CD4   CD5   CD6   CD7   CD8   CD9  CD10  CD11  CD12
 $RAVEN-GrdLss$   1     1     2     3     4     5     6     7     8     9    10    11    12
     0.90700      1    13    14    15    16    17    18    19    20    21    22    23    24
     0.90700      1    25    26    27    28    29    30    31    32    33    34    35    36

\end{lstlisting}

It is also possible to modify input values in parameter exposure input files.\\

\textbf{Example}:
We want to do the sampling of 3 correlation parameters (Dittus-Boelter parameter modification, DB1 $\times$ Re$^{DB2}$ $\times$ Pr$^{DB3}$):

\begin{itemize}
  \item DB1, DB2, DB3 values will be sampled and vuq\_param.txt will be modified with sampled values.
\end{itemize}

vuq$\_$param.txt and vuq$\_$mult.txt files are modified similarly with defined variable names.

\begin{lstlisting}[basicstyle=\tiny]
k_chen_1 = 0.24
k_chen_2 = 0.75
k_db_1 = $RAVEN-DB1$
k_db_2 = $RAVEN-DB2$
k_db_3 = $RAVEN-DB3$
k_db_4 = 7.86
k_wf_1 = 1.691
k_wf_2 = 0.43
\end{lstlisting}

\noindent It can be seen that each variable is connected with a proper distribution
defined in the \\\xmlNode{Distributions} block (from the previous example).

%%%%%%%%%%%%%%%%%%%%%%%%%%%%%%%%%%%%%%%%%%%%%%%%%%%%%%%%%%%
\subsubsection{Steps}
For a CTF interface, the \xmlNode{MultiRun} step type will most likely be
used. But \xmlNode{SingleRun} step can also be used for plotting and data extraction purposes.
%
First, the step needs to be named: this name will be one of the names used in
the \xmlNode{sequence} block.
%
In our example, \texttt{Grid\_Sampler} and \texttt{MC\_Sampler}.
%
\begin{lstlisting}[style=XML,morekeywords={name,debug,re-seeding}]
     <MultiRun name='Grid_Sampler' verbosity='debug'>
\end{lstlisting}

With this step, we need to import all the files needed for the simulation:
\begin{itemize}
  \item CTF input files
\end{itemize}
\begin{lstlisting}[style=XML,morekeywords={name,class,type}]
    <Input class="Files" type="ctf">CTFinput</Input>
    <Input class="Files" type="vuq_param">vuq_param</Input>
    <Input class="Files" type="vuq_mult" >vuq_mult</Input>
\end{lstlisting}
We then need to define which model will be used:
\begin{lstlisting}[style=XML]
    <Model  class='Models' type='Code'>MyCobraTF</Model>
\end{lstlisting}
We then need to specify which Sampler is used, and this can be done as follows:
\begin{lstlisting}[style=XML]
    <Sampler class='Samplers' type='Grid'>Grid_Sampler</Sampler>
\end{lstlisting}
And lastly, we need to specify what kind of output the user wants.
%
For example the user might want to make a database (in RAVEN the database
created is an HDF5 file).
%
Here is a classical example:
\begin{lstlisting}[style=XML,morekeywords={class,type}]
    <Output  class='Databases' type='HDF5'>Grid_out</Output>
\end{lstlisting}
Following is the example of two MultiRun steps which use different sampling
methods (Grid and Monte Carlo), and creating two different databases for each
one:
\begin{lstlisting}[style=XML]
<Steps verbosity='debug'>
  <MultiRun name='Grid_Sampler' verbosity='debug'>
    <Input   class='Files' type="ctf">CTFinput</Input>
    <Input   class="Files" type="vuq_param">vuq_param</Input>
    <Input   class="Files" type="vuq_mult" >vuq_mult</Input>
    <Model   class='Models'    type='Code'>MyCobraTF</Model>
    <Sampler class='Samplers'  type='Grid'>Grid_Sampler</Sampler>
    <Output  class='Databases' type='HDF5'>Grid_out</Output>
    <Output  class='DataObjects' type='PointSet'   >GridCTFPointSet</Output>
    <Output  class='DataObjects' type='HistorySet'>GridCTFHistorySet</Output>
  </MultiRun>
  <MultiRun name='MC_Sampler' verbosity='debug' re-seeding='210491'>
    <Input   class='Files' type=''>CTFinput</Input>
    <Input   class="Files" type="">vuq_param</Input>
    <Input   class="Files" type="" >vuq_mult</Input>
    <Model   class='Models'    type='Code'>MyCobraTF</Model>
    <Sampler class='Samplers'  type='MonteCarlo'>MC_Sampler</Sampler>
    <Output  class='Databases' type='HDF5'      >MC_out</Output>
    <Output  class='DataObjects' type='PointSet'   >MonteCarloCobraPointSet</Output>
    <Output  class='DataObjects' type='HistorySet'>MonteCarloCobraHistorySet</Output>
  </MultiRun>
</Steps>
\end{lstlisting}

