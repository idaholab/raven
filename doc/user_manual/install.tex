\section{Installation Overview}

The installation of the RAVEN code is a straightforward procedure;
depending on the usage purpose and machine architecture, the
installation process slightly differs.

In the following sections, all the different installation procedures
are reported.

There are two main requirements to installing RAVEN, installing the
RAVEN dependencies (Section \ref{raven_dependencies}) and installing
RAVEN itself (Section \ref{raven_installation}).  For any particular
installation, only one of the raven dependency procedures and one of
the raven installation paths needs to be taken.

For macOS (OSX) it is recommended that the dependencies be installed with
Miniconda (Section \ref{miniconda}).  For Linux, it is recommended
that the distribution package manager be used if possible (Section
\ref{sysprep_linux}) or Miniconda (Section \ref{miniconda}).

There are several different ways of getting RAVEN as described in
Section \ref{raven_installation}.  They vary depending on how easy
they are to use and how easy they are to develop with (these tend to
be inversely correlated).

\newcommand{\goToRavenInstallation}{Now go on to Section \ref{raven_installation} for Raven installation.
}


\section{RAVEN Dependencies Installation}
\label{raven_dependencies}

RAVEN is built upon several freely available open-source software packages,
which must  be installed before it will function properly:

\begin{center}
    \begin{tabular}{ | l | p{10cm} |}
    \hline Package & Purpose \\
    \hline git & Source code control tool \\
    \hline g++ & C++ language compiler suite (Needed to build
        support code in the RAVEN package) \\
    \hline libtool & Generic library management tool \\
    \hline python & Scripting language RAVEN is written in \\
    \hline swig & Simplified Wrapper and Interface Generator
        (Used to create Python interfaces to supporting C++ code) \\
    \hline hdf5 & Library providing interface to HDF5 database
        file format used by RAVEN \\
    \hline h5py & Python interface to HDF5 database library \\
    \hline numpy & N-Dimensional array package for Python \\
    \hline scipy &  Scientific computing package for Python \\
    \hline scikit-learn & Machine learning library for Python \\
    \hline matplotlib & Plotting library for Python \\
    \hline
    \end{tabular}
\end{center}


RAVEN is supported on three separate computing platforms:
Linux, OSX (Apple Macintosh), and Microsoft Windows.  Depending
on which of these systems is used, the preparation of the system
to run RAVEN varies.

\subsection{Preparing a Linux System for RAVEN}
\label{sysprep_linux}

The installation of RAVEN dependencies on a Linux system  can be performed
using two alternative methods: 

\begin{itemize}
    \item Native distribution's package manager
    \item Miniconda package manager
\end{itemize}

Using one of the above automates the process and automatically includes any needed
dependencies of the requested packages.  Below are instructions for
doing so for two popular Linux distributions, Ubuntu and Fedora.  

\subsubsection{Ubuntu}

\paragraph{Advanced Package Tool (APT)}
The Ubuntu distribution of Linux makes use of the Advanced Package
Tool (APT) to automate the installation of pre-configured software.
Ubuntu 16.4 or newer provide all the packages needed for RAVEN.  The
following command uses the APT to add the needed packages and any
needed dependencies not already on the system:

\begin{lstlisting}[language=bash]
 sudo apt-get install libtool git python-dev swig g++ \
 python3-dev python-numpy python-sklearn python-h5py
\end{lstlisting}


\paragraph{Miniconda}
The Minoconda package manager is a cross platform installation package specifically 
used for Python dependencies installation.
The package manager can be downloaded and installed from \url{https://conda.io/miniconda.html}.
After the installation of Minconda, the following command needs to be executed for the installation of
the needed packages and any needed dependencies not already on the system:

\begin{lstlisting}[language=bash]
 conda create --name raven_libraries -y numpy=1.11.0 \
  h5py=2.6.0 scipy=0.17.1 scikit-learn=0.17.1 \
  matplotlib=1.5.1 python=2.7 hdf5 swig pylint lxml
\end{lstlisting}

\paragraph{Optional LateX installation}
Optionally, if you want to be able to edit and rebuild the manual, you can
install \TeX~Live and its related packages:
\begin{lstlisting}[language=bash]
  sudo apt-get install texlive-latex-base \
  texlive-extra-utils texlive-latex-extra texlive-math-extra
\end{lstlisting}

\goToRavenInstallation

\subsubsection{Fedora}

\paragraph{Red Hat Package Manager (RPM)}
The Fedora distribution of Linux makes use of the ARed Hat Package Manager (RPM)
to automate the installation of pre-configured software. The
following command uses the RPM to add the needed packages and any
needed dependencies not already on the system:

\begin{lstlisting}[language=bash]
dnf install swig libtool gcc-c++ redhat-rpm-config python-devel \
  python3-devel numpy h5py scipy python-scikit-learn \
  python-matplotlib-qt4
\end{lstlisting}

Note: The 'dnf' command replaces 'yum' used on older versions of
Fedora Linux.

\paragraph{Miniconda}
The Minoconda package manager is a cross platform installation package specifically 
used for Python dependencies installation.
The package manager can be downloaded and installed from \url{https://conda.io/miniconda.html}.
After the installation of Minconda, the following command needs to be executed for the installation of
the needed packages and any needed dependencies not already on the system:

\begin{lstlisting}[language=bash]
 conda create --name raven_libraries -y numpy=1.11.0 \
  h5py=2.6.0 scipy=0.17.1 scikit-learn=0.17.1 \
  matplotlib=1.5.1 python=2.7 hdf5 swig pylint lxml
\end{lstlisting}

\paragraph{Optional LateX installation}
In addition, if you would like to be able to build the documentation
included in the RAVEN software distribution it is necessary to install
\TeX~Live and its related packages:
\begin{lstlisting}[language=bash]
dnf install texlive texlive-subfigure texlive-stmaryrd \
  texlive-titlesec texlive-preprint texlive-placeins \
  texlive-bigints texlive-relsize texlive-appendix
\end{lstlisting}

\goToRavenInstallation

\subsection{Preparing an Apple Macintosh OSX System for RAVEN}
\label{sysprep_osx}

When using an Apple Macintosh computer, the above dependencies are met
by following three steps: Installing the XCode command line tools from Apple,
installing the XQuartz  X-Window system server, and then installing and using the Miniconda
package system to add the rest of the dependencies.

\subsubsection{Installing XCode Command Line Tools}

The XCode command line tools package from Apple Computer provides the C++
compilers and git source code control tools needed to obtain and build RAVEN.
It is freely available from the Apple store. In order to obtain it the following command should be launched in an open terminal:
\begin{lstlisting}[language=bash]
 xcode-select --install
\end{lstlisting}

\subsubsection{Installing XQuartz}
XQuartz is an implementation of the X Server for the Mac OSX operating system.
XQuartz is freely available on the web and can be downloaded from the link 
 \url{https://dl.bintray.com/xquartz/downloads/XQuartz-2.7.9.dmg}.
 \\After downloaded, install the package.


\subsubsection{Install RAVEN libraries}
\label{miniconda}

For OSX, the simplest and most robust way to install the RAVEN
dependencies is with Miniconda.  Miniconda is a package manager for
python packages and can be used to install the third party packages
that RAVEN depends on. Minconda can be downloaded and installed from
the following link \url{https://conda.io/miniconda.html}.
\\Once installed, open a Terminal and launch the following command:
\begin{lstlisting}[language=bash]
conda create --name raven_libraries -y numpy=1.11.0 \
 h5py=2.6.0 scipy=0.17.1 scikit-learn=0.17.1 \
 matplotlib=1.5.1 python=2.7 hdf5 swig pylint lxml
\end{lstlisting}

This command will install all the libraries and dependencies needed for executing RAVEN 
in a Miniconda enviroment called ``raven\_libraries''.

\goToRavenInstallation


\subsection{Preparing a Windows System for RAVEN}
\label{sysprep_windows}

Since RAVEN requires a UNIX-like shell to function, one must be installed for a Microsoft
Windows system to run it.  A freely available software package called MSYS2 is used to
provide this functionality.  More information about MSYS2 is available at
\url{https://sourceforge.net/p/msys2/wiki/MSYS2%20introduction/}.

\subsubsection{Prerequisites to use RAVEN on Windows}
\begin{itemize}
    \item A system running a 64-bit version of Microsoft Windows. Installation and operation
        has been verified on Windows 7, 10, and Windows Server 2012 R2 Standard. While there
        is also a 32-bit version of MSYS2 available, the RAVEN installation described here will not work with it.
    \item At least 9 Gigabytes of available disk space:
    \begin{itemize}
        \item 0.5 GB for MSYS2, including supporting tools and git source code control
        \item 1.5 GB for Python language and supporting packages
        \item 1.5 GB for RAVEN and the MOOSE framework
        \item 5.0 GB for the Visual Studio compiler needed to build RAVEN
    \end{itemize}
\end{itemize}

\subsubsection{Installation and Configuration of the MSYS2 environment}
\begin{enumerate}
    \item Obtain and run the latest basic 64-bit MSYS2 installer from \url{ https://msys2.github.io/} (As of this writing it is named
	msys2-x86\_64-20161025.exe and is approximately 67 Megabytes in size).
    \item The page with the download also contains installation instructions. Perform the steps described there up to
	step 6 to install a minimal MSYS2 system and bring it up to date. Make sure that you install to path 
        C:\textbackslash{}msys64.  This installation will create shortcuts in the Windows start menu that may be used 
        to start UNIX-Like shells:
		\begin{itemize}
	    		\item MSYS2 Shell
	    		\item MinGW-w64 Win32 Shell
	    		\item MinGW-w64 Win64 Shell
		\end{itemize}
        When working with RAVEN, it is recommended to use "MinGW-w64 Win64 Shell", although any of them should work.
    \item Use the MSYS2 package manager {\it pacman} to install a few tools that will be needed later.  Enter the following command in an MSYS shell window:

\begin{lstlisting}[language=bash]
USER@HOSTNAME MINGW64 ~
$ pacman -S git winpty make
\end{lstlisting}
	The package manager will then download and install those packages (and their dependencies) from the MSYS2 
	repository.
\end{enumerate}

\subsubsection{Install Python Language and Package Support}
\begin{enumerate}
	\item Download the latest 64-bit installer for Windows Python 2.7 from 
		\url{https://conda.io/miniconda.html} and install it.  \item The installer 
		will ask whether Python should be installed for only the logged in user or
		for all users.  Either option will work for RAVEN.
	\item Locate and test the Python installation.   Open a Windows command prompt and enter the 
		command "{\it where python}", which attempts to locate a the Python language interpreter 
		in the current system path.  This looks like:

\begin{lstlisting}[language=bash, basicstyle=\small]
C:\Users\USERID> where python
C:\Users\USERID\AppData\Local\Continuum\Miniconda2\python.exe
\end{lstlisting}

	\item Setup MSYS2 to find Python.  MSYS2 has its own separate PATH which must also be adjusted 
		so that Python and its associated tools may be found. This is done by converting the 
		file system location of Python determined in the previous step to its MSYS2-compatible 
		equivalent and using the result to setup MSYS2 so that it too can find it in the future.
		\newline \newline
		This is done by turning all backslashes ('\textbackslash') in the path to be converted to 
		forward slashes ('/'), and changing the drive letter from its '\textless letter\textgreater:' 
		form to '/ \textless letter\textgreater'. In addition, any spaces in the path must 
		be escaped using a backslash ('\textbackslash') when converted.
		\newline \newline
		For example:

\begin{lstlisting}[language=bash]
C:\Users\USERID\AppData\Local\Continuum\Miniconda2
\end{lstlisting}
		becomes
\begin{lstlisting}[language=bash]
/c/Users/USERID/AppData/Local/Continuum/Miniconda2
\end{lstlisting}
		for MSYS2. Here is an example with spaces that need to be escaped:
\begin{lstlisting}[language=bash]
C:\Program Files\Common Files
\end{lstlisting}
		converted to MSYS2 form would become
\begin{lstlisting}[language=bash]
/c/Program\ Files/Common\ Files
\end{lstlisting}
		\medskip
		Three separate paths must be added to MSYS2 to enable all of the Python tools needed 
		to be found.  These are:

		\smallskip
\begin{tabular}{| l | l |}
	\hline
	{\bf Path} & {\bf Purpose} \\\hline			
	\textless Converted path from above\textgreater	& Python executable \\\hline
	\textless Converted path from above\textgreater /Scripts &	Conda (Needed to manage Python packages) \\\hline
	\textless Converted path from above\textgreater /Library/bin &	Swig (Needed to build RAVEN) \\\hline
\end{tabular}

		\medskip
		These paths are added using shell commands that append new entries to the existing PATH
		variable without overwriting it.  These commands take the following form:

\begin{lstlisting}[language=bash, basicstyle=\tiny]
export PATH=/c/Users/USERID/AppData/Local/Continuum/Miniconda2:$PATH
export PATH=/c/Users/USERID/AppData/Local/Continuum/Miniconda2/Scripts:$PATH
export PATH=/c/Users/USERID/AppData/Local/Continuum/Miniconda2/Library/bin:$PATH
\end{lstlisting}
		
		To configure these needed paths in MSYS2 so that they persist, file "~/.bashrc" will need
		to be edited.  This may be done either using an MSYS2-based editor such as {\it vim} 
		(VI-iMproved, which is included in the installation) or a Windows-based editor like 
		{\it Wordpad} (included with Windows).  Another excellent open source editor for 
		Windows is {\it Notepad++} \url{https://notepad-plus-plus.org/}, which is also good
		for editing RAVEN input files.

	\item Test Python in MSYS2.  At this point open a new MSYS2 shell window and see if Python is 
		now found in the PATH:
		\newline 
		Note: Due to the way that Python interacts with the MSYS2 shell, when using Python by 
		itself in MSYS2 the {\it winpty} utility is provided. (If Python is run without winpty, 
		it may appear to sit there and do nothing. Pressing \textless Ctrl\textgreater -C will 
		interrupt it.)


\begin{lstlisting}[language=bash, basicstyle=\tiny]
USER@HOSTNAME MINGW64 ~
$ winpty python
Python 2.7.13 |Anaconda 4.0.0 (64-bit)| (default, Dec 19 2016, 13:29:36) [MSC v.1500 64 bit (AMD64)] on win32
Type "help", "copyright", "credits" or "license" for more information.
Anaconda is brought to you by Continuum Analytics.
Please check out: http://continuum.io/thanks and https://anaconda.org
>>>
>>> quit()

\end{lstlisting}

	\item Install needed Python packages.  RAVEN requires several Python packages to function properly. 
		Now the {\it conda} command will be used to download and install them in an automated manner. The 
		following asks {\it conda} to obtain the specified versions of the listed packages, as well as all 
		of their dependencies.
		\smallskip

\begin{lstlisting}[language=bash]
conda install numpy=1.11.0 h5py=2.6.0 scipy=0.17.1 \
	scikit-learn=0.17.1 matplotlib=1.5.1 python=2.7 \
	hdf5 swig pylint lxml
\end{lstlisting}

\end{enumerate}

\subsubsection{Compiler Installation and Configuration}
\begin{enumerate}
	\item Download and install Visual Studio.  A C++ language compiler that supports C++11 features 
		is needed to perform this step. Microsoft's Visual Studio Community Edition is free and 
		available from \url{https://www.visualstudio.com/downloads/}.  

		The current version (as of this writing) is 2017. The 2015 and 2017 versions have been 
		successfully used to build RAVEN. Professional and Enterprise versions of these will 
		also work. If one of these is already present on your system, it is not necessary to 
		obtain another one. Note that because C++11 language features are required, the 
		"Microsoft Visual C++ Compiler for Python 2.7" often used for building Python
		add-ons will {\bf not} work.

		After downloading and running the Visual Studio installer, it will ask what features 
		to install. For building RAVEN, "Desktop development with C++" is needed at a minimum. 
		Installation of other Visual Studio features should be fine.

	\item Let the build system know where to find the compiler.  When the build system attempts 
		to search for an installed compiler, this process often fails with the error message 
		"Unable to find vcvarsall.bat".  This happens because Python version 2.7 has not been 
		updated to automatically locate modern Visual Studio installations. To solve this it 
		is necessary to help the Python build system find the C++ compiler on the system. 
		The easiest way to do this is create a Windows batch (.BAT) file that will redirect 
		the build system to the information it needs. First, locate the file VCVARSALL.BAT 
		file installed as part of Visual Studio on your system. This location will usually 
		be something like the following: 
		\smallskip

\begin{tabular}{| l | l |}
	\hline
	{\bf Visual Studio Version} & {\bf Directory containing VCVARSALL.BAT} \\\hline			
	2015 & C:\textbackslash Program Files (x86)\textbackslash Microsoft Visual Studio 14.0\textbackslash VC \\\hline
	2017 & C:\textbackslash Program Files (x86)\textbackslash Microsoft Visual \\
               & Studio\textbackslash 2017\textbackslash Community\textbackslash VC\textbackslash
                Auxiliary\textbackslash Build \\\hline
\end{tabular}

		\medskip
		Once the target file has been located it is necessary to create a couple of directories 
		and one file. The first directory created must be named "VC" and should be created 
		somewhere outside of the RAVEN source tree (such as your MinGW home directory):

\begin{lstlisting}[language=bash]
USER@HOSTNAME MINGW64 ~
$ mkdir VC
\end{lstlisting}

		The next directory to be created must be inside the one just created. It is suggested 
		to name it "target", because it is there that we will point the Python build system:

\begin{lstlisting}[language=bash]
USER@HOSTNAME MINGW64 ~
$ cd VC

USER@HOSTNAME MINGW64 ~/VC
$ mkdir target

USER@HOSTNAME MINGW64 ~/VC
$ ls
target

\end{lstlisting}

		The file to be created is named "VCVARSALL.BAT", and it must be written in the VC 
		directory that was just made. The Python build system will be configured to find this 
		file, which then redirects it to the actual file. Use a text editor (such as 
		{\it vim} or {\it notepad} as described above) to create the file VCVARSALL.BAT:

\begin{lstlisting}[language=bash]
USER@HOSTNAME MINGW64 ~/VC
$ vim VCVARSALL.BAT
\end{lstlisting}

		or

\begin{lstlisting}[language=bash]
USER@HOSTNAME MINGW64 ~/VC
$ notepad VCVARSALL.BAT
\end{lstlisting}

		One line will need to be added to the new file VCVARSALL.BAT:

\begin{lstlisting}[language=bash, basicstyle=\tiny]
CALL "<Full Path To VCVARSALL.BAT File Installed by Visual Studio>" %1 %2 %3 %4 %5
\end{lstlisting}

		For example, in the case of Visual Studio 2017 Community installed in the default 
		location this would be:

\begin{lstlisting}[language=bash, basicstyle=\tiny]
CALL "C:\Program Files (x86)\Microsoft Visual Studio\2017\Community\VC\Auxiliary\Build\VCVARSALL.BAT" %1 %2 %3 %4 %5
\end{lstlisting}

		Note the double quotes around the path and file name. These are necessary because 
		there are spaces in some of the directory names that make up the full location of 
		VCVARSALL.BAT.

		After creating the new {\it VCVARSALL.BAT} in the directory {\it VC}, one more thing 
		needs to be done to inform the Python build system where this file just created is. 
		During the build process, an {\it environment variable} "VS90COMNTOOLS" will be checked. 
		The value of VS90COMNTOOLS will need to be set to the {\it target} directory just below 
		the location of VCVARSALL.BAT file just created. 

		For example, if VCVARSALL.BAT was created in directory VC under your MinGW home 
		directory, the variable VS90COMNTOOLS should point to \textasciitilde /VC/target.

\begin{lstlisting}[language=bash]
USER@HOSTNAME MINGW64 ~
$ export VS90COMNTOOLS=~/VC/target

USER@HOSTNAME MINGW64 ~
$ echo $VS90COMNTOOLS
/home/user/VC/target
\end{lstlisting}

\end{enumerate}

\subsubsection{For More Information}
Note: An illustrated version of this procedure may be found on the Wiki page at
\url{https://github.com/idaholab/raven/wiki/installationWindows}.

\subsection{Manual Dependency Install}
\label{sysprep_manual}

If other options don't work, the dependencies can be installed
manually.  This is sometimes tricky, pay attention to the order you
install them.  RAVEN uses the following packages (newer versions
usually work):

\begin{enumerate}
  \input{libraries.tex}
\item hdf5 1.8 or newer
\item swig 2.0 or newer
\item gcc 4.9 or newer (or other C++11 compiler)
\item TeX Live 2016 or newer (to build manuals)
\end{enumerate}

\goToRavenInstallation

\subsection{How RAVEN finds Dependencies}

RAVEN, when run from either \texttt{raven\_framework} or in 
\texttt{run\_tests} runs a script called\\ \texttt{setup\_raven\_libs}
which sets the up the dependencies if they are not already present.

If conda is available, it will try and activate a conda environment
called \texttt{raven\_libraries}

Otherwise, the \texttt{setup\_raven\_libs} bash script will try and
source a script:\\
\verb'$HOME/.raven/environments/raven_libs_profile'
or if that is not available:\\
\verb'/opt/raven_libs/environments/raven_libs_profile' These scripts
can be created if alternative ways of finding the dependencies are
needed.

\section{RAVEN Installation}
\label{raven_installation}

Once the RAVEN dependencies have been installed (See Section
\ref{raven_dependencies}), the rest of RAVEN can be installed.

There are two different ways to get RAVEN.  There are trade offs
between how easy it is to setup and how easy it is to develop with the
method.  If you want to do development on RAVEN or keep up with RAVEN
changes go to Section \ref{submodule_git}, which describes obtaining
the software directly from the online repository.  Otherwise, RAVEN
may be installed from a stand alone source code package (Section
\ref{raven_source_package}).


\subsection{Submodule Git}
\label{submodule_git}

This install method uses git to obtain the software and uses
submodules to get MOOSE.  This can be used for RAVEN development.

First RAVEN needs to be cloned, and then the submodules initialized.

\begin{lstlisting}[language=bash]
git clone https://github.com/idaholab/raven.git
cd raven
git submodule update --init moose
\end{lstlisting}

Next follow the compilation and testing instructions in Section \ref{raven_compilation}.

To update the software, the git pull and submodule update commands can
be used:

\begin{lstlisting}[language=bash]
git pull
git submodule update
\end{lstlisting}


\subsection{RAVEN Source Code Package}
\label{raven_source_package}

Untar the source install (if there is more than one version of the
source tarball, the full filename will need to be used instead of *):

\begin{lstlisting}[language=bash]
tar -xvzf raven_framework_*_source.tar.gz
cd raven
\end{lstlisting}

Next follow the compilation and testing instructions in Section
\ref{raven_compilation}.

\subsection{RAVEN Compilation}
\label{raven_compilation}

The RAVEN modules should be compiled:

\begin{lstlisting}[language=bash]
#change into the raven directory if needed.
make
\end{lstlisting}

Then the testing should be done:

\begin{lstlisting}[language=bash]
./run_tests
\end{lstlisting}

The output should describe why any tests failed.

At the end, there should be a line that looks similar to the output below:
\begin{lstlisting}[language=bash]
8 passed, 19 skipped, 0 pending, 0 failed
\end{lstlisting}

Normally there are skipped tests because either some of the codes are
not available, or some of the test are not currently working.  The
output will explain why each is skipped.

If all the tests pass, you are ready to read about Running RAVEN in
Section \ref{HowToRun}.

If the tests did not pass, check Section
\ref{troubleshooting_installation} on troubleshooting.

\subsection{Troubleshooting the Installation}
\label{troubleshooting_installation}

Often the problems result from one or more of the libraries being
incorrect or missing.  In the raven directory, the command:

\begin{lstlisting}[language=bash]
./run_tests --library_report
\end{lstlisting}
can be used to check if all the libraries are available, and which
ones are being used.  If amsc, distribution1D or interpolationND are
missing, then the RAVEN modules need to be compiled or recompiled.
Otherwise, the RAVEN dependencies need to be fixed.

Note, that when using RAVEN remotely in a graphical session with X11
forwarded to the client, some tests may depend on live X11 forwarding
to the remote client. If the user is not using X11 forwarding then
RAVEN will work fine and not use X11.  However, when the user has
forwarded their X11 environment to a ssh client, the connection may
timeout. The standard timeout of X for an untrusted connection is 20
minutes.  The full test suite including those involving graphical
output can take longer than the aforementioned timeout. One way to
alleviate this is to login to the remote host of RAVEN using a trusted
connection by using the -Y flag:

\begin{lstlisting}[language=bash]
ssh -Y hostname
\end{lstlisting}

This should only be done when the user is using a secure connection to
a known host though, as there are security concerns to the client
machine when allowing a remote computer access to its graphical user
interface.


\subsection{In-use Testing}

In use testing can be done by re-running the installation tests as
described in Section \ref{raven_compilation}.
