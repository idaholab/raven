\subsection{Microsoft Windows}
\label{sec:install windows}

The process of establishing the required environment for Windows is notably more involved than the other two
systems; however, it is straightforward.  First, RAVEN has the following prerequisites on Windows:

\begin{itemize}
    \item A system running a 64-bit version of Microsoft Windows. Installation and operation
        has been verified on Windows 7, 10, and Windows Server 2012 R2 Standard.
    \item At least 9 Gigabytes of available disk space:
    \begin{itemize}
        \item 0.5 GB for MSYS2, including supporting tools and git source code control
        \item 1.5 GB for Python language and supporting packages
        \item 1.5 GB for RAVEN and the MOOSE framework
        \item 5.0 GB for the Visual Studio compiler needed to build RAVEN
    \end{itemize}
\end{itemize}

\subsubsection{A Visual Guide}
Note: An illustrated version of this procedure may be found on the \wiki.

\subsubsection{MSYS2 environment}
Since RAVEN requires a UNIX-like shell to function, the freely-available software package called MSYS2 is
used to
provide this functionality.  More information about MSYS2 is available at
\url{https://sourceforge.net/p/msys2/wiki/MSYS2%20introduction/}.
\nb While there is also a 32-bit version of MSYS2 available, the RAVEN installation described here will not work with it.

\begin{enumerate}
    \item Obtain and run the latest basic 64-bit MSYS2 installer from \url{ https://msys2.github.io/} (As of this writing it is named
	msys2-x86\_64-20161025.exe and is approximately 67 Megabytes in size).
    \item The page with the download also contains installation instructions. Perform the steps described there up to
	step 6 to install a minimal MSYS2 system and bring it up to date. Make sure that you install to path
        C:\textbackslash{}msys64.  This installation will create shortcuts in the Windows start menu that may be used
        to start UNIX-Like shells:
		\begin{itemize}
	    		\item MSYS2 Shell
	    		\item MinGW-w64 Win32 Shell
	    		\item MinGW-w64 Win64 Shell
		\end{itemize}
        When working with RAVEN, it is recommended to use "MinGW-w64 Win64 Shell", although any of them should work.
    \item Use the MSYS2 package manager {\it pacman} to install a few tools that will be needed later.  Enter the following command in an MSYS shell window:

\begin{lstlisting}[language=bash]
USER@HOSTNAME MINGW64 ~
$ pacman -S git winpty make
\end{lstlisting}
	The package manager will then download and install those packages (and their dependencies) from the MSYS2
	repository.
\end{enumerate}

\subsubsection{Install Python Language and Package Support}
\begin{enumerate}
	\item Download the latest 64-bit installer for Windows Python 3 from
		\url{https://conda.io/miniconda.html} and install it.  \item The installer
		will ask whether Python should be installed for only the logged in user or
		for all users.  Either option will work for RAVEN.
	\item Locate and test the Python installation.   Open a Windows command prompt and enter the
		command "{\it where python}", which attempts to locate a the Python language interpreter
		in the current system path.  This looks like:

    \begin{lstlisting}[language=bash, basicstyle=\small]
    C:\Users\USERID> where python
    C:\Users\USERID\AppData\Local\Continuum\Miniconda3\python.exe
    \end{lstlisting}

	\item Setup MSYS2 to find Python.  MSYS2 has its own separate PATH which must also be adjusted
		so that Python and its associated tools may be found. This is done by converting the
		file system location of Python determined in the previous step to its MSYS2-compatible
		equivalent and using the result to setup MSYS2 so that it too can find it in the future.
		\newline \newline
		This is done by turning all backslashes ('\textbackslash') in the path to be converted to
		forward slashes ('/'), and changing the drive letter from its '\textless letter\textgreater:'
		form to '/ \textless letter\textgreater'. In addition, any spaces in the path must
		be escaped using a backslash ('\textbackslash') when converted.
		\newline \newline
		For example:

\begin{lstlisting}[language=bash]
C:\Users\USERID\AppData\Local\Continuum\Miniconda3
\end{lstlisting}
		becomes
\begin{lstlisting}[language=bash]
/c/Users/USERID/AppData/Local/Continuum/Miniconda3
\end{lstlisting}
		for MSYS2. Here is an example with spaces that need to be escaped:
\begin{lstlisting}[language=bash]
C:\Program Files\Common Files
\end{lstlisting}
		converted to MSYS2 form would become
\begin{lstlisting}[language=bash]
/c/Program\ Files/Common\ Files
\end{lstlisting}
		\medskip
		Three separate paths must be added to MSYS2 to enable all of the Python tools needed
		to be found.  These are:

		\smallskip
\begin{tabular}{| l | l |}
	\hline
	{\bf Path} & {\bf Purpose} \\\hline
	\textless Converted path from above\textgreater	& Python executable \\\hline
	\textless Converted path from above\textgreater /Scripts &	Conda (Needed to manage Python packages) \\\hline
	\textless Converted path from above\textgreater /Library/bin &	Swig (Needed to build RAVEN) \\\hline
\end{tabular}

		\medskip
		These paths are added using shell commands that append new entries to the existing PATH
		variable without overwriting it.  These commands take the following form:

\begin{lstlisting}[language=bash, basicstyle=\tiny]
export PATH=/c/Users/USERID/AppData/Local/Continuum/Miniconda3:$PATH
export PATH=/c/Users/USERID/AppData/Local/Continuum/Miniconda3/Scripts:$PATH
export PATH=/c/Users/USERID/AppData/Local/Continuum/Miniconda3/Library/bin:$PATH
\end{lstlisting}

		To configure these needed paths in MSYS2 so that they persist, file "~/.bashrc" will need
		to be edited.  This may be done either using an MSYS2-based editor such as {\it vim}
		(VI-iMproved, which is included in the installation) or a Windows-based editor like
		{\it Wordpad} (included with Windows).  Another excellent open source editor for
		Windows is {\it Notepad++} \url{https://notepad-plus-plus.org/}, which is also good
		for editing RAVEN input files.

	\item Test Python in MSYS2.  At this point open a new MSYS2 shell window and see if Python is
		now found in the PATH:
		\newline
		Note: Due to the way that Python interacts with the MSYS2 shell, when using Python by
		itself in MSYS2 the {\it winpty} utility is provided. (If Python is run without winpty,
		it may appear to sit there and do nothing. Pressing \textless Ctrl\textgreater -C will
		interrupt it.)


\begin{lstlisting}[language=bash, basicstyle=\tiny]
USER@HOSTNAME MINGW64 ~
$ winpty python
Python 3.6.7 |Anaconda 4.0.0 (64-bit)| (default, Dec 19 2016, 13:29:36) [MSC v.1500 64 bit (AMD64)] on win32
Type "help", "copyright", "credits" or "license" for more information.
Anaconda is brought to you by Continuum Analytics.
Please check out: http://continuum.io/thanks and https://anaconda.org
>>>
>>> quit()

\end{lstlisting}

% Windows should use establish_conda_env.sh like everyone else
%	\item Install needed Python packages.  RAVEN requires several Python packages to function properly.
%		Now the {\it conda} command will be used to download and install them in an automated manner. The
%		following asks {\it conda} to obtain the specified versions of the listed packages, as well as all
%		of their dependencies.
%		\smallskip
%
%    \begin{lstlisting}[language=bash]
%    conda install numpy=1.11.0 h5py=2.6.0 scipy=0.17.1 \
%      scikit-learn=0.17.1 matplotlib=1.5.1 python=3 \
%      hdf5 swig pylint lxml
%    \end{lstlisting}

\end{enumerate}

\subsubsection{Compiler Installation and Configuration}
\begin{enumerate}
	\item Download and install Visual Studio.  A C++ language compiler that supports C++11 features
		is needed to perform this step. Microsoft's Visual Studio Community Edition is free and
		available from \url{https://www.visualstudio.com/downloads/}.

		The current version (as of this writing) is 2017. The 2015 and 2017 versions have been
		successfully used to build RAVEN. Professional and Enterprise versions of these will
		also work. If one of these is already present on your system, it is not necessary to
		obtain another one. Note that because C++11 language features are required, the
		"Microsoft Visual C++ Compiler for Python 2.7" often used for building Python
		add-ons will {\bf not} work.

		After downloading and running the Visual Studio installer, it will ask what features
		to install. For building RAVEN, "Desktop development with C++" is needed at a minimum.
		Installation of other Visual Studio features should be fine.

	\item Let the build system know where to find the compiler.  When the build system attempts
		to search for an installed compiler, this process often fails with the error message
		"Unable to find vcvarsall.bat".  This happens because Python version 2.7 has not been
		updated to automatically locate modern Visual Studio installations. To solve this it
		is necessary to help the Python build system find the C++ compiler on the system.
		The easiest way to do this is create a Windows batch (.BAT) file that will redirect
		the build system to the information it needs. First, locate the file VCVARSALL.BAT
		file installed as part of Visual Studio on your system. This location will usually
		be something like the following:
		\smallskip

\begin{tabular}{| l | l |}
	\hline
	{\bf Visual Studio Version} & {\bf Directory containing VCVARSALL.BAT} \\\hline
	2015 & C:\textbackslash Program Files (x86)\textbackslash Microsoft Visual Studio 14.0\textbackslash VC \\\hline
	2017 & C:\textbackslash Program Files (x86)\textbackslash Microsoft Visual \\
               & Studio\textbackslash 2017\textbackslash Community\textbackslash VC\textbackslash
                Auxiliary\textbackslash Build \\\hline
\end{tabular}

		\medskip
		Once the target file has been located it is necessary to create a couple of directories
		and one file. The first directory created must be named "VC" and should be created
		somewhere outside of the RAVEN source tree (such as your MinGW home directory):

\begin{lstlisting}[language=bash]
USER@HOSTNAME MINGW64 ~
$ mkdir VC
\end{lstlisting}

		The next directory to be created must be inside the one just created. It is suggested
		to name it "target", because it is there that we will point the Python build system:

\begin{lstlisting}[language=bash]
USER@HOSTNAME MINGW64 ~
$ cd VC

USER@HOSTNAME MINGW64 ~/VC
$ mkdir target

USER@HOSTNAME MINGW64 ~/VC
$ ls
target

\end{lstlisting}

		The file to be created is named "VCVARSALL.BAT", and it must be written in the VC
		directory that was just made. The Python build system will be configured to find this
		file, which then redirects it to the actual file. Use a text editor (such as
		{\it vim} or {\it notepad} as described above) to create the file VCVARSALL.BAT:

\begin{lstlisting}[language=bash]
USER@HOSTNAME MINGW64 ~/VC
$ vim VCVARSALL.BAT
\end{lstlisting}

		or

\begin{lstlisting}[language=bash]
USER@HOSTNAME MINGW64 ~/VC
$ notepad VCVARSALL.BAT
\end{lstlisting}

		One line will need to be added to the new file VCVARSALL.BAT:

\begin{lstlisting}[language=bash, basicstyle=\tiny]
CALL "<Full Path To VCVARSALL.BAT File Installed by Visual Studio>" %1 %2 %3 %4 %5
\end{lstlisting}

		For example, in the case of Visual Studio 2017 Community installed in the default
		location this would be:

\begin{lstlisting}[language=bash, basicstyle=\tiny]
CALL "C:\Program Files (x86)\Microsoft Visual Studio\2017\Community\VC\Auxiliary\Build\VCVARSALL.BAT" %1 %2 %3 %4 %5
\end{lstlisting}

		Note the double quotes around the path and file name. These are necessary because
		there are spaces in some of the directory names that make up the full location of
		VCVARSALL.BAT.

		After creating the new {\it VCVARSALL.BAT} in the directory {\it VC}, one more thing
		needs to be done to inform the Python build system where this file just created is.
		During the build process, an {\it environment variable} "VS90COMNTOOLS" will be checked.
		The value of VS90COMNTOOLS will need to be set to the {\it target} directory just below
		the location of VCVARSALL.BAT file just created.

		For example, if VCVARSALL.BAT was created in directory VC under your MinGW home
		directory, the variable VS90COMNTOOLS should point to \textasciitilde /VC/target.

\begin{lstlisting}[language=bash]
USER@HOSTNAME MINGW64 ~
$ export VS90COMNTOOLS=~/VC/target

USER@HOSTNAME MINGW64 ~
$ echo $VS90COMNTOOLS
/home/user/VC/target
\end{lstlisting}

\end{enumerate}



Once the compiler installation and configuration is complete, you are prepared to install the RAVEN libraries
(see section \ref{sec:install conda}).


