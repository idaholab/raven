\subsection{Linux Ubuntu Installation}
The following instructions are for installing RAVEN on a Linux machine running Ubuntu 16.04 or greater.  Some
explanations of alternatives for other Linux distributions may be provided on the \wiki.

To install the prerequisite packages, the following terminal command should be executed (note this requires
administrative privileges):

\begin{lstlisting}[language=bash]
 sudo apt-get install libtool git python-dev swig g++ \
 python3-dev python-numpy python-sklearn python-h5py
\end{lstlisting}



%%%%%% OLD %%%%%%
\subsection{Preparing a Linux System for RAVEN}
\label{sysprep_linux}

\subsubsection{Ubuntu}

\paragraph{Miniconda}
The Minoconda package manager is a cross platform installation package specifically
used for Python dependencies installation, and is the RAVEN-preferred way to keep up to date on
Python libraries..
The package manager can be downloaded and installed from \url{https://conda.io/miniconda.html}.
After the installation of Minconda, the installation of the RAVEN-dependent libraries needs to
be performed.  For this task, we've prepared an installation process at:
\begin{lstlisting}[language=bash]
  raven/scripts/establish_conda_env.sh
\end{lstlisting}
Navigating to that folder and running that process
in a \texttt{bash} shell will set up a special environment for RAVEN libraries and install the
dependencies.

In the event an alternative to the installation process in \texttt{establish\_conda\_env.sh} is needed,
the conda command that needs to be executed can be generated by navigating to
\texttt{raven/scripts/TestHarness/testers/} and running the following command:
\begin{lstlisting}[language=bash]
  python RavenUtils.py --conda-install
\end{lstlisting}
If the \texttt{raven\_libraries} environment already exists, then instead the conda command
needed is printed by
\begin{lstlisting}[language=bash]
  python RavenUtils.py --conda-update
\end{lstlisting}
In either case, the screen output can be copied and pasted as a shell command.
% XXX WORKING

%the following command needs to be executed for the installation of
%the needed packages and any needed dependencies not already on the system:
%\begin{lstlisting}[language=bash]
% conda create --name raven_libraries -y numpy=1.11.0 \
%  h5py=2.6.0 scipy=0.17.1 scikit-learn=0.17.1 \
%  matplotlib=1.5.1 python=2.7 hdf5 swig pylint lxml
%\end{lstlisting}

\paragraph{Advanced Package Tool (APT)}
The Ubuntu distribution of Linux makes use of the Advanced Package
Tool (APT) to automate the installation of pre-configured software.
Ubuntu 16.4 or newer provide all the packages needed for RAVEN.


The
following command will list all the required packages along with versions
that are the standard for RAVEN.

First, navigate to \texttt{raven/scripts/TestHarness/testers/},
then run the python command
\begin{lstlisting}[language=bash]
  python RavenUtils.py --conda-update
\end{lstlisting}
Ignoring the \texttt{conda} part of the command, the remaining are the names of the libraries that
RAVEN is dependent on, along with testing versions if any.  In addition, RAVEN needs the following
tools installed:

\begin{lstlisting}[language=bash]
 sudo apt-get install libtool git python-dev swig g++ \
 python3-dev python-numpy python-sklearn python-h5py
\end{lstlisting}


\paragraph{Optional LateX installation}
Optionally, if you want to be able to edit and rebuild the manual, you can
install \TeX~Live and its related packages:
\begin{lstlisting}[language=bash]
  sudo apt-get install texlive-latex-base \
  texlive-extra-utils texlive-latex-extra texlive-math-extra
\end{lstlisting}

\goToRavenInstallation

\subsubsection{Fedora}

\paragraph{Red Hat Package Manager (RPM)}
The Fedora distribution of Linux makes use of the ARed Hat Package Manager (RPM)
to automate the installation of pre-configured software. The
following command uses the RPM to add the needed packages and any
needed dependencies not already on the system:

\begin{lstlisting}[language=bash]
dnf install swig libtool gcc-c++ redhat-rpm-config python-devel \
  python3-devel numpy h5py scipy python-scikit-learn \
  python-matplotlib-qt4
\end{lstlisting}

Note: The 'dnf' command replaces 'yum' used on older versions of
Fedora Linux.

\paragraph{Miniconda}
The Minoconda package manager is a cross platform installation package specifically
used for Python dependencies installation.
The package manager can be downloaded and installed from \url{https://conda.io/miniconda.html}.
After the installation of Minconda, the following command needs to be executed for the installation of
the needed packages and any needed dependencies not already on the system:

\begin{lstlisting}[language=bash]
 conda create --name raven_libraries -y numpy=1.11.0 \
  h5py=2.6.0 scipy=0.17.1 scikit-learn=0.17.1 \
  matplotlib=1.5.1 python=2.7 hdf5 swig pylint lxml
\end{lstlisting}

\paragraph{Optional LateX installation}
In addition, if you would like to be able to build the documentation
included in the RAVEN software distribution it is necessary to install
\TeX~Live and its related packages:
\begin{lstlisting}[language=bash]
dnf install texlive texlive-subfigure texlive-stmaryrd \
  texlive-titlesec texlive-preprint texlive-placeins \
  texlive-bigints texlive-relsize texlive-appendix
\end{lstlisting}

\goToRavenInstallation

