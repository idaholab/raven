\subsection{Installing RAVEN}
\label{sec:clone raven}

Once the RAVEN dependencies have been installed  and conda is present
(see section \ref{sec:install overview}), the rest of RAVEN can be installed.

The installation of RAVEN involves the following steps:
\begin{itemize}
  \item Obtain the source code,
  \item Install the prerequisite Python libraries using conda,
  \item Compile
\end{itemize}




\subsubsection{Obtaining RAVEN Source Code}
RAVEN is hosted publicly as a \texttt{Git} repo on \texttt{GitHub}
and can be viewed at \url{https://github.com/idaholab/raven/wiki}.
In the event that access to \texttt{GitHub} is impossible, contact the user list and other arrangements may be
possible.  In general, however, using the git repository assures the most consistent usage and update process.

To clone RAVEN, navigate in a terminal to the desired destination, for example \texttt{~/projects}.  Then run
the commands
\begin{lstlisting}[language=bash]
git clone https://github.com/idaholab/raven.git
cd raven
git submodule update --init
\end{lstlisting}
This will obtain RAVEN as well as other submodules that RAVEN uses.  In the future, whenever we declare a path
starting with \texttt{raven/}, we refer to the cloned directory.




\subsubsection{Installing Python Libraries}
RAVEN depends heavily on Python, and uses conda to maintain a separate environment to prevent conflicts with
other Python library installations.  This separate environment is called \texttt{raven\_libraries}.

In order to establish this environment, navigate to \texttt{raven}, then
\begin{itemize}

  \item \textbf{Any unix-based systems (e.g. Macintosh, Linux, etc.)}:
\begin{lstlisting}[language=bash]
cd scripts
./establish_conda_env.sh --install
\end{lstlisting}
  \item \textbf{Windows}:
  \begin{lstlisting}[language=bash]
cd scripts
bash.exe establish_conda_env.sh --install
\end{lstlisting}
  
\end{itemize}
Assure that there are no errors in this process, then continue to compiling RAVEN.

\nb If \texttt{conda} is not installed in the default location, then the path to the conda definitions
needs to be provided, for example

\begin{itemize}

  \item \textbf{Any unix-based systems (e.g. Macintosh, Linux, etc.)}:
\begin{lstlisting}[language=bash]
cd scripts
./establish_conda_env.sh --install
   --conda-defs /path/to/miniconda3/etc/profile.d/conda.sh
\end{lstlisting}
  \item \textbf{Windows}:
  \begin{lstlisting}[language=bash]
cd scripts
bash.exe establish_conda_env.sh --install
  --conda-defs \path/\to\miniconda3\etc\profile.d\conda.sh
\end{lstlisting}
  
\end{itemize}

replacing \texttt{/path/to} with the install path for \texttt{conda}.

\nb Various options exist for \texttt{establish\_conda\_env.sh}, which
can be found by using the \texttt{--help} option.  These options
include \texttt{--mamba} which uses the mamba instead of conda for
resolving dependencies, \texttt{--load} which can be used with
\verb'source ./scripts/establish_conda_env.sh --load' to switch to the
raven environment in a shell, \texttt{--installation-manager PIP} which
uses pip instead of conda.

\subsubsection{Compiling RAVEN}
Once Python libraries are established and the source code present, navigate to \texttt{raven} and run


\begin{itemize}

  \item \textbf{Any unix-based systems (e.g. Macintosh, Linux, etc.)}:
\begin{lstlisting}[language=bash]
./build_raven
\end{lstlisting}
  \item \textbf{Windows}:
  \begin{lstlisting}[language=bash]
bash.exe build_raven
\end{lstlisting}
  
\end{itemize}

This will compile several dependent libraries.  This step has the highest potential for revealing problems in
the operating system setup, particularly for Windows.  See troubleshooting on the \wiki for help sorting out
difficulties.


\subsubsection{Testing RAVEN}
\label{sec:testing raven}
To test the installation of RAVEN, navigate to \texttt{raven}, then run the command

\begin{itemize}

  \item \textbf{Any unix-based systems (e.g. Macintosh, Linux, etc.)}:
\begin{lstlisting}[language=bash]
../run_tests -j2
\end{lstlisting}
  \item \textbf{Windows}:
  \begin{lstlisting}[language=bash]
bash.exe ./run_tests -j2
\end{lstlisting}
  
\end{itemize}

where \texttt{-j2} signifies running with 2 processors.  If more processors are available, this can be
increased, but using all or more than all of the available processes can slow down the testing dramatically.
This command runs RAVEN's regression tests, analytic tests, and unit tests.  The number of tests changes
frequently as the code's needs change, and the time taken to run the tests depends strongly on the number of
processors and processor speed.

At the end of the tests, a number passed, skipped, and failing will be reported.  Having some skipped tests is
expected; RAVEN has many tests that apply only to particular configurations or codes that are not present on
all machines.  However, no tests should fail; if there are problems, consult the troubleshooting section on
the \wiki.


\subsubsection{Updating RAVEN}
RAVEN updates frequently, and new features are added while bugs are fixed on a regular basis.  To update
RAVEN, navigate to \texttt{raven}, then run the commands
\begin{itemize}

  \item \textbf{Any unix-based systems (e.g. Macintosh, Linux, etc.)}:
\begin{lstlisting}[language=bash]
git pull
./scripts/establish_conda_env.sh --install
./build_raven
\end{lstlisting}
  \item \textbf{Windows}:
  \begin{lstlisting}[language=bash]
git pull
bash.exe scripts/establish_conda_env.sh --install
bash.exe build_raven
\end{lstlisting}
  
\end{itemize}

\subsubsection{In-use Testing}
Whenever RAVEN is installed on a new computer or whenever there is a significant change to the operating system, 
in-use tests shall be conducted.
Acceptable performance of RAVEN shall be confirmed by running the installation tests as described in  \ref{sec:testing raven}.
