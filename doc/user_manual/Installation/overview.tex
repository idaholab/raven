\section{Installation Overview}

The installation of the RAVEN code is a straightforward procedure;
depending on the usage purpose and machine architecture, the
installation process slightly differs.

In the following sections, the recommended installation procedure is outlined.  For alternatives, we encourage
checking the \wiki.  The machines on which
RAVEN is tested and developed, however, use the standard installation procedures outlined below.

The standard installation procedure for RAVEN includes using Miniconda (often simply referred to as
\emph{conda}) to install the Python libraries required to run RAVEN.  If conda cannot be made available on an
operating system, refer to the wiki (listed above) for alternatives.  To install miniconda, follow the
instructions for your operating system at \url{https://conda.io/miniconda.html}.  Since RAVEN currently relies
on some Python 2.7-only libraries, make sure to install the 2.7 64-bit version of miniconda.
