\section{Samplers  \\ \vspace{2 mm} {\small }}

Samplers own the sampling strategy (Type) and they generate the input values using the associate distribution. They do not have distributions inside.

There are different kind of samplers:
\begin{itemize}
\item \textbf{MonteCarlo}
\item \textbf{Grid}
\item \textbf{LHS}
\item \textbf{Adaptive}
\item \textbf{DynamicEventTree}
\item \textbf{AdaptiveDynamicEventTree}
\end{itemize}

For MonteCarlo method: 
\begin{itemize}
\item \textbf{MonteCarlo}
\begin{itemize}
\item name: name of sampling method used
\item limit: number of samplings the MonteCarlo method will use
\item initial seed: (optional) initial number of the iterations
\end{itemize}
\item variable 
\begin{itemize}
\item name: name of the variable that the code will do the sampling of
\end{itemize}
\item distribution: 
\begin{itemize}
\item what kind of distribution the variable follows (it is chosen from the distribution block)
\end{itemize}
\end{itemize}

Example:

\begin{lstlisting}[style=XML]
<MonteCarlo name='MC_Sampler' limit='1000'> 
 <variable name='pressure'> 
  <distribution>***</distribution> 	
 </variable> 
</MonteCarlo> 
\end{lstlisting}
In the example the code is sampling 1000 times (limit) the variable "pressure" with a distribution taken from the distribution block.

For the Grid method:
\begin{itemize}
\item \textbf{Grid}
\begin{itemize}
\item name: name of the sampling method used
\item initial seed (optional)
\end{itemize}
\item variable
\begin{itemize}
\item name: name of the variable that the code will do the sampling of
\end{itemize}
\item distribution: what kind of distribution the variable follows (it is chosen from the distribution block) 
\item grid
\begin{itemize}
\item type
\begin{itemize}
\item value
\item CDF
\end{itemize}
\item construction 
\begin{itemize}
\item equal
\begin{itemize}
\item the size of the step, given in the input node, is the same for all steps
\item requires lowerbound or upperbound
\item requires steps
\end{itemize}
\item custom
\begin{itemize} 
\item no lowerbound or upperbound 
\item no number of steps
\item in the input node it is necessary to input each step of the grid separated by a space
\end{itemize}
\end{itemize}
\item lowerbound or upperbound
\item steps
\item input node
\end{itemize}
\end{itemize}

Example: 

\begin{lstlisting}[style=XML]
<Grid name='Grid_Sampler'> 
 <variable name='pressure'>  
  <distribution>***</distribution> 
  <grid	type='value' construction='equal' steps='100' lowerBound='1.0'>0.2</grid>   
 </variable> 
</Grid> 
\end{lstlisting}
In the example the code is sampling the variable "pressure" with a distribution "***" which was chosen by the distribution block. The grid ranges from 1.0 (lowerbound) to 21.0 (100 steps of equal size of 0.2).  
% % % % % % % % % % % % % % % % % % % % % % % % % % % % % % % % % %
\\
For the LHS method:
\begin{itemize}
\item \textbf{LHS}
\begin{itemize}
\item name: name of the sampling method used
\item initial seed (optional)
\end{itemize}
\item variable
\begin{itemize}
\item name: name of the variable that the code will do the sampling of
\end{itemize}
\item distribution: what kind of distribution the variable follows (it is chosen from the distribution block) 
\item grid
\begin{itemize}
\item type
\begin{itemize}
\item value
\item CDF
\end{itemize}
\item construction 
\begin{itemize}
\item equal
\begin{itemize}
\item the size of the step, given in the input node, is the same for all steps
\item requires lowerbound or upperbound
\item requires steps
\end{itemize}
\item custom
\begin{itemize} 
\item no lowerbound or upperbound 
\item no number of steps
\item in the input node it is necessary to input each step of the grid separated by a space
\end{itemize}
\end{itemize}
\item lowerbound or upperbound
\item steps
\item input node
\end{itemize}
\end{itemize}

Example:
\begin{lstlisting}[style=XML]
<LHS name='***' initial_seed='***'> 
 <variable name='***'>  
  <distribution>***</distribution> 	
  <grid	type='***' construction='***' steps='***' lowerBound='***'>****</grid>   
 </variable> 
</LHS> 
\end{lstlisting}
The input structure is the same as the \textbf{Grid} input structure

For the Adaptive method:
\begin{itemize}
\item \textbf{Adaptive}
\begin{itemize}
\item name
\item initial seed (optional)
\end{itemize}
\item Convergence
\begin{itemize}
\item limit: 'Integer'
\item persistence: 'Integer'
\item weight='probability' or 'None': 
\item subGridTol='None' or 'Float' :This is the tolerance used to construct the testing sub grid
\item forcelteration='True' or 'False': this flag control if at least a self.limit number of iteration should be done
\end{itemize}
\item variable
\item distribution
\end{itemize}

\begin{lstlisting}[style=XML]
<Adaptive name='***' initial_seed='***'> 
 <Convergence limit='***' persistence='***' weight='***' subGridTol='***' forcelteration='***'>***</Convergence>  
  <variable name='***'>
   <distribution>***</distribution>
  </variable> 
</Adaptive>   
\end{lstlisting}



For DynamicEventTree:
***

For AdaptiveDynamicEventTree:
***


























