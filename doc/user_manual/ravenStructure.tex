\section{Raven Input Structure}
The RAVEN code does not have a fixed calculation flow, since all of its basic
objects can be combined in order to create a user-defined calculation flow.
%
Thus, its input (xml) is organized in different xml blocks, each with a
different functionality.
%
The main input blocks are as follows:
\begin{itemize}
\item \textbf{Simulation}: The root node containing the entire input, all of
  the following blocks fit inside the \emph{Simulation} block.
\item \textbf{RunInfo}: Specifies the calculation settings (number of parallel
  simulations, etc.).
\item \textbf{Distributions}: Defines distributions needed for describing
  parameters, etc.
\item \textbf{Samplers}: Sets up the strategies used for exploring an uncertain
  domain.
\item \textbf{Functions}: Details interfaces to external user-defined functions
  and modules.
\item \textbf{Models}: Specifies codes, ROMs, post-processing analysis, etc.
  the user will be building and/or running.
\item \textbf{Steps}: Combines other blocks to detail a step in the RAVEN
  workflow including I/O and computations to be performed.
\item \textbf{Datas}: Specifies internal data objects used by RAVEN.
\item \textbf{Databases}: Lists the HDF5 databases used as input/output to a
  RAVEN run.
\item \textbf{OutStream system}: Visualization and Printing system block.
\end{itemize}

Each of these blocks are explained in dedicated sections in the following
chapters.
