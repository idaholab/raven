\subsubsection{FastFourierTransform}
\label{FastFourierTransformPP}
The \xmlNode{FastFourierTransform} PostProcessor provides access to the Numpy fast Fourier transform function
\texttt{numpy.fft.fft}
and provides the frequencies, periods, and amplitudes from performing the transform. The periods are simply
the inverse of the frequencies, and the frequency units are the deltas between pivot values in the provided
input. For example, if data is collected every 3600 seconds, the units of frequency are per-hour.  This
PostProcessor expects uniformly-spaced pivot values. Note that for each realization in the input data object,
a separate fft will be created for each target.

The \xmlNode{FastFourierTransform} PostProcessor can act on any target in a DataObject that depends on a
single index, and generates three histories per sample per target: an independent variable
\xmlString{target\_fft\_frequency}, and two dependent values \xmlString{target\_fft\_period} and
\xmlString{target\_fft\_amplitude}, which both depend on the frequency by default. In all three outputs,
\emph{target} is replaced by the name of the target for which the fft was requested.

\ppType{FastFourierTransform}{FastFourierTransform}
%
\begin{itemize}
  \item \xmlNode{target}, \xmlDesc{comma separated strings, required field}, specifies the names of the
    target(s) for which the fast Fourier transform should be calculated.
 \end{itemize}
\textbf{Example:}

\begin{lstlisting}[style=XML]
<Simulation>
 ...
  <Models>
    ...
    <PostProcessor name="pp" subType="FastFourierTransform">
      <target>x, y</target>
    </PostProcessor>
    ...
  </Models>
 ...
</Simulation>
\end{lstlisting}
