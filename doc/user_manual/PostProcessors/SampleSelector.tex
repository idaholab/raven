\subsubsection{SampleSelector}
\label{SampleSelectorPP}
The \xmlNode{SampleSelector} PostProcessor is a tool to select a row from a dataset, depending on different
criteria. The different criteria that can be used are listed below.

The \xmlNode{SampleSelector} PostProcessor can  act on any \xmlNode{DataObjects}, and generates a
\xmlNode{DataObject} with a single realization in return.

\ppType{SampleSelector}{SampleSelector}
%
\begin{itemize}
  \item \xmlNode{criterion}, \xmlDesc{string, required field}, specifies the criterion to select the
    realization from the input DataObject. Options are as follows:
    \begin{itemize}
      \item \xmlString{min}, choose the realization that has the lowest value of the \xmlNode{target}
        variable. The target must be scalar.
      \item \xmlString{max}, choose the realization that has the highest value of the \xmlNode{target}
        variable. The target must be scalar.
      \item \xmlString{index}, choose the realization that has the provided index. The index must be an
        integer and is zero-based, meaning the first entry is at index 0, the second entry is at index 1, etc.
        The realization order is taken from the order in which they were entered originally into the input
        DataObject. If this option is used, the \xmlNode{criterion} node must have an \xmlAttr{value}
        attribute that gives the index.
    \end{itemize}
  \item \xmlNode{target}, \xmlDesc{string, optional field}, required if the criterion targets a particular
    variable (such as the minimum and maximum criteria). Specifies the name of the
    target for which the criterion should be evaluated.
\end{itemize}

\textbf{Example:}

\begin{lstlisting}[style=XML]
<Simulation>
 ...
  <Models>
    ...
    <PostProcessor name="select_min" subType="SampleSelector">
      <target>x</target>
      <criterion>min</criterion>
    </PostProcessor>
    ...
    <PostProcessor name="select_index" subType="SampleSelector">
      <criterion value='3'>index</criterion>
    </PostProcessor>
    ...
  </Models>
 ...
</Simulation>
\end{lstlisting}
