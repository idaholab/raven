\subsubsection{ImportanceRank}
\label{ImportanceRank}
The \textbf{ImportanceRank} PostProcessor is specifically used
to compute sensitivity indices and importance indices with respect to input parameters
associated with multivariate normal distributions. In addition, the user can also request the transformation
matrix and the inverse transformation matrix when the PCA reduction is used.
%
\ppType{ImportanceRank}{ImportanceRank}
%
\begin{itemize}
  \item \xmlNode{what}, \xmlDesc{comma separated string, required field},
  %
  List of quantities to be computed.
  %
  Currently the quantities available are:
  \begin{itemize}
    \item \xmlString{SensitivityIndex}: used to measure the impact of sensitivities on the model.
    \item \xmlString{ImportanceIndex}: used to measure the impact of sensitivities and input uncertainties on the model.
    \item \xmlString{PCAIndex}: the indices of principal component directions, used to measure the impact
    of principal component directions on input covariance matrix.
    \nb \xmlString{PCAIndex} can be only requested when subnode \xmlNode{latent} is defined in \xmlNode{features}.
    \item \xmlString{transformation}: the transformation matrix used to map the latent variables to the manifest variables in the original input space.
    \item \xmlString{InverseTransformation}: the inverse transformation matrix used to map the manifest variables to the latent variables in the transformed space.
    \item \xmlString{ManifestSensitivity}: the sensitivity coefficients of \xmlNode{target} with respect to \xmlNode{manifest} variables defined in \xmlNode{features}.

    \nb In order to request \xmlString{transformation} matrix or \xmlString{InverseTransformation} matrix or \xmlString{ManifestSensitivity},
    the subnodes \xmlNode{latent} and \xmlNode{manifest} under \xmlNode{features} are required (more details can be found in the following).
    %
  \end{itemize}
  %
  \nb For each computed quantity, RAVEN will define a unique variable name so that the data can be accessible by the users
  through RAVEN entities \textbf{DataObjects} and \textbf{OutStreams}. These variable names are defined as follows:
  \begin{itemize}
    \item \xmlString{SensitivityIndex}: `sensitivityIndex' + `\_' + `targetVariableName' + `\_' + `latentFeatureVariableName'
    \item \xmlString{ImportanceIndex}: `importanceIndex' + `\_' + `targetVariableName' + `\_' + `latentFeatureVariableName'
    \item \xmlString{PCAIndex}: `pcaIndex' + `\_' + `latentFeatureVariableName'
    \item \xmlString{transformation}: `transformation' + `\_' + `manifestFeatureVariableName' + `\_' + `latentFeatureVariableName'
    \item \xmlString{InverseTransformation}: `inverseTransformation' + `\_' + `latentFeatureVariableName' + `\_' + `manifestFeatureVariableName'
    \item \xmlString{ManifestSensitivity}: `manifestSensitivity' + `\_' + `targetVariableName' + `\_' + `manifestFeatureVariableName'
  \end{itemize}
  %
  If all the quantities need to be computed, the user can input in the body of \xmlNode{what} the string \xmlString{all}.
  \nb \xmlString{all} equivalent to \xmlString {SensitivityIndex, ImportanceIndex, PCAIndex}.

  Since the transformation and InverseTransformation matrix can be very large, they are not printed with option \xmlString{all}.
  In order to request the transformation matrix (or inverse transformation matrix) from this post processor,
  the user need to specify \xmlString{transformation} or \xmlString{InverseTransformation} in \xmlNode{what}. In addition,
  both  \xmlNode{manifest} and \xmlNode{latent} subnodes are required and should be defined in node \xmlNode{features}. For example, let $\mathbf{L, P}$ represent
  the transformation and inverse transformation matrices, respectively. We will define vectors $\mathbf x$ as manifest variables and vectors $\mathbf y$
  as latent variables. If a absolute covariance matrix is used in given distribution, the following equation will be used:

  $
  \mathbf{\delta x} = \mathbf L * \mathbf y
  $

  $
  \mathbf y = \mathbf P * \mathbf \delta \mathbf x
  $

  If a relative covariance matrix is used in given distribution, the following equation will be used:

  $
  \frac{\mathbf \delta \mathbf x}{\mathbf \mu} = \mathbf L * \mathbf y
  $

  $
  \mathbf y = \mathbf P * {\frac{\mathbf \delta \mathbf x}{\mathbf \mu}}
  $

  where $\mathbf{\delta x}$ denotes the changes in the input vector $\mathbf x$, and $\mathbf \mu$ denotes the mean values of the input vector $\mathbf x$.

  %
  %
  \item \xmlNode{features}, \xmlDesc{XML node, required parameter}, used to specify the information for the input variables.
  In this xml-node, the following xml sub-nodes need to be specified:
    \begin{itemize}
      \item \xmlNode{manifest},\xmlDesc{XML node, optional parameter}, used to indicate the input variables belongs to the original input space.
      It can accept the following child node:
        \begin{itemize}
          \item \xmlNode{variables},\xmlDesc{comma separated string, required field}, lists manifest variables.
          \item \xmlNode{dimensions}, \xmlDesc{comma separated integer, optional field}, lists the dimensions corresponding to the manifest variables.
          If not provided, the dimensions are determined by the order indices of given manifest variables.
        \end{itemize}
      \item \xmlNode{latent},\xmlDesc{XML node, optional parameter}, used to indicate the input variables belongs to the transformed space.
      It can accept the following child node:
        \begin{itemize}
          \item \xmlNode{variables},\xmlDesc{comma separated string, required field}, lists latent variables.
          \item \xmlNode{dimensions}, \xmlDesc{comma separated integer, optional field}, lists the dimensions corresponding to the latent variables.
          If not provided, the dimensions are determined by the order indices of given latent variables.
        \end{itemize}
      \nb At least one of the subnodes, i.e. \xmlNode{manifest} and \xmlNode{latent} needs to be specified.
    \end{itemize}
  %
  \item \xmlNode{targets}, \xmlDesc{comma separated string, required field}, lists output responses.
  %
  \item \xmlNode{mvnDistribution}, \xmlDesc{string, required field}, specifies the
  multivariate normal distribution name. The \xmlNode{MultivariateNormal} node must be present. It requires two attributes:
    \begin{itemize}
      \item \xmlAttr{class}, \xmlDesc{required string attribute}, is the main
        ``class'' the listed object is from, the only acceptable class for
        this post-processor is \xmlString{Distributions};
      \item \xmlAttr{type}, \xmlDesc{required string attribute}, is the type of distributions,
        the only acceptable type is \xmlString{MultivariateNormal}
    \end{itemize}
\end{itemize}
  %
  %
  Here is an example to show the user how to request the transformation matrix, the inverse transformation matrix, the
  manifest sensitivities and other quantities.
  %

\textbf{Example:}
\begin{lstlisting}[style=XML,morekeywords={name,subType,debug}]
<Simulation>
  ...
  <Models>
    ...
    <PostProcessor name='aUserDefinedName' subType='ImportanceRank'>
      <what>SensitivityIndex,ImportanceIndex,Transformation, InverseTransformation,ManifestSensitivity</what>
      <features>
        <manifest>
          <variables>x1,x2</variables>
          <dimensions>1,2</dimensions>
        </manifest>
        <latent>
          <variables>latent1</variables>
          <dimensions>1</dimensions>
        </latent>
      </features>
      <targets>y</targets>
      <mvnDistribution>MVN</mvnDistribution>
    </PostProcessor>
    ...
  </Models>
  ...
</Simulation>
\end{lstlisting}

The calculation results can be accessible via variables ``sensitivityIndex\_y\_latent1, importanceIndex\_y\_latent1,
manifestSensitivity\_y\_x1, manifestSensitivity\_y\_x2, transformation\_x1\_latent1, transformation\_x2\_latent1,
inverseTransformation\_latnet1\_x1, inverseTransformation\_laent1\_x2'' through RAVEN entities \textbf{DataObjects}
and \textbf{OutStreams}.
