\subsubsection{ValueDuration}
\label{ValueDurationPP}
The \xmlNode{ValueDuration} PostProcessor is a tool to construct a particular kind of histogram, where the
independent variable is the number of times a variable exceeds a particular value, and the dependent variable
is the values themselves.  An example of this is the Load Duration Curve in energy modeling. This approach is
similar to that used in Lebesgue integration. Note that for each realization in the input
\xmlNode{HistorySet}, a separate load duration curve will be created for each target.

The \xmlNode{ValueDuration} PostProcessor can only act on \xmlNode{HistorySet} data objects, and generates a
\xmlNode{HistorySet} in return.  Two output variables are created for each \xmlAttr{target}:
\xmlString{counts\_x} and \xmlString{bins\_x}, where \xmlString{x} is replaced by the name of the target.
These must be specified in the output data object in order to be collected.

To plot a traditional Load Duration Curve, the x-axis should be the bins variable, and the y-axis should be
the counts variable.

\ppType{ValueDuration}{ValueDuration}
%
\begin{itemize}
  \item \xmlNode{target}, \xmlDesc{comma separated strings, required field}, specifies the names of the
    target(s) for which Value Duration histograms should be generated.
  \item \xmlNode{bins}, \xmlDesc{integer, required field}, specifies the number of bins that the values of the
    targets should be counted into.
\end{itemize}

\textbf{Example:}

\begin{lstlisting}[style=XML]
<Simulation>
 ...
  <Models>
    ...
    <PostProcessor name="pp" subType="ValueDuration">
      <target>x, y</target>
      <bins>100</bins>
    </PostProcessor>
    ...
  </Models>
 ...
</Simulation>
\end{lstlisting}
