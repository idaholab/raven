\subsubsection{ComparisonStatistics}
\label{ComparisonStatistics}
The \textbf{ComparisonStatistics} post-processor computes statistics
for comparing two different dataObjects.  This is an experimental
post-processor, and it will definitely change as it is further
developed.

There are four nodes that are used in the post-processor.

\begin{itemize}
\item \xmlNode{kind}: specifies information to use for comparing the
  data that is provided.  This takes either uniformBins which makes
  the bin width uniform or equalProbability which makes the number
  of counts in each bin equal.  It can take the following attributes:
  \begin{itemize}
  \item \xmlAttr{numBins} which takes a number that directly
    specifies the number of bins
  \item \xmlAttr{binMethod} which takes a string that specifies the
    method used to calculate the number of bins.  This can be either
    square-root or sturges.
  \end{itemize}
\item \xmlNode{compare}: specifies the data to use for comparison.
  This can either be a normal distribution or a dataObjects:
  \begin{itemize}
  \item \xmlNode{data}: This will specify the data that is used.  The
    different parts are separated by $|$'s.
  \item \xmlNode{reference}: This specifies a reference distribution
    to be used.  It takes distribution to use that is defined in the
    distributions block.  A name parameter is used to tell which
    distribution is used.
  \end{itemize}
\item \xmlNode{fz}: If the text is true, then extra comparison
  statistics for using the $f_z$ function are generated.  These take
  extra time, so are not on by default.
\item \xmlNode{interpolation}: This switches the interpolation used
  for the cdf and the pdf functions between the default of quadratic
  or linear.
\end{itemize}

The \textbf{ComparisonStatistics} post-processor generates a variety
of data.  First for each data provided, it calculates bin boundaries,
and counts the numbers of data points in each bin.  From the numbers
in each bin, it creates a cdf function numerically, and from the cdf
takes the derivative to generate a pdf.  It also calculates statistics
of the data such as mean and standard deviation. The post-processor
can generate a CSV file only.

The post-processor uses the generated pdf and cdf function to
calculate various statistics.  The first is the cdf area difference which is:
\begin{equation}
  cdf\_area\_difference = \int_{-\infty}^{\infty}{\|CDF_a(x)-CDF_b(x)\|dx}
\end{equation}
This given an idea about how far apart the two pieces of data are, and
it will have units of $x$.

The common area between the two pdfs is calculated.  If there is
perfect overlap, this will be 1.0, if there is no overlap, this will
be 0.0.  The formula used is:
\begin{equation}
  pdf\_common\_area = \int_{-\infty}^{\infty}{\min(PDF_a(x),PDF_b(x))}dx
\end{equation}

The difference pdf between the two pdfs is calculated.  This is calculated as:
\begin{equation}
  f_Z(z) = \int_{-\infty}^{\infty}f_X(x)f_Y(x-z)dx
\end{equation}
This produces a pdf that contains information about the difference
between the two pdfs.  The mean can be calculated as (and will be
calculated only if fz is true):
\begin{equation}
  \bar{z} = \int_{-\infty}^{\infty}{z f_Z(z)dz}
\end{equation}
The mean can be used to get an signed difference between the pdfs,
which shows how their means compare.

The variance of the difference pdf can be calculated as (and will be
calculated only if fz is true):
\begin{equation}
  var = \int_{-\infty}^{\infty}{(z-\bar{z})^2 f_Z(z)dz}
\end{equation}

The sum of the difference function is calculated if fz is true, and is:
\begin{equation}
  sum = \int_{-\infty}^{\infty}{f_z(z)dz}
\end{equation}
This should be 1.0, and if it is different that
points to approximations in the calculation.


\textbf{Example:}
\begin{lstlisting}[style=XML]
<Simulation>
   ...
   <Models>
      ...
      <PostProcessor name="stat_stuff" subType="ComparisonStatistics">
      <kind binMethod='sturges'>uniformBins</kind>
      <compare>
        <data>OriData|Output|tsin_TEMPERATURE</data>
        <reference name='normal_410_2' />
      </compare>
      <compare>
        <data>OriData|Output|tsin_TEMPERATURE</data>
        <data>OriData|Output|tsout_TEMPERATURE</data>
      </compare>
      </PostProcessor>
      <PostProcessor name="stat_stuff2" subType="ComparisonStatistics">
        <kind numBins="6">equalProbability</kind>
        <compare>
          <data>OriData|Output|tsin_TEMPERATURE</data>
        </compare>
        <Distribution class='Distributions' type='Normal'>normal_410_2</Distribution>
      </PostProcessor>
      ...
   </Models>
   ...
   <Distributions>
      <Normal name='normal_410_2'>
         <mean>410.0</mean>
         <sigma>2.0</sigma>
      </Normal>
   </Distributions>
</Simulation>
\end{lstlisting}
