\subsubsection{SparseSensing}
\label{SparseSensing}
The \textbf{SparseSensing} post-processor incorporates ``PySensors'', a Scikit-learn style Python package for the sparse placement of sensors for reconstruction tasks and classification tasks which will be added here soon.

Sparse sensor placement concerns the problem of selecting a small subset of sensor or measurement locations in a way that allows one to perform some task nearly as well as if one had access to measurements at every location.

This post-processor provides objects designed for the tasks of reconstruction and classification. See

\begin{itemize}
    \item Manohar, Krithika, et al. ``Data-driven sparse sensor placement for reconstruction: Demonstrating the benefits of exploiting known patterns.'' IEEE Control Systems Magazine 38.3 (2018): 63-86 for more information about the PySensors approach to reconstruction problems and,
    \item Brunton, Bingni W., et al. ``Sparse sensor placement optimization for classification.'' SIAM Journal on Applied Mathematics 76.5 (2016): 2099-2122 for classification and,
    \item de Silva, Brian M., et al. ``PySensors: A Python package for sparse sensor placement.'' arXiv preprint arXiv:2102.13476 (2021) contains a full literature review along with examples and additional tips for using PySensors effectively.
\end{itemize} 

Some important terms related to sensor placement problems are : 

\begin{itemize}
    \item \textbf{Reconstruction} deals with predicting the values of a quantity of interest at different locations other than those where sensors are located. For example, one might predict the temperature at a point point in the middle of a fuel rod based on readings taken at various other positions. 

    \item \textbf{Classification} is the problem of predicting which category an example belongs to, given a set of training data (e.g. determining whether digital photos are of dogs or cats).

    \item \textbf{Bases} in which measurement data are represented can have a dramatic effect on performance. PySensors implements the three bases most commonly used for sparse sensor placement: raw measurements, SVD/POD/PCA modes, and random projections. These bases can be easily incorporated into either classification or reconstruction tasks. 
\end{itemize}

\ppType{SparseSensing post-processor}{SparseSensing}
\begin{itemize}
	\item \xmlNode{Goal}, The goal of the sparse sensor optimization (i.e., reconstruction or classification) should be filled in subType.
	\item \xmlNode{TrainingData}, The Dataobject containing the training data
	\item \xmlNode{features}, Features/inputs of the data model
	\item \xmlNode{target}, target of data model
	\item \xmlNode{basis}, The type of basis onto which the data are projected
	\item \xmlNode{nModes}, The number of modes retained
	\item \xmlNode{nSensors}, The number of sensors used
	\item \xmlNode{optimizer}, The type of optimizer used
\end{itemize}

\textbf{Example:}
\begin{lstlisting}[style=XML]
<Simulation>
   ...
   <Models>
      ...
      <PostProcessor name="mySPSL" subType="SparseSensing" verbosity="debug">
      <Goal subType="reconstruction">
        <TrainingData>myDO</TrainingData>
        <features>X (m),Y (m),Temperature (K)</features>
        <target>Temperature (K)</target>
        <basis>svd</basis> <!--Ddefault: svd-->
        <nModes>4</nModes> <!--default: opt, allows the algorithm to pick nModes-->
        <nSensors>4</nSensors><!--default: opt, allows the algorithm to pick nSensors-->
        <optimizer>QR</optimizer><!--default: QR-->
      </Goal>
    </PostProcessor>
    ...
  </Models>
   ...
</Simulation>
\end{lstlisting}