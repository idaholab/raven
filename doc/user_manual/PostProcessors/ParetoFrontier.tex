\subsubsection{ParetoFrontier}
\label{ParetoFrontierPP}
The \textbf{ParetoFrontier} PostProcessor is designed to identify the points lying on the Pareto Frontier in a multi-dimensional trade-space.
This post-processor receives as input a \textbf{DataObject} (a PointSet only) which contains all data points in the trade-space space and it
returns the subset of points lying in the Pareto Frontier as a PointSet.

It is here assumed that each data point of the input PointSet is a realization of the system under consideration for a
specific configuration to which corresponds several objective variables (e.g., cost and value).

%
\ppType{ParetoFrontier}{ParetoFrontier}
%
\begin{itemize}
  \item   \xmlNode{objective},\xmlDesc{string, required parameter}, ID of the objective variable that represents a dimension of the trade-space space.
          The \xmlNode{costID} requires one identifying attribute:
          \begin{itemize}
            \item \xmlAttr{goal}, \xmlDesc{string, required field}, Goal of the objective variable characteristic: minimization (min) or maximization (max)
            \item \xmlAttr{upperLimit}, \xmlDesc{string, optional field}, Desired upper limit of the objective variable for the points in the Pareto frontier
            \item \xmlAttr{lowerLimit}, \xmlDesc{string, optional field}, Desired lower limit of the objective variable for the points in the Pareto frontier
          \end{itemize}
\end{itemize}

The following is an example where a set of realizations (the ``candidates'' PointSet) has been generated by changing two parameters
(var1 and var2) which produced two output variables: cost (which it is desired to be minimized) and value (which it is desired to be maximized).
The \textbf{ParetoFrontier} post-processor takes the ``candidates'' PointSet and populates a Point similar in structure
(the ``paretoPoints'' PointSet).

\textbf{Example:}
\begin{lstlisting}[style=XML,morekeywords={anAttribute},caption=ParetoFrontier input example (no expand)., label=lst:ParetoFrontier_PP_InputExample]
  <Models>
    <PostProcessor name="paretoPP" subType="ParetoFrontier">
      <objective goal='min' upperLimit='0.5'>cost</objective>
      <objective goal='max' lowerLimit='0.5'>value</objective>
    </PostProcessor>
  </Models>

  <Steps>
    <PostProcess name="PP">
      <Input     class="DataObjects"  type="PointSet"        >candidates</Input>
      <Model     class="Models"       type="PostProcessor"   >paretoPP</Model>
      <Output    class="DataObjects"  type="PointSet"        >paretoPoints</Output>
    </PostProcess>
  </Steps>

  <DataObjects>
    <PointSet name="candidates">
      <Input>var1,var2</Input>
      <Output>cost,value</Output>
    </PointSet>
    <PointSet name="paretoPoints">
      <Input>var1,var2</Input>
      <Output>cost,value</Output>
    </PointSet>
  </DataObjects>
\end{lstlisting}

\nb it is possible to specify both upper and lower limits for each objective variable.
When one or both of these limits are specified, then the Pareto frontier is filtered such that all Pareto frontier points that
satisfy those limits are preserved.
