\section{OutStream system \\ \vspace{2 mm} {\small Author: Andrea Alfonsi}}
The PRA and UQ framework provides the capabilities to visualize and dump out the data that are  generated, imported (from a system code) and post-processed during the analysis. These capabilities are contained in the "OutStream" system. Actually, two different OutStream types are available:
\vspace{-5mm}
\begin{itemize}
\itemsep0em
\item Print, module that lets the user dump the data contained in the internal objects 
\item Plot, module, based on MatPlotLib~\cite{MatPlotLib}, aimed to provide advanced plotting capabilities 
\end{itemize}
\vspace{-5mm}

Both the types listed above only accept as ``input'' a \textit{Data} object type. This choice has been taken since the ``\textit{Datas}'' system (see section~\ref{sec:Datas}) has the main advantages, among the others, of ensuring a standardized approach for exchanging the data/meta-data among the different framework entities. Every module can project its outcomes into a \textit{Data} object. This provides, to the user, the capability to visualize/dump all the modules' results. 
As already mentioned [put reference to the xml input section], the RAVEN framework input is based on the \textbf{E}xtensible \textbf{M}arkup \textbf{L}anguage (\textbf{XML}) format. Thus, in order to activate the ``\textit{OutStream}'' system, the input needs to contain a block identified by the ``\textbf{$<OutStreamManager>$}'' tag (as shown below).  

\begin{lstlisting}[style=XML]
-----------------------------------------------------------
<OutStreamManager>
    <!-- "OutStream" objects that need to be created-->
</OutStreamManager>
-----------------------------------------------------------
\end{lstlisting}
In the ``OutStreamManager'' block an unlimited number of ``Plot'' and ``Print'' sub-blocks can be inputted. The input specifications and the main capabilities for both types are reported in the following sections.
%%%%%%%%%
%
% PRINTING SYSTEM
%
%%%%%%%%%
\subsection{Printing system \label{sec:printing}}
The Printing system has been created in order to let the user dump  the data, contained in the internal data objects (see [reference to Data(s) section]), out at anytime during the calculation. Currently, the only available output is a \textbf{C}omma \textbf{S}eparated \textbf{V}alue (\textbf{CSV}) file. In the near future, an XML formatted file option will be available. This will facilitate  the exchanging of results and provide the possibility to dump the solution of an analysis and "restart" another one constructing a \textit{Data} from scratch.
The XML code, that is reported below, shows different ways to request a \textit{Print} OutStream. The user needs to provide a name for each sub-block (XML attribute). These names are then used in the \textit{Steps'} blocks in order to activate the Printing options at anytime.
As shown in the examples below, every \textit{Print} block must contain, at least, the two required tags:
\vspace{-5mm}
\begin{itemize}
\itemsep0em
\item $<type>$, the output file type (csv or xml). \textit{Note, only \textbf{csv} is currently available}
\item $<source>$, the \textit{Data} name (one of the \textit{Data} defined in the ``\textit{Datas}'' block)
\end{itemize}
\vspace{-5mm}
If only these two tags are provided (as in the ``first-example'' below), the output file will be filled with the whole content of the ``d-name'' \textit{Data}. 
\begin{lstlisting}[style=XML]
-----------------------------------------------------------
<OutStreamManager>
  <Print name='first_example'>
    <type>csv</type>
    <source>d-name</source>
  </Print>
  <Print name='second-example'>
    <type>csv</type>
    <source>d-name</source>
    <variables>Output</variables>
  </Print>
  <Print name='third-example'>
    <type>csv</type>
    <source>d-name</source>
    <variables>Input</variables>
  </Print>
  <Print name='forth-example'>
    <type>csv</type>
    <source>d-name</source>
    <variables>Input|var-name-in,Output|var-name-out</variables>
  </Print>
</OutStreamManager>
-----------------------------------------------------------
\end{lstlisting}
If just few parts of the $<source>$ are important for a particular analysis, the additional XML tag $<variables>$ can be provided. In this block, the variables that need to be dumped must be inputted, in a comma separated format. The available options, for the $<variables>$ sub-block, are listed below:
\vspace{-5mm}
\begin{itemize}
\itemsep0em
\item \textbf{Output}, the output space will be dumped out (see ``second-example'')
\item \textbf{Input}, the input space will be dumped out (see ``third-example'')
\item \textbf{Input|var-name-in/Output|var-name-out}, only the particular variables ``var-name-in'' and ``var-name-out'' will be reported in the output file (see ``forth-example'')
\end{itemize}
\vspace{-5mm}
Note that all the XML tags are case-sensitive but not their content. 
%%%%%%%%%
%
% PLOTTING SYSTEM
%
%%%%%%%%%
\subsection{Plotting system \label{sec:plotting}}
The Plotting system provides all the capabilities to visualize the analysis outcomes, in real-time or at the post-processing stage. The system is based on the  Python library MatPlotLib~\cite{MatPlotLib}. MatPlotLib is a  2D/3D plotting library which produces publication quality figures in a variety of hardcopy formats and interactive environments across platforms. This external tool has been wrapped in the RAVEN framework, and is usable by the user. Since it was unfeasible to support, in the source code, all the interfaces for all the available plot types, the RAVEN Plotting system directly provide a formatted input structure for 11 different plot types (2D/3D). The user may request a plot not present among the supported ones, since the RAVEN Plotting system has the capability to construct on the fly the interface for a Plot, based on XML instructions. This capability will be discussed in the sub-section~\ref{sec:Interpretedplotting}.
%%%%%%%%%%%%%
% Plot Input Strucutre sub-sub-section 
%%%%%%%%%%%%%
\subsubsection{Plot input structure \label{sec:PlotInputStructure}}
In order to create a plot, the user needs to add, within the $<OutStreamManager>$ block,  a $<Plot>$ sub-block. As for the \textit{Print}  OutStream, the user needs to specify a name as attribute of the plot. This name will then be used to request the plot in the \textit{Steps'} block. In addition, the Plot block may need the following attributes:
\vspace{-5mm}
\begin{itemize}
\itemsep0em
\item \textbf{dim}, \textit{required integer attribute}, define the dimensionality of the plot: 2 (2D) or 3 (3D)
\item \textbf{interactive}, \textit{optional bool attribute (default=False)'}, specify if the Plot needs to be interactively created (real-time screen visualization)
\item \textbf{overwrite}, \textit{optional bool attribute (default=False)'}, if the plot needs to be dumped into picture file/s, does the code need to overwrite them every time a new plot (with the same name) is requested?
\end{itemize}
\vspace{-5mm}

As shown, in the XML input example below, the body of the Plot XML input contains two main sub-nodes:
\vspace{-5mm}
\begin{itemize}
\itemsep0em
\item$<actions>$, where general control options for the figure layout are defined (see [])
\item $<plot\_settings>$, where the actual plot options are provided
\vspace{-5mm}
\end{itemize}
These two main sub-block are discussed in the following paragraphs.
%%%%%%%%%%%%%
% Actions' block sub-sub-sub section
%%%%%%%%%%%%%
\paragraph{``Actions'' input block \label{sec:actionsBlock}} 
The input in the $<actions>$ sub-node is common to all the Plot types, since, in it, the user specifies all the controls that need to be applied to the figure style. This block must be unique in the definition of the $<Plot>$ main block. In the following list, all the predefined ``actions'' are reported:
 \vspace{-5mm}
\begin{itemize}
\itemsep0em
\item $<how>$, comma separated list of output types:
     \begin{itemize}
        \item \textit{screen}, show the figure on the screen in interactive mode
        \item \textit{pdf}, save the figure as a Portable Document Format file (PDF)
        \item \textit{png}, save the figure as a Portable Network Graphics file (PNG)
        \item \textit{eps}, save the figure as a Encapsulated Postscript file (EPS)
        \item \textit{pgf}, save the figure as a LaTeX PGF Figure file (PGF)
        \item \textit{ps}, save the figure as a Postscript file (PS)
        \item \textit{gif}, save the figure as a Graphics Interchange Format (GIF)
        \item \textit{svg}, save the figure as a Scalable Vector Graphics file (SVG)
        \item \textit{jpeg}, save the figure as a jpeg file (JPEG)
        \item \textit{raw}, save the figure as a Raw RGBA bitmap file (RAW)
        \item \textit{bmp}, save the figure as a Windows bitmap file (BMP)
        \item \textit{tiff}, save the figure as a Tagged Image Format  file (TIFF)
        \item \textit{svgz}, save the figure as a Scalable Vector Graphics file (SVGZ)
      \end{itemize}
\item $<title>$, as the name suggests , within this block the user can specify the title of the figure. In the body, few other keywords (required and not) are present:

% TITLE
 \begin{itemize}
        \item \textit{$<text>$}, string type, title of the figure
        \item \textit{$<kwargs>$},  within this block the user can specify optional parameters with the following format:
        \begin{lstlisting}[style=XML]
        --------------------------
         <kwargs>
           <param1>value1</param1>
           <param2>value2</param2>
         </kwargs>
        -------------------------
       \end{lstlisting}
         The kwargs block is able to convert whatever string into a python type (for example $<param1>{'1stKeyword':45}</param1>$ will be converted into a dictionary, $<param2>[56,67]</param2>$ into a list, etc.). For reference regarding the available kwargs, see ``matplotlib.pyplot.title'' method in~\cite{MatPlotLib}.
      \end{itemize}
% LABEL FORMAT
\item $<label\_format>$, within this block the default scale formating can be modified. In the body, few keywords can be specified (all optional):
 \begin{itemize}
        \item \textit{$<style>$}, string, the style of the number notation, 'sci' or 'scientific' for scientific, 'plain'  for plain notation. Default = scientific
        \item \textit{$<scilimits>$}, tuple, (m, n), pair of integers; if style is ‘sci’, scientific notation will be used for numbers outside the range 10`m`:sup: to 10`n`:sup:. Use (0,0) to include all numbers. NB. The value for this keyword, needs to be inputted between brackets [for example, (5,6)]. Default = (0,0)
        \item \textit{$<useOffset>$}, bool or double, if True, the offset will be calculated as needed; if False, no offset will be used; if a numeric offset is specified, it will be used. Default = False
        \item \textit{$<axis>$}, string, the axis where to apply the defined format, 'x','y' or 'both'. Default = 'both'. NB. If this action will be used in a 3-D plot, the user can input 'z' as well and 'both' will apply this format to all three axis.
      \end{itemize}
% FIGURE PROPERTIES
\item $<figure\_properties>$, within this block the user specifies how to customize the figure style/quality. Thus, through this ``action'' the user has got full control on the quality of the figure, its dimensions, etc. This control is performed by the following keywords: 
 \begin{itemize}
        \item \textit{$<figsize>$}, tuple (optional), (width, hight), in inches
        \item \textit{$<dpi>$}, integer, dots per inch
        \item \textit{$<facecolor>$}, string, set the figure background  color (please refer to ``matplotlib.figure.Figure'' in~\cite{MatPlotLib} for a list of all the colors available) 
        \item \textit{$<edgecolor>$}, string, the figure edge background color (please refer to ``matplotlib.figure.Figure'' in~\cite{MatPlotLib} for a list of all the colors available) 
        \item \textit{$<linewidth>$}, self explainable keyword
        \item \textit{$<frameon>$}, bool, if False, suppress drawing the figure frame
      \end{itemize}
% RANGE
\item $<range>$, the range ``action'' allows to specify the ranges of all the axis. All the keywords in the body of this block are optional:
     \begin{itemize}
        \item \textit{$<ymin>$}, double  (optional), lower boundary for y axis
        \item \textit{$<ymax>$}, double  (optional), upper boundary for y axis
        \item \textit{$<xmin>$}, double  (optional), lower boundary for x axis
        \item \textit{$<xmax>$}, double  (optional), upper boundary for x axis 
        \item \textit{$<zmin>$}, double  (optional), lower boundary for z axis. NB. Obviously, this keyword is effective  in 3-D plots only
        \item \textit{$<zmax>$}, double  (optional), upper boundary for z axis. NB. Obviously, this keyword is effective in 3-D plots only
      \end{itemize}
% CAMERA
\item $<camera>$, the camera item is available in 3-D plots only. Through this ``action'', it is possible to orientate the plot as wished.  The controls are:
     \begin{itemize}
        \item \textit{$<elevation>$}, double  (optional), stores the elevation angle in the z plane
        \item \textit{$<azimuth>$}, double  (optional), stores the azimuth angle in the x,y plane
      \end{itemize}
% SCALE
\item $<scale>$, the scale block allows the specification of the axis scales:
     \begin{itemize}
        \item \textit{$<xscale>$}, string  (optional), scale of the x axis. Three options are available: ``linear'',``log'',``symlog''. Default = linear
        \item \textit{$<yscale>$}, string  (optional), scale of the y axis. Three options are available: ``linear'',``log'',``symlog''. Default = linear
        \item \textit{$<zscale>$}, string  (optional), scale of the z axis. Three options are available: ``linear'',``log'',``symlog''. Default = linear. NB. Obviously, this keyword is effective in 3-D plots only
      \end{itemize}
% ADD_TEXT
\item $<add\_text>$, same as title
% AUTOSCALE
\item $<autoscale>$, the autoscale block is a convenience method for simple axis view autoscaling. It turns autoscaling on or off, and then, if autoscaling for either axis is on, it performs the autoscaling on the specified axis or axes. The following keywords are available:
     \begin{itemize}
        \item \textit{$<enable>$}, bool (optional), True turns autoscaling on, False turns it off. None leaves the autoscaling state unchanged. Default = True
        \item \textit{$<axis>$}, string  (optional),  string, the axis where to apply the defined format, 'x','y' or 'both'. Default = 'both'. NB. If this action will be used in a 3-D plot, the user can input 'z' as well and 'both' will apply this format to all three axis.
        \item \textit{$<tight>$}, bool  (optional), if True, set view limits to data limits; if False, let the locator and margins expand the view limits; if None, use tight scaling if the only artist is an image, otherwise treat tight as False.
      \end{itemize}
%HORIZONTAL_LINE
\item $<horizontal\_line>$, this ``action''  provides the ability to draw a horizontal line in the current figure. This capability might be useful, for example, if the user wants to highlight a trigger, function of a variable. The following keywords are settable:
    \begin{itemize}
        \item \textit{$<y>$}, double (optional), y coordinate. Default = 0
        \item \textit{$<xmin>$}, double (optional), starting coordinate on the x axis. Default = 0
        \item \textit{$<xmax>$}, double (optional), ending coordinate on the x axis. Default = 1
        \item \textit{$<kwargs>$},  within this block the user can specify optional parameters with the following format:
        \begin{lstlisting}[style=XML]
        --------------------------
         <kwargs>
           <param1>value1</param1>
           <param2>value2</param2>
         </kwargs>
        -------------------------
       \end{lstlisting}
         The kwargs block is able to convert whatever string into a python type (for example $<param1>{'1stKeyword':45}</param1>$ will be converted into a dictionary, $<param2>[56,67]</param2>$ into a list, etc.). For reference regarding the available kwargs, see ``matplotlib.pyplot.axhline'' method in~\cite{MatPlotLib}.
      \end{itemize}
 NB. This capability is not available  for 3-D plots.
%VERTICAL_LINE
\item $<vertical\_line>$, similarly to the ``horizontal\_line'' action,  this block  provides the ability to draw a vertical line in the current figure. This capability might be useful, for example, if the user wants to highlight a trigger, function of a variable. The following keywords are settable:
    \begin{itemize}
        \item \textit{$<x>$}, double (optional), x coordinate. Default = 0
        \item \textit{$<ymin>$}, double (optional), starting coordinate on the y axis. Default = 0
        \item \textit{$<ymax>$}, double (optional), ending coordinate on the y axis. Default = 1
        \item \textit{$<kwargs>$},  within this block the user can specify optional parameters with the following format:
        \begin{lstlisting}[style=XML]
        --------------------------
         <kwargs>
           <param1>value1</param1>
           <param2>value2</param2>
         </kwargs>
        -------------------------
       \end{lstlisting}
         The kwargs block is able to convert whatever string into a python type (for example $<param1>{'1stKeyword':45}</param1>$ will be converted into a dictionary, $<param2>[56,67]</param2>$ into a list, etc.). For reference regarding the available kwargs, see ``matplotlib.pyplot.axvline'' method in~\cite{MatPlotLib}.
      \end{itemize}
 NB. This capability is not available  for 3-D plots.
%HORIZONTAL_RECTANGLE
\item $<horizontal\_rectangle>$, this ``action''  provides the possibility to draw, in the current figure, a horizontally orientated rectangle . This capability might be useful, for example, if the user wants to highlight a zone in the plot. The following keywords are settable:
    \begin{itemize}
        \item \textit{$<ymin>$}, double (required), starting coordinate on the y axis
        \item \textit{$<ymax>$}, double (required), ending coordinate on the y axis
        \item \textit{$<xmin>$}, double (optional), starting coordinate on the x axis. Default = 0
        \item \textit{$<xmax>$}, double (optional), ending coordinate on the x axis. Default = 1
        \item \textit{$<kwargs>$},  within this block the user can specify optional parameters with the following format:
        \begin{lstlisting}[style=XML]
        --------------------------
         <kwargs>
           <param1>value1</param1>
           <param2>value2</param2>
         </kwargs>
        -------------------------
       \end{lstlisting}
         The kwargs block is able to convert whatever string into a python type (for example $<param1>{'1stKeyword':45}</param1>$ will be converted into a dictionary, $<param2>[56,67]</param2>$ into a list, etc.). For reference regarding the available kwargs, see ``matplotlib.pyplot.axhspan'' method in~\cite{MatPlotLib}.
      \end{itemize}
 NB. This capability is not available  for 3-D plots.
%VERTICAL_RECTANGLE
\item $<vertical\_rectangle>$, this ``action''  provides the possibility to draw, in the current figure, a vertically orientated rectangle . This capability might be useful, for example, if the user wants to highlight a zone in the plot. The following keywords are settable:
    \begin{itemize}
        \item \textit{$<xmin>$}, double (required), starting coordinate on the x axis
        \item \textit{$<xmax>$}, double (required), ending coordinate on the x axis
        \item \textit{$<ymin>$}, double (optional), starting coordinate on the y axis. Default = 0
        \item \textit{$<ymax>$}, double (optional), ending coordinate on the y axis. Default = 1
        \item \textit{$<kwargs>$},  within this block the user can specify optional parameters with the following format:
        \begin{lstlisting}[style=XML]
        --------------------------
         <kwargs>
           <param1>value1</param1>
           <param2>value2</param2>
         </kwargs>
        -------------------------
       \end{lstlisting}
         The kwargs block is able to convert whatever string into a python type (for example $<param1>{'1stKeyword':45}</param1>$ will be converted into a dictionary, $<param2>[56,67]</param2>$ into a list, etc.). For reference regarding the available kwargs, see ``matplotlib.pyplot.axvspan'' method in~\cite{MatPlotLib}.
      \end{itemize}
 NB. This capability is not available  for 3-D plots.
%AXES_BOX
\item $<axes\_box>$, this keyword controls the axes' box. No body and its value can be 'on' or 'off'.
\item $<axis\_properties>$, this block is used to set axis properties. There are not fixed keywords. If only a single property needs to be set, it can be specified as body of this block, otherwise a dictionary-like string needs to be provided. For reference regarding the available keys, refer to ``matplotlib.pyplot.axis'' method in~\cite{MatPlotLib}.
\item $<grid>$, this block is used to define a grid that needs to be added in the plot. The following keywords can be inputted:
    \begin{itemize}
        \item \textit{$<b>$}, double (required), starting coordinate on the x axis
        \item \textit{$<which>$}, double (required), ending coordinate on the x axis
        \item \textit{$<axis>$}, double (optional), starting coordinate on the y axis. Default = 0
        \item \textit{$<kwargs>$},  within this block the user can specify optional parameters with the following format:
        \begin{lstlisting}[style=XML]
        --------------------------
         <kwargs>
           <param1>value1</param1>
           <param2>value2</param2>
         </kwargs>
        -------------------------
       \end{lstlisting}
         The kwargs block is able to convert whatever string into a python type (for example $<param1>{'1stKeyword':45}</param1>$ will be converted into a dictionary, $<param2>[56,67]</param2>$ into a list, etc.). For reference regarding the available kwargs, see ``matplotlib.pyplot.grid'' method in~\cite{MatPlotLib}.
      \end{itemize}
\vspace{-5mm}
\end{itemize}
%%%%%%%%%%%%%
%Plot block sub-sub-sub section
%%%%%%%%%%%%%
\paragraph{``plot\_settings'' input block \label{sec:plotSettings}} 
The sub-block identified by the keyword $<plot\_settings>$ is used to define the plot characteristics.Within this sub-section at least a $<plot>$ block must be present. the $<plot>$ sub-section may not be unique within the $<plot\_settings>$  definition; the number of  $<plot>$ sub-blocks is equal to the number of plots that need to be placed in the same figure. For example, in the following XML cut, a ``line'' and a ``scatter'' type are combined in the same figure. 
\begin{lstlisting}[style=XML]
-----------------------------------------------------------
<OutStreamManager>
  <Plot name='example2PlotsCombined' dim='2'>
    <actions>
      <!-- Actions -->
    </actions>
    <plot\_settings>
       <plot>
        <type>line</type>
        <x>d-type|Output|x1</x>
        <y>d-type|Output|y1</y> 
      </plot>
       <plot>
        <type>scatter</type>
        <x>d-type|Output|x2</x>
        <y>d-type|Output|y2</y> 
      </plot>
      <xlabel>label X</xlabel>
      <ylabel>label Y</ylabel>
    </plot\_settings>
  </Plot>
</OutStreamManager>
-----------------------------------------------------------
\end{lstlisting}
As already mentioned, within the $<plot\_settings>$  block, at least a $<plot>$ sub-block needs to be inputted. Independently from the plot type, some keywords are mandatory:
\begin{itemize}
     \item \textit{$<type>$}, string, the plot type (for example, line, scatter, wireframe, etc.)
     \item \textit{$<x>$}, string, the parameter needs to be considered as x coordinate
     \item \textit{$<y>$}, string, the parameter needs to be considered as y coordinate
     \item \textit{$<z>$}, string (required in 3-D plots only), the parameter needs to be considered as z coordinate
\end{itemize}
In addition other plot-dependent keywords, reported in section~\ref{sec:23Dplotting}, can be provided. 
\\Under the  $<plot\_settings>$ block other keywords, common to all the plots the user decided to combine in the figure, can be inputted, such as:
\begin{itemize}
     \item \textit{$<xlabel>$}, string, x axis label
     \item \textit{$<ylabel>$}, string, y axis laber
     \item \textit{$<zlabel>$}, string (available in 3-D plots only), z axis label
\end{itemize}
%%%%%%%%%%%%%
%Predefined Plotting System block sub-sub-sub section
%%%%%%%%%%%%%
\paragraph{Predefined Plotting System: 2D/3D \label{sec:23Dplotting}} 
sgagagga
%As already mentioned above, the Plotting system provides specialized input structure for 11 different plot types. 
-----------------------------------------------------------
\begin{lstlisting}[style=XML]
<OutStreamManager>
  <Plot name='2DHistoryPlot' dim='2' interactive='False' overwrite='False'>
    <actions>
      <how>pdf,png,eps</how>
      <title>
        <text> </text>
      </title>
    </actions>
    <plot_settings>
       <plot>
        <type>line</type>
        <x>stories|Output|time</x>
        <y>stories|Output|pipe1_Hw</y> 
        <kwargs>
         <color>green</color>
         <label>pipe1-Hw</label>
        </kwargs>
      </plot>
       <plot>
        <type>line</type>
        <x>stories|Output|time</x>
        <y>stories|Output|pipe1_aw</y> 
        <kwargs>
         <color>blue</color>
         <label>pipe1-aw</label>
        </kwargs>
      </plot>
      <xlabel>time [s]</xlabel>
      <ylabel>evolution</ylabel>
    </plot_settings>
  </Plot>
</OutStreamManager>
\end{lstlisting}
-----------------------------------------------------------

\subsubsection{Interpreted Plotting instruction \label{sec:Interpretedplotting}}











