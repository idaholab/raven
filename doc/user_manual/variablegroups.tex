\section{VariableGroups}
\label{sec:VariableGroups}

The \xmlNode{VariableGroups} block is an optional input for the convenience of the user.  It allows the
possibility of creating a collection of variables instead of re-listing all the variables in places throughout
the input file, such as DataObjects, ROMs, and ExternalModels.
%
Each entry in the \xmlNode{VariableGroups} block has a distinct name and list of each constituent variable in
the group.
%
Additionally, set operations can be used to construct variable groups from other variable groups, by listing
them in node text along with the operation to perform.
The following types of
set operations are included in RAVEN:
\begin{itemize}
  \item \texttt{+}, Union, the combination of all variables in the \xmlString{base} set and listed set,
  \item \texttt{-}, Complement, the relative complement of the listed set in the \xmlString{base} set,
  \item \texttt{\^}, Intersection, the variables common to both the \xmlString{base} and listed set,
  \item \texttt{\%}, Symmetric Difference, the variables in only either the \xmlString{base} or listed set,
    but not both.
\end{itemize}
Multiple set operations can be performed by separating them with commas in the text of the group node, whether
they be variable groups or single variables. In the event a circular dependency loop is detected, an error will be
raised. VariableGroups are evaluated in the order of entries listed in their node text.

When using the variable groups in a node, they can be listed alone or as part of a comma-separated list.  The
variable group name will only be substituted in the text of nodes, not attributes or tags.

Each \xmlNode{Group} node has the following attributes:
\vspace{-5mm}
\begin{itemize}
  \itemsep0em
  \item \xmlAttr{name}, \xmlDesc{required string attribute}, user-defined name
  of the group. This is the identifier that will be used elsewhere in the RAVEN input.
  %
\end{itemize}
\vspace{-5mm}

An example of constructing and using variable groups is listed here.  The variable groups \xmlString{x\_odd},
\xmlString{x\_even}, \xmlString{x\_first},  and  \xmlString{y\_group} are constructed independently, and the
remainder are examples of other operations.
\begin{lstlisting}[style=XML,morekeywords={name,file}] %moreemph={name,file}]
<Simulation>
  ...
  <VariableGroups>
    <Group name="x_odd"     >x1,x3,x5</Group>
    <Group name="x_even"    >x2,x4,x6</Group>
    <Group name="x_first"   >x1,x2,x3</Group>
    <Group name="y_group"   >y1,y2</Group>
    <Group name="add_remove">x_first,-x1,+ x4,+x5</Group>
    <Group name="union"     >x_odd,+x_even</Group>
    <Group name="complement">x_odd,-x_first</Group>
    <Group name="intersect" >x_even,^x_first</Group>
    <Group name="sym_diff"  >x_odd,% x_first</Group>
  </VariableGroups>
  ...
  <DataObjects>
    <PointSet name="dataset">
      <Input>union</Input>
      <Output>y_group</Output>
    </PointSet>
  </DataObjects>
  ...
</Simulation>
\end{lstlisting}
